% Options for packages loaded elsewhere
\PassOptionsToPackage{unicode}{hyperref}
\PassOptionsToPackage{hyphens}{url}
\PassOptionsToPackage{dvipsnames,svgnames,x11names}{xcolor}
%
\documentclass[
  a5paper,
  smalldemyvopaper,11pt,twoside,onecolumn,openright,extrafontsizes]{memoir}

\usepackage{amsmath,amssymb}
\usepackage{iftex}
\ifPDFTeX
  \usepackage[T1]{fontenc}
  \usepackage[utf8]{inputenc}
  \usepackage{textcomp} % provide euro and other symbols
\else % if luatex or xetex
  \usepackage{unicode-math}
  \defaultfontfeatures{Scale=MatchLowercase}
  \defaultfontfeatures[\rmfamily]{Ligatures=TeX,Scale=1}
\fi
\usepackage{lmodern}
\ifPDFTeX\else  
    % xetex/luatex font selection
\fi
% Use upquote if available, for straight quotes in verbatim environments
\IfFileExists{upquote.sty}{\usepackage{upquote}}{}
\IfFileExists{microtype.sty}{% use microtype if available
  \usepackage[]{microtype}
  \UseMicrotypeSet[protrusion]{basicmath} % disable protrusion for tt fonts
}{}
\makeatletter
\@ifundefined{KOMAClassName}{% if non-KOMA class
  \IfFileExists{parskip.sty}{%
    \usepackage{parskip}
  }{% else
    \setlength{\parindent}{0pt}
    \setlength{\parskip}{6pt plus 2pt minus 1pt}}
}{% if KOMA class
  \KOMAoptions{parskip=half}}
\makeatother
\usepackage{xcolor}
\setlength{\emergencystretch}{3em} % prevent overfull lines
\setcounter{secnumdepth}{5}
% Make \paragraph and \subparagraph free-standing
\makeatletter
\ifx\paragraph\undefined\else
  \let\oldparagraph\paragraph
  \renewcommand{\paragraph}{
    \@ifstar
      \xxxParagraphStar
      \xxxParagraphNoStar
  }
  \newcommand{\xxxParagraphStar}[1]{\oldparagraph*{#1}\mbox{}}
  \newcommand{\xxxParagraphNoStar}[1]{\oldparagraph{#1}\mbox{}}
\fi
\ifx\subparagraph\undefined\else
  \let\oldsubparagraph\subparagraph
  \renewcommand{\subparagraph}{
    \@ifstar
      \xxxSubParagraphStar
      \xxxSubParagraphNoStar
  }
  \newcommand{\xxxSubParagraphStar}[1]{\oldsubparagraph*{#1}\mbox{}}
  \newcommand{\xxxSubParagraphNoStar}[1]{\oldsubparagraph{#1}\mbox{}}
\fi
\makeatother


\providecommand{\tightlist}{%
  \setlength{\itemsep}{0pt}\setlength{\parskip}{0pt}}\usepackage{longtable,booktabs,array}
\usepackage{calc} % for calculating minipage widths
% Correct order of tables after \paragraph or \subparagraph
\usepackage{etoolbox}
\makeatletter
\patchcmd\longtable{\par}{\if@noskipsec\mbox{}\fi\par}{}{}
\makeatother
% Allow footnotes in longtable head/foot
\IfFileExists{footnotehyper.sty}{\usepackage{footnotehyper}}{\usepackage{footnote}}
\makesavenoteenv{longtable}
\usepackage{graphicx}
\makeatletter
\def\maxwidth{\ifdim\Gin@nat@width>\linewidth\linewidth\else\Gin@nat@width\fi}
\def\maxheight{\ifdim\Gin@nat@height>\textheight\textheight\else\Gin@nat@height\fi}
\makeatother
% Scale images if necessary, so that they will not overflow the page
% margins by default, and it is still possible to overwrite the defaults
% using explicit options in \includegraphics[width, height, ...]{}
\setkeys{Gin}{width=\maxwidth,height=\maxheight,keepaspectratio}
% Set default figure placement to htbp
\makeatletter
\def\fps@figure{htbp}
\makeatother

\makeatletter
\@ifpackageloaded{bookmark}{}{\usepackage{bookmark}}
\makeatother
\makeatletter
\@ifpackageloaded{caption}{}{\usepackage{caption}}
\AtBeginDocument{%
\ifdefined\contentsname
  \renewcommand*\contentsname{Table of contents}
\else
  \newcommand\contentsname{Table of contents}
\fi
\ifdefined\listfigurename
  \renewcommand*\listfigurename{List of Figures}
\else
  \newcommand\listfigurename{List of Figures}
\fi
\ifdefined\listtablename
  \renewcommand*\listtablename{List of Tables}
\else
  \newcommand\listtablename{List of Tables}
\fi
\ifdefined\figurename
  \renewcommand*\figurename{Figure}
\else
  \newcommand\figurename{Figure}
\fi
\ifdefined\tablename
  \renewcommand*\tablename{Table}
\else
  \newcommand\tablename{Table}
\fi
}
\@ifpackageloaded{float}{}{\usepackage{float}}
\floatstyle{ruled}
\@ifundefined{c@chapter}{\newfloat{codelisting}{h}{lop}}{\newfloat{codelisting}{h}{lop}[chapter]}
\floatname{codelisting}{Listing}
\newcommand*\listoflistings{\listof{codelisting}{List of Listings}}
\makeatother
\makeatletter
\makeatother
\makeatletter
\@ifpackageloaded{caption}{}{\usepackage{caption}}
\@ifpackageloaded{subcaption}{}{\usepackage{subcaption}}
\makeatother

\ifLuaTeX
  \usepackage{selnolig}  % disable illegal ligatures
\fi
\usepackage{bookmark}

\IfFileExists{xurl.sty}{\usepackage{xurl}}{} % add URL line breaks if available
\urlstyle{same} % disable monospaced font for URLs
\hypersetup{
  pdftitle={Het Genesisboek},
  pdfauthor={Aaron van Wirdum},
  colorlinks=true,
  linkcolor={blue},
  filecolor={Maroon},
  citecolor={Blue},
  urlcolor={Blue},
  pdfcreator={LaTeX via pandoc}}


\title{Het Genesisboek}
\usepackage{etoolbox}
\makeatletter
\providecommand{\subtitle}[1]{% add subtitle to \maketitle
  \apptocmd{\@title}{\par {\large #1 \par}}{}{}
}
\makeatother
\subtitle{Het verhaal van de mensen en projecten die Bitcoin
inspireerden}
\author{Aaron van Wirdum}
\date{2024-09-12}

\begin{document}
\frontmatter
\maketitle

\renewcommand*\contentsname{Table of contents}
{
\hypersetup{linkcolor=}
\setcounter{tocdepth}{2}
\tableofcontents
}

\mainmatter
\bookmarksetup{startatroot}

\chapter*{Over dit boek}\label{over-dit-boek}
\addcontentsline{toc}{chapter}{Over dit boek}

\markboth{Over dit boek}{Over dit boek}

Bitcoin is niet uit het niets ontstaan. Gedurende tientallen jaren
voorafgaand aan de uitvinding van Satoshi Nakamoto hebben diverse
groepen computerwetenschappers, privacyactivisten en niet-traditionele
economen geprobeerd een digitale vorm van geld te creëren die
onafhankelijk van overheidscontrole kon opereren. Het Genesisboek
vertelt het verhaal van de mensen en projecten die de uitvinding van 's
werelds eerste succesvolle peer-to-peer elektronisch betaalsysteem
inspireerden.

\begin{quote}
``Het Genesisboek neemt je mee op een eeuwenlange reis door de minder
bekende verhalen van visionairs, wiens inzichten en innovaties de basis
vormden voor de revolutionaire creatie van Bitcoin. Van economen die de
gevestigde orde uitdaagden tot cypherpunks die nieuwe wegen verkenden op
het gebied van privacy, weeft Aaron van Wirdum zorgvuldig een verhaal
met technologische overwinningen, tegenslagen en bijzondere doorbraken.
De anekdotes over mensen die durfden te dromen buiten de gebaande paden
en grenzen verlegden om de financiële wereld te veranderen, zullen je
zeker fascineren.''

\textbf{--- Jameson Lopp}
\end{quote}

\begin{quote}
``Waarom verschilt Bitcoin zo van eerdere versies? Dit boek werpt licht
op de \emph{problemen} die slimme en hardwerkende mensen in de tijd vóór
Bitcoin bezig hielden. Dit is de juiste manier om het verhaal van een
technologie te vertellen. Alle belangrijke kwesties worden behandeld en
ze zijn in een logische volgorde gepresenteerd. Dit is het beste boek
over Bitcoin dat ooit geschreven is.''

\textbf{--- Paul Sztorc}
\end{quote}

\begin{quote}
``Tot nu toe waren er veel boeken over Bitcoin, maar geen enkel boek
pakte de veelzijdige culturele achtergrond zo compleet, gestructureerd
en elegant aan. Aaron van Wirdum, bekend om zijn talent om technische
materie helder over te brengen naar een breder publiek, heeft dat nu
gedaan. Dit is een must-read als je wilt begrijpen waar Bitcoin vandaan
komt.''

\textbf{--- Giacomo Zucco}
\end{quote}

\begin{quote}
``Ik had al enige tijd het vermoeden dat Van Wirdum de beste historicus
van Bitcoin is, en deze pageturner bevestigt dat.

Het is een indrukwekkend werk. Het Genesisboek biedt een toegankelijk en
essentieel overzicht van de geschiedenis, en onthult de vele verbanden
tussen het Weense klassiek liberalisme, de Anglo-Saksische
cypherpunkbeweging en de opkomst van Bitcoin.

Waar andere boeken in de branche vaak de nadruk leggen op veelgeprezen
ondernemers, heeft Van Wirdum de technische kennis om de belangrijke
figuren te belichten die de fundamenten hebben gelegd waarop het
Bitcoin-gebouw later is opgericht.

In zestien informatieve hoofdstukken biedt Het Genesisboek een
combinatie van diepgaand onderzoek en filosofische inzichten die je
alleen verwacht van een veteraan in de sector (Van Wirdum was een van de
eerste schrijvers die in de Bitcoin-industrie aan de slag ging), met de
aantrekkelijke schrijfstijl die je zoekt in een gerespecteerd
tijdschrift.

Je kunt Bitcoin niet begrijpen zonder zijn bijzondere oorsprong te leren
kennen, en ik ben blij dat dit boek bestaat om die kennis toegankelijk
te maken voor een breed publiek.''

\textbf{--- Tuur Demeester}
\end{quote}

\bookmarksetup{startatroot}

\chapter*{Inleiding}\label{inleiding}
\addcontentsline{toc}{chapter}{Inleiding}

\markboth{Inleiding}{Inleiding}

E-goud was in volle bloei. Tegen 2005 waren er op het innovatieve online
betalingssysteem van Douglas Jackson meer dan een miljoen accounts
aangemaakt, die samen verantwoordelijk waren voor bijna € 2 miljard aan
transacties per jaar. De volledig gedekte digitale tokens die
e-goudklanten gebruikten om al deze transacties uit te voeren,
vertegenwoordigden 3,8 ton goud, opgeborgen in kluizen over de hele
wereld. Als een van de eerste werkende implementaties van elektronisch
geld, was e-goud in minder dan tien jaar tijd uitgegroeid tot de
populairste digitale valuta op het internet.

Maar Jackson stond een onaangename verrassing te wachten.

In december 2005 voerde de Amerikaanse geheime dienst, net voor het
einde van het jaar, een inval uit bij het bedrijf van Jackson en zijn
woning in Melbourne, Florida. Ze namen boeken en administratieve
gegevens mee, en federale agenten namen alles wat zelfs maar enigszins
interessant leek in beslag: naast juridische documenten en zakelijke
contracten, omvatte dit ook het adresboek van zijn vrouw, de paspoorten
van hun kinderen en de creditcards van op het nachtkastje. Op datzelfde
moment werden de servers van e-goud, in een AT\&T-gebouw in Orlando,
zo'n 110 kilometer verder naar het noordwesten, offline gehaald, en
werden alle transactiegegevens in beslag genomen.

De Amerikaanse geheime dienst, bijgestaan door de IRS en de FBI, was van
mening dat de betalingswerker een broedplaats was geworden voor
criminelen, die simpelweg met behulp van een emailadres e-goudaccounts
konden aanmaken --- dus vrijwel anoniem. In een tijdperk waarin
creditcardfraude op het internet hoogtij vierde, zou het betaalsysteem
van Jackson als een magneet voor oplichters hebben gefunctioneerd. Erger
nog, wetshandhavers vermoedden dat kinderpornografen en terroristen van
de relatieve anonimiteit van e-goud misbruik maakten.

Jackson werd aangeklaagd en beschuldigd van het witwassen van geld en
het runnen van een niet-gelicentieerde betaaldienst.\footnote{\hspace{0pt}Lawrence
  H. White, `The Troubling Suppression of Competition from Alternative
  Monies: The Cases of the Liberty Dollar and e-gold', Cato Journal, 34,
  No.~2: 281--301.}

\section*{Digitaal goud}\label{digitaal-goud}
\addcontentsline{toc}{section}{Digitaal goud}

\markright{Digitaal goud}

Jackson had nooit beoogd dat e-goud misbruikt zou worden voor illegale
doeleinden. Hij geloofde ook niet dat dit op een serieuze schaal
gebeurde. Sterker nog, hij beweerde dat e-goud een beter
fraudedetectiesysteem had dan elke andere betalingswerker die er
bestond. Daarnaast was hij altijd meer dan bereid om met politie samen
te werken. E-goud was ook een van de oprichtende leden van het `National
Center for Missing \& Exploited Children's Financial Coalition Against
Child Pornography'. De gegevens die deze coalitie verzamelde, zo stelde
Jackson, gaven aan dat e-goud vrijwel niet werd gebruikt voor dergelijke
doeleinden.

In plaats daarvan was Jackson, als succesvol en onafhankelijk oncoloog
en veteraan van het medisch korps van het Amerikaanse leger, in de jaren
negentig al geïnteresseerd geraakt in het monetaire beleid en de invloed
daarvan op de economie. Hij ontdekte dat moderne valuta's -- dollars,
ponden, yen -- niet langer door iets gedekt werden, waardoor ze in
feite, uit het niets, met een druk op de knop gecreëerd konden worden.
Toen hij zich verder in dit onderwerp verdiepte, raakte hij ervan
overtuigd dat dit de economie op zeer schadelijke manieren beïnvloedde.

Jackson was dus van plan om een alternatief aan te bieden.

Tijdens zijn onderzoek naar valuta's, ontwikkelde Jackson een hernieuwde
waardering voor het `klassieke geld' -- goud. Hij ontdekte dat mensen
ten minste sinds de predynastieke Egyptische tijd waarde hechtten aan
dit glanzende, gele metaal, en dat met goede reden: het natuurlijke
element werd niet beïnvloed door de willekeur van mensen.

Echter, het pre-dynastieke Egypte was allang verdwenen en zelfs Jackson
moest toegeven dat het kostbare metaal niet echt praktisch was voor
dagelijkse transacties. Nu het nieuwe millennium naderde, besefte
Jackson dat mensen niet zouden terugkeren naar de tijd waarin ze
betaalden met gouden munten. Sterker nog, zelfs koperen munten en
papiergeld zouden waarschijnlijk binnenkort ouderwets lijken.

Nee, de toekomst van geld moest digitaal zijn.

Met dat toekomstbeeld zag Jackson (heel letterlijk) een gouden kans. Hij
bundelde de krachten met advocaat Barry Downey en richtte in 1996 `Gold
\& Silver Reserve Inc.' op, onder zijn leiderschap. De start-up zou een
betalingssysteem voor de eenentwintigste eeuw runnen, maar dan wel
gebaseerd op dat klassieke geld. Ze zouden een elektronisch equivalent
voor goud leveren: \emph{e-gold}.

Het basisidee was simpel. \emph{Gold \& Silver Reserve Inc.} huurde
kluizen die ze vulden met het fysieke, gouden metaal zelf. Voor elk stuk
goud in deze kluizen gaf het bedrijf een digitale `token' uit --- in
feite een nummer in een database. Deze tokens vertegenwoordigden een
claim op het goud. Als iemand tokens had die gelijk stonden aan tien
gram goud, was tien gram goud in een van de kluizen wettelijk van hen.

De belangrijkste innovatie was dat \emph{Gold \& Silver Reserve Inc.}
ook een server in stand hield die een openbaar toegankelijk
boekhoudsysteem voor de tokens huisvestte. Mensen van over de hele
wereld konden inloggen op de server en een persoonlijke rekening
aanmaken, waardoor ze tokens naar en van elke andere rekening konden
sturen en ontvangen. Bij elke transactie heeft \emph{Gold \& Silver
Reserve Inc.} de rekeningsaldi dienovereenkomstig bijgewerkt.

Dankzij de kracht van het internet, konden e-goud gebruikers elkaar dus
in wezen over grote afstanden, direct, betalen tegen minimale kosten. Op
de grenzeloze \emph{informatiesnelweg} kon iedereen met een
internetverbinding iemand anders betalen, zonder beperkingen met
betrekking tot nationale grenzen of bankregels.

Jackson creëerde e-goud, zo zei hij vaak, als een instelling om het
materiële welzijn van de mensheid te bevorderen door toegang te bieden
tot wereldwijde markten.

`In tegenstelling tot andere, is e-goud een betalingssysteem dat mensen
van elke regio of economische achtergrond wereldwijd laat opereren: een
migrant kan gemakkelijk waarde naar huis sturen en een handelaar kan
betalingen accepteren van iemand in een derdewereldland die misschien
geen toegang heeft tot een creditcard of bankrekening.'\footnote{Huis
  van Afgevaardigden, `Deleting Commercial Pornography Sites From the
  Internet: The U.S. Financial Industry's Efforts to Combat This
  Problem', Hearing Before the Subcommittee on Oversight and
  Investigations of the Committee on Energy and Commerce, One Hundred
  Ninth Congress, Second Session, 21 september
  2006,~\href{https://www.govinfo.gov/content/pkg/CHRG-109hhrg31467/html/CHRG-109hhrg31467.htm}{online}}

Bovendien beweerde Jackson dat e-goud de mogelijkheid bood om een soort
geld te gebruiken dat bestand is tegen inflatie. Omdat het digitaal was,
was e-goud eigenlijk voor veel mensen toegankelijker dan echt goud.

Op de lange termijn had e-goud zelfs het potentieel om de ruggengraat te
worden van een geheel nieuw, financieel systeem, suggereerde Jackson
optimistisch.

`Hoe vinden we een bankensysteem uit dat niet de oorzaak zal zijn van
catastrofale verstoringen, dat zelf het minst waarschijnlijk is om
schommelingen te introduceren en dat het meest waarschijnlijk de juiste
aanpassingen zal maken\ldots{} is het meest prangende, onopgeloste
economische probleem van onze tijd', citeerde hij op een gegeven moment
uit het boek \emph{The Rationale of Central Banking and the Free Banking
Alternative} van econome Vera Smith.

Toe te voegen: `Een systeem en munteenheid zoals e-goud, vooral na
opkomst en integratie in de financiële mainstream als een
reserve-activum dat als betaalmiddel wordt gebruikt, kunnen dit probleem
zeker en vast oplossen.'\footnote{\hspace{0pt}White, `Troubling
  Suppression', 289.}

\section*{Juridische kwesties}\label{juridische-kwesties}
\addcontentsline{toc}{section}{Juridische kwesties}

\markright{Juridische kwesties}

Vroeg in de jaren 2000 was e-goud snel aan het groeien, en Jackson
werkte onvermoeid om zijn dienst te verbeteren. Hij stelde meer soorten
edelmetalen beschikbaar, en voegde tevens nieuwe betalingsfuncties toe,
zoals geautomatiseerde, maandelijkse betalingen. Bovendien maakte hij
toegang tot het systeem mogelijk via mobiele telefoons via het toen
nieuwe WAP-protocol.\footnote{\hspace{0pt}e-gold, `e-gold News',
  December 1999, geraadpleegd
  \href{https://web.archive.org/web/20001209053900fw_/http://www.e-gold.com/news.html.}{online}}

Maar Jackson was iets vergeten: hij had zijn bedrijf niet geregistreerd
als betaaldienst. Daardoor voerde hij ook niet alle types van
\emph{Know-Your-Customer} (KYC) en \emph{Anti-Money Laundering} (AML)
controles uit die vereist zijn voor een betaaldienst. Hij was zich er
niet van bewust dat hij dat moest doen.

Het was niet zo dat hij nonchalant was. Het werd pas een federale
misdaad om zonder licentie als betaaldienst te opereren (voor elke staat
die er een vereist) na de invoering van de \emph{Patriot Act}, die in
het leven werd geroepen als reactie op de terroristische aanslagen van
11 september 2001, enkele jaren na de lancering van e-goud. Nog
belangrijker: het was niet duidelijk dat Jacksons onderneming überhaupt
als een betaaldienst beschouwd zou worden: het e-goudsysteem verstuurde
geen dollars of andere nationale valuta, waarvoor zulke regels doorgaans
van toepassing waren.

Desondanks had Jackson geprobeerd meer duidelijkheid te krijgen over de
kwestie. \emph{Gold \& Silver Reserve Inc.} had zelf aan de relevante
overheidsinstanties voorgesteld dat e-goud gecategoriseerd kon worden
als een valuta voor reguleringsdoeleinden, waardoor het bedrijf zich ook
als een valutawisseldienst kon registreren. Maar als reactie hierop had
het Amerikaanse ministerie van Financiën opnieuw bevestigd dat de
definities van valuta niet van toepassingen waren op e-goud.

Bovendien had Jackson vrijwillig een conformiteitsonderzoek op de
Bankgeheimwet gestart bij een agentschap van het ministerie van
Financiën, puur om erachter te komen hoe zij dachten dat zijn bedrijf
gereguleerd zou moeten worden.\footnote{\hspace{0pt}e-gold, `e-gold
  Welcomes US Government Review of its Status as a Privately Issued
  Currency', 20 januari 2006, geraadpleegd
  \href{https://web.archive.org/web/20060304203618if_/http://www.e-gold.com/letter2.html.}{online}}

Toen de invallen plaatsvonden, wachtte hij nog op een antwoord.

De daaropvolgende juridische procedures brachten ernstige schade toe aan
de e-goudonderneming. Bankrekeningen werden bevroren en bedrijfsgelden
in beslag genomen. De juridische strijd die zich ontvouwde tussen
Jackson en de Amerikaanse regering duurde twee jaar en putte zijn
middelen uit: de juridische kosten zouden uiteindelijk oplopen tot in de
miljoenen. En voor zover het bedrijf van Jackson nog kon functioneren,
moest dat nu gebeuren onder een zweem van verdenking.

Intussen had de Amerikaanse overheid beslagleggingsbevelen uitgevaardigd
om achtenvijftig grote e-goudaccounts te sluiten op verdenking van
witwassen. De doelwitten van de actie waren onafhankelijke
e-goudbeurzen, waarvan sommige in het buitenland waren gevestigd. Door
gebruik te maken van de \emph{Racketeering Act} uit 1961 (een instrument
voor handhaving van de wet, oorspronkelijk opgesteld om georganiseerde
misdaad te bestrijden) werd 1.000 kilogram goud dat deze accounts
ondersteunde (ongeveer een kwart van de totale voorraad van e-goud) in
beslag genomen en geliquideerd.

Toen er in 2008 eindelijk een voorlopig vonnis kwam, bepaalde de rechter
dat e-goud inderdaad een betaaldienst was, en verwierp daarmee Jacksons
verzoek om de zaak te seponeren. Aangezien hij nu geconfronteerd werd
met de mogelijkheid op een aanzienlijke gevangenisstraf en enorme
boetes, besloot Jackson een schikking te treffen.\footnote{\hspace{0pt}US
  Department of Justice, `Digital Currency Business e-gold Pleads Guilty
  to Money Laundering and Illegal Money Transmitting Charges', 21 juli
  2008,
  \href{https://www.justice.gov/archive/opa/pr/2008/July/08-crm-635.html}{online}}

In een van de weinige positieve wendingen in het hele verhaal toonde de
rechter enige mildheid in haar uiteindelijke vonnis. Ze stelde dat `de
intentie om illegale activiteiten te ondernemen er niet was'.\footnote{\hspace{0pt}e-gold
  `Transcript of sentence before the honorable Rosemarie M. Collyer
  United States District Judge', 114, 20 november 2008,
  \href{https://legalupdate.e-gold.com/2008/11/transcript-of-sentence-before-the-honorable-rosemarie-m-collyer-united-states-district-judge.html}{online}}
Toch werd Jackson veroordeeld tot zesendertig maanden voorwaardelijke
vrijlating (huisarrest), waarvan zes werden afgedwongen door middel van
een enkelband. Hij moest ook 300 uur openbare dienstverlening uitvoeren
en een boete van € 200 betalen. Zijn bedrijf kreeg ondertussen een boete
van € 600.000. Twee van zijn werknemers -- mede-oprichters Barry Downey
en Douglas' broer Reid Jackson -- werden veroordeeld tot zesendertig
maanden voorwaardelijke vrijlating, 300 uur dienstverlening aan de
gemeenschap en een boete van € 2.500 plus een boete van € 100.

En natuurlijk moest e-goud een licentie verkrijgen voor het opereren van
een betaaldienst. Het enige probleem? Als veroordeelde misdadiger kwam
Jackson niet langer in aanmerking voor zo'n licentie - iets wat hij niet
onmiddellijk had beseft toen hij instemde met de schikking. Net toen hij
dacht dat hij eindelijk de juridische strijd voor altijd kon achterlaten
en zijn bedrijven op welke manier dan ook kon proberen te redden,
ontdekte Jackson dat dit niet onder zijn leiding kon gebeuren.

Uiteindelijk heeft e-goud nooit opnieuw de deuren geopend.

Jackson had e-goud opgericht om het materiële welzijn van mensen te
verbeteren door een alternatief te bieden voor conventionele, ongedekte
valuta's zoals de Amerikaanse dollar. Opgesloten in zijn eigen huis, met
meer dan een miljoen dollar aan juridische kosten op zijn naam, en zijn
bedrijf noodgedwongen gesloten, had hij op de pijnlijke les geleerd dat
het aanbieden van een dergelijk alternatief niet zonder slag of stoot
ging.

\section*{Satoshi Nakamoto}\label{satoshi-nakamoto}
\addcontentsline{toc}{section}{Satoshi Nakamoto}

\markright{Satoshi Nakamoto}

Het lot van Douglas Jackson en e-goud diende als een niet te miskennen
waarschuwing voor iedereen met ambities om een alternatieve vorm van
geld aan te bieden. Overheden (en met name de Amerikaanse overheid)
konden besluiten hard op te treden, wat ernstige persoonlijke en
financiële schade zou kunnen berokkenen. Voor de meesten was dit
waarschijnlijk het risico niet waard.

Echter, dit hield een onbekend persoon of groep, alleen bekend als
`Satoshi Nakamoto', niet tegen. Rond dezelfde tijd dat Jackson zijn
dagen thuis doorbracht met een enkelband om, was Nakamoto bezig met het
voorbereiden van de lancering van een eigen elektronisch geldsysteem.

Het ontwerp van Nakamoto's digitale valutasysteem was echter zeer
verschillend van e-goud. En hoewel er niet veel bekend is over de
achtergrond of beweegredenen van Satoshi Nakamoto, is het duidelijk dat
deze mysterieuze entiteit (de naam is vrijwel zeker een pseudoniem) zijn
eigen systeem bewust zodanig ontwierp om een soortgelijk lot als e-goud
te vermijden.

Dit ontwerp was waarschijnlijk ook niet het resultaat van een spontane
ingeving. Al jaren, zelfs ruim voor Jackson e-goud lanceerde, probeerde
een kleine maar toegewijde groep technici een digitale vorm van contant
geld te creëren: ze wisselden ideeën uit, ontwikkelden technieken, en
ontworpen diverse voorstellen. Dit alles in de hoop dichter bij een
werkende oplossing te komen. Maar het succes bleef uit.

Totdat Nakamoto eindelijk de puzzelstukjes op hun plek wist te krijgen.

In dit boek gaan we terug naar de ideeën en technologieën die Satoshi
Nakamoto (waarschijnlijk) hebben geïnspireerd in het ontwikkelen van dit
elektronisch geldsysteem.

In Deel I onderzoekt het boek de gevarieerde oorsprong van sommige van
deze grondleggende ideeën en technologieën die de basis vormden voor
elektronisch geld. Deze lopen uiteen van heterodoxe zienswijzen over
monetaire economie tot een opstandige revolutie in de cryptografie, en
van de opkomst van de hackercultuur in de jaren '60 en '70 tot
techno-utopische visies op ruimtekolonisatie, moleculaire
nanotechnologie en eeuwig leven.

Deel II vertelt het verhaal van de Cypherpunks: een verzameling
cryptografen, hackers en privacyactivisten die gedurende de jaren '90
privacyhulpmiddelen ontwikkelden en verspreidden voor het internet, en
die probeerden een elektronische versie van contant geld te creëren. Dit
deel van het boek focust ook op enkele specifieke pogingen om dergelijke
elektronische betalingssystemen te ontwikkelen.

Tot slot beschrijft Deel III van het boek hoe Satoshi Nakamoto zijn
elektronische geldsysteem ontwierp en ontwikkelde, wat de inspiratiebron
was voor dit ontwerp, en hoe het zich verhoudt tot andere vormen van
(digitaal) geld.

Samen vormen zij het verhaal van de monetaire hervormers,
computerwetenschappers, activisten voor privacy, futurologen,
ondernemers en andere pioniers die, elk op hun eigen manier, bijdroegen
aan de opkomst van het eerste succesvolle peer-to-peer elektronische
geldsysteem ter wereld: Bitcoin.

\part{Deel I: Grondslagen}

\chapter{Spontane orde}\label{spontane-orde}

Friedrich August von Hayek wilde net als zijn vader hoogleraar in de
biologie worden, maar de Eerste Wereldoorlog zorgde voor een volledige
verandering van zijn levenspad.\footnote{\hspace{0pt}Deirdre N.
  McCloskey, How to be Human -- Though an Economist, 33.} Hij werd
geboren in 1899 en groeide op tijdens de laatste jaren van het
Oostenrijks-Hongaarse Rijk. Nadat hij achttien was geworden, werd hij
opgeroepen om te vechten aan het Italiaanse front. Het laatste deel van
de oorlog bracht hij door als waarnemer in vliegtuigen.

Toen hij in 1918 na het einde van de oorlog thuis kwam, vond Hayek (het
aristocratische voorvoegsel `von' werd na de ineenstorting van de
dubbele monarchie weggelaten) zijn woonplaats Wenen terug in totale
verwoesting. De oorlog was verloren, de economie vernietigd en het rijk
was niet meer. Het moreel in de stad was gebroken.

Om de zaken nog erger te maken, spendeerde de nieuwe Oostenrijkse
regering zoveel om de naoorlogse kosten te dekken dat de waarde van hun
nationale munteenheid enorm kelderde. Hoewel de krone al meer dan 90
procent van zijn koopkracht had verloren tijdens de oorlog, zou dit in
de naoorlogse jaren echt uit de hand lopen. Terwijl een Amerikaanse
dollar voor ongeveer negen kronen verkocht werd in 1917, kon diezelfde
dollar tegen 1923 meer dan 70.000 van de Oostenrijkse munteenheden
opbrengen. Het nationale geld was in feite vernietigd.\footnote{\hspace{0pt}Lawrence
  H. Officer, `Exchange Rates Between the United States Dollar and
  Forty-one Currencies', MeasuringWorth, 2023.}

Nadat hij van dichtbij geconfronteerd was met de verschrikkingen van de
Grote Oorlog, waarin bijna achttien miljoen mannen en vrouwen het leven
lieten, besloot Hayek om zijn tijd en energie te besteden aan het
proberen te voorkomen van een herhaling van dergelijke dramatische
conflicten in de toekomst. Hij was vastberaden om betere manieren te
vinden om de maatschappij te organiseren.

Als leergierig persoon uit een hoogopgeleid gezin -- beide grootvaders
van hem waren ook academici -- schreef Hayek zich in aan de Universiteit
van Wenen, de oudste universiteit in de Duitstalige wereld en een van de
meest gerenommeerde academische instituten in heel Europa. Gemotiveerd
door zijn nieuwe missie, koos Hayek ervoor om politieke wetenschappen en
recht te studeren, terwijl hij naast zijn hoofdstudies ook lessen in
filosofie, psychologie en economie volgde.

Hij schreef zich niet meteen in voor alle economielessen. Voor de
socialistisch-geïnspireerde Hayek leek de economieprofessor aan de
universiteit een beetje te hard te denken vanuit vrije marktprincipes.
Pas toen diezelfde economieprofessor Hayek in dienst nam om een
tijdelijk overheidskantoor in de stad te bemannen, besloot hij eindelijk
zijn lessen een kans te geven.\footnote{\hspace{0pt}Eamonn Butler,
  `Hayek: His Contribution to the Political and Economic Thought of Our
  Time'.}

De naam van de professor was Ludwig von Mises. Hayek kwam al snel te
weten dat Mises een vooraanstaand econoom was binnen een relatief nieuwe
school van economisch denken.\footnote{\hspace{0pt}Bruce Caldwell and
  Hansjoerg Klausinger, `Hayek: A Life, 1899--1950, Chs. 6--9'.}

\section{Oostenrijkse economie}\label{oostenrijkse-economie}

De Eerste Wereldoorlog was het gewelddadige hoogtepunt van een tijdperk
dat sterk doordrenkt was van nationalisme, de ideologie die stelt dat
collectieven van mensen met een gemeenschappelijke herkomst,
geschiedenis, cultuur of taal --- \emph{naties} --- zichzelf als staten
zouden moeten organiseren en handelen in het belang van deze staten.

Het nationalisme had in de negentiende eeuw ook invloed op de
economische wetenschap. Waar de \emph{klassieke economie}, met haar
sterke nadruk op vrije markten zoals voorgesteld door baanbrekende
economen zoals David Hume, Adam Smith, en David Ricardo, dominant was in
de late achttiende eeuw, begonnen Europese universiteiten gedurende de
jaren 1800 de methoden van de \emph{historische school van economie} te
omarmen. De meest invloedrijke experts pleitten voor staatsinterventies
in de economie, zoals arbeidswetten, beschermde heffingen, en
progressieve belastingen.\footnote{\hspace{0pt}Ludwig von Mises, `The
  Historical Setting of the Austrian School of Economics', 12.}

De methodologie van de historische economische school -- de verzameling
methoden die gebruikt worden om de economie te bestuderen -- sloot
algemene economische theorieën uit en stelde dat de `regels' waaraan
economieën voldoen verschillen per cultuur en tijd. In tegenstelling tot
het opstellen van modellen of theorema's, verzamelden historische
economen grote hoeveelheden historische data voor empirische analyse.

Maar professor Carl Menger van de Universiteit van Wenen had deze aanpak
in de jaren 1870 al afgewezen. Hij geloofde dat mensen en menselijke
interacties te complex waren om enkel op basis van empirische gegevens
waardevolle wetenschappelijke inzichten te kunnen afleiden. Een
ontelbare hoeveelheid factoren beïnvloedt de gedachten en acties van een
typisch persoon, redeneerde hij --- het is onbegonnen werk om het aantal
factoren die een hele samenleving beïnvloeden te bestuderen. Geen enkele
hoeveelheid empirische data zou groot genoeg kunnen zijn om al deze
factoren te omvatten, geloofde Menger. Elke conclusie die uit zo'n
dataset wordt getrokken, zou nooit overtuigend zijn.

In plaats daarvan betoogde Menger dat economen zouden moeten proberen om
economische verschijnselen te begrijpen en uit te leggen door deductieve
redenering. Door te beginnen met basisprincipes, kon logisch redeneren
leiden tot onweerlegbare inzichten die het wetenschappelijke begrip van
economische processen \emph{a priori} zouden uitbreiden (De Latijnse
term \emph{a priori} verwijst naar kennis die onafhankelijk is van
ervaring, zoals wiskunde, in tegenstelling tot \emph{a posteriori}
kennis die afhankelijk is van empirisch bewijs, zoals meer typisch is in
de meeste andere wetenschappelijke vakgebieden.).

Menger bracht deze aanpak in de praktijk in zijn boek uit 1871 getiteld
\emph{Grundsätze der Volkswirtschaftslehre} (`Grondbeginselen van de
Economie'). Hierin schetste hij de theorie van het marginale nut, die
stelt dat de prijs van goederen en diensten deels afhangt van hoeveel
extra voldoening men krijgt door er meer van te hebben.\footnote{\hspace{0pt}Mises,
  `Historical Setting', 12--13, 19--20.}

Het boek en de methode van Menger betekende een fundamentele
verschuiving in denkwijze. Tot dan toe hadden economen, zowel uit de
klassieke als uit de historische school, altijd aangenomen dat de waarde
van een product werd afgeleid van zijn productiekosten. Zij stelden dat
een paar schoenen waardevol is omdat de productie ervan kosten met zich
meebrengt --- met name de kosten van arbeid, leer en benodigdheden. De
reden dat het leer en de benodigdheden kosten met zich meebrengen, is op
hun beurt omdat het produceren van het leer en de benodigdheden eveneens
arbeid vereist (en mogelijk ook andere kosten). Dit werd de
arbeidswaardetheorie genoemd.

Volgens de theorie van het marginale nut, stelde Menger dat waarde
subjectief is: individuen waarderen producten en diensten als deze een
persoonlijke behoefte of verlangen vervullen. De waarde van een paar
schoenen komt niet voort uit de productiekosten, maar uit het feit dat
\emph{mensen het dragen van schoenen waarderen}.

Dit betekent dat de waarde van een bepaald product kan variëren van
persoon tot persoon. Iemand die helemaal geen schoenen heeft, zal een
nieuw paar waarschijnlijk meer waarderen dan iemand die al meerdere
paren bezit. Op dezelfde manier kan \emph{dezelfde} persoon hetzelfde
product op verschillende tijdstippen anders waarderen. Nadat de persoon
zonder schoenen in het vorige voorbeeld een paar heeft verworven,
waardeert hij waarschijnlijk een tweede, identiek paar schoenen niet
even hoog als het eerste.\footnote{\hspace{0pt}Ludwig von Mises, Human
  Action: A Treatise on Economics, The Scholar's Edition, 21, 38--54.}

Met deze \emph{subjectieve theorie van waarde}, plaatste Menger het
individu opnieuw in het centrum van de economie. Hij stelde dat niet
landen of andere collectieven de drijvende kracht waren achter
economische beslissingen, maar mensen en hun subjectieve voorkeuren. In
plaats van de staat als uitgangspunt te nemen voor analyse, geloofde
Menger dus dat de studie van economie moest beginnen door te begrijpen
wat de kleinste onderdelen van elk economisch systeem beïnvloedt.
Precies, de individuen.

Met zijn benadering, die wellicht het best te beschrijven is als een
herleving van de klassieke economie gericht op individuele subjectieve
ervaring, won Carl Menger de steun van verschillende van zijn collega's
aan de Universiteit van Wenen. Door de publicatie van zijn tweede boek
in de jaren 1880,\footnote{\hspace{0pt}Carl Menger, `Untersuchungen über
  die Methode der Sozialwissenschaften und der Politischen Oekonomie
  insbesondere'.} had Menger een filosofisch debat aangewakkerd over de
methodologie van de economische wetenschap binnen Duitstalige
universiteiten.

Tijdens deze soms vijandige \emph{Methodenstreit} (`strijd der
methodes'), begonnen Duitse economen -- die sterk neigden naar de
historische school -- ietwat minachtend naar Mengers aanpak te verwijzen
als de `Oostenrijkse school van economie'. Hoewel oorspronkelijk bedoeld
als een sneer (in die tijd associeerden de Duitsers het predicaat
`Oostenrijks' met het verlies van Oostenrijk in de
Oostenrijks-Pruisische oorlog van 1866), bleef de naam in gebruik.
Economen die Mengers methodologie aannamen, werden sindsdien aangeduid
als Oostenrijkse economen, zelfs wanneer ze niet uit Oostenrijk
afkomstig waren.\footnote{\hspace{0pt}Mises, `Historical Setting',
  3--19.}

De vijandige sfeer tijdens de Methodenstreit in de late negentiende eeuw
bereikte zijn hoogtepunt met de \emph{de facto} verbanning van de
Oostenrijkse economie uit Duitse universiteiten. De boycott zou decennia
van kracht blijven, en verhinderdde in grote mate dat Mengers ideeën
zich door de nieuwe, verenigde natiestaat verspreidden. In plaats
daarvan bleef het nationalisme domineren, terwijl een andere
collectivistische ideologie zich zonder veel wezenlijk tegengas ook
begon te verspreiden doorheen de Duitse universiteiten: het socialisme
was in opkomst.

\section{Economische berekening}\label{economische-berekening}

Oorspronkelijk gepopulariseerd door de Duitse auteur en sociale
commentator Karl Marx, geloofden socialisten dat de economische
geschiedenis van de wereld het best te begrijpen was als een
klassenstrijd tussen degenen die kapitaal bezitten (goederen die kunnen
worden gebruikt als productiemiddel, zoals fabrieken en hun machines) en
de arbeidersklasse, die alleen hun arbeid te verkopen hebben. Marx
voorspelde dat deze strijd in het voordeel van de kapitaalbezittende
klasse (de \emph{kapitalisten}) zou uitdraaien, omdat ze steeds meer
kapitaal en eindeloos groeiende winsten zouden genieten, totdat de
arbeidersklasse (het proletariaat) onvermijdelijk in opstand zou komen.

Volgens Marx was de uiteindelijke oplossing voor economische
ongelijkheid het socialisme, een economisch systeem waarin de
productiemiddelen onder gemeenschappelijk eigendom worden gebracht en
hun opbrengsten over de hele samenleving worden verdeeld. Dit zou in
eerste instantie onder toezicht van de staat moeten gebeuren, om
geleidelijk aan vervangen te worden door een anarchistische vorm van
zelfbestuur.

Hoewel de ideeën van Marx pas echt populair leken te worden na zijn dood
in 1883, hadden ze ook een behoorlijk aantal critici. Een vaak gehoord
bezwaar was dat mensen geen stimulans zouden hebben om te werken in een
socialistisch systeem, aangezien ze toch een vast aandeel van alle
geproduceerde goederen zouden ontvangen, terwijl de goederen die ze zelf
hielpen te maken, verspreid zouden worden over de rest van de
samenleving. Een tweede bezwaar was het risico dat socialistische
leiders zich tegen hun eigen bevolking zouden keren en veel van de onder
staatsbeheer geproduceerde goederen voor zichzelf zouden opeisen, in
plaats van ze eerlijk te verdelen.

Desondanks heeft het de opkomst van de socialistische leer in het
Russische Rijk niet gestopt. In 1917, midden in de Eerste Wereldoorlog,
wierpen revolutionairen, georganiseerd via arbeidersraden bekend als
`Soviets', de zittende regering omver en richtten de Sovjet-Unie op als
een communistische staat.

Ongeveer drie jaar na deze gebeurtenissen bracht Mises, de professor die
Hayek zijn overheidsbaan had aangeboden, een baanbrekende nieuwe kritiek
op het socialisme.\footnote{\hspace{0pt}Ludwig von Mises, `Economic
  Calculation in the Socialist Commonwealth'.} Belangrijk om te
vermelden, is dat deze kritiek stand zou houden \emph{zelfs} als mensen
gemotiveerd waren om te werken, en \emph{zelfs} als socialistische
leiders zich inzetten voor een eerlijke verdeling van de economische
winsten. Mises beweerde echter dat het fundamentele probleem van het
socialisme het ontbreken van een direct feedbackmechanisme was om
producenten te informeren of ze überhaupt waarde toevoegden aan de
samenleving.

Laten we een autofabriek nemen om dit argument te illustreren. In een
vrije markt voegt een fabriek die auto's produceert en winst maakt
duidelijk waarde toe aan de maatschappij: mensen zijn bereid meer te
betalen voor auto's dan de fabriek moet betalen voor de benodigde
hulpbronnen --- staal, machines, arbeidskrachten --- om ze te
produceren. Winst wijst erop dat de productie van de fabriek meer wordt
gewaardeerd dan de input.

Een autofabriek die verlies draait daarentegen, voegt duidelijk geen
waarde toe aan de maatschappij, aangezien mensen de gebruikte middelen
meer waarderen dan het eindproduct. Uiteindelijk zal zo'n fabriek de
deuren moeten sluiten. De middelen die de fabriek gebruikte, kunnen nu
worden gekocht (of in het geval van arbeid, ingehuurd) door
winstgevendere bedrijven om beter ingezet te worden (De Oostenrijkse
econoom Joseph Schumpeter zou dit later `creatieve destructie' noemen.).

In een socialistische samenleving zou een staatsgeleide autofabriek op
bevel van een centrale planner auto's produceren. Wanneer de auto's op
bevel geproduceerd worden, is er geen terugkoppeling van de maatschappij
in de vorm van winst of verlies. De autofabriek kan mogelijk middelen
verspillen aan het maken van auto's die mensen niet waarderen, of niet
zo hoog waarderen als andere producten die gemaakt hadden kunnen worden
met dezelfde middelen.

Zonder vrije markt is er geen \emph{economische berekening} mogelijk,
wat de fundamentele taak van ieder economisch systeem onmogelijk maakt:
de efficiënte verdeling van schaarse middelen over de
samenleving.\footnote{Hoewel dit argument inderdaad is aangevoerd in de
  context van consumptiegoederen, is een meer precieze verwoording van
  dit argument, uitgewerkt in latere economische debatten, dat dit
  vooral van toepassing is op kapitaalgoederen.}

`Zonder economische berekening kan er geen economie zijn', concludeerde
Mises. `Daarom kan er in een socialistische staat, waarin het nastreven
van economische berekening onmogelijk is, in onze betekenis van het
woord, geen economie bestaan.'\footnote{\hspace{0pt}Mises, `Economic
  Calculation', 18.}

\section{Prijzen}\label{prijzen}

Mises, en met name zijn concept van economische berekening, zou een
grote invloed hebben op Hayek. Aan de Universiteit van Wenen groeide hij
uit tot een enthousiaste student van de Oostenrijkse school van
economie. Hij bestudeerde de werken van Menger, evenals andere `eerste
generatie'-Oostenrijkers zoals Eugen von Böhm-Bawerk. Hij werd ook een
vaste bezoeker van privé-seminaries die Mises tweemaal per maand
organiseerde in zijn overheidskantoor. Daar kwam een kleine groep
geleerden samen om economische theorie, filosofie en welke andere
onderwerpen Mises en zijn gasten die week ook interessant vonden, te
bespreken.

Mises hielp Hayek persoonlijk bij het opstarten van zijn academische
carrière in de economie. In 1927, nadat hij zijn studie aan de
Universiteit van Wenen had afgerond, werd Hayek benoemd tot directeur
van Mises' pas opgerichte Oostenrijks Instituut voor
Conjunctuuronderzoek. Dit bood de jonge econoom een perfecte omgeving om
de theorie van zijn voormalige professor over economische berekening
verder te ontwikkelen.

Hayek concentreerde zich met name op de functie en het effect van
\emph{prijzen}. Zoals hij in de volgende jaren zou uitleggen, zijn
prijzen volgens hem de gedecentraliseerde en maatschappelijk schaalbare
communicatiemiddelen van de markt.\footnote{Friedrich A. Hayek, `Prices
  and Production'.} Hoewel men prijzen meestal ziet als een eenvoudige
functie van vraag en aanbod van goederen en diensten binnen een
economie, liet Hayek zien dat prijzen in feite een breed scala aan
relevante informatie bevatten die mensen nodig hebben om economische
beslissingen te nemen.

Laten we als (vereenvoudigd) voorbeeld opnieuw de autofabriek van Mises
nemen. Zoals eerder genoemd, heeft deze fabriek hulpbronnen nodig zoals
staal, machines en arbeid om auto's te produceren --- maar we
concentreren ons voor nu alleen op staal. Stel dat deze specifieke
fabrieksoperator zijn staal koopt van een staalproducent in een
nabijgelegen stad. Deze staalproducent haalt op zijn beurt ijzererts uit
een mijn halverwege het land. Tegelijkertijd koopt een lokale autodealer
de auto's van de fabriek, in de tegenovergestelde richting van de
toeleveringsketen, om ze vervolgens aan klanten te verkopen.

Iedereen in deze leveringsketen heeft de informatie die ze nodig hebben
om hun eigen bedrijf uit te baten, en ze communiceren dit met alle
andere marktdeelnemers door middel van prijzen.

De autohandelaar heeft een goed beeld van hoe hij auto's moet verkopen;
hij weet bijvoorbeeld hoeveel vraag er is naar nieuwe auto's, en hij
weet wat hij nodig heeft om ze te verkopen --- misschien een toonzaal op
een gunstige locatie en wat was om de auto's er mooi en glanzend uit te
laten zien. De prijzen die klanten bereid zijn te betalen voor auto's,
en de prijs die hij moet betalen voor een toonzaal en was, zullen dus
bepalen welke prijs hij zelf bereid is te betalen aan de autofabriek
voor nieuwe auto's.

Intussen weet de staalproducent hoeveel hij moet betalen voor erts, wat
zijn oven kost om het erts om te zetten in staal en wat hij aan
salarissen moet uitgeven. Zolang zijn klanten, zoals de autofabriek,
meer betalen voor zijn staal dan wat het hem kost om het te produceren,
zal hij staal blijven produceren.

Hoewel iedereen in de toeleveringsketen van elkaar afhangt, hoeft
niemand \emph{exact} te weten hoe iemand anders zijn werk doet. De
kosten van een toonzaal kunnen van invloed zijn op hoeveel de autodealer
bereid is aan de fabriek te betalen voor een nieuwe auto. Maar de
fabrieksmanager hoeft zich niet echt te verdiepen in de vastgoedmarkt
voor toonzalen. Hij hoeft zich ook niet te buigen over de schaarste van
erts. Deze informatie wordt weerspiegeld in de prijzen die de autodealer
biedt voor nieuwe auto's, en de prijs die de staalproducent vraagt voor
nieuw staal.

Bij uitbreiding, kunnen prijzen helpen bij de herverdeling van
hulpbronnen wanneer er iets verandert in de economie.

Als de ijzerertsmijn bijvoorbeeld gedeeltelijk moet sluiten vanwege een
brand, wordt het aanbod van erts kleiner, en de algehele vraag naar het
overgebleven erts zal de prijs van ijzer opdrijven. De staalproducent
zou dan op zijn beurt de prijs van zijn staal moeten verhogen om
winstgevend te blijven. Deze verhoogde prijs voor staal communiceert
eigenlijk de relevante informatie aan de autofabriek die deze nodig
heeft om economische beslissingen te nemen (De autofabriek zou
bijvoorbeeld kunnen besluiten staal te kopen bij een andere producent
die zijn erts van een andere mijn krijgt).

Op dezelfde manier, als de consumentenvraag naar keukenapparatuur
stijgt, zou de fabriek van keukenapparatuur meer staal willen kopen,
waardoor de staalprijs stijgt wanneer het de autofabriek overbiedt. De
staalproducent zou de staallevering verplaatsen van de autofabriek naar
de fabriek van keukenapparatuur, niet omdat hij iets weet over (de vraag
naar) auto's of keukenapparatuur, maar simpelweg omdat het prijssysteem
hem heeft geïnformeerd dat dit winstgevender zou zijn. (Op de lange
termijn zou de staalproducent ook gestimuleerd worden om meer staal te
produceren.)

Hayek legde uit dat relevante informatie in de economie kenbaar wordt
gemaakt door middel van het prijssysteem, wat ervoor zorgt dat markten
efficiënt middelen kunnen toewijzen over de samenleving daar waar ze het
meest gewaardeerd worden.

`In wezen, in een systeem waarin de kennis van de relevante feiten is
verspreid over vele mensen, kunnen prijzen fungeren om de afzonderlijke
acties van verschillende mensen te coördineren op dezelfde manier als
subjectieve waarderingen het individu helpen om de onderdelen van zijn
plan te coördineren', schreef de Oostenrijker: `een wonder.'\footnote{Friedrich
  A. Hayek, `The Use of Knowledge in Society', American Economic Review.
  XXXV, No.~4: 526--27.}

En belangrijk is dat dit allemaal mogelijk is zonder centrale planning.
De vrije markt, zo betoogde Hayek, valt het best te begrijpen als een
vorm van zelforganisatie van onderaf: een \emph{spontane orde}.

\section{Rentetarieven}\label{rentetarieven}

Waar Mises Hayeks begrip van spontane orde vormde doorheen ruimte -- de
toewijzing van middelen van het ene punt in de samenleving naar het
andere -- hielpen de werken van Von Böhm-Bawerk om Hayeks begrip van
spontane orde in de tijd te vormen.

Von Böhm-Bawerk kwam in de jaren 1890 op de proppen met een nieuw idee
in het vakgebied van de economie, dat van groot belang werd voor de
Oostenrijkse School: \emph{tijdsvoorkeur}. Von Böhm-Bawerk stelde dat
mensen doorgaans liever vroeger dan later goederen en diensten willen
ontvangen. De mate van deze voorkeur verschilt echter van persoon tot
persoon. Iedereen heeft zijn eigen -- en subjectieve -- tijdsvoorkeur.

Deze tijdsvoorkeuren, bepleitte von Böhm-Bawerk, worden weerspiegeld op
de markt door middel van rentetarieven.

Stel je voor dat zowel Marie als Joris graag een nieuwe auto willen
hebben. Ze hebben beiden liever vandaag dan volgend jaar een nieuwe
auto. Maar Marie, wier auto net kapot is gegaan en die elke dag naar
haar werk moet rijden, hechte veel meer waarde aan een nieuwe auto
vandaag dan aan een nieuwe auto volgend jaar. Joris daarentegen heeft
nog steeds een redelijk goede auto en werkt van thuis uit, dus heeft hij
niet zo'n haast om een nieuwe te krijgen. Marie heeft een hogere
tijdsvoorkeur dan Joris.

Stel je voor dat een nieuwe auto € 20.000 kost. Marie heeft helaas geen
geld, terwijl Joris € 20.000 aan spaargeld heeft. Op het eerste zicht
zou dit suggereren dat Joris eerder dan Marie een nieuwe auto zal kopen:
Joris kan het zich vandaag al veroorloven, terwijl Marie eerst nog geld
moet sparen voordat ze een nieuwe auto kan betalen.

Maar er is een andere optie. Joris zou Marie € 20.000 kunnen lenen.

Of dit een goede deal is voor hen beiden, kan eenvoudig worden ontdekt
door middel van rentetarieven. Laten we zeggen dat Marie, omdat ze een
hoge tijdsvoorkeur heeft, vandaag in principe een auto 10 procent meer
zou waarderen dan een auto volgend jaar. Dat wil zeggen dat ze zou
bereid zijn om € 22.000 voor een auto van € 20.000 te betalen als ze
deze vandaag kan hebben in plaats van een jaar vanaf nu. Ze is daarom
bereid om 10 procent rente te betalen op een lening van € 20.000. Joris,
die een lage tijdsvoorkeur heeft, zou een nieuwe auto vandaag slechts 1
procent meer waarderen dan een auto volgend jaar, een verschil van
slechts € 200.

Joris zou dus kunnen besluiten om zijn aankoop uit te stellen en in
plaats daarvan € 20.000 te lenen aan Marie. Na een jaar zal hij het
geleende bedrag plus een extra € 2.000 aan rente terugkrijgen. Dit zorgt
ervoor dat Marie de auto vandaag al kan kopen, terwijl de extra € 2.000
voor Joris de `kosten' van het uitstellen van de aankoop van € 200
ruimschoots goedmaakt. Beide partijen zouden er voordeel uit halen.
Rentetarieven stellen hen in staat onderling hun middelen in de tijd te
verdelen, zodat ze het beste overeenkwamen met hun individuele
tijdsvoorkeuren.

Hoewel dit natuurlijk een zeer vereenvoudigd voorbeeld is, doen
kredietmarkten iets vergelijkbaars op grotere schaal. Geldverstrekkers
en -ontleners komen een rentetarief overeen waar het aanbod en de vraag
naar geld elkaar treffen, gebaseerd op de gezamenlijke tijdsvoorkeuren.
Als zodanig zijn rentetarieven in feite ook prijzen. Ze weerspiegelen de
prijs van geld.

En net zoals alle prijzen, communiceert de prijs van geld relevante
informatie. Hayek was van mening dat de gemiddelde rente iets onthult
over de hele economie. Als de rentetarieven hoog zijn, wijst dit erop
dat veel mensen een hoge tijdsvoorkeur hebben en niet erg bereid zijn om
geld uit te lenen. Ze geven er de voorkeur aan om goederen en diensten
eerder dan later aan te schaffen. Omgekeerd, als de rentetarieven laag
zijn, suggereert dit dat veel mensen een relatief lage tijdsvoorkeur
hebben en bereid zijn hun aankopen uit te stellen als dat betekent dat
ze ondertussen wat rente kunnen verdienen.

Hayek was dus van mening dat rentetarieven producenten informeerden over
de productiefase waar ze hun middelen aan moesten besteden. Lage
rentetarieven signaleerden aan producenten dat ze dit `goedkope geld'
moesten benutten om productieprocessen op lange termijn te verbeteren
door te investeren in goederen van een hogere orde, zoals een nieuwe
oven om staal te produceren, die later kan worden gebruikt in de
productie van auto's (of keukenapparatuur). Daarentegen maken hoge
rentevoeten het lenen van geld duur, wat producenten aanmoedigt om reeds
beschikbare middelen te gebruiken en zich te concentreren op het
voltooien van de productie in een laat stadium (het laatste deel van het
proces waar de uiteindelijke consumptiegoederen zoals auto's worden
gemaakt en tentoongesteld in toonzalen voor mensen om te kopen).

Het mooie hiervan, zag Hayek in, is dat de tijdsvoorkeuren van mensen
netjes overeenkomen met de productiecapaciteit van de economie. Als
tijdsvoorkeuren laag zijn, investeren mensen hun geld (of in de meeste
gevallen zouden ze het `sparen' op een bankrekening en de bank
investeert het voor hen), en producenten worden gestimuleerd om te
investeren in hun langetermijnproductieprocessen. Dus wanneer
tijdsvoorkeuren in de toekomst toenemen, kunnen mensen hun geld en de
rente die ze verdienen uitgeven aan de vruchten van al deze verhoogde
productiviteit.

Rentevoeten, legde Hayek uit, bevorderen spontane orde door de tijd
heen!

Dat is natuurlijk zo, \emph{als} rentetarieven in feite de
tijdsvoorkeuren nauwkeurig weerspiegelen. Hayek merkte echter op dat het
in de praktijk vaak niet toegelaten werd om zo te zijn.

\section{De Federale Reserve}\label{de-federale-reserve}

Een aantal jaren voordat zijn student afstudeerde aan de Universiteit
van Wenen, hielp Mises Hayek opnieuw om een tijdelijke functie te
verkrijgen als onderzoeksassistent bij de Universiteit van New York.

Toen de jonge Oostenrijker in 1923 in de Verenigde Staten aankwam, waren
de \emph{Roaring Twenties} in volle gang. De Amerikaanse economie
floreerde, en mensen waren maar al te blij om geld te lenen om auto's
van Ford te kopen, nieuwe technologische wonderen zoals wasmachines, of
vastgoed te kopen in de voorsteden van grote steden. Of ze gebruikten
het geld om te investeren in aandelen: de Dow Jones-beursindex bereikte
jaar na jaar nieuwe hoogtepunten.

Hayeks onderzoek zou zich richten op de economische rol van één
instituut in het bijzonder: het relatief nieuwe centrale banksysteem van
de Verenigde Staten, genaamd de Federal Reserve. De `Fed', zoals dit
centraal banksysteem vaak wordt genoemd, was in 1913 opgezet om
vertrouwen en stabiliteit te brengen in het Amerikaanse bankensysteem.

Er werd gedacht dat een anker voor vertrouwen en stabiliteit
noodzakelijk was, omdat commerciële banken werkten op basis van
\emph{fractionele reserves}: ze hadden minder echt geld in hun kluizen
dan wat spaarders op hun bankrekeningen hadden toegeschreven. Het
verschil werd uitgeleend aan kredietnemers en bracht rente op voor zowel
banken als hun spaarders. Maar dit kon ook economische instabiliteit
veroorzaken, want als te veel spaarders het vertrouwen in een bank
verloren en ervoor kozen om tegelijkertijd hun geld op te nemen, zou de
bank krap bij kas kunnen komen te zitten, waardoor ze niet alle
opnameverzoeken konden honoreren.

In het scenario van zo'n \emph{bankrun}, kon de Federal Reserve nu
optreden als \emph{kredietverstrekker in uiterste nood} door een lening
te verstrekken aan de bank die in de problemen kwam. Zo'n lening zou de
bank van voldoende liquiditeit (contant geld) voorzien om de storm te
doorstaan, zodat er geen reden zou zijn voor spaarders om zich zorgen te
maken.

Maar Hayek, die helemaal uit Oostenrijk was gekomen om de rol van de Fed
als kredietverstrekker in uiterste nood te onderzoeken, en haar invloed
op de Amerikaanse economie, was kritisch.

Hayek vond dat de garanties van de nieuwe instelling economische
prikkels verstoorden. Hij vreesde dat gunstige economische
vooruitzichten de commerciële banken konden aansporen om leningen ruimer
te verstrekken dan voorheen, waardoor de geldhoeveelheid in wezen
toenam. Meer bankleningen betekent meer geld om in de economie uit te
geven.\footnote{Als persoon A € 100 bij de bank stort en de bank hiervan
  € 90 uitleent aan persoon B, zal persoon A nog steeds denken dat hij
  of zij € 100 heeft, terwijl persoon B € 90 zal hebben, voor een totaal
  van € 190. Bovendien, als persoon B de € 90 opnieuw bij de bank stort
  en de bank hiervan € 81 uitleent aan persoon C, zullen drie mensen
  denken dat ze samen € 271 bezitten. Dit kan zo doorgaan, wat lijkt
  alsof er steeds meer geld in omloop komt. Dit concept staat bekend als
  de `geldmultiplier'. In werkelijkheid kan de geldmultiplier zelfs nog
  agressiever zijn dan dit conventionele voorbeeld van fractioneel
  bankieren suggereert, omdat banken geen stortingen hoeven te ontvangen
  voordat ze leningen kunnen verstrekken; ze kunnen leningen uitgeven
  door eenvoudigweg krediet te creëren in de bankrekeningen van klanten.}
Dit `nieuwe geld' zou de algemene prijzen omhoog drijven --
\emph{inflatie} -- waardoor bedrijfswinsten over de hele lijn hoger
uitvallen en op hun beurt de gunstige economische vooruitzichten
bevestigen. De mogelijkheid van commerciële banken om in wezen nieuw
geld aan te maken door middel van leningen, zou een feedback-loop van
uitbundige kredietcreatie kunnen veroorzaken.

Echter, wanneer deze kredietcreatie onvermijdelijk vertraagt, zou de
muziek stoppen. Als de hoeveelheid nieuw geld die in de economie wordt
geïnjecteerd vermindert, zouden de algehele prijzen dalen -- deflatie --
en bedrijven zouden worden geconfronteerd met hun te optimistische
inschatting van economische vooruitzichten. Ze zouden hun producten niet
voor zoveel geld kunnen verkopen als ze hadden verwacht. Sommige
bedrijven zouden moeten inkrimpen, en de werkloosheid zou toenemen, wat
de economie nog meer zou vertragen, omdat mensen minder geld te besteden
zouden hebben. Andere bedrijven zouden failliet gaan, waardoor ze hun
leningen niet kunnen terugbetalen, wat op zijn beurt commerciële banken
in de problemen zou brengen om aan alle deposito-eisen te voldoen. Ze
zouden moeten stoppen met het verstrekken van nieuwe leningen, wat de
economie alleen maar verder zou vertragen, resulterend in meer ontslagen
en leningen die niet terugbetaald worden, en zo verder.

Economen zouden later naar een dergelijke dynamiek verwijzen als een
\emph{deflatoire schuldenspiraal}. Deze lagen aan de basis van enkele
bankencrises die de oprichting van de Federal Reserve in eerste
instantie motiveerden. \footnote{Gary Richardson and Tim Sablik,
  `Banking Panics of the Gilded Age: 1863--1913', Federal Reserve
  History, 4 december 2015,
  \href{https://www.federalreservehistory.org/essays/banking-panics-of-the-gilded-age.}{online}}

Volgens Hayek was er geen reden om aan te nemen dat het oprichten van
een `kredietverstrekker in uiterste nood' deze kwalijke dynamiek zou
beperken. Integendeel, het zou het zelfs kunnen versterken.

In het oude systeem, merkte hij op, hadden commerciële banken tenminste
nog een goede reden om voorzichtig te zijn en niet te veel leningen te
verstrekken:

`In afwezigheid van enige centrale bank, is de voornaamste beperking
voor individuele banken tegen het uitgeven van buitensporig krediet in
een stijgende economische activiteitsfase, de noodzaak om voldoende
liquiditeit te behouden om de vraag in een periode van krap geld het
hoofd te bieden met hun eigen middelen.'\footnote{\hspace{0pt}Friedrich
  A. Hayek, `Monetary Policy in the United States after the Recovery
  from the Crisis of 1920', in The Collected Works of F.A. Hayek, Good
  Money: part I, ed.~Stephen Kresge, 145.}

Het oprichten van een `kredietverstrekker in uiterste nood' kan
inderdaad een toeloop op de bank en paniek voorkomen. Maar Hayek stelde
dat dit tegelijkertijd de prikkel voor banken wegneemt om in eerste
instantie enige terughoudendheid te tonen bij het verstrekken van
leningen.

`Het moet dus onvermijdelijk leiden tot een gestage toename in het
gebruik van krediet en daardoor de herhaling van recessies nog
onvermijdelijker maken', concludeerde Hayek.\footnote{\hspace{0pt}Hayek,
  `Monetary Policy', 146.}

Dit soort verkeerde afstemming van economische prikkels, waar bepaalde
economische spelers -- in dit geval, banken -- beloond worden voor het
nemen van meer risico's, maar niet de volledige kosten van deze risico's
dragen, wordt in de economie \emph{moreel risico} genoemd. Hayek vond
dat de Federal Reserve dit morele risico in de economie introduceerde.

En Hayek geloofde dat dit niet eens de belangrijkste manier was waarop
de Federal Reserve onhoudbare, door krediet geïnduceerde, economische
bubbels stimuleerde.

\section{De Oostenrijkse
conjunctuurcyclus}\label{de-oostenrijkse-conjunctuurcyclus}

Het lezen van het jaarverslag 1923 van de Federal Reserve bepaalde de
koers van Hayeks carrière als econoom voor de komende decennia.

In het document legde de Amerikaanse centrale bank uit hoe ze haar
beheer over de geldhoeveelheid gebruikte om economische activiteit te
stabiliseren. Meer in het bijzonder verklaarde de Amerikaanse monetaire
autoriteit hoe ze rentetarieven gebruikte als beleidsinstrument: een
zeer nieuw concept op dat moment, geïntroduceerd door wat zelf een zeer
nieuwe instelling was.\footnote{Stephen Kresge, `The Collected Works of
  F.A. Hayek, Good Money: part I', 13.}

Het idee was vrij eenvoudig. Door valuta in het bankensysteem te
injecteren (doorgaans door overheidsobligaties te kopen), kon de Fed
commerciële banken voorzien van meer reserves en daardoor bereid (en in
staat) maken om leningen te verstrekken tegen steeds lagere
rentetarieven. Dit zou bedrijven en mensen stimuleren om te lenen.
Anderzijds, door reserves uit het bankensysteem te halen (door
overheidsobligaties te verkopen), konden banken worden ontmoedigd om
leningen te verstrekken, waardoor de rentetarieven stijgen en
economische activiteit wordt afgeremd.

De Federal Reserve geloofde dat ze de conjunctuurcyclus konden afvlakken
door de rentetarieven zorgvuldig te beheren. Als de Fed de rentetarieven
tijdens recessies licht kon verlagen, en ze tijdens oplevingen een
beetje kon verhogen, konden ze de markt een kleine stimulans bieden als
die in een dip zat, en een beetje afremmen als hij op hol sloeg.

Hayek was een grote criticus van dit beleid.

De Oostenrijker was van mening dat de Federal Reserve valse signalen
naar de markt zond door de rentetarieven kunstmatig laag te houden.
Iedereen maakte gebruik van goedkoop geld om te investeren in bedrijven,
die dit kapitaal gebruikten om meer middelen toe te wijzen aan hun
productieprocessen. Echter, waarschuwde Hayek, er was geen
overeenkomstige uitgestelde consumptie om de toekomstige productiegroei
op te vangen. De rentetarieven waren niet laag omdat veel mensen een
lage tijdsvoorkeur hadden en hun geld spaarden voor toekomstige
uitgaven, maar omdat de Fed ze zo laag hield.

Wanneer rentetarieven uiteindelijk zouden stijgen en de creatie van
nieuw krediet zou vertragen, zouden bedrijven worden gedwongen om de
productie te voltooien, maar zouden ze ontdekken dat er geen echte vraag
was om dit te evenaren. Zonder klanten om hun goederen te kopen, of in
ieder geval niet tegen de prijzen die ze hadden verwacht, zouden
bedrijven gedwongen zijn om werknemers te ontslaan en mogelijk in
gebreke te blijven op hun leningen. Zo start een deflatoire
schuldenspiraal.

Toen Hayek in de Verenigde Staten aankwam, bloeide de economie: zowel
consumptie als investeringen schoten omhoog. Maar hij kwam tot de
conclusie dat dit niet stand kon houden. De manipulatie van
rentetarieven verstoorde de opkomst van de spontane economische orde in
de tijd. Hayek was bezorgd dat de Federal Reserve in plaats van het
afvlakken van de hoogte- en dieptepunten van de economische cyclus, deze
eigenlijk versterkte.

En inderdaad, toen de Fed aan het eind van het decennium eindelijk de
rente verhoogde, droogden de investeringen op terwijl er geen toename in
consumptie was om dit te compenseren. De Roaring Twenties eindigden in
1929 met een knal en de Amerikaanse aandelenmarkt stortte in. In de
daaropvolgende jaren verloor de Dow Jones-aandelenindex bijna 90 procent
van zijn waarde, gingen tienduizenden bedrijven failliet, steeg de
werkloosheid sterk en (ondanks de verantwoordelijkheid van de Federal
Reserve als kredietverstrekker in uiterste nood) gingen ook duizenden
banken op de fles.

Alhoewel het pijnlijk was, geloofde Hayek dat de beste aanpak destijds
was om de deflatoire schuldenspiraal zijn gang te laten gaan. Daar waar
kunstmatig lage rentetarieven een valse economische bloei hadden
ingeluid, zou de economische malaise de economie herkalibreren naar
duurzamere niveaus. Terwijl niet-rendabele bedrijven onderuitgingen,
konden rendabele bedrijven hun middelen (inclusief hun werknemers)
overnemen en ze beter benutten. Dit proces zou waarschijnlijk een tijdje
duren, maar zou uiteindelijk tot een gezondere economie leiden.

Midden in de scherpe economische crisis die later bekend zou worden als
de Grote Depressie, was de voorgestelde oplossing van Hayek echter niet
erg geliefd. De meeste mensen waren van mening dat er \emph{iets moest
gedaan worden}.

\section{De rivaliteit}\label{de-rivaliteit}

En er kon iets gedaan worden, zo stelde een academicus uit Cambridge
genaamd John Maynard Keynes. De Britse econoom zou tijdens de Grote
Depressie snel naam maken door een onconventionele, maar dringend
gewenste oplossing aan te dragen om de economie weer op de been te
krijgen. In schril contrast met de pijnlijke oplossing die Hayek
voorstelde, verspreidde Keynes een boodschap waar veel mensen
reikhalzend naar uitkeken.\footnote{Het meest uitgebreid uiteengezet in
  John Maynard Keynes, `The General Theory of Employment, Interest and
  Money'.}

Keynes negeerde Hayeks analyse van de oorzaken van de depressie en
beweerde dat de economische malaise gewoon het ongelukkige resultaat was
van een daling van de totale vraag. Hij redeneerde dat de economie
stagneerde omdat mensen, voornamelijk om psychologische redenen, minder
geld uitgaven dan eerder. Hij omschreef het fenomeen als `dierlijke
instincten'. Om uit de depressie te geraken, moesten mensen weer meer
geld uitgeven.

In wat de basis zou worden van nog een nieuwe school van economisch
denken - het \emph{Keynesianisme} - beargumenteerde de Britse econoom
dat als het grote publiek geen geld zou uitgeven, de overheid het hun
plaats moest doen. De overheid kon bijvoorbeeld investeren in openbare
infrastructuurwerken, zelfs als dat zou betekenen dat daarvoor geld
geleend moest worden. Geld lenen zou volgens Keynes sowieso goedkoop
moeten zijn, aangezien de centrale bank de rentetarieven moest verlagen.

Door geld te besteden aan openbare infrastructuurwerken, zou de overheid
banen creëren. Dat zorgt ervoor dat mensen lonen hebben om uit te geven
en geld weer in de economie kan gaan circuleren. Wanneer mensen weer op
eigen houtje gaan uitgeven, zou de overheid vervolgens haar uitgaven
moeten verminderen. Keynes stelde voor dat beleidsmakers een
\emph{anti-cyclische} aanpak van overheidsuitgaven zouden moeten
adopteren.

Eén specifieke beleidsmaker was volledig klaar voor deze uitdaging.
Franklin D. Roosevelt (FDR), die in 1932 de eerste Amerikaanse
presidentsverkiezingen sinds de beurskrach won, had zijn campagne
gevoerd met de belofte om via zijn presidentieel mandaat actief een
einde te maken aan de depressie. Toen hij zijn ambt opnam, vormden
Keynes' ideeën het economische kader om zijn beleid te ondersteunen (zij
het enigszins \emph{nadat} dit beleid werd aangekondigd\footnote{\hspace{0pt}George
  Selgin, `The New Deal and Recovery, Part 15: The Keynesian Myth', Cato
  Institute, 16 maart 2022,
  \href{https://www.cato.org/blog/new-deal-recovery-part-15-keynesian-myth.}{online}}).
Via een reeks overheidsprogramma's, die `de New Deal' werden genoemd,
begon FDR al snel miljarden dollars uit te geven aan wegen, luchthavens,
bruggen, dammen en nog veel meer.

Hayek was echter helemaal niet overtuigd van de ideeën van Keynes.
Aangezien hij geloofde dat de economische malaise slechts een correctie
was van de onhoudbare boom die eraan voorafging, was hij van mening dat
overheidsuitgaven de uiteindelijk onhoudbare situatie alleen maar langer
lieten voortduren.

Daarbovenop was er een wellicht nog belangrijker bezwaar tegen de
Keynesiaanse anti-cyclische benadering, die niet eens echt gerelateerd
was aan economie. Dit bezwaar was van politieke aard: Hayek geloofde
niet dat politici te vertrouwen waren om te beslissen wanneer een
economie in een opwaartse of neerwaartse trend zit. In plaats daarvan
zouden ze in de verleiding komen om geld te lenen en in de economie uit
te geven wanneer daar vraag naar is\ldots{} wat evengoed continu het
geval kan zijn.

`Er zullen altijd delen van het land of bevolkingsgroepen zijn die van
mening zijn dat ze het moeilijk genoeg hebben om hulp te mogen
ontvangen', schreef Hayek. `Kan onder deze omstandigheden een rationeel
anti-cyclisch beleid ontwikkeld worden als het in handen van politieke
organen wordt gegeven?'\footnote{\hspace{0pt}Friedrich A. Hayek, `The
  Gold Problem', in `The Collected Works of F.A. Hayek, Good Money: part
  I', ed.~Stephen Kresge, 184.}

Voor Hayek was het antwoord een overduidelijke `nee'.

Dit leidde tot wat vaak beschouwd wordt als een van de grootste
intellectuele confrontaties van de twintigste eeuw. Gedurende de jaren
'30 stonden Hayek, die op dat moment professor was aan de London School
of Economics, en Keynes, nog steeds bij King's College in Cambridge,
vaak tegenover elkaar in publieke debatten en ook in hun
privécorrespondentie. Hun respectieve universiteiten in het zuidoosten
van Engeland dienden als het strijdtoneel voor de opkomende titanen van
de economie en hun twee tegenovergestelde economische visies.

En er was in belangrijke opzichten een scherp contrast. Terwijl Keynes
geloofde dat de economie onder andere regels werkt wanneer deze op
nationale schaal wordt geanalyseerd (het macro-niveau), hield Hayek vol
dat alles uiteindelijk voortkomt uit individuen en hun subjectieve
keuzes (het micro-domein). Waar Keynes graag focuste op prijsgemiddelden
en totalen, was Hayek meer geïnteresseerd in prijsverschillen. En
terwijl Keynes betoogde dat overheden een actieve rol moesten spelen in
het beheren van de economie, volhield Hayek dat de vrije markt het beste
aan zichzelf kon worden overgelaten.

Als Hayek het boegbeeld was van spontane orde van onderaf, dan had hij
in Keynes en zijn interventiebeleid van bovenaf zijn hedendaagse,
intellectuele rivaal gevonden.

\chapter{Vrije en open source
software}\label{vrije-en-open-source-software}

Richard Stallman was in de vroege jaren '60, al vanaf jonge leeftijd,
gefascineerd door computers. Toen hij op zomerkamp was, leende hij
programmeerhandleidingen van zijn begeleiders. Er was destijds geen
computer te bekennen -- ze kostten gemakkelijk meer dan € 100.000 per
stuk -- maar dat kleine detail ging zijn plezier niet bederven. Tijdens
zijn reis schreef hij computerprogramma's volledig uit op papier.

Het zou nog een paar jaar duren voordat de jonge New Yorker voor het
eerst kennismaakte met het echte werk. In 1970, net klaar met de
middelbare school, kreeg de toen zeventienjarige Stallman een zomerbaan
bij het Wetenschappelijk Centrum van IBM in Manhattan, om een numeriek
analyseprogramma te schrijven. Hij rondde het project binnen enkele
weken af, wat hem de rest van de zomer de gelegenheid gaf om bij het
onderzoekscentrum een teksteditor en een programmeertaalprocessor te
ontwerpen, gewoon voor de lol.

Na die zomer schreef Stallman zich in om natuurkunde te studeren aan
Harvard. Hij kon doorgaan met programmeren in het vrij nieuwe
computercentrum van de universiteit, en begon na verloop van tijd ook
rond te kijken naar andere gehoste computers aan verschillende
universiteiten en computerfaciliteiten in Cambridge. Hij ontdekte dat
een bijzonder krachtige machine gevestigd was in het Kunstmatige
Intelligentie Lab (AI Lab) van MIT. Dit onderzoekscentrum van MIT was
opgericht door twee pioniers op het gebied van AI - John McCarthy en
Marvin Minsky - en werd, zonder extra voorwaarden, gefinancierd door het
Amerikaanse ministerie van Defensie.

De student van Harvard besloot dat hij de documentatie van de
MIT-computer wilde bestuderen om meer informatie over de machine te
verzamelen en te begrijpen hoe deze verschilde van wat ze hadden op
Harvard. Maar toen hij het AI Lab bezocht, ontdekte Stallman dat ze
helemaal geen dergelijke documentatie hadden.

In plaats daarvan gaven ze hem een baan.

Dit weerspiegelde de tamelijk anarchistische cultuur in het AI Lab. De
leiders van het onderzoekscentrum hadden niet veel interesse in ervaring
of kwalificaties, maar waardeerden vaardigheid en potentieel. Het was
duidelijk dat dit wonderkind van Harvard, die het lab bezocht om hun
computerdocumentatie te bestuderen, goed bij hen zou passen.\footnote{Richard
  Stallman, `Richard Stallman: High School Misfit, Symbol of Free
  Software, MacArthur-Certified Genius'. Interview door Michael Gross,
  mgross.com, 1999,
  \href{https://www.mgross.com/writing/books/my-generation/bonus-chapters/richard-stallman-high-school-misfit-symbol-of-free-software-macarthur-certified-genius/.}{online}}

\section{De hackercultuur}\label{de-hackercultuur}

De anarchistische cultuur in het AI Lab was ongeveer een decennium
eerder voor het eerst ontstaan.\footnote{Dit deel is grotendeels
  gebaseerd op Steven Levy, `Hackers: Heroes of the Computer
  Revolution'.}

Het begon toen het Lincoln Lab, een militair onderzoeks- en
ontwikkelingscentrum voor geavanceerde technologie verbonden aan het
MIT, rond 1960 een kleine revolutie ontketende door de universiteit de
TX-0 te schenken, een vroege, volledig getransistoriseerde computer. In
tegenstelling tot eerdere computers op de universiteit die altijd een
speciale operator nodig hadden, was deze machine voor het eerst
toegankelijk voor studenten.

De machine, die een hele kamer in beslag nam en een ton woog, had al
snel de fascinatie gewekt van een specifieke groep studenten: de
knutselende techneuten uit de modeltreinclub van de universiteit. Ze
hadden nooit echt veel interesse gehad in de modeltreinen zelf, maar
vonden het vooral leuk om het elektrische systeem van draden,
schakelaars en hergebruikte telecomapparatuur te ontwerpen die de
snelheid en richting van hun treintjes beheersten. Ze beseften dat er
een veel interessanter spel was gearriveerd.

De jonge mannen (het waren aanvankelijk allemaal mannen) waren
vastbesloten om de machtige TX-0 te beheersen vanaf het eerste moment
dat die op de campus arriveerde. En inderdaad, al snel ontdekten ze hoe
ze toegang konden krijgen tot en de broncode konden bewerken van de
verschillende programma's die in de machine waren ingebouwd. Kort daarna
wisten ze te achterhalen hoe ze zelf hele nieuwe programma's konden
schrijven.

Het duurde niet lang voordat ze hele nachten rond de TX-0 doorbrachten,
op momenten dat ze de machine volledig voor zichzelf hadden. Verenigd
door hun gedeelde passie, daagden de jongens elkaar uit om de computer
steeds ingewikkeldere taken te laten uitvoeren. Wie pronkte met elegante
manieren om code te schrijven, had recht om op te scheppen. Bijzonder
slimme oplossingen werden in hun interne jargon `hacks' genoemd; de
jongens identificeerden zichzelf dan ook trots als `hackers'.

Naarmate de jongens hun programmeervaardigheden verbeterden in een geest
van kameraadschap, gingen ze de computer meer en meer zien als een
levensstijl. Niets was voor hen belangrijker dan hacken, en niets was
leuker. Het benutten van het potentieel van deze krachtige machines gaf
hen een geweldig gevoel van zelbeschikking.

Daarmee groeide ook een gevoel van verantwoordelijkheid.

De hackers voelden instinctief aan dat computers een blijvende invloed
zouden hebben op de wereld. Na verloop van tijd ontwikkelden ze een
filosofisch en ethisch kader rond programmeren en technologie. Dit
vormde de basis voor een unieke subcultuur, gericht op technologie en
gekenmerkt door experimenten en innovatie. Zelfs toen de groep hackers
veranderde -- nieuwe studenten kwamen naar MIT, terwijl oudere studenten
vertrokken -- bleef de hackercultuur behouden.

Als een cruciaal onderdeel van deze cultuur, waren hackers erop gesteld
om zaken zelf in handen te nemen: ze wilden alles waarvan ze dachten dat
het verbeterd kon worden aanpassen, en herstellen wat kapot was. Om
toestemming vragen werd beschouwd als een verspilling van tijd; een goed
idee moest onmiddellijk uitgevoerd worden en mogelijke beperkingen
moesten worden genegeerd. Bureaucratie was de natuurlijke vijand van de
hacker.

Als beperkingen als een uitdaging gezien werden, nou, hackers hielden
van uitdagingen, en ze hielden ervan om deze uitdagingen te
overwinnen\ldots{} het liefst met een beetje elegantie en flair. Hackers
geloofden dat computers gebruikt konden worden om schoonheid te creëren:
code kon esthetische waarde hebben, en hackers bewonderden goed
geschreven programma's en originele oplossingen.

En misschien wel het allerbelangrijkst, ze deelden die.

Hackers waren ervan overtuigd dat vrij toegankelijke code en bestanden
iedereen ten goede kwamen, en ze waren trots wanneer mensen de
programma's gebruikten die ze hadden geschreven. Ze geloofden dat ze een
ethische plicht hadden om hun code te delen, en om toegang tot
informatie voor anderen te vergemakkelijken. Er waren geen wachtwoorden,
geen beperkingen, en geen `persoonlijke' documenten.

De hackers zouden uiteindelijk een speciaal besturingssysteem
ontwikkelen om dat doel te dienen, het Incompatible Time-Sharing System
(ITS) (een woordspeling op het Compatible Time-Sharing System dat eraan
voorafging), dat hen in staat stelde om samen te coderen. Als iemand
inlogde en ontdekte dat een van zijn kameraden een nieuwe teksteditor of
strategisch spel had ontwikkeld, konden ze simpelweg het bestand openen
en zelf onmiddellijk bijdragen aan het project. Of, als twee hackers
tegelijkertijd de computer gebruikten, konden ze de code gelijktijdig
debuggen en verbeteren.

Dit was de vrije en coöperatieve cultuur die Stallman in het AI Lab van
MIT ontdekte.

\section{Anarchisme}\label{anarchisme}

De computer van het AI Lab was inderdaad, door iedereen, vrij, en zonder
beperkingen, te gebruiken. Niet alleen werknemers, maar ook bezoekers
van het lab konden de machine op elk gewenst moment gebruiken, en
toegang krijgen tot elk programma of bestand dat zich op de machine
bevond. De machine fungeerde als een gedeelde voorziening, voor iedereen
beschikbaar.

Het leidde wel tot moeilijkheden. Iedereen die een ITS-computer
gebruikte, kon bijvoorbeeld elk bestand verwijderen, zelfs als ze deze
niet zelf hadden gemaakt. Op dezelfde manier verstoorde het crashen van
de machine elk actief gebruikersproces.

Maar in de praktijk waren dergelijke incidenten zeldzaam. Het
vernietigen van iemand anders zijn werk paste niet in de hackercultuur,
en hoewel het crashen van de computer hinderlijk was voor andere
gebruikers, gaf het hen ook de kans om samen te werken bij het debuggen
van de code en het vinden van een oplossing voor wat de crash
veroorzaakte.

Dit was heel anders dan de computeromgevingen bij het IBM
Wetenschappelijk Centrum en Harvard waaraan Stallman gewend was geraakt.
Die machines waren ontworpen met beveiligingsfuncties die vereisten dat
sommige mensen meer bevoegdheden hadden dan anderen: bepaalde
programma's waren alleen toegankelijk voor bepaalde gebruikers, zoals
systeembeheerders of sommige professoren. De `elite', zij die
bevoorrechte accounts hadden, kon eenzijdig beslissen wat anderen wel en
niet konden doen op de machines, wat betekende dat reguliere gebruikers
vaak om hulp of toestemming moesten vragen.

De nieuwe collega's van Stallman in het AI Lab walgden van dit soort
beleidslijnen. Volgens hen hadden de beheerders in feite politiestaten
opgezet in hun respectieve computeromgevingen, waarbij ze zichzelf de
autoriteit toe-eigenden om andere gebruikers te besturen en controleren.

Nu hij hun vrije alternatief had ervaren, was Stallman het volledig met
hen eens. Terwijl het AI Lab bewees dat hun vorm van anarchie een
productieve werkomgeving kon bevorderen, raakte hij ervan overtuigd dat
de beperkende en gecontroleerde systemen feitelijk een digitale vorm van
fascisme vertegenwoordigden.\footnote{\hspace{0pt}Richard Stallman,
  `Talking to the Mailman', Interview door Rob Lucas, New Left Review,
  Sept--Oct 2018,
  \href{https://newleftreview.org/issues/ii113/articles/richard-stallman-talking-to-the-mailman}{online}}

`De gebruikers van ons systeem waren vrije mensen, aan wie gevraagd werd
om zich verantwoordelijk te gedragen. In plaats van een elite van macht,
hadden we een elite van kennis, bestaande uit iedereen die gemotiveerd
was om te leren', schreef Stallman later. `Omdat niemand anderen kon
domineren op onze machine, functioneerde het lab als een anarchie. Het
zichtbare succes hiervan bekeerde mij tot het anarchisme. Voor de meeste
mensen betekent 'anarchie' `verspillende, destructieve wanorde', maar
voor een anarchist als ik betekent het vrijwillige organisatie naar
behoefte, met de nadruk op doelen, in tegenstelling tot regels en
vereisten tot uniformiteit omwille van de uniformiteit.'\footnote{\hspace{0pt}Richard
  Stallman, `RMS Berättar', Linköping University,
  \href{http://www.lysator.liu.se/history/garb/txt/87-2-rms.txt}{online}}

Hoewel Stallman geen anarchist was in de meest complete betekenis van
het woord -- hij geloofde namelijk nog steeds dat de staat vele
belangrijke functies uitvoerde, waaronder het financieren van het AI Lab
-- dacht hij dat het anarchistische model ook in andere
computeromgevingen kon werken. En inderdaad, rond deze tijd begon de
hackercultuur zich ook te verspreiden buiten MIT, met name naar Stanford
University, die een eigen AI Lab kreeg. Tegen het begin van de jaren '70
had de hackergemeenschap een nieuwe basis gevestigd in de San Francisco
Bay Area.

En Stallman geloofde dat de hackercultuur ook buiten het academische
domein levensvatbaar zou zijn. Met het AI Lab als een geslaagd
voorbeeld, zou het vrije en samenwerkende ethos wellicht een model
kunnen worden voor de opkomende computerrevolutie.

\section{Problemen in het paradijs}\label{problemen-in-het-paradijs}

Later werd echter duidelijk dat de verspreiding van de hackercultuur
niet zo eenvoudig was.

Bijna tien jaar later was Stallman nog steeds werkzaam in het lab. Hij
merkte dat de hackercultuur daadwerkelijk verdreven begon te worden uit
haar oorspronkelijke thuisbasis. Mensen in en rond het AI Lab gingen
steeds meer wachtwoorden en, erger nog, auteursrechtelijke licenties
omarmen. Tegelijkertijd wilden de beheerders van MIT dat
computergebruikers formulieren invulden voordat ze de machines konden
bedienen, een praktijk die Stallman actief probeerde tegen te
houden.\footnote{\hspace{0pt}Stallman, `Talking to the Mailman'.}

En toch, vergeleken met wat er ging komen, waren dit kleine problemen.

In 1979 wilden Richard Greenblatt, een van de meest gerespecteerde
hackers van het lab, en Russell Noftsker, een voormalig labbeheerder,
een van de meest prominente projecten van het AI Lab op de markt
brengen. Hun plan was om een start-up op te richten om speciale
computers te verkopen die ontworpen waren voor LISP, de programmeertaal
voor AI die in het onderzoeksinstituut in ontwikkeling was.

Het werd echter al snel duidelijk dat Greenblatt en Noftsker zeer
verschillende ideeën hadden over de start-up. Greenblatt wilde dicht bij
de geest van het AI Lab en haar anarchistische cultuur blijven. Dit
betekende dat hij uit de buurt wilde blijven van investeerders en zo
dicht mogelijk wilde aanleunen bij de gekende hackercultuur. Noftsker
vond de benadering van Greenblatt echter onrealistisch. Hij zag een meer
traditioneel bedrijf voor zich, dat zijn producten zou beschermen met
softwarelicenties en auteursrechten.

Greenblatt en Noftsker slaagden er niet in om een compromis te bereiken
en besloten uit elkaar te gaan. Ze startten ieder hun eigen onderneming:
Greenblatts LISP Machine Incorporated (LMI) en Noftskers Symbolics
werden rivalen.

Aanvankelijk deelden zowel LMI als Symbolics de code die ze bij het AI
Lab produceerden, en daardoor ook met elkaar. Echter, begin 1982,
verbrak Symbolics deze driehoeksrelatie. Noftsker besloot dat het AI Lab
wel nog de door Symbolics aangepaste versie van de LISP-software kon
gebruiken, maar LMI mocht dit niet meer. Het was een ultimatum. De
beslissing van Noftsker betekende dat elke hacker in het lab een kant
moest kiezen.

Hoewel Greenblatt, die veel werk had verricht om het LISP-project te
realiseren, de kennis en de capaciteiten had, beschikte hij over te
weinig middelen. Ondertussen had het bedrijfsplan van Symbolics Noftsker
in staat gesteld om fondsen te werven van investeerders. Hij gebruikte
het geld om enkele van de beste hackers van het AI Lab in te huren. Om
er zeker van te zijn dat de nieuw aangeworven hackers exclusief voor
zijn start-up zouden werken, verbood hij alle medewerkers van Symbolics
om bij te dragen aan het AI Lab.

In één klap waren veel van de beste programmeurs van het computerlab
verdwenen, en ze namen hun werk met hen mee.

Het AI Lab was effectief uitgekocht. MIT's toevluchtsoord voor vrije
samenwerking was frontaal in botsing gekomen met meedogenloze zakelijke
belangen. Omdat de kleinschalige utopie van de hackers geroofd was van
haar meest waardevolle middelen, bleef slechts een uitgehold
overblijfsel van het onderzoekscentrum achter. Het lab had tijdens een
kortstondig gouden tijdperk gediend als een toonbeeld van effectief
anarchisme, maar was dat na de beslissing van Noftsker niet langer.

Voor Stallman betekende dit het einde van het lab.

De ontgoochelde hacker vatte het kort daarna samen in een brief:

`De personen die nog in het lab waren, waren de professoren, studenten
en non-hackeronderzoekers, die niet wisten hoe ze het systeem of de
hardware moesten onderhouden, of dit zelfs niet wilden weten. Machines
begonnen te breken en werden nooit gefixt; soms werden ze gewoon
weggegooid. Noodzakelijke veranderingen in software konden niet worden
doorgevoerd. De niet-hackers reageerden hierop door over te schakelen
naar commerciële systemen, wat fascisme en licentieovereenkomsten met
zich meebracht. Ik slenterde door het lab, door de kamers die 's nachts
leeg waar ze vroeger vol waren en dacht: 'Oh mijn arme AI-lab! Je gaat
dood en ik kan je niet redden.' Iedereen verwachtte dat als er meer
hackers werden opgeleid, Symbolics ze zou wegkapen, dus het leek niet
eens de moeite waard om het te proberen\ldots{} de hele cultuur werd
uitgewist\ldots'\footnote{\hspace{0pt}Levy, `Hackers', 448.}

Stallman had ooit gedroomd van een toekomst geïnspireerd door de vrije
en collaboratieve hackercultuur, maar hij geloofde nu dat hij in plaats
daarvan de laatste ademtochten ervan aan het aanschouwen was.

`Ik ben de laatste overlevende van een dode cultuur', klaagde Stallman
met een gevoel van drama. `En ik hoor echt niet meer thuis in de wereld.
En op sommige manieren voel ik dat ik eigenlijk dood zou moeten
zijn.'\footnote{\hspace{0pt}Levy, `Hackers', 472.}

\section{Vrije software}\label{vrije-software}

Toch was Stallman nog niet helemaal bereid om op te geven.

Stallman wees voornamelijk Noftsker aan als schuldige voor de ondergang
van het AI Lab. De hacker zette zich vervolgens in om alle
software-upgrades van Symbolics opnieuw te implementeren. Hij hield hun
documentatie van nieuwe functies bij en schreef vervolgens code die
dezelfde functies bood. In feite deed hij in zijn eentje het werk van
zes ontwikkelaars van de start-up. Hij deelde zijn code met LMI,
waardoor het bedrijf van Greenblatt een kans had om tegen Symbolics te
concurreren. Hij hield dit lang genoeg vol zodat Greenblatt nieuwe
programmeurs kon aannemen en zijn bedrijf weer op de rails kon
krijgen.\footnote{\hspace{0pt}Stallman, `High School Misfit'.}

Vervolgens besloot Stallman dat het tijd was voor een nieuwe start. Hij
had zichzelf ervan overtuigd dat de hackercultuur de wereld nog steeds
kon veranderen, maar concludeerde dat er een nieuw plan nodig was: `een
ambitieus project dat de fundamenten aanvalt van de manier waarop de
commerciële, vijandige manier van leven wordt voortgezet.'\footnote{\hspace{0pt}Stallman,
  `RMS Berättar'.}

Specifiek wilde Stallman de algemene trend naar \emph{propriëtaire
software} omkeren, software die door licenties en auteursrechten beperkt
werd, en die in de jaren `80 steeds gebruikelijker werd. In lijn met de
hackerethiek geloofde Stallman dat een computerprogramma maximaal nut
biedt als mensen het kunnen verbeteren. En aangezien computers het quasi
kosteloos maakten om informatie te kopiëren, stelde hij dat het
verhinderen van het delen van software 'de gehele mensheid saboteert',
zo betoogde de hacker.\footnote{\hspace{0pt}Steven Levy, `Hackers',
  441--42.}

Erger nog, propriëtaire software kan doorgaans niet geïnspecteerd
worden. Als mensen geen toegang hebben tot de in mensentaal leesbare
broncode van de software die ze op hun computers draaien, kunnen ze niet
zeker weten wat hun eigen machine eigenlijk doet. Een programma kan
kwaadaardig zijn en bijvoorbeeld zijn gebruiker beperken, censureren,
bespioneren of op een andere manier misbruiken.\footnote{\hspace{0pt}Angela
  Watercutter, `Why Free Software Is More Important Now Than Ever
  Before', Wired, 20 september 2013,
  \href{https://www.wired.com/2013/09/why-free-software-is-more-important-now-than-ever-before/}{online}}

Stallman was van mening dat het gebruik van propriëtaire software in
feite betekende dat je de controle overliet aan degene die het had
ontwikkeld.\footnote{Voor de nauwkeurigheid dient opgemerkt te worden
  dat dit deel van het argument technisch gezien pas duidelijk naar
  voren kwam toen Stallman het GNU-project een jaar of zo later
  lanceerde: het was nog geen deel van zijn oorspronkelijke motivatie om
  het project überhaupt te starten. Dit kleine anachronisme is in de
  tekst gebleven ten behoeve van de leesbaarheid.}

`Als gebruikers het programma niet beheersen, beheerst het programma de
gebruikers', redeneerde hij. `Met eigendomsrechtelijk beschermde
software is er altijd een entiteit -- de 'eigenaar' van het programma --
die controle heeft over het programma en daarmee macht uitoefent over de
gebruikers.'\footnote{\hspace{0pt}Richard Stallman, `Free Software Is
  Even More Important Now', gnu.org,
  \href{https://www.gnu.org/philosophy/free-software-even-more-important.en.html}{online}}

In plaats daarvan wilde Stallman dat computers instrumenten van
zelfbeschikking en vrijheid waren. Hij geloofde dat gebruikers te allen
tijde de controle over hun eigen apparaten moesten hebben.

Om dit te verwezenlijken, ontwikkelde hij een filosofie die vereiste dat
elk computerprogramma vier essentiële vrijheden moest bieden:\footnote{\hspace{0pt}GNU
  Operating System, `What is Free Software?'
  \href{https://www.gnu.org/philosophy/free-sw.html}{online}}

\begin{itemize}
\item
  De vrijheid om het programma te gebruiken zoals je wenst, voor welk
  doel dan ook (vrijheid 0)\footnote{Eigenlijk werd vrijheid 0 pas
    expliciet toegevoegd in de jaren 1990. Daarvoor dacht Stallman dat
    het een automatische juridische consequentie was van de
    oorspronkelijke drie vrijheden.}.
\item
  De vrijheid om te bekijken hoe het programma werkt en het zo aan te
  passen dat het je computerwerk naar wens uitvoert (vrijheid 1).
  Toegang tot de broncode is een noodzakelijke voorwaarde om dit te
  realiseren.
\item
  De vrijheid om kopieën te herverdelen zodat je anderen kunt helpen
  (vrijheid 2).
\item
  De vrijheid om kopieën van je aangepaste versies met anderen te delen
  (vrijheid 3). Door dit te doen kun je de hele gemeenschap de kans
  geven om te profiteren van je veranderingen. Toegang tot de broncode
  is een noodzakelijke voorwaarde.
\end{itemize}

Of samengevat, gebruikers hadden over het algemeen `de vrijheid om
software te draaien, kopiëren, verspreiden, bestuderen, veranderen en te
verbeteren.'\footnote{\hspace{0pt}GNO Operating System, `What is Free
  Software?'} Dit vereist dat de menselijk-leesbare broncode van een
programma gepubliceerd is, en dat deze niet onderhevig is aan beperkende
auteursrechtenlicenties. Stallman zou software, die deze vier essentiële
vrijheden bood, classificeren als vrije software (vrij in de betekenis
van `vrijheid', benadrukte de hacker graag, en niet zoals in `gratis
bier'.).

\section{GNU}\label{gnu}

Om echt de belofte van vrije software waar te maken, begreep Stallman
dat elk programma op een computer de vereiste vrijheden moest bieden.
Dit omvatte -- in de eerste plaats -- het besturingssysteem. Een
tekstverwerker die zich houdt aan de vier vrijheden kan zijn gebruiker
niet stiekem bespioneren, maar als het besturingssysteem waarop de
tekstverwerker draait ook niet-vrije software is, kan niet worden
uitgesloten dat het besturingssysteem bespioneert.

Daarom kondigde Stallman in 1983 zijn ongelooflijk ambitieus project aan
om een alternatief te bieden voor het populaire Unix besturingssysteem.
Waar Unix propriëtaire software was, bestond Stallmans besturingssysteem
volledig uit vrije software. Passend noemde hij het project GNU: GNU is
Niet Unix! (inderdaad, een recursief acroniem).\footnote{\hspace{0pt}Richard
  Stallman, `Free Unix!' September 27, 1983,
  \href{https://www.gnu.org/gnu/initial-announcement.en.html.}{online}}

GNU belichaamde de hackerethiek en wees propriëtaire software volledig
af.

`Ik heb veel andere programmeurs gevonden die enthousiast zijn over GNU
en willen helpen', schreef Stallman in het GNU Manifesto dat hij uitgaf
na de aankondiging, waarin hij het doel en de staat van het project
beschreef. `Veel programmeurs zijn ongelukkig over de commercialisering
van systeemsoftware. Het stelt hen wellicht in staat om meer geld te
verdienen, maar het zorgt ervoor dat ze zich in conflict voelen met
andere programmeurs in plaats van als kameraden. {[}\ldots{]} GNU dient
als voorbeeld om te inspireren en als een vaandel om anderen rond te
verzamelen om ons te vergezellen in het delen.'\footnote{\hspace{0pt}GNU
  Operating System, `The GNU Manifesto', 1985,
  \href{https://www.gnu.org/gnu/manifesto.html.en}{online}}

Inderdaad, GNU was meer dan alleen een stukje software, maar
vertegenwoordigde het onstaan van een nieuwe sociale beweging: de vrije
softwarebeweging.

Om de beweging te ondersteunen, richtte Stallman in 1985 ook de
non-profitorganisatie Free Software Foundation op. De stichting zou
pleiten voor vrije software en geld inzamelen om vrije softwareprojecten
te financieren. Daarnaast leidde de Free Software Foundation de
introductie van speciale vrije softwarelicenties onder de nieuwe paraplu
van `copyleft', ontworpen om vrije software te stimuleren.

Dit omvatte, het meest opmerkelijk, de GNU General Public License: een
licentie die het recht verleent om broncode te verspreiden en te
wijzigen zolang dit wordt gedaan onder even vrije voorwaarden. Met
andere woorden, andere ontwikkelaars van vrije software konden vrije
software die onder deze licentie werd uitgebracht op welke manier dan
ook in hun eigen project integreren, maar ontwikkelaars van propriëtaire
software konden dit niet.

Met het GNU-project in volle gang en de nieuwe licenties
geïmplementeerd, stond vrije software op het punt om een kracht te
worden waar je maar best rekening mee hield.

\section{De kathedraal en de bazaar}\label{de-kathedraal-en-de-bazaar}

Traditioneel werden vrije softwareprojecten uitgevoerd door kleine
groepjes ontwikkelaars vanuit speciale technologiehubs, zoals het AI Lab
van MIT. Maar toen hij aan GNU begon te werken, nodigde Stallman andere
ontwikkelaars uit om ook aan zijn project mee te werken. Door gebruik te
maken van het ontluikende internet, konden hackers zelfs vanuit de hele
wereld code bijdragen.

Hoewel Stallman aan bijdragers voor hun werk normaal geen financiële
compensatie beloofde, waren veel ontwikkelaars desondanks bereid om te
helpen van GNU een realiteit te maken. Misschien hoopten sommigen van
hen respect of status te verdienen van hun programmeercollega's door bij
te dragen, zoals altijd een factor was binnen de hacker-gemeenschap.
Anderen droegen misschien bij omdat ze GNU zelf wilden gebruiken. Nog
anderen vonden misschien de uitdaging op zichzelf interessant genoeg om
er deel van uit te maken. En, misschien wilden sommigen gewoon de wereld
een betere plaats maken, en zagen dit project als een middel om dat doel
te bereiken.

Wat hun redenen ook waren, ze droegen bij. Wat nog meer is, hun
bijdragen waren waardevol. Deze vrijwillige programmeurs leverden,
enigszins opmerkelijk, hoogwaardige code, waardoor Stallman in staat was
om vele afzonderlijke delen van het GNU-besturingssysteem een paar jaar
later te voltooien --- een indrukwekkende prestatie.

Rond datzelfde moment maakte de Finse software-ingenieur Linus Torvalds
graag gebruik van de vrijheid geboden door de GNU General Public
License. Hij nam een groot deel van Stallmans GNU-code, maar voegde zijn
eigen kernel toe (een programma dat zich in het hart van een
besturingssysteem van een computer bevindt), en in 1992 bracht Torvalds
Linux uit.\footnote{Om de zware afhankelijkheid van het project van GNU
  te benadrukken, geven sommigen de voorkeur aan de naam `GNU/Linux'.}
Het was het eerste werkende besturingssysteem dat volledig uit vrije
software bestond.

Maar Torvalds' voornaamste vernieuwing was mogelijk niet het
Linux-kernel zelf. Het was de manier waarop hij het maakte. De jaren
ervoor ontwikkelde de software-ingenieur een proces dat expliciet is
ontworpen voor samenwerking via het internet.

Zoals uitvoerig besproken door Linux-bijdrager Eric S. Raymond in zijn
essay `The Cathedral \& The Bazaar' uit 1997 en het (later) gelijknamige
boek, was de grootste aanpassing die Torvalds maakte de benadering van
het project omtrent beveiliging.

Tot die tijd beschouwden vrije software-ontwikkelaars bugs en andere
kwetsbaarheden als grote risico's die moesten worden aangepakt door
toegewijde experts die hun software zorgvuldig beoordeelden. Dit omvatte
ook de code die ze ontvingen van externe bijdragers. Ze zouden de code
pas vrijgeven als ze er zeker van waren dat het veilig was om deze te
gebruiken. Raymond bestempelde deze top-down benadering als het
'kathedraal'model.

In plaats daarvan gebruikte Torvalds wat Raymond het 'bazaar'model
noemde. Dit model gebruikte een flexibelere methode om bijdragen te
integreren, waardoor ontwikkelaars meer direct hun wijzigingen konden
uploaden naar verschillende versies van de software. Andere bijdragers
konden vervolgens deze software downloaden, testen en mogelijk de
wijzigingen in hun eigen versies overnemen.

Zo'n flexibel systeem zou kunnen leiden tot versies van de software met
meer fouten dan de software van hun tegenhangers in het
kathedraal-model. Maar omdat het ontwikkelproces openlijk plaatsvindt,
zijn andere bijdragers meestal sneller in staat om deze fouten te
ontdekken en ze te corrigeren. Indien nodig, wordt de oplossing direct
opgenomen in een nieuwe versie; onder het bazaarmodel vinden
software-updates sneller en vaker plaats.

`Met een groot genoeg aantal bèta-testers en mede-ontwikkelaars, wordt
bijna elk probleem snel vastgesteld en is de oplossing voor iemand
duidelijk', schreef Raymond in zijn essay, waarbij hij een van de
belangrijkste lessen die hij over de jaren had geleerd, samenvatte. `Of,
minder formeel, ``Met genoeg ogen, zijn alle bugs
oppervlakkig.''\,'\footnote{\hspace{0pt}Eric S. Raymond, `The Cathedral
  \& The Bazaar: Musings on Linux and Open Source by an Accidental
  Revolutionary'.}

Hij gaf de uitspraak de naam de `Wet van Linus'.

Opmerkelijk genoeg, geloofde Raymond dat dit ontwikkelingsmodel
voordelen kon bieden zelfs aan bedrijven en mensen die Stallmans
bezorgdheid over eigendomssoftware niet deelden, maar gewoon
kwaliteitscode wilden tegen lage kosten. Hij vermoedde echter dat velen
van hen (vooral bedrijven) terughoudend waren om gratis software te
gebruiken, juist omdat ze afgeschrikt werden door de ideologische
verhalen eromheen. Om minder nadruk te leggen op Stallmans originele
motivaties en meer op de pragmatische voordelen, leidde Raymond daarom
in de late jaren `90 de poging om vrije software te rebranden als 'open
source software'.

Stallman was zelf echter niet akkoord met de herbenoeming. Voor hem was
vrijheid het voornaamste, en de term `open source' deed afbreuk aan deze
boodschap.

Vandaag de dag verwijzen de termen `vrije software' en `open source
software' in vrijwel alle gevallen naar hetzelfde concept, maar het
verschil in terminologie blijft de filosofische kloof vertegenwoordigen.
De term `vrije en open source software' (FOSS) wordt gebruikt om beide
zijden van het schisma expliciet te omvatten.\footnote{Richard Stallman
  is ook geen fan van deze terminologie. Als er een term moet worden
  gebruikt die beide kanten van de kloof omvat, geeft hij de voorkeur
  aan `Free/Libre and Open Source Software' omdat dit duidelijker
  overbrengt dat het `vrije' deel over `vrijheid' gaat.}

\section{Gemeenschappelijk begrip}\label{gemeenschappelijk-begrip}

Het bazaarmodel kan hoogwaardige code opleveren. Maar die kwaliteit is
geen vanzelfsprekendheid. Volgens de Wet van Linus, vereist hoogwaardige
code voldoende `ogen', oftewel bijdragers.

FOSS-projecten hebben meestal niet de middelen om financiële beloningen
te geven aan potentiële bijdragers. Daarnaast worden bestaande
machtsverhoudingen vaak genegeerd in het kader van vrije en
open-source-ontwikkeling. Zoals Raymond ook al uitsprak in zijn essay,
was dwang natuurlijk volledig buiten de orde in `het anarchistische
paradijs dat we het internet noemen.' \footnote{\hspace{0pt}Raymond,
  `The Cathedral \& The Bazaar', 52.} Het aantrekken van bijdragers is
daarom een cruciale vaardigheid gebleken voor ontwikkelaars van vrije en
open software.

Geïnspireerd door de negentiende-eeuwse Russische anarchist Pyotr
Alexeyevich Kropotkin, legde Raymond uit dat projectleiders moesten
leren om effectieve belangengemeenschappen te werven en te motiveren op
basis van een gemeenschappelijk begrip. Om ontwikkelaars te overtuigen
een bijdrage te leveren, moet de leiding van een project bedenken hoe
bij te dragen aan hoe het project hen ten goede zou komen. De prikkels
moeten worden afgestemd op een gedeeld doel, stelde Raymond voor, een
`ernstige inspanning van vele samenkomende wilskrachten.'\footnote{\hspace{0pt}Raymond,
  `The Cathedral \& The Bazaar', 52.}

Dit betekent in de praktijk dat niemand echt de leiding heeft over
FOSS-projecten in de bazaarstijl. Een projectleider kan het project niet
in een richting sturen die niet gedragen wordt door de rest, zonder de
ontwikkelaars te verliezen die hij zo hard nodig heeft. In het
bazaarmodel wordt software beheerd door zijn schare van bijdragers, elk
met hun eigen persoonlijke reden om betrokken te zijn.

Wanneer deze prikkels wel in lijn liggen en er een groep bijdragers
bereid is aan een gemeenschappelijk doel te werken, kunnen de resultaten
geweldig zijn. Hoewel niemand ooit echt de leiding heeft, zijn deze
grootschalige samenwerkingsverbanden tussen vreemden met sterk
uiteenlopende niveaus van kennis en vaardigheden, erin geslaagd om zeer
complexe programma's te produceren, waarvan de Linux-kernel slechts één
voorbeeld is van vele.

Op deze manier lijkt de ontwikkeling van vrije en open source software
sterk op die andere vorm van grootschalige, leiderloze samenwerking:
vrije markten. Net zoals vrije markten bestaan FOSS-projecten alleen uit
vrijwillige interacties, benutten ze de kennis die verspreid is onder de
deelnemers en wat misschien wel het meest interessant is, kunnen ze
beter presteren dan top-down organisatievormen.

Net zoals vrije markten, kunnen vrije en open softwareprojecten een
spontane orde vormen.

`In veel opzichten gedraagt de Linux-wereld zich als een vrije markt of
een ecologie, een verzameling van individuen die hun eigenbelang
nastreven. Tijdens dit proces ontstaat een zichzelf corrigerende
spontane orde die veel complexer en efficiënter is dan wat elke vorm van
centrale planning ooit zou kunnen bereiken.'\footnote{\hspace{0pt}Raymond,
  `The Cathedral \& The Bazaar', 52.}

\chapter{Neutraal geld}\label{neutraal-geld}

Friedrich Hayek legde het idee van de vrije markt uit in termen van
spontane orde in de jaren dertig. Hij voerde aan dat individuen, door in
hun eigen belang te handelen, in staat waren middelen op een efficiënte
manier over de samenleving te verdelen met behulp van het prijssysteem:
een verbazingwekkend concept.

Het was dus volkomen logisch dat Hayek bijzonder geïnteresseerd was in
het goed waarin goederen en diensten worden uitgedrukt --- geld.

Vreemd genoeg, lijkt geld een fundamenteel principe van de Oostenrijkse
economie tegen te spreken. Deze stroming is opgebouwd op het idee dat
waarde subjectief is: mensen kennen waarde toe aan producten en diensten
als ze een persoonlijke behoefte vervullen. Schoenen worden gewaardeerd
omdat je ze kunt dragen, appels worden gewaardeerd omdat je ze kunt eten
en auto's worden gewaardeerd omdat je erin kunt rijden. Maar geld lijkt
op het eerste gezicht geen directe behoefte te vervullen. Je draagt het
niet, je eet het niet, je rijdt er niet in.

Geld lijkt dus vrij waardeloos. Desondanks wordt geld algemeen
geaccepteerd als betalingsmiddel in handel.

Deze schijnbare tegenstrijdigheid was in het begin van de twintigste
eeuw al aangepakt door Hayeks mentor aan de Universiteit van Wenen,
Ludwig von Mises.\footnote{\hspace{0pt}Ludwig von Mises, `The Theory of
  Money and Credit', vertaling. J.E. Batson.} De uitleg van Mises, het
regressie-theorema genoemd, accepteert dat mensen daadwerkelijk geen
geld willen. Ze willen wat geld kan kopen. Ze verlangen naar koopkracht.

Mises redeneerde dat de verwachte koopkracht van geld afgeleid is van
eerdere prestaties. Als € 10 je gisteren in een restaurant een lunch kon
kopen, zullen mensen aannemen dat ze er morgen ook een lunch voor kunnen
kopen. En de reden dat € 10 hen gisteren een lunch kon bieden, is dat de
restauranteigenaar wist dat hij daarmee de dag daarvoor tien broden kon
kopen bij de bakkerij, en waarschijnlijk dus ook de dag erna. De bakker
accepteerde op zijn beurt € 10 in ruil voor zijn brood, want daarmee kon
hij de dag ervoor een pond meel kopen bij de lokale molenaar\ldots{} en
zo verder.

Maar dit laat uiteraard nog steeds een belangrijk deel van de
tegenstrijdigheid onopgelost: wanneer begonnen mensen \emph{voor het
eerst} geld te accepteren en, vooral, waarom? Als we ver genoeg terug in
de tijd gaan -- regressie --, moet er ooit iemand de eerste zijn geweest
die geld begon te accepteren, zonder enige vorige ervaring om op te
vertrouwen bij het inschatten van toekomstige koopkrachtverwachtingen.

Mises loste deze vraag op door de theorie van Carl Menger te accepteren,
die stelde dat geld oorspronkelijk uit ruilhandel ontstond.

\section{Van ruilhandel naar geld}\label{van-ruilhandel-naar-geld}

In een ruileconomie -- een economie zonder geld -- ruilen mensen direct
goederen en diensten. Als de schoenmaker een paar schoenen heeft, maar
liever een brood wil, en de bakker een brood heeft maar liever een paar
schoenen wil, dan ruilen ze hun producten met elkaar. Na deze transactie
zijn ze beiden (subjectief gezien) beter af dan daarvoor.

Zo'n ruilhandel economie lijdt echter onder een probleem wat bekend
staat als de `dubbele toevalligheid van behoeften'. Een ruil kan alleen
plaatsvinden als twee personen precies het product willen dat de ander
te bieden heeft. De schoenmaker kan alleen schoenen ruilen voor een
brood als de bakker toevallig een nieuw paar nodig heeft\ldots{} maar
dit gebeurt waarschijnlijk niet heel vaak.

Meer gespecialiseerde vaklieden hebben het nog moeilijker om in een
ruileconomie te bemachtigen wat ze willen, omdat minder mensen hun
product nodig hebben. Een horlogemaker kan bijna nooit een horloge
ruilen voor een brood of een paar schoenen, omdat bakkers en
schoenmakers niet vaak een nieuw horloge nodig hebben.

Maar het tegenovergestelde is ook waar: sommige producten zouden
relatief gemakkelijk te verhandelen moeten zijn. Neem bijvoorbeeld zout,
en laten we aannemen dat veel mensen in deze economie vrij regelmatig
zout nodig hebben om een maaltijd op smaak te brengen, of om voedsel te
conserveren. In de woordenschat van Menger en de Oostenrijkse school, is
zout \emph{verkoopbaarder} (of `verhandelbaarder') dan een horloge.

En zout heeft ook andere voordelen. Het is behoorlijk duurzaam want zout
bederft niet. Het is redelijk draagbaar; zout kan makkelijk in een tas
worden meegenomen. Het is deelbaar; zout kan moeiteloos in kleinere
porties worden verdeeld, en die kleinere porties kunnen net zo
gemakkelijk weer worden samengevoegd tot een grotere hoeveelheid.
Bovendien is zout ook gemakkelijk te herkennen, en het is redelijk
fungibel, wat betekent dat verschillende porties onderling uitwisselbaar
zijn; zout is zout. En tot slot, afhankelijk van waar (en wanneer) je
bent, kan zout ook schaars zijn; het kan moeilijk zijn om er meer van te
verkrijgen.

Menger bedacht daarom dat het voor de horlogemaker verstandig zou zijn
om een lading zout te accepteren, wanneer dit hem wordt aangeboden in
ruil voor een horloge. Zelfs als hij zelf geen behoefte aan zout zou
hebben, dan zou de bakker dit zeker wel hebben. De horlogemaker kan
vervolgens het zout met de bakker ruilen en zo eindelijk aan dat brood
komen dat hij nodig heeft.

En het zou niet alleen voor de horlogemaker verstandig zijn om zout in
ruil aan te nemen, maar ook voor de schoenmaker. De bakker zou
waarschijnlijk vaker zout dan een paar schoenen accepteren in
ruilhandel. Dit zou op zijn beurt de horlogemaker nog meer mogelijkheden
geven met het zout dat hij in ruil krijgt; hij kan het uitgeven bij
zowel de bakker als de schoenmaker.

Naarmate meer mensen in deze ruileconomie zout zouden gaan accepteren in
de verwachting dat anderen dat ook zullen doen, zou dit een
zelfversterkende cyclus in gang zetten. Voor elke extra persoon die zout
accepteert in ruilhandel, wordt zout aantrekkelijker voor iedereen om
als ruilmiddel te accepteren. Zout zou zich op deze manier als een
gangbaar \emph{ruilmiddel} kunnen ontwikkelen.

Hoewel sommige mensen in deze economie zelf geen zout nodig zouden
hebben, noch direct iets hadden om het aan uit te geven, zouden ze leren
vertrouwen dat het uiteindelijk nuttig voor hen zal zijn. Daarom
beginnen ze met het opslaan van zout voor toekomstig gebruik. Zo wordt
zout ook een \emph{opslag van waarde}.

En uiteindelijk zouden mensen in deze economie beginnen de waarde van
producten en diensten in zout te meten. Een horloge kost misschien een
kilo zout, een paar schoenen een pond, en een brood een ons. Zout wordt
gebruikt om prijzen vast te stellen, waardoor het een
\emph{rekeneenheid} wordt.

Zodra het deze drie eigenschappen verwerft - ruilmiddel, opslag van
waarde en rekeneenheid - is zout geld geworden.

Dit zou op zijn beurt de vraag naar zout ook vergroten. In deze economie
werd zout oorspronkelijk alleen gewaardeerd om zijn inherente
eigenschappen - het vermogen om een maaltijd te kruiden of voedsel te
conserveren.\footnote{Dit wordt soms beschouwd als `intrinsieke waarde',
  een veelgebruikte uitdrukking in de economie om aan te geven dat een
  goed economische gebruikswaarde heeft los van zijn monetaire rol.
  Oostenrijkse economen verwerpen echter over het algemeen het idee dat
  producten überhaupt intrinsieke waarde hebben: waarde is volgens hen
  altijd subjectief.} Maar zodra het als geld wordt aangenomen, zouden
veel mensen erop gebrand zijn meer zout te vergaren, omdat dit hen in
staat zou stellen elk ander product te kopen. Het proces van
\emph{monetarisatie} zou een \emph{monetaire} premie toevoegen aan de
waarde van zout.

Deze premie verklaart waarom mensen bereid zijn geld - in dit voorbeeld,
zout - aan te nemen in ruil voor goederen en diensten die op zichzelf
meer verlangens of behoeften vervullen, waardoor de schijnbare
tegenstrijdigheid wordt overwonnen die werd geïntroduceerd door de
subjectieve waarde theorie.

Zout is in het verleden inderdaad als vorm van betaling gebruikt. Enkele
duizenden jaren geleden werden de soldaten van het Romeinse Rijk in zout
uitbetaald; het hedendaagse woord `salaris' is afgeleid van het Latijnse
woord \emph{salarium}, dat `zoutgeld' betekent. Toen de Italiaanse
ontdekkingsreiziger Marco Polo in de dertiende eeuw naar China reisde,
ontdekte hij dat de plaatselijke bevolking elkaar betaalde met een soort
pannenkoeken gemaakt van zout. En er waren zelfs bepaalde Ethiopische
stammen die zout als geld gebruikten, en dat zo recent als de twintigste
eeuw.

Desalniettemin, gedurende duizenden jaren, ontdekten diverse
beschavingen van over de wereld dat er een beter ruilmiddel bestond dan
zout. Tegen de tijd dat Mises zijn regressietheorema publiceerde, was
het grootste deel van de wereld overgegaan op het gebruik van
goud.\footnote{Recenter archeologisch onderzoek, gepubliceerd in David
  Graeber's `Debt: The First 5,000 Years', suggereert dat er nooit een
  pure ruilhandel-economie is geweest zoals beschreven in de
  regressietheorie. In plaats daarvan gebruikten de oudste menselijke
  beschavingen schuld als hun eerste vorm van geld. Schuld werkt echter
  alleen als valuta in omgevingen met een hoge mate van vertrouwen en op
  reputatie gebaseerde systemen. In omgevingen met weinig vertrouwen was
  goud vaak de valuta bij uitstek, en het is heel goed mogelijk dat het
  edelmetaal deze status in de loop van de tijd verwierf via een proces
  dat lijkt op wat wordt beschreven in de regressietheorie.}

\section{Goud}\label{goud}

De eigenschappen van goud maken het bijzonder geschikt als geld. Goud is
ongelooflijk duurzaam: het rot niet, roest niet en bederft ook niet. Het
is ook redelijk draagbaar. Gouden munten kunnen makkelijk worden
meegenomen. Het is deelbaar, want met de juiste gereedschappen kan goud
worden gesmolten tot kleinere stukjes, en deze kleinere porties kunnen
weer worden samengesmolten tot grotere baren. De eigenschappen van goud
maken het ook relatief makkelijk te herkennen, terwijl het perfect
inwisselbaar is. En wellicht het belangrijkste, het winnen van goud uit
de aardkorst is een moeilijk en duur proces, en wordt steeds lastiger
naarmate de makkelijkst toegankelijke goudmijnen uitgeput raken, wat
enige mate van schaarste garandeert. En als bonus, het glanzende, gele
metaal wordt door velen als mooi beschouwd.

In de laatste eeuwen werd goud echter vrijwel nooit echt als munteenheid
in transacties gebruikt; mensen gebruikten namelijk over het algemeen
bankbiljetten. Deze bankbiljetten konden worden ingewisseld voor goud,
dat in eerste instantie veilig bewaard werd door de banken die de
biljetten uitgaven. Bankklanten vonden het handiger om de biljetten als
ruilmiddel te gebruiken terwijl het goud opgesloten bleef in de
bankkluizen.

Dit zorgde ervoor dat het merendeel van al het goud dat in de reserves
van banken werd gehouden, nooit werd opgevraagd. Het stimuleerde banken
om meer bankbiljetten uit te geven -- en uit te lenen -- dan ze
daadwerkelijk konden rechtvaardigen met het goud in hun kluizen. Het
wijdverbreide gebruik van papiergeld leidde tot het tijdperk van de
fractionele reservebank. Na verloop van tijd groeide dit uit tot een
complex systeem van kredietverlening, corresponderende banken en
clearinginstellingen, nauw verweven met de aandelenmarkten en het
bredere financiële systeem.

En uiteindelijk kwam dit allemaal onder toezicht te staan van centrale
banken zoals de Federal Reserve. Deze centrale banken waren
verantwoordelijk geworden voor het beheren van de goudreserves van hun
landen, waar tegen zij nationale valuta's uitgaven --- papieren
bankbiljetten die konden worden ingewisseld voor een vastgestelde
hoeveelheid goud. Het echte goud werd eigenlijk alleen maar gebruikt
voor internationale handel. Aan het begin van de twintigste eeuw was de
wereld georiënteerd op de goudstandaard.

Maar de Eerste Wereldoorlog had effectief een einde gemaakt aan \emph{de
klassieke goudstandaard}. De meeste regeringen schaften de
inwisselbaarheid van hun valuta af, waardoor ze hun oorlogsinspanningen
vrijer konden financieren.\footnote{\hspace{0pt}Nicholas Dimsdale,
  `British Monetary Policy and the Exchange Rate 1920-1938', Oxford
  Economic Papers 33, New Series: 307--49.} In plaats van een
vertegenwoordiging van goud, werden de ongedekte nationale valuta's
simpelweg door overheidsbesluit als geld beschouwd, een vorm van geld
genaamd `fiatgeld'. (Het Latijnse woord `fiat' betekent `laat het zo
zijn', en wordt meestal geassocieerd met overheidsdecreten.)

In de eerste jaren nadat de oorlog was beëindigd, fluctueerden de
fiatvaluta's vrij in waarde ten opzichte van elkaar. Dit betekende dat
als iemand van bijvoorbeeld de Verenigde Staten een product uit Engeland
wilde kopen (importeren), ze eerst een deel van hun dollars moesten
inwisselen voor ponden. Als dit op een grote schaal gebeurde, zou de
extra vraag naar ponden op zijn beurt de wisselkoers tegenover de dollar
verhogen (het zou meer dollars kosten om dezelfde hoeveelheid ponden te
kopen).

Een sterke pond zou vervolgens het importeren van producten en diensten
uit Engeland duurder maken voor Amerikanen, waardoor de export van
Engeland wordt geremd. Tegelijkertijd zou een zwakke dollar het
importeren van producten en diensten uit de VS aantrekkelijker maken
voor Britten, wat potentieel kan resulteren in een verhoogde vraag naar
Amerikaanse goederen, en op zijn beurt weer kan zorgen voor een hogere
vraag naar dollars. Schommelingen in de valutawaarde zouden dus op een
bepaalde manier de handelsbalans tussen de twee landen stabiliseren.

Er werd aangenomen dat dit een tijdelijke situatie zou zijn; de meeste
landen hadden het voornemen terug te keren naar een goudstandaard.
Ondanks dat, betoogden invloedrijke economen uit die tijd dat deze
nieuwe goudstandaard wat anders zou moeten werken dan de klassieke
goudstandaard. Aangezien het merendeel van het goud dat nationale
munteenheden dekte nooit werd opgevraagd, konden centrale banken
daadwerkelijk meer geld uitgeven dan ze in goud konden verantwoorden (In
de VS mocht de Fed dit doen, zolang zij binnen de
\emph{gouddekkingratio} bleven: de verhouding tussen goud in reserve en
uitgegeven dollars moest ten minste 40 procent zijn.).

Deze flexibiliteit bood de mogelijkheid voor een nieuw soort monetair
beleid: centrale banken konden geld toevoegen aan --, of onttrekken uit
het bankensysteem om rentetarieven te manipuleren, met als doel de
waarde van geld te stabiliseren.

Het was dit beleid waar Hayek zo fel kritiek op had.

\section{De stabilisators}\label{de-stabilisators}

De eerste pleitbezorger voor dit nieuwe type van monetair beleid was
Irving Fisher, een toonaangevende econoom in het begin van de twintigste
eeuw. Fisher was een van de eerste economen die zich zorgen maakten over
deflatie, of specifieker, de deflatoire schuldenspiraal. Hij geloofde
dat alleen stabiele prijzen `de kwalen van monetaire instabiliteit'
konden voorkomen.\footnote{\hspace{0pt}Irving Fisher, `The Purchasing
  Power of Money'.}

Fisher wist dat de prijzen van specifieke goederen en diensten
natuurlijk soms veranderen, omdat de vraag en aanbod van verschillende
producten en diensten om allerlei redenen over tijd fluctueren. Maar hij
geloofde dat het algemene, gemiddelde prijsniveau stabiel moest blijven
na verloop van tijd. Om de stabiliteit van de koopkracht van een
munteenheid te bepalen, had Fisher dus een manier nodig om gemiddelde
prijzen vast te stellen. En hij wist precies waar hij moest zoeken.

In eerder onderzoek leverde de econoom empirisch bewijs voor de
kwantiteitstheorie van geld. Deze theorie stelt dat het algemene
prijsniveau van goederen en diensten proportioneel is met de hoeveelheid
geld in omloop. Om dit te bewijzen, gebruikte Fisher indexen, waarbij
elke index bestond uit een scala aan goederen en diensten en hun
gemiddelde prijs op een specifiek moment in de tijd. Het vergelijken van
deze indexen over verschillende tijdlijnen gaf inzicht in de
ontwikkeling van het algemene prijsniveau.

Fisher bedacht dat vergelijkbare indices gebruikt konden worden om een
stabiele vorm van geld te creëren, waarbij een dollar door de tijd heen
hetzelfde aandeel van een index zou moeten kopen. Zo'n index zou een
selectie van alle producten bevatten die de gemiddelde consument koopt:
een \emph{consumentenprijsindex} (CPI). Het ene jaar zou een dollar
misschien meer aardappelen en minder wortelen kunnen kopen, en het
volgende jaar meer wortelen en minder aardappelen, maar de gemiddelde
koopkracht van een dollar, gemeten volgens de CPI, zou ongeveer
hetzelfde moeten blijven.

Fisher beweerde dat de Federal Reserve de stabiliteit van de dollar kon
nastreven door de rentetarieven te manipuleren. Hij had in 1920 de
Stable Money Association opgericht om deze beleidswijziging te
realiseren. Een collectief bestaande uit economen, politici en
zakenleiders pleitte voor een stabilisatiebeleid via parlementaire
ondervragingen, bekend als de `stabilisatiehoorzittingen', terwijl ze
daarnaast ook hun zaak bepleitten op internationale conferenties en
andere bijeenkomsten waar monetair beleid een gespreksonderwerp was. Dit
hielp om de ideeën van Fisher te verspreiden tot in de hoogste rangen
van het Federal Reserve Systeem en daarbuiten.

De Stable Money Association vond al gauw een bondgenoot in Hayek's
hedendaagse rivaal, John Maynard Keynes. Maar Keynes ging zelfs nog
verder dan Fisher en de Stable Money Association. Hij stelde voor om de
goudstandaard volledig los te laten. In zijn verhandeling uit 1923,
\emph{A Tract on Monetary Reform}, betoogde de econoom van King's
College dat het edelmetaal niet geschikt was om stabiele prijzen te
garanderen, omdat het zelf onderhevig was aan marktsentimenten.

Hoewel dit inderdaad gedeeltelijk verzacht kon worden als de centrale
banken een flexibeler aanpak zouden adopteren om muntstabiliteit te
bereiken, betoogde Keynes dat de gouddekkingratio uiteindelijk de
speelruimte van de centrale banken fundamenteel zou beperken. En in de
praktijk zou deze genegeerd worden wanneer dat als noodzakelijk werd
beschouwd.

`In werkelijkheid is de goudstandaard al een barbaars overblijfsel.
{[}\ldots{]}. We zijn allemaal, van de gouverneur van de Bank van
Engeland af, nu vooral geïnteresseerd in het behouden van de stabiliteit
van bedrijven, prijzen en werkgelegenheid en het is niet waarschijnlijk
dat we, gedwongen om een keuze te maken, bewust deze dingen zullen
opofferen voor het achterhaalde dogma'.\footnote{\hspace{0pt}John
  Maynard Keynes, `A Tract on Monetary Reform', 173.}

Als geld daarentegen volledig bevrijd zou worden van de beperkingen van
goud, kon het monetaire beleid zo flexibel zijn als de monetaire
autoriteiten nodig achtten.

\section{De verwerping van stabiel
geld}\label{de-verwerping-van-stabiel-geld}

Hayek verwierp het gebruik van rentetarieven door centrale banken als
middel, omdat hij geloofde dat dit enkel leidde tot een volatielere
conjunctuurcyclus. Maar hij verwierp ook het doel om prijzen te
stabiliseren. Hayek keurde deze stabilisatoren af.

Prijzen, legde Hayek uit, bevatten een breed scala aan informatie.
Daardoor is het mogelijk dat de prijzen van identieke goederen op
verschillende locaties van elkaar verschillen. Een krat bananen zou
bijvoorbeeld goedkoper kunnen zijn in Colombia, waar bananen groeien,
vergeleken met IJsland, waar de bananen eerst naartoe vervoerd moeten
worden. De kosten van het transport (en daarmee de kosten van brandstof
en meer) zouden verwerkt zijn in de prijs van bananen in IJsland.

Daarom redeneerde Hayek dat, technisch gezien, een krat bananen in
Colombia en een krat bananen in IJsland in economische zin als twee
verschillende producten moeten worden beschouwd. Het interspatiale
prijssysteem --- verschillende prijzen op verschillende locaties ---
maakte een efficiënte toewijzing van bronnen over de ruimte mogelijk.

Daarnaast betoogde de econoom dat iets vergelijkbaars gold voor
anderszins identieke producten op verschillende tijdstippen; zoals het
interspatiale prijssysteem zorgt voor een efficiënte verdeling van
middelen over ruimte, zorgt het intertemporele prijssysteem voor een
efficiënte verdeling van middelen doorheen de tijd.

Hayek:

`Strikt genomen zouden goederen die technisch gelijk zijn maar alleen op
verschillende tijdstippen beschikbaar zijn, in economische zin beschouwd
moeten worden als verschillende goederen, net zoals goederen die
technisch hetzelfde zijn maar zich op verschillende plaatsen
bevinden'.\footnote{\hspace{0pt}Friedrich A. Hayek, `Intertemporal Price
  Equilibrium and Movements in the Value of Money', in `The Collected
  Works of F.A. Hayek, Good Money: part I', ed.~Stephen Kresge, 195.}

Vrije markten bevorderen innovatie, en de meeste producten worden door
deze innovatie goedkoper om te produceren. Het produceren van een krat
bananen, om bij dat voorbeeld te blijven, wordt na verloop van tijd
betaalbaarder doordat bananenkwekers profiteren van verbeterde
technologie voor het beheer van hun plantages. Het is dan logisch dat
een krat bananen over tien jaar goedkoper zal zijn dan een krat bananen
nu: de verschillen in productiekosten zullen tot uiting komen in de
respectievelijke prijzen.

Hayek geloofde dus niet dat het stabiliseren van prijzen op basis van
indices überhaupt wenselijk was. Als prijzen kunstmatig stabiel worden
gehouden, zou dit het intertemporele prijssysteem verstoren, en
uiteindelijk ook de toewijzing van middelen doorheen de tijd verstoren.

Laten we zeggen dat een bananenboer verwacht dat de prijs van bananen
(net zoals alle andere consumentenproducten, gemiddeld genomen) stabiel
zal blijven in de toekomst, terwijl hij ook weet dat zijn
productiekosten zullen dalen. Dit zou hem aanzetten om te investeren in
toekomstige productie ten koste van de productie vandaag: hij zal later
dezelfde hoeveelheid bananen kunnen verbouwen voor minder totale kosten,
terwijl hij elke krat bananen zal kunnen verkopen voor dezelfde prijs
als vandaag, en zo zijn totale winst kan verhogen.

Als alle producenten in de economie op dezelfde manier denken, als ze
allemaal dezelfde prikkels volgen en investeren in productie voor de
toekomst ten koste van productie vandaag, zou dit op korte termijn
leiden tot een tekort aan totale economische productie en op lange
termijn tot een overschot.

Hayek schreef:

`Als, tijdens een algemene uitbreiding van de productie, de verwachting
met zekerheid is dat de prijzen van producten niet zullen dalen, maar
stabiel zullen blijven of zelfs zullen stijgen, zodat op een later
tijdstip dezelfde of zelfs een hogere prijs kan worden verkregen voor
het product dat tegen een lagere prijs is geproduceerd, moet het
resultaat zijn dat de productie voor de latere periode, waarin het
aanbod al op een relatief adequaat niveau is, nog verder wordt
uitgebreid ten koste van die voor de eerdere periode, waarin het aanbod
relatief minder adequaat is'.\footnote{\hspace{0pt}Hayek, `Intertemporal
  Price Equilibrium', 207.}

Om een spontane orde te hebben doorheen de tijd, moeten prijzen de
mogelijkheid krijgen om te dalen.

Hayek concludeerde daarom:

`De acceptatie van de noodzaak voor een intertemporeel prijssysteem is
niet alleen onverenigbaar met, het staat lijnrecht tegenover de
heersende opvatting dat constante prijzen doorheen de tijd een
voorwaarde zijn voor een ongestoorde economie'.\footnote{\hspace{0pt}Hayek,
  `Intertemporal Price Equilibrium', 190.}

Natuurlijk erkende de Oostenrijker ook dat een daling van de prijzen in
sommige gevallen een negatieve impact kan hebben op de economie: hij
waarschuwde tegen de manipulatie van rentetarieven juist omdat het
uiteindelijk zou leiden tot deflatie. Echter, Hayek stelde dat deflatie
alleen een probleem is als de daling van de prijzen daadwerkelijk wordt
veroorzaakt door een afname van de geldhoeveelheid. In dat geval zouden
bedrijven inderdaad minder verdienen dan verwacht, wat, zoals ook Fisher
had aangegeven, een deflatoire schuldenspiraal in gang zou kunnen
zetten.

Hayek benadrukte dat als de daling van de prijzen niet veroorzaakt werd
door een krimpende geldhoeveelheid, maar in plaats daarvan het resultaat
was van goedkopere productieprocessen, dit probleem niet zou moeten
bestaan.

`Een daling van het prijspeil als gevolg van continue verbeteringen in
alle productietakken heeft niet dezelfde problematische gevolgen als een
deflatie. Theorie is tot nu toe nauwelijks verder gekomen dan dit
onderscheid tussen de effecten van prijswijzigingen die enerzijds
afkomstig zijn van de 'goederenkant' en anderzijds van de
`geldkant'\,'.\footnote{\hspace{0pt}Hayek, `Intertemporal Price
  Equilibrium', 214.}

\section{De goudwisselstandaard}\label{de-goudwisselstandaard}

De Vereniging voor Stabiel Geld had zich in de periode kort na de Eerste
Wereldoorlog snel laten gelden. Slechts een jaar na de oprichting wist
de groep hun standpunt met succes te verdedigen op de Economische en
Financiële Conferentie van Genua in 1922. Daar kwamen vertegenwoordigers
van 34 grote geïndustrialiseerde landen bijeen om de grote economische
en politieke problemen van het naoorlogse Europa op te lossen.

De vertegenwoordigers op de conferentie stemden in met het aannemen van
een \emph{goudwisselstandaard}. Nationale valuta's zouden een vaste
wisselkoers ten opzichte van goud aanhouden, maar centrale banken kregen
relatieve flexibiliteit om een monetair beleid te voeren dat
prijsstabiliteit nastreeft door het manipuleren van rentetarieven.

Deze beleidslijnen voor prijsstabilisatie moesten per land worden
beheerd: de stabilisatoren stelden voor dat de koopkracht van een valuta
stabiel zou moeten blijven binnen de eigen nationale economie. Maar dit
betekende dat als het totale prijspeil in verschillende landen zou
beginnen te variëren, de waarde van hun nationale valuta tegen elkaar
kon fluctueren, wat de internationale handel potentieel zou kunnen
beïnvloeden.

De stabilisatoren erkenden dit, maar vonden dat de afweging het waard
was.

`{[}\ldots{]} Wanneer stabiliteit van het interne prijsniveau en
stabiliteit van de externe wisselkoersen onverenigbaar zijn, is de
eerste doorgaans te verkiezen', schreef Keynes, en `wanneer het dilemma
acuut is, is het behoud van het eerste ten koste van het laatste,
gelukkig misschien, de weg van de minste weerstand'.\footnote{\hspace{0pt}Keynes,
  `A Tract on Monetary Reform', 164.}

De stabilisatoren geloofden ook dat internationale handel vrij soepel
kon doorgaan onder een goudwisselstandaard, als de centrale banken zich
zouden houden aan wat Keynes de `regels van het spel' had genoemd. In
een notendop, landen met een handelsoverschot (die dus meer exporteren
dan importeren) en daardoor een instroom van goud hebben, zouden volgens
deze regels de rentetarieven moeten verlagen. Dit zou meer leningen
stimuleren, wat zou resulteren in meer valuta in omloop en uiteindelijk
hogere prijzen in het algemeen. Landen met een handelstekort zouden naar
verwachting de rentetarieven verhogen om het tegenovergestelde effect te
bereiken.

Het verhoogde prijspeil in landen met een handelsoverschot zou de
producten van dat land minder aantrekkelijk moeten maken voor export,
wat zou moeten helpen om de omvang van het handelsoverschot te
verminderen. Tegelijkertijd zouden lagere prijzen in de landen met
handelstekorten de producten van deze landen aantrekkelijker moeten
maken voor export, wat zou moeten helpen om hun handelstekort te
verminderen. Net als onder een systeem met zwevende fiatvaluta zou de
verandering in totale prijzen tussen landen moeten helpen om het
handelsevenwicht tussen hen in evenwicht te brengen.

Ook Hayek erkende dat een dergelijk handelsevenwicht behouden kon worden
onder de goudwisselstandaard, net zoals onder de fiatgelstandaard. Maar
hij was het er niet mee eens dat dit een goede zaak was.

\section{Monetair nationalisme}\label{monetair-nationalisme}

In 1937 nam Hayek zowel de goudwisselstandaard als het zwevende
fiatgeldsysteem onder de loep in een reeks lezingen getiteld `Monetair
nationalisme en internationale stabiliteit'. De lezingen wezen de
internationale valuta-afspraken door de stabilisatoren, die Hayek had
bestempeld als monetair nationalisme, volledig af.

Hayek stelde dat, onder zowel de goudwisselstandaard als ook een zwevend
valutasysteem, de bedrijven en individuen die profiteren van een
exporttoename (of in tegendeel, die lijden onder een exportdaling) niet
noodzakelijkerwijs dezelfde bedrijven of individuen zijn die
verantwoordelijk zijn voor de toename (of afname) van
grensoverschrijdende handel.

`Als we kijken naar de methoden die beschikbaar zijn voor het bankwezen
om een uitbreiding of inkrimping van krediet te realiseren, is er geen
reden om aan te nemen dat zij het geld dat vernietigd moet worden exact
kunnen halen bij die personen waar het tijdig vrijgegeven zou worden als
er geen bankensysteem zou zijn, of dat ze het extra geld zullen plaatsen
in handen van diegenen die het geld zouden ontvangen als het direct
vanuit het buitenland naar het land kwam', schreef Hayek. \footnote{\hspace{0pt}Friedrich
  A. Hayek, `Monetary Nationalism and International Stability', in `The
  Collected Works of F.A. Hayek, Good Money: part II', ed.~Stephen
  Kresge: 55.}

Specifiek zal de verandering in rentetarieven onder de
goudwisselstandaard waarschijnlijk niet reflecteren wat het rentetarief
in een vrije markt zou zijn. Dit zorgt voor winnaars en verliezers:
schuldenaren profiteren van lagere rentetarieven, terwijl schuldeisers
eronder lijden. Er is echter geen reden om aan te nemen dat de handelaar
die zijn export verhoogt een schuldenaar is. Hij zou net zo goed een
schuldeiser kunnen zijn, in welk geval hij indirect schade lijdt door
zijn eigen toename in internationale verkopen.

In plaats van alleen de twee handelspartijen te beïnvloeden die zaken
doen over grenzen heen, zorgde internationale handel onder de
goudwisselstandaard in wezen voor een herverdeling van middelen via de
kredietmarkten, zo merkte Hayek op.

`Er zijn daarentegen sterke argumenten om te geloven dat de last van de
verandering geheel of tot een mate die op geen enkele manier
gerechtvaardigd is door de onderliggende verandering in de werkelijke
situatie, zal vallen op de investeringsactiviteit in beide
landen'.\footnote{\hspace{0pt}Hayek, `Monetary Nationalism', 55--56.}

Op een soortgelijke manier worden onder een systeem van zwevende
fiatvaluta hulpbronnen ook verdeeld aan meer partijen dan enkel zij die
direct betrokken zijn bij de stijging (of daling) van de export, legde
Hayek uit.

Laten we inzien waarom. Stel dat er een verschuiving in de vraag
plaatsvindt van de auto-industrie in de VS naar de auto-industrie in
Engeland. Dit betekent dat Amerikanen dollars moeten wisselen voor
ponden om een nieuwe auto te kopen, wat invloed heeft op de relatieve
waardes van de dollar en het pond. Als gevolg hiervan worden \emph{alle}
producten uit de Verenigde Staten goedkoper vanuit het perspectief van
Engeland, terwijl alle producten uit Engeland duurder worden vanuit het
perspectief van de Verenigde Staten. Alle exporteurs in de Verenigde
Staten zouden daarom meer verkoop genieten ten koste van bedrijven in
Engeland.

Hoewel in dit voorbeeld de initiële verschuiving in vraag enkel
plaatsvindt tussen de twee auto-industrieën, zouden Britten ook
aangemoedigd worden om Amerikaans voedsel, kleding, of elektronica te
kopen. Niet omdat deze producten daadwerkelijk beter of goedkoper te
produceren zijn, maar puur door de werking van het systeem van zwevende
fiatvaluta.

In beide vormen verstoorde het monetair nationalisme de spontane orde
over grenzen heen, concludeerde Hayek.

\section{Neutraal geld}\label{neutraal-geld-1}

Hayek verwierp stabilisatiebeleid en hij verwierp ook monetair
nationalisme. In plaats daarvan pleitte hij voor een homogeen,
internationaal soort geld met een vaste hoeveelheid. Of wat hij
\emph{neutraal geld} noemde.

Hayek verwierp resoluut de aanname van stabilisatoren dat monetaire
stabiliteit op nationaal niveau gemeten zou moeten worden. Hij betoogde
dat, indien geld net zo gemakkelijk van het ene naar het andere land kan
bewegen als binnen regio's van hetzelfde land, een algehele
prijsstijging binnen een land simpelweg een toename in vraag naar
goederen en diensten uit dat land weerspiegelt. Dit zou precies aan de
markt signaleren dat middelen het best aan dat land kunnen worden
toegewezen, zodat het meer goederen en diensten kan produceren.

Een homogeen geld zou eenvoudig functioneren in de internationale
handel: de betaling zou effect hebben op de verzender en de ontvanger en
niemand anders, ongeacht in welke landen zij wonen. Een grenzeloze
valuta zou daarom het beste het interspatiële prijssysteem mogelijk
maken, zo redeneerde de Oostenrijker, mensen in staat stellend om
prijzen te vergelijken over verschillende locaties.

Geld met een vaste hoeveelheid zou ondertussen het intertemporele
prijssysteem het beste faciliteren, omdat dit mensen in staat stelt om
prijzen op verschillende tijdstippen nauwkeurig te vergelijken:

`Alleen met een geldsysteem waarin elke verandering in de hoeveelheid
geld uitgesloten was, zou het mogelijk zijn om een structuur van
geldprijzen op opeenvolgende momenten in de tijd te bedenken die
correspondeert met het systeem van intertemporele evenwicht', schreef
Hayek. \footnote{\hspace{0pt}Hayek, `Intertemporal Price Equilibrium',
  212.}

Misschien wel het allerbelangrijkste is, als de geldhoeveelheid vast
was, zouden veranderingen in de productiekosten door de tijd heen
duidelijk weerspiegeld worden in overeenkomstige veranderingen in
prijzen. Als alle andere factoren buiten beschouwing worden gelaten,
zouden de prijzen dalen als de kosten voor de productie van goederen
dalen. Een geleidelijke daling van de prijzen -- deflatie -- was volgens
Hayek de natuurlijke uitkomst van elke gezonde economie.

Echter, Hayek vond wel dat er een heel echt probleem was met een zo
homogeen geld met vaste geldhoeveelheid. Hij geloofde niet dat het kon
worden bewerkstelligd.

Ten eerste, zelfs als een dergelijke munt gecreëerd zou kunnen worden,
zouden mensen nog steeds kunnen kiezen om krediet en andere
alternatieven te gebruiken in plaats van daadwerkelijk geld, wat
neerkomt op een feitelijke verhoging van de geldhoeveelheid.

`Het is uiteraard onmogelijk om {[}de hoeveelheid ruilmiddelen voor
altijd vast te stellen{]}, gezien de altijd aanwezige mogelijkheid om
een surrogaatgeld in plaats van echt geld te gebruiken', concludeerde de
Oostenrijker droevig. `De hoeveelheid van dat surrogaat kan niet nauw
gekoppeld zijn aan dat van het echte geld, en het creëren ervan zou
exact hetzelfde effect hebben als dat van elke andere uitbreiding van de
geldhoeveelheid'.\footnote{\hspace{0pt}Hayek, `Intertemporal Price
  Equilibrium', 217.}

Maar nog belangrijker, Hayek was van mening dat een homegene valuta met
een vaste geldhoeveelheid überhaupt niet kon worden aangenomen, omdat
hij er niet van overtuigd was dat er enige internationale autoriteit zou
zijn die het vertrouwen zou krijgen om dergelijke valuta uit te geven.
Iets zo belangrijk als een wereldwijde monetaire standaard zou de
sterkste garantie moeten bieden dat het universeel aanvaardbaar en
toegankelijk zal blijven. Maar Hayek dacht dat er geen bekende instantie
was die deze garantie kon bieden.

`{[}\ldots{]} zolang er afzonderlijke soevereine staten zijn, zal er
altijd de dreiging van oorlog opdoemen, of het risico van het instorten
van de internationale monetaire regelingen om een andere
reden'.\footnote{\hspace{0pt}Hayek, `Monetary Nationalism', 87.}

Neutraal geld was onmogelijk omdat, op een zeer fundamenteel niveau,
naties elkaar niet konden vertrouwen.

\section{Valuta-oorlogen}\label{valuta-oorlogen}

Inderdaad, dit gebrek aan vertrouwen had tegen het einde van de jaren
dertig ook bijgedragen aan de ineenstorting van de goudwisselstandaard.

Voor zijn eigen kritiek op de goudwisselstandaard, ging Hayek uit van
een ideaal scenario, waarbij de deelnemende landen zich hielden aan de
regels van het spel. Deze regels schreven voor wanneer en hoe de
centrale banken hun rentetarieven dienen aan te passen. Echter, het was
al duidelijk dat zelfs dit ideaal scenario zich niet in de werkelijkheid
had voltrokken, aangezien meerdere van de deelnemende landen in plaats
daarvan betrokken waren bij opeenvolgende competitieve devaluaties, in
wat soms omschreven wordt als een `valutaoorlog'.

Als een nationale munteenheid devalueert, worden goederen en diensten
relatief goedkoop vanuit het perspectief van andere landen. Een daling
van de munteenheid kan daardoor de export stimuleren en zo tenminste
tijdelijk de nationale economie ten goede komen. Maar er is ook een
keerzijde. Wanneer de internationale vraag verschuift naar het land dat
zijn munteenheid heeft gedevalueerd, betekent dit ook dat de vraag
verschuift \emph{weg} van andere landen. Hun economieën hebben hierdoor
vaak te lijden.

De snelste manier voor deze andere landen om hun concurrentievermogen
terug te krijgen, is door hun eigen valuta te devalueren. Dit zou de
handelsbalansen in hun oorspronkelijke staat moeten herstellen. Maar als
gevolg hiervan zou al het geld, in al deze economieën, minder waard zijn
dan voorheen. Dit treft vanzelfsprekend spaarders, schuldeisers en
mensen met een vast inkomen. Als iedereen meedoet met een valutaoorlog,
dan is er niemand die wint.

Dit weerhield landen er echter niet van om precies dat te doen. Nog
voordat de goudwisselstandaard werd ingevoerd, was Duitsland deze reeks
van devaluaties in 1921 al vrij spectaculair begonnen door zijn
munteenheid te hyperinflateren, om hun herstelbetalingen voor de oorlog
te kunnen betalen. Hayeks geboorteland Oostenrijk volgde al snel.
Frankrijk was de volgende, hoewel (als een van de overwinnaars van de
Eerste Wereldoorlog) niet in dezelfde extreme mate: het devalueerde de
franc vlak voor de invoering van de nieuwe goudstandaard in 1925. Als
reactie hierop schortte Engeland in 1931, slechts een paar jaar na
invoering van de nieuwe internationale monetaire afspraken, de
goud-omwisselbaarheid op om het pond te devalueren.

Als een van de weinige landen die tijdens de oorlog een beperkte vorm
van goudconvertibiliteit had gehandhaafd, was de Verenigde Staten
aanvankelijk terughoudend om deel te nemen aan enigerlei devaluatie van
de munteenheid. Op dat moment was een troy ounce goud precies \$ 20,67
waard, en de Federal Reserve was in de vroege jaren 1930 nog steeds
verplicht om zich te houden aan de gouddekkingsgraad van 40 procent.

Maar deze dekkingsratio begon een belemmerende factor te vormen toen
president Franklin Roosevelt de Keynesiaanse methoden toepaste in een
poging om de Amerikaanse economie uit de economische depressie te
helpen. Uiteindelijk besloot hij de obstakels die hij tegenkwam op een
ongekende manier te verwijderen.

Via Uitvoeringsbevel 6102, goedgekeurd met warme steun van
Keynes,\footnote{\hspace{0pt}Nicholas Wapshott, `Keynes \& Hayek: The
  Clash That Defined Modern Economics': 159.} verbood FDR in 1933
resoluut `het hamsteren van gouden munten, goudstaaf en goudcertificaten
binnen de continentale Verenigde Staten'.\footnote{\hspace{0pt}Franklin
  D. Roosevelt, `Relating to the Hoarding, Export, and Earmarking of
  Gold Coin, Bullion, or Currency and to Transactions in Foreign
  Exchange', 28 augustus 1993, beschikbaar via `The American Presidency
  Project'.} Alle Amerikaanse burgers werden bevolen elk goud dat ze
hadden in te ruilen voor dollars bij hun lokale Federale Reserve lidbank
tegen het vaste tarief van \$ 20.67. Bij niet-naleving stond een boete
van \$ 10.000 en een gevangenisstraf van maximaal tien jaar.

Een paar maanden later devalueerde Roosevelt de dollar tot \$ 35 per
troy ounce, waardoor de gouddekkingsratio van de Federal Reserve in
wezen met ongeveer 69 procent in één nacht toenam. Dit betekende een
nieuwe klap voor de goudwisselstandaard.

Daarna was Europa weer aan de beurt, te beginnen met de Fransen die in
1936 hun munteenheid voor de tweede keer devalueerden. En toen in de
volgende jaren meer landen het voorbeeld van Engeland volgden om goud
volledig te laten vallen om hun munteenheden te devalueren, werd de
goudwisselstandaard volledig verlaten, nauwelijks een decennium nadat
deze was ingevoerd.\footnote{\hspace{0pt}Joris Rickards, `Currency Wars:
  The Making of the Next Global Crisis', 56--77.}

De reeks van muntdevaluaties, gecombineerd met een diepe economische
depressie, eiste haar tol, met name in de landen die de oorlog in 1918
hadden verloren. De vernietiging van spaargeld, wijdverspreide
werkloosheid en een gebrek aan perspectief in grote delen van Europa
resulteerde in veel onzekerheid, wanhoop en uiteindelijk, woede.

Het bood een vruchtbare voedingsbodem voor een nieuwe en bijzonder
gewelddadige, nationalistische, racistische en autoritaire
collectivistische ideologie. Het fascisme kreeg over het hele continent
voet aan de grond.

\chapter{Cryptografie}\label{cryptografie}

Whitfield Diffie had altijd al een voorliefde voor codes. Al sinds zijn
docent in het vijfde leerjaar hem de substitutiecipher leerde (een
basistechniek in de wiskundige tak van cryptografie), was hij
gefascineerd door deze methode van geheimhouding. Het feit dat
tekstversleutelingsalgoritmen in die tijd -- de jaren vijftig -- de
specialiteit waren van het leger, geheime agenten en spionnen, droeg
alleen maar bij tot de mysterieuze aantrekkingskracht.\footnote{Een
  groot deel van dit hoofdstuk is gebaseerd op Steven Levy's `Crypto:
  How the Code Rebels Beat the Government -- Saving Privacy in the
  Digital Age'.}

De jonge Whitfield was al snel in de ban van elk cryptografieboek dat
zijn vader, een universiteitsprofessor, kon vinden in de bibliotheek van
het City College in New York. Hij stortte zich op standaardwerken zoals
het boek \emph{Cryptoanalysis} uit 1939 van Helen Forché Gaines, dat
diverse methoden beschreef om berichten om te zetten in onleesbare
versleutelde tekst, bij voorkeur zo dat alleen de beoogde ontvanger ze
kon ontcijferen.

Met de zeer simpele \emph{Caesar-cipher} (vermoedelijk gebruikt door
Julius Caesar), kunnen berichten bijvoorbeeld in \emph{ciphertext}
worden omgezet door elke letter te vervangen door een andere letter. De
versleutelingssleutel `+1' vervangt bijvoorbeeld elke letter door de
volgende letter in het alfabet - `a' wordt `b', `b' wordt `c',
enzovoort. Het woord `Secret' verandert in `Tfdsfu'. Om de versleutelde
tekst te ontcijferen, wordt dezelfde versleutelingssleutel gebruikt,
maar in omgekeerde volgorde: elke letter wordt vervangen door de vorige
letter in het alfabet ( `Tfdsfu' wordt weer `Secret').

Deze +1 versleutelingssleutel is natuurlijk niet erg sterk. Een
tegenstander die vastbesloten is om de resulterende ciphertext te
ontcijferen - een \emph{cryptanalist} - zou het waarschijnlijk bij hun
eerste poging raden. En zelfs al zouden ze dat niet doen, er zijn een
aantal patronen in een versleutelde tekst die gespecialiseerde
codekrakers kunnen helpen te bepalen welke vervangende letters
waarschijnlijk overeenkomen met welke originele letters. Vooral in
langere teksten kunnen aanwijzingen worden gevonden in de frequentie van
specifieke letters en de lengte van woorden, om maar een paar
voorbeelden te noemen.

Moderne encryptiesleutels gebruikten daarom veel geavanceerdere
technieken, en gebruikten bijvoorbeeld enkele delen van een tekst toe om
andere delen te versleutelen. Tegen het midden van de twintigste eeuw
was encryptie, en het breken ervan, essentieel voor militaire operaties
en werd uitgevoerd door toegewijde specialisten die het vakgebied hadden
uitgebreid tot het punt waarop ciphertext volledig willekeurig kon
lijken. Zonder patronen om te analyseren, waren zelfs de beste
cryptografen ter wereld niet in staat om deze codes te breken.

Desalniettemin bleef het basisidee grotendeels onveranderd. Net als bij
de Caesar-cipher waren de geheime sleutels altijd \emph{symmetrisch}: de
ontsleutelingssleutel was dezelfde als de versleutelingssleutel, alleen
in omgekeerde volgorde gebruikt. Om veilig te communiceren, moesten
mensen eerst een sleutel delen.

Een sleutel delen via een onbeveiligd communicatiekanaal was echter geen
optie. Als afluisteraars de sleutel zouden onderscheppen, konden ze alle
daaropvolgende berichten die met die sleutel versleuteld waren,
ontcijferen. Hiermee zou dus het hele doel van het versleutelen van
berichten in de eerste plaats teniet gedaan worden. De sleutels werden
daarom meestal persoonlijk gedeeld. Beide partijen moesten eerst fysiek
samenkomen voordat ze versleutelde berichten konden uitwisselen.

Dit was natuurlijk niet altijd gemakkelijk, of zelfs mogelijk. Grote
afstanden of extreme situaties zoals oorlog konden het proces
aanzienlijk bemoeilijken. Toch geloofden cryptografen, zoals Diffie in
zijn beginjaren leerde, dat er geen andere manier was: men moest eerst
sleutels persoonlijk uitwisselen.

\section{MIT}\label{mit}

Ongeveer tien jaar nadat hij voor het eerst kennismaakte met de methode
van de substitutiecipher, begon Whitfield Diffie wiskunde te studeren
aan de Massachusetts Institute of Technology. Dit viel ook samen met de
komst van de allereerste computers op de universiteitscampus. Hoewel hij
zichzelf toen, in zijn vroege twintiger jaren en als een naar eigen
zeggen vredelievend persoon, meer als een pure wiskundige zag dan als
een informaticus, besloot Diffie toch om programmeervaardigheden aan te
leren. Op deze manier wilde hij zijn vaardigheden uitbreiden met
praktischere kennis.

Het zou goed uitpakken voor hem. Na zijn afstuderen aan de technische
universiteit in 1965, accepteerde Diffie een baan bij Mitre, een
defensiecontractant die slechts een paar jaren daarvoor was afgesplitst
van MIT's Lincoln Laboratory. De baan hielp hem om tijdens de
Vietnamoorlog onder de dienstplicht uit te komen, terwijl het werk zelf
ook niets met de oorlog te maken had. Diffie zou helpen bij de
ontwikkeling van het computeralgebrasysteem Macsyma.

Diffie hoefde zelfs niet naar het kantoor van Mitre te komen. In plaats
daarvan kon hij werken vanuit het AI Lab van MIT, waar hij zich volledig
verdiepte in de nieuwe hacker-cultuur en haar vrije en collaboratieve
filosofie.

Toch week Diffie op sommige punten af van de typische hacker-ethiek. Hij
was niet van mening dat onbeperkte vrijheid in alle computeromgevingen
gewenst was en vond dat software in bepaalde contexten ook privacy moest
bieden. Naarmate mensen, bedrijven en regeringen hun activiteiten naar
het digitale domein gingen verplaatsen, zag hij in dat het belangrijk
zou worden om gevoelige data te beschermen, zoals persoonlijke
gezondheidsgegevens, bedrijfsfinanciën of militaire geheimen.

Diffie nam het initiatief om een virtuele `kluis' te bouwen. Net zoals
een fysieke kluis, moest de digitale variant eenvoudig te openen zijn
door de legitieme eigenaar van de gegevens, maar de toegang voor
iedereen anders beperken. Hij geloofde dat dit het type probleem was dat
sterke versleuteling kon oplossen.

Hoewel de hacker zijn kinderpassie eigenlijk niet echt had behouden (hij
geloofde dat alle relevante paden in het domein van cryptografie al
waren verkend) hielp zijn meerdere bij het AI Lab, wiskundige Roland
Silver, hem weer op de been. En toen Diffie ontdekte hoeveel vooruitgang
er was geboekt in het veld van crypto sinds hij het als kind
bestudeerde, werd zijn interesse opnieuw aangewakkerd.

Maar Diffie leerde nu ook dat de werkelijke voorhoede van cryptografie
waarschijnlijk verborgen bleef achter gesloten deuren. De National
Security Agency (NSA), de Amerikaanse inlichtingendienst die destijds in
het geheim werkte en officieel niet bestond, had al jaren de beste
cryptografen van het land binnengehaald. Het was waarschijnlijk dat elk
echt baanbrekend onderzoek, en de superieure cryptografische technieken
die daaruit voortkwamen, geclassificeerd bleef.

Het idee dat de NSA belangrijke informatie voor het publiek zou kunnen
achterhouden, zat Diffie helemaal niet lekker.

\section{Stanford}\label{stanford}

Toen de hackercultuur voor het eerst begon te verspreiden buiten de
campus van MIT, vond het een vroege thuis in de San Francisco Bay Area,
op de Stanford University. En zo zou ook Diffie zijn weg daarheen
vinden. Toen hij in 1969 vierentwintig werd en de leeftijdsgrens voor de
dienstplicht naderde, verliet de afgestudeerde van MIT Mitre om in
plaats daarvan te werken voor het AI Lab van Stanford. Hier vond hij
nieuwe uitdagingen die pasten bij zijn hernieuwde interesse in
cryptografie.

De eerste van deze uitdagingen werd geïnspireerd door John McCarthy. De
mede-oprichter van het AI Lab van MIT en originele ontwerper van de LISP
programmeertaal was verder gegaan met het oprichten van Stanford's AI
Lab, en hij leidde het onderzoeksinstituut toen Diffie daar voor het
eerst aankwam. Tegen die tijd had McCarthy een interesse ontwikkeld in
de toekomst van digitale handel, wat op zijn beurt Diffie ertoe aanzette
om te dromen van het geautomatiseerde kantoor, waar software wordt
gebruikt om werkgerelateerde documenten digitaal te creëren, verzamelen,
opslaan, bewerken en te verspreiden. Dit leidde hem tot het overwegen
van het vraagstuk van \emph{authenticatie}.

In de fysieke wereld worden documenten doorgaans geverifieerd door
middel van persoonlijke, geschreven handtekeningen. Mensen ondertekenen
een brief om te bewijzen dat ze echt degenen zijn die hem hebben
geschreven, of ze voegen hun handtekening toe aan een contract om het
juridisch bindend te maken. Maar, zo bedacht Diffie, naarmate er meer
documenten digitaal zouden worden, zouden mensen een digitaal equivalent
van een handtekening nodig hebben om te bewijzen dat zij echt degenen
zijn die de inhoud van deze documenten goedkeurden.

Het creëren van een dergelijke vorm van digitale authenticatie was
echter niet zo eenvoudig: Diffie en McCarthy brachten talloze uren door
in het onderzoeksinstituut, peinzend over mogelijke oplossingen. Het
hoofdprobleem was dat zelfs specifieke individuele gegevens,
bijvoorbeeld een lang, persoonlijk nummer, makkelijk waren om te
kopiëren. Iedereen kon zo'n digitale handtekening overnemen van één
contract en toevoegen aan een ander contract. Dat zou ze nutteloos
maken.

Een andere uitdaging ontstond in de schoot van het Advanced Research
Projects Agency (ARPA) van het ministerie van Defensie, dat in 1972 was
begonnen met het verbinden van grote onderzoeksinstellingen in het land
door middel van een computernetwerk: ARPAnet. Als onderdeel van dit
project, zocht Larry Roberts, de directeur van Information Processing
Techniques bij ARPA, naar manieren om berichten over het netwerk privé
te houden. Nadat de NSA hem elke hulp had geweigerd - het geheime
overheidsagentschap weigerde aan een dergelijk openbaar project te
werken - hoopte hij dat een van zijn belangrijkste onderzoekers
misschien een idee had.

Toen McCarthy, een van de belangrijkste onderzoekers, het probleem
besprak met de hackers in Stanford's AI Lab, erkende Diffie het belang
van dit vraagstuk. Als in de toekomst de communicatie steeds meer
elektronisch zou plaatsvinden (en dit was al meer en meer het geval) zou
het persoonlijk delen van encryptiesleutels waarschijnlijk onhaalbaar
worden, waardoor privégesprekken onmogelijk zouden zijn. Tenzij mensen
toegang zouden hebben tot hulpmiddelen om hun communicatie te
beveiligen, vreesde Diffie dat iedereen op elk moment potentieel in de
gaten kon worden gehouden. Een verontrustend toekomstbeeld.

Terwijl hij oplossingen probeerde te vinden voor deze uitdagingen,
groeide Diffie's hernieuwde interesse in cryptografie langzaam uit tot
een obsessie. Zijn enthousiasme werd nog aangewakkerd door het lezen van
\emph{The Codebreakers}, een boek uit 1967 van David Kahn dat de gehele
geschiedenis van cryptografie gedetailleerd beschrijft en deels
gebaseerd is op informatie van twee NSA-overlopers die naar de
Sovjet-Unie waren gevlucht. Hij was steeds meer vastberaden om de
cryptografische technieken en inzichten te achterhalen die nog steeds
door inlichtingendiensten werden onderdrukt.

Dit was echter niet de enige reden waarom de hacker van het AI Lab
uiteindelijk besloot om alles op alles te zetten.

\section{Rondreis door Amerika}\label{rondreis-door-amerika}

Toen Diffie in de zomer van 1973 een oude vriendin, de dierentrainster
Marie Fischer uit Brooklyn, bezocht, had hij niet verwacht verliefd te
worden. Maar toen dat gebeurde, veranderden zijn plannen drastisch. In
plaats van terug te keren naar de Westkust, besloot hij om zijn baan in
het AI Lab van Stanford op te zeggen en tijd met zijn nieuwe vriendin op
de weg door te brengen. De twee begonnen aan een rondreis door het land
in een oude Datsun 510.

Dit gaf Diffie toevallig ook de tijd en gelegenheid om zich echt te
wijden aan het ontdekken van superieure cryptografische technieken.
Levend van zijn spaargeld, nam hij Fischer mee op zijn zoektocht naar
aanwijzingen. Ze bezochten David Kahn, de auteur van \emph{The
Codebreakers}, in Great Neck op Long Island, snuffelden in bibliotheken
op zoek naar referentiewerken, en maakten afspraken met experts in
cryptografie uit zowel de academische wereld als het bedrijfsleven over
heel de VS.

Diffie hoopte uiteindelijk een \emph{logisch formele theorie} te
ontwikkelen, een wiskundig systeem dat als basis zou dienen voor
cryptografie. Om dit te bereiken, geloofde hij dat hij bij de basis
moest beginnen.

Het eenvoudigste cryptografische fenomeen dat hij kon vinden was de
\emph{eenrichtingsfunctie}: een vergelijking waarvan het eenvoudig is om
de oplossing in één richting uit te rekenen, maar aanzienlijk moeilijker
om in de omgekeerde richting te berekenen.

Een zeer simpele eenrichtingsfunctie --- aangeduid als een
\emph{polynoom} --- zou herkenbaar moeten zijn voor iedereen die algebra
heeft gestudeerd op de middelbare school. Het zou er bijvoorbeeld uit
kunnen zien als x\^{}2 - 5x + 8. Als de invoer (x) in dit voorbeeld 16
is, is het vrij makkelijk te berekenen dat de vergelijking het resultaat
184 als uitkomst geeft. Echter, wanneer alleen de uitkomst van 184
bekend is, kan de vergelijking niet eenvoudig omgekeerd worden gebruikt
om te berekenen dat de oorspronkelijke invoer (x) 16 was. Een
eenrichtingsfunctie is het wiskundige equivalent van een
eenrichtingsstraat.

Bovendien leerde Diffie over het concept van een \emph{valdeur}, dat
naar men aannam een onderdeel kon zijn van sommige soorten
eenrichtingsfuncties. Een valdeur was in feite een geheim stuk
informatie, meestal een andere vergelijking, dat de omgekeerde
berekening ook gemakkelijk zou maken. Als de vergelijking in het
voorbeeld hierboven een valdeur bevatte, kon de uitkomst 184 worden
gebruikt om de invoer 16 net zo gemakkelijk te berekenen als het
oorspronkelijk was om 184 uit 16 te produceren. Als een
eenrichtingsfunctie een eenrichtingsstraat is, dan is de valdeurfunctie
een geheime tunnel in de tegenovergestelde richting zijn.

Deze concepten fascineerden Diffie. Eenrichtingsfuncties en valdeuren
leken intuïtief iets te zijn wat van grote waarde kon zijn in het veld
van cryptografie, hoewel hij niet precies wist hoe.

Via een gemeenschappelijke kennis leidde Diffie's vermoeden hem
uiteindelijk naar Martin Hellman, een dertigjarige assistent-professor
aan Stanford. Hellman deelde zowel Diffie's interesse in cryptografie
als zijn ideologische overtuiging dat deze technologie veel breder
beschikbaar moest zijn: hij had net een baan van de NSA afgewezen omdat
hij wilde dat zijn werk de bevolking ten goede zou komen. En ook Hellman
had nagedacht over hoe eenrichtingsfuncties breder toegepast konden
worden in de cryptografie.

Toen Diffie en Hellman elkaar voor het eerst ontmoetten in 1974 en het
idee bespraken in Hellman's kantoor aan Stanford, vonden ze niet meteen
de oplossing waar ze naar op zoek waren. Maar in elkaar vonden ze iemand
die geïnteresseerd was in hetzelfde probleem. Vanaf dat moment zouden ze
hun denkvermogen bundelen door ideeën aan elkaar af te toetsen en nieuwe
inzichten te delen.

Toen Diffie na meer dan een jaar reizen besloot om zich te vestigen in
de Bay Area, werden hij en Hellman goede vrienden en al snel collega's:
Hellman nam Diffie aan als deeltijdse onderzoeker aan de universiteit.

\section{Publieke
sleutel-cryptografie}\label{publieke-sleutel-cryptografie}

Op een doodgewone middag, toen Diffie op het huis van zijn voormalige
werkgever John McCarthy pastte, viel het kwartje eindelijk.

\emph{Twee} sleutels.

De oplossing was om \emph{twee} sleutels te gebruiken.

Cryptografen hadden altijd als vanzelfsprekend beschouwd dat
encryptiesleutels geheim moesten blijven, omdat ze ook dienden als
decryptiesleutels. Maar Diffie negeerde deze `vanzelfsprekende waarheid'
en kwam met het idee van \emph{sleutelparen}. In plaats van slechts één
geheime sleutel, zou iedereen twee sleutels hebben, namelijk een
privésleutel die inderdaad geheim moest blijven, en een publieke sleutel
die vrijuit gedeeld kon worden.

Diffie was van mening dat de sleutels wiskundig met elkaar verbonden
moesten zijn, waarbij de publieke sleutel in essentie zou worden
afgeleid van de privésleutel door middel van een soort
eenrichtingsfunctie. Zijn visie was erop gericht dat een verzender --
laten we haar `Alice' noemen, zoals cryptografen graag doen -- een
bericht zou versleutelen met haar privésleutel, waarna de beoogde
ontvanger, `Bob', het zou kunnen ontcijferen met behulp van haar
publieke sleutel.

Als Bob het bericht inderdaad kon ontcijferen met Alice's publieke
sleutel, zou dit bewijzen dat het bericht specifiek was versleuteld met
Alice's privésleutel. Dit zou in feite een vorm van authenticatie
mogelijk maken, aangezien de versleutelde versie van een bericht zou
dienen als Alice's digitale handtekening.

Deze digitale handtekeningen zouden zelfs krachtiger zijn dan geschreven
handtekeningen, aangezien een cryptografische handtekening alleen geldig
zou zijn in combinatie met het precieze stuk data dat was ondertekend.
Als een digitaal contract na ondertekening wordt gewijzigd, zou de
cryptografische handtekening niet meer overeenkomen. Op een bepaalde
manier zouden zowel de handtekening als de data zelf onmogelijk na te
maken zijn.

Bovendien zag Diffie in dat het omgekeerde ook kon werken. Alice zou een
bericht naar Bob kunnen versleutelen met \emph{Bob}'s publieke sleutel,
waarna Bob, en alleen Bob, het zou kunnen ontcijferen met zijn
privésleutel. \emph{Publieke sleutel-cryptografie} beloofde zowel
digitale authenticatie als veilige communicatie te bieden!

Toen het concept die avond aan Hellman werd uitgelegd, was hij het ermee
eens dat Diffie potentieel iets belangrijks had bedacht, ook al bestond
het idee alleen nog maar in ontwerpfase en moest de exacte wiskunde nog
worden uitgewerkt. In de weken die volgden, legde het duo de vroege
wiskundige basis om het idee tastbaarder te maken.

Dit resulteerde in het eerste gezamenlijk geschreven artikel van Diffie
en Hellman. `Multiuser Cryptographic Techniques', dat in het voorjaar
van 1976 werd gepubliceerd en kort daarna werd gepresenteerd op de
National Computer Conference in New York. In het artikel gaven de twee
onderzoekers toe dat er nog grote vragen onbeantwoord waren. Ze wisten
nog niet precies hoe encryptie of decryptie zou werken, noch hoe een
publieke sleutel zou worden afgeleid van een privésleutel.

`Op dit moment hebben we noch een bewijs dat er publieke sleutelsystemen
bestaan', gaven Diffie en Hellman toe in hun paper, `noch een
demonstratiesysteem'.\footnote{Whitfield Diffie and Martin E. Hellman,
  `Multiuser Cryptographic Techniques', AFIPS\ldots: Proceedings of the
  June 7-10, 1976, national computer conference and exposition:
  109--112.}

Maar ze kondigden aan dat ze bezig waren met iets groots: ze brachten
het idee van cryptografie met publieke sleutels naar voren.

\section{Ralph Merkle}\label{ralph-merkle}

Het artikel was nauwelijks gepubliceerd toen Hellman een brief van een
afgestudeerde student aan de Universiteit van Californië, Berkeley
ontving.

De student had zelf ook een werkstuk geschreven, zo legde hij uit in
zijn brief. Maar, `De mensen met wie ik probeer te praten, begrijpen
totaal niet wat er aan de hand is, of zien elke poging tot oplossing als
onmogelijk', schreef hij. Zijn frustratie druipte van de pagina. Hij
concludeerde: `Ik zie de mogelijkheid om samen te werken ontstaan, en ik
zou geïnteresseerd zijn in die mogelijkheid'.\footnote{\hspace{0pt}Levy,
  Crypto, 76.}

Ondertekend: Ralph C. Merkle.

Hoewel Merkle zo'n zeven of acht jaar jonger was dan Diffie en Hellman,
leek zijn verhaal niet zo verschillend van dat van hen. Hij was altijd
goed geweest met cijfers, stond consequent aan de top van elk wiskunde
klas, en had sinds de aanvang van zijn universiteitsloopbaan een
bijzondere interesse ontwikkeld voor computers.

In zijn laatste semester als student maakte hij tijdens een cursus over
computerbeveiliging kennis met het veld van cryptografie. Maar toen de
docent de cipher van Caeser en andere vormen van symmetrische
versleuteling besprak, realiseerde hij zich meteen dat de noodzakelijke
sleuteluitwisseling in persoon de toepasbaarheid ervan ernstig beperkte.
Merkle geloofde dat in een wereld waarin steeds meer communicatie
digitaal zou gaan verlopen, er dringend behoefte was aan een betere
oplossing.

In plaats van het hele land door te reizen op zoek naar antwoorden,
beperkte Merkle zijn zoektocht naar een oplossing tot zijn eigen
creatieve geest. En uiteindelijk bedacht hij een plan dat, althans tot
op zekere hoogte, het probleem zou kunnen oplossen.

Zo zou het werken.

Eerst zou Alice een groot aantal cryptografische puzzels maken,
misschien wel miljoenen of zelfs meer. De oplossing voor elke puzzel zou
bestaan uit een uniek getal en een even unieke geheime sleutel. Alice
zelf zou de oplossing voor elke puzzel al kennen; ze zou weten welke
getallen bij welke geheime sleutels horen. Maar elke individuele puzzel
zou ook door iemand anders kunnen worden opgelost, met een beetje
rekenkracht.

Alice zou vervolgens alle puzzels naar Bob sturen. Bob zou op zijn beurt
willekeurig een puzzel kiezen, en deze oplossen met een beetje
rekenkracht om het unieke nummer en de bijbehorende geheime sleutel te
vinden. Daarna zou hij het nummer (maar niet de bijbehorende geheime
sleutel) terugsturen naar Alice.

Op basis van het unieke nummer dat Bob terugstuurt, weet Alice
onmiddellijk welke geheime sleutel Bob daarbij heeft gevonden. Deze
geheime sleutel zou dan de versleutelingsleutel zijn die ze met elkaar
delen. Net als elke andere symmetrische versleutelingsleutel, zou deze
worden gebruikt om berichten tussen hen te coderen en te decoderen.

Een meeluisteraar die alle puzzels die Alice naar Bob had gestuurd zag,
en ook welk uniek nummer Bob terug naar Alice stuurde, zou nog steeds de
bijbehorende geheime sleutel niet weten.

Om erachter te komen welke sleutel Alice en Bob hebben gekozen, zou een
meeluisteraar beginnen met het willekeurig oplossen van alle puzzels
(via \emph{brute kracht}) om dat ene unieke nummer te vinden dat Bob
terugstuurde, wat tevens de geheime sleutel zou onthullen. Dit zou
echter een rekenkundig intensief proces zijn. Afhankelijk van de
hoeveelheid puzzels die aanvankelijk werden gemaakt (en de
moeilijkheidsgraad om elke puzzel op te lossen), zou het veel
rekenkracht kunnen vergen en veel tijd in beslag kunnen nemen.

Alice en Bob zouden daarom een asymmetrisch voordeel hebben ten opzichte
van de afluisteraar. Ze hoefden nauwelijks berekeningen uit te voeren om
het eens te worden over een geheime sleutel, terwijl de afluisteraar
heel veel berekeningen zou moeten uitvoeren voordat hij hun gesprek zou
kunnen ontcijferen.

Deze oplossing vereiste echter wel dat Alice en Bob veel data met elkaar
deelden in de vorm van puzzels. En de veiligheid van de oplossing zou
lineair toenemen met het totale aantal puzzels. Om het systeem tien keer
moeilijker te kraken, zouden ze tien keer zoveel puzzels moeten delen;
om het systeem honderd keer moeilijker te kraken, zouden ze honderd keer
zoveel puzzels moeten delen, en zo voorts. In de praktijk suggereren
data bronbeperkingen van normale gebruikers dat een goed gefinancierde
aanvaller met een supercomputer in veel gevallen berichten binnen enkele
dagen zou kunnen ontcijferen.

Desondanks had Merkle een oplossing bedacht die twee mensen de
mogelijkheid bood om tamelijk privé te communiceren zonder dat ze elkaar
vooraf persoonlijk hoefden te ontmoeten. Hoewel het niet perfect was,
was hij van mening dat deze techniek voor sleuteluitwisseling zeker
innovatief en potentieel nuttig was.

Merkle was echter niet in staat om iemand anders te overtuigen. Zijn
idee ontving weinig lof aan de Universiteit van Berkeley, en zijn
artikel werd afgewezen door \emph{Communications of the ACM}, het
prestigieuze blad van de Association for Computing Machinery (ACM). Het
verzenden van geheime sleutels via een onveilig netwerk werd door de
recensenten als onacceptabel beschouwd. Bovendien merkten zij op, was er
geen eerdere literatuur die sleuteluitwisseling had vastgesteld als een
belangrijk probleem.

Omdat hij het potentieel ervan aan niemand om hem heen kon uitleggen,
stond de teleurgestelde student op het punt om zijn idee helemaal los te
laten, totdat hij een proefdruk van Diffie en Hellman's paper in handen
kreeg. Merkle zag meteen dat het duo een soortgelijk probleem probeerde
op te lossen. Het bood een vorm van bevestiging die hij wanhopig nodig
had en hij besloot om contact op te nemen.

In tegenstelling tot Merkle's professor in Berkeley en de recensenten
van het tijdschrift, bewonderden Diffie en Hellman de vindingrijkheid
van het voorstel. Waar het idee van Diffie draaide om sleutelparen,
zorgde Merkle's aanpak er op een slimme manier voor dat Alice en Bob
overeen konden komen over een gedeelde sleutel op zo'n manier dat alleen
zij (gemakkelijk) konden berekenen wat deze sleutel is. Hoewel Diffie en
Hellman uiteindelijk concludeerden dat het schema niet robuust genoeg
was voor wat ze probeerden te bereiken, nodigde het puzzelschema hen uit
om het probleem vanuit een nieuw oogpunt te bekijken.

Hellman besloot om Merkle een zomerstage aan te bieden, waarbij hij de
student uit Berkeley bijstond in het herschrijven van zijn paper tot een
versie die uiteindelijk geaccepteerd werd door het ACM-journaal. En door
te bewijzen dat hij een creatieve en scherpzinnige denker was, werd
Merkle bovendien ook betrokken bij verdere discussies rondom publieke
sleutelcryptografie.

Nu waren ze met drie bezig met het probleem.

\section{De doorbraak}\label{de-doorbraak}

Uiteindelijk was het Hellman die de puzzelstukjes op hun plek liet
vallen.

Hoewel zijn oplossing niet aan alle vereisten voldeed waar hij en Diffie
op mikten (want er was geen digitale authenticatie), bedacht Hellman wel
een schema dat twee partijen privé liet communiceren, zonder de noodzaak
om persoonlijk op voorhand een encryptiesleutel te delen. Vergelijkbaar
met het puzzelschema van Merkle is het idee achter de Diffie-Hellman
sleuteluitwisseling (zoals deze oplossing bekend zou worden) dat Alice
en Bob een gezamenlijk geheim kunnen afspreken: in essentie, een
symmetrische encryptiesleutel die alleen zij kennen.

Om dit gedeelde geheim te genereren, zouden Alice en Bob sleutelparen
gebruiken, zoals oorspronkelijk door Diffie werd geopperd. Hun
privésleutels zouden in wezen gewoon zeer grote willekeurige getallen
zijn, die zelfs de snelste supercomputers niet binnen een miljoen jaar
zouden kunnen raden. Elke publieke sleutel kan vervolgens afgeleid
worden van een privésleutel door middel van een eenrichtingsfunctie. Het
berekenen van de publieke sleutel uit de privésleutel zou gemakkelijk
zijn, terwijl het berekenen van de privésleutel uit de publieke sleutel
in principe onmogelijk zou zijn.

Om het gedeelde geheim te produceren, zouden Alice en Bob elk hun eigen
\emph{privésleutel} vermenigvuldigen met de \emph{publieke} sleutel van
de ander. Dit zou hen beiden hetzelfde resultaat moeten geven: het
gedeelde geheim.

Dit werkt omdat in beide gevallen het gedeelde geheim in essentie een
combinatie is van beide privésleutels die telkens \emph{eenmaal} door
een eenrichtingsfunctie zijn gehaald. Wanneer Alice haar privésleutel
vermenigvuldigt met de publieke sleutel van Bob, is de
eenrichtingsfunctie al `ingebed' in de publieke sleutel van Bob, en
hetzelfde klopt als Bob zijn privésleutel vermenigvuldigt met de
publieke sleutel van Alice. Hoewel Alice en Bob de wiskunde in een
andere `volgorde' uitvoeren, moet de uitkomst hetzelfde zijn.

(Dit is, weliswaar enigzins vereenvoudigd, vergelijkbaar met hoe de
uitkomst van 2 x (3 x 5) en 3 x (2 x 5) beide 30 zijn. In deze analogie
is de 2 de privésleutel van Alice, de 3 is de privésleutel van Bob, x 5
is de eenrichtingsfunctie, en 30 is het gedeelde geheim.)

Tegelijk kan niemand anders het gedeelde geheim van Alice en Bob vinden.
Immers, als de twee \emph{publieke} sleutels worden vermenigvuldigd, is
het resultaat in wezen een combinatie van beide privésleutels, maar dan
\emph{twee keer} door een eenrichtingsfunctie heen gegaan, en dus een
keer te veel. (Als we de analogie voortzetten, is dit te vergelijken met
(2 x 5) x (3 x 5), wat 150 zou zijn in plaats van 30.)

Het genereren van het gedeelde geheim vereist dus toegang tot ofwel de
privésleutel van Alice of van Bob. Zolang zij deze geheim houden, kunnen
ze genieten van wiskundig gegarandeerde privécommunicatie.

De Diffie-Hellman sleuteluitwisseling doorbrak het status-quo in het
veld van de cryptografie en was in staat om iets te realiseren dat lang
onmogelijk werd geacht. Hun techniek had het potentieel om het volledige
veld van cryptografie dramatisch te transformeren. De twee
onconventionele cryptografen waren zich hier maar al te goed van bewust
toen ze hun tweede paper, `Nieuwe richtingen in Cryptografie', indienden
bij \emph{IEEE Transactions on Information Theory}, een prestigieus
wetenschappelijk tijdschrift uitgegeven door een professionele
vereniging voor elektronische techniek.

`Vandaag staan we aan de vooravond van een revolutie in cryptografie',
kondigden Diffie en Hellmann aan in hun paper, gepubliceerd in de editie
van november 1976 van het tijdschrift. Vooruitziend dat hun doorbraak
het begin van een grote omwenteling zou markeren, vervolgden ze:
`theoretische ontwikkelingen in informatietheorie en informatica beloven
aantoonbaar veilige cryptosystemen mogelijk te maken, waardoor deze
eeuwenoude kunst verandert in een wetenschap'.\footnote{\hspace{0pt}Whitfield
  Diffie and Martin E. Hellman, `New Directions in Cryptography', IEEE
  Transactions On Information Theory, vol.~IT-22, no. 6: 644.}

Het werd al snel duidelijk dat Diffie en Hellman een stroomversnelling
hadden veroorzaakt. Een nieuwe generatie cryptografen stond op het punt
om een waterval aan innovatie in de wereld van de cryptografie te
introduceren.

\section{RSA}\label{rsa}

Onder de eersten die geïnspireerd raakten door het paper van Diffie en
Hellman waren drie jonge wiskundigen van het MIT. Niet alleen hielden ze
allen van getallen, ze waren ook gedreven om lastige problemen op te
lossen. Ron Rivest, een assistent-professor aan de universiteit, Adi
Shamir, een gastprofessor, en Leonard Adleman, een computerspecialist
aan hetzelfde instituut, begonnen samen aan het ontwerpen van een
eenrichtingsfunctie specifiek voor publieke sleutel-cryptografie.

Enkele maanden later, in 1977, waren ze succesvol.

RSA, zoals hun algoritme werd genoemd (elke letter vertegenwoordigt een
van de uitvinders), benutte eeuwenoude wiskundige inzichten in
priemfactoren (de vermenigvuldiging van priemgetallen) om een
eenrichtingsfunctie met een ingebouwde valdeur te creëren. Door
berekeningen te doen met een maximaal getal, zouden twee opeenvolgende,
maar verschillende vermenigvuldigingen altijd het originele invoergetal
teruggeven. In feite zou de eerste vermenigvuldiging de data
versleutelen, terwijl de tweede het zou ontsleutelen.

Als een vereenvoudigde analogie, het is alsof je wiskunde toepast op een
klok. Aangezien een klok maar tot twaalf telt, is het resultaat van tien
plus vijf drie. Drie plus zeven, brengt ons vervolgens weer terug naar
tien. In deze analogie is tien de oorspronkelijke data, het optellen van
vijf staat voor het versleutelen, het aanvankelijke resultaat van drie
is de versleutelde data, en wanneer er weer zeven wordt opgeteld
vertegenwoordigt dit de functie van de valdeur, wat neerkomt op het
ontcijferen (decrypten).

RSA maakte het versleutelen en ontsleutelen van berichten compatibel met
het oorspronkelijke idee van Diffie. Als Alice naar Bob een versleuteld
bericht wilde sturen, kon ze simpelweg het bericht versleutelen met
Bob's publieke sleutel. Bob kon het bericht vervolgens ontcijferen met
alleen zijn privésleutel. Bob had de publieke sleutel van Alice niet
meer nodig, en beiden hadden geen gedeeld geheim meer nodig.

Misschien nog belangrijker, werkte het algoritme net zo goed in
omgekeerde volgorde. Twee opeenvolgende vermenigvuldigingen leverden
altijd het oorspronkelijke getal terug, ongeacht welke van de twee
vermenigvuldigingen eerst werd uitgevoerd. Voortbordurend op de
klok-analogie, waarbij tien plus vijf plus zeven ons terugbracht naar
tien, brengt de omgekeerde volgorde (tien plus zeven plus vijf) ons ook
terug naar tien.

In de praktijk betekende dit dat een privésleutel kon worden gebruikt om
data te versleutelen, die vervolgens kon worden ontcijferd met de
bijpassende publieke sleutel. Hierdoor maakte RSA een cryptografisch
handtekeningenschema mogelijk: Alice kan een bericht versleutelen, dat
Bob (of iemand anders) kan ontcijferen met haar publieke sleutel, wat
bewijst dat het bericht daadwerkelijk is versleuteld met Alice's
privésleutel.

Terwijl de Diffie-Hellman sleuteluitwisseling privécommunicatie mogelijk
maakte, faciliteerde RSA voor het eerst ook een vorm van digitale
authenticatie.

En tot slot, RSA-encryptie schaalde (ongeveer) exponentieel. Gewoon een
paar cijfers toevoegen aan de originele priemgetallen die in dit schema
werden gebruikt, zou het veel moeilijker maken om later te ontdekken
welke priemgetallen werden gebruikt, en daarmee ook veel moeilijker om
de encryptie te breken. Met publieke sleutels zo klein dat zelfs
informele computergebruikers ze gemakkelijk konden delen, was
beveiliging voor miljoenen jaren wiskundig gegarandeerd.

Met RSA was Diffie's visie voor publieke sleutelcryptografie echt
werkelijkheid geworden.

\section{David Chaum}\label{david-chaum}

Een andere jonge cryptograaf die geïnspireerd werd door de doorbraak van
Diffie en Hellman was David Chaum.

Chaum groeide op in de buitenwijken van Los Angeles in de jaren 60 en
begin jaren 70, waar hij al op jonge leeftijd een natuurlijke interesse
voor beveiligingstechnologie ontwikkelde. Aanvankelijk ging zijn
belangstelling uit naar hardware, zoals deursloten, inbraakalarmen en
fysieke kluizen. Aan het einde van zijn jeugdjaren resulteerde dit in
het ontwerpen van een nieuw type slot. Chaum kwam zelfs dichtbij het
verkopen van zijn ontwerp aan een grote fabrikant.

Maar tegen die tijd begon zijn interesse tevens te groeien op het
relatief nieuwe gebied van informatietechnologie. Hij wisselde
regelmatig van universiteit, beginnend aan de University of California,
Los Angeles nog voordat hij de middelbare school had afgerond, schakelde
over naar de University of Sonora in Mexico om dicht bij zijn vriendin
van dat moment te zijn, en studeerde uiteindelijk af aan de University
of California, San Diego. Chaum studeerde informatica en wiskunde, en
tegen de late jaren '70 leerde hij over cryptografie en de recente
doorbraken in dat veld.

Chaum zag het potentieel van de publieke sleutelcryptografie natuurlijk
ook, omdat hij eveneens glimpsen van de toekomst had opgevangen. Hij
voorzag dat computers steeds gebruikelijker zouden worden, tot het punt
waarop elk huishouden er een in huis zou hebben. En terwijl ARPAnet
langzaam begon te transformeren naar het (meer algemeen toegankelijke)
vroege internet, verwachtte hij dat elektronische communicatie de wereld
zou transformeren.

Net zoals Diffie dat voor hem deed, herkende Chaum ook dat deze
transformatie naar een tamelijk dystopische toekomst kon leiden. Hij
begreep dat als berichten, documenten of bestanden over het internet
werden verzonden, al deze data het risico liep om op een nooit eerder
geziene schaal gemonitord, onderschept en geëxploiteerd te worden door
tirannen. Chaum maakte zich zorgen dat mensen zich anders gaan gedragen
als ze geloven dat ze mogelijk bekeken worden. Massasurveillance zou een
gevangenis voor de geest creëren, conformiteit bevorderen, en
uiteindelijk fundamentele vrijheden vernietigen, aldus Chaum.

`De cyberruimte kent niet dezelfde fysieke beperkingen', legde Chaum
later uit aan een journalist van het technologiemagazine \emph{Wired}.
`Er zijn geen muren\ldots{} het is een andere, angstaanjagende, vreemde
plek en met identificatie wordt het een nachtmerrie van panoptisch
toezicht. Juist? Alles wat je doet kan door iedereen gekend zijn, kan
voor altijd worden opgenomen. Het is volkomen tegenstrijdig met het
basisprincipe dat ten grondslag ligt aan de mechanismen van de
democratie'.\footnote{Steven Levy, `E-Money', Wired, 1 december 1994,
  \href{https://www.wired.com/1994/12/emoney/.}{online}}

Maar hij zag dat er nu een alternatief was: een andere mogelijke
toekomst. Chaum besefte dat de gloednieuwe ontwikkelingen in de
cryptografie konden worden ingezet als verdediging. De maatschappij
stond op een kruispunt, en innovaties zoals publieke sleutelcryptografie
boden hoop voor een wereld waarin mensen de macht hadden over hun eigen
data.

Gedeeltelijk omdat hij wist dat Ralph Merkle naar Berkeley was gegaan
(maar niet op de hoogte was dat Merkle's initiële puzzelschema daar niet
bepaald goed was ontvangen), koos Chaum voor de universiteit in de Bay
Area om zijn doctoraat te behalen. Hier was hij van dichtbij getuige van
hoe het veld van cryptografie tegen het einde van dat decennium
evolueerde van een niche interesse, exclusief voor kleine
universiteitsafdelingen en academische tijdschriften, naar een kleine
revolutie in het veld van de informatica, aangedreven door een
toegewijde en groeiende gemeenschap van gelijkgestemde wiskundigen.

Het resulteerde in de allereerste \emph{Crypto}-conferentie in 1981,
georganiseerd aan de Universiteit van Californië, Santa Barbara. De
grootste vernieuwers in het veld -- Diffie, Hellman, Merkle, Rivest,
Shamir, Adelman -- en ongeveer vijftig andere cryptografen waren
aanwezig, met velen die elkaar voor het eerst in persoon ontmoetten. Ze
presenteerden hun nieuwste documenten, bespraken mogelijke verbeteringen
aan bestaande schema's, of leerden elkaar gewoon wat beter kennen
terwijl ze een avond doorbrachten met barbecuen op het strand.

Echter, dit gebeurde allemaal tot grote ontsteltenis van de NSA. Net als
de nieuwe lichting cryptografen die samengekomen waren bij Crypto 1981,
herkende ook de overheidsorganisatie het potentieel van publieke
sleutelcryptografie. Maar in tegenstelling tot het optimisme en de
vreugde op de conferentie, waren mensen binnen de
inlichtingengemeenschap bezorgd dat sterke encryptie hun hele werkwijze
in gevaar kon brengen. Ze wilden deze cryptografische revolutie stoppen
voordat het te ver ging, en in de nasleep van de conferentie begon de
NSA waarschuwingen te geven aan wetenschappelijke organisaties om geen
presentaties te faciliteren, zoals die tijdens Crypto 1981 zo openlijk
tentoongesteld werden.

Toen hij hiervan hoorde, besloot Chaum terug te vechten. Hij ondernam
actie om ervoor te zorgen dat de conferentie geen eenmalig evenement zou
blijven. Uitgerust met een lijst van namen en contactinformatie die door
Adleman was verstrekt, begon de promovendus contact op te nemen met alle
grote namen in het veld van crypto. Om geen aandacht te trekken van de
NSA, stuurde hij fysieke brieven of ontmoette de mensen persoonlijk,
maar vermeden telefoongesprekken over het onderwerp. Uiteindelijk wist
Chaum iedereen terug samen te brengen in Santa Barbara op Crypto 1982,
terwijl hij ook een Europese conventie (\emph{Eurocrypt}) in Duitsland
organiseerde in datzelfde jaar.

Daarnaast richtte Chaum de \emph{International Association for
Cryptologic Research} op, een non-profit organisatie die belast werd met
het bevorderen van onderzoek in de cryptografie. Hij kondigde de
oprichting van deze organisatie aan tijdens de Crypto conferentie van
1982, wat opnieuw voor veel irritatie zorgde bij de NSA.

De grootste bijdragen van Chaum aan het veld van de cryptografie waren
echter niet de evenementen die hij organiseerde of de organisatie die
hij oprichtte, maar de technieken die hij ontwierp.

\section{Remailers}\label{remailers}

Publieke sleutelcryptografie maakte het mogelijk voor twee mensen die
elkaar nog nooit hadden ontmoet om berichten uit te wisselen die alleen
zij konden lezen. Dit bood privacy in communicatie --- een baanbrekende
ontwikkeling.

Maar het was geen wondermiddel, besefte Chaum. Zelfs met de publieke
sleutelcryptografie, bleef een aanzienlijke privacy-risico bestaan:
verkeersanalyse kon onthullen wie met wie praatte, en wanneer.

Dergelijke \emph{metadata} kunnen meer over iemand onthullen dan iemand
prettig zou vinden. Een onderzoeksjournalist wil misschien niet dat zijn
bronnen bekend worden, bijvoorbeeld. Of burgers uit landen met
autoritaire regimes willen misschien niet dat iemand weet dat ze
communiceren met een politieke dissident. Of misschien geeft een
werknemer die informeert naar werkgelegenheden bij een concurrerend
bedrijf de voorkeur aan het feit dat zijn baas niet ontdekt dat hij
contact heeft gehad met dit bedrijf.

In zijn paper uit 1981 getiteld `Ontraceerbare elektronische post,
retouradressen en digitale pseudoniemen', stelde Chaum een oplossing
voor dit probleem voor, eveneens gebaseerd op publieke
sleutelcryptografie.\footnote{David Chaum, `Untraceable Electronic Mail,
  Return Addresses, and Digital Pseudonyms', Communications of the ACM
  24, 2: 84--90.}

Dit is hoe het werkte.

Als Alice een bericht aan Bob wil sturen, zou ze eerst zijn publieke
sleutel moeten nemen en het bericht daarmee versleutelen. Op deze manier
zou alleen Bob het bericht moeten kunnen ontcijferen.

Maar ze zou het bericht niet rechtstreeks naar Bob sturen: een speurder
die hun verbindingen monitort, zou dan kunnen zien dat de twee
communiceren, wat Alice juist probeert te vermijden.

In plaats daarvan zou Alice het versleutelde bericht nemen en daaraan
het e-mailadres van Bob toevoegen. Vervolgens zou ze dit hele pakket
(het versleutelde bericht voor Bob en zijn e-mailadres) nog een keer
versleutelen, maar nu met de publieke sleutel die geassocieerd is met
een speciale mixserver. Het bericht zou nu twee lagen van codering
hebben: één laag voor het originele bericht en een andere laag voor het
hele pakket. Alice zou dit dubbel versleutelde pakket dan naar de mixer
sturen.

De server van de mixer zou op zijn beurt het pakket ontcijferen met zijn
privésleutel, om het originele gecodeerde bericht en Bob's e-mailadres
te vinden. De mixer zou dit gebruiken om het gecodeerde bericht naar Bob
te sturen, die het vervolgens zou ontcijferen om Alice's originele
bericht te lezen. Als tussenpersoon brak de mixer de directe link tussen
Alice en Bob, waardoor het moeilijker werd voor de speurneus om te
concluderen dat ze communiceerden. Tenzij het duidelijk is uit de inhoud
van het bericht, zou Bob zelf ook niet weten dat het van Alice afkomstig
was.

Dit basisopzet functioneert zolang je op de mixer kan rekenen en zolang
deze niet aan de gluurder (of aan Bob) onthult dat hij het versleutelde
pakket eerst van Alice heeft ontvangen. Maar Chaum maakte duidelijk dat
het zelfs niet nodig is om de mixer te vertrouwen.

Om het risico van een onbetrouwbare mixer te verminderen, kan Alice
meerdere mixers gebruiken. In dat geval zou Alice een gecodeerd pakket
naar mixer 1 sturen, dat mixer 1 kan ontcijferen om het e-mailadres van
mixer 2 en een ander gecodeerd bericht te vinden. Dit bericht kan mixer
2 vervolgens ontcijferen om het e-mailadres van mixer 3 te vinden en nog
een ander gecodeerd bericht. Dit bericht kan mixer 3 dan ontcijferen om
inderdaad het e-mailadres van Bob en nog een gecodeerd bericht te
vinden. Dit laatste bericht kan Bob ten slotte ontcijferen om het
originele bericht te lezen. Door elk één laag van de encryptie te
verwijderen en het pakket naar de volgende ontvanger door te sturen,
zouden de mixers weten van wie ze het pakket hebben ontvangen en naar
wie ze het hebben doorgestuurd, maar geen van hen zou weten waar het
pakket oorspronkelijk vandaan kwam en wie de eindontvanger was.

Een luistervink die Bob's communicatie probeert te monitoren, zou
mogelijk kunnen weten dat mixer 3 uiteindelijk het bericht naar Bob
stuurde, maar zelfs als mixer 3 deze informatie met de indringer deelt
op verzoek (of gerechtelijk bevel), helpt dit de indringer slechts één
stap vooruit. Als er ook maar één van de mixers weigert mee te werken,
stopt het spoor. (En als de mixers in verschillende rechtsgebieden
verblijven, kunnen zelfs wetshandhavingsautoriteiten uit een bepaald
land het moeilijk hebben om de oorsprong van een bericht na te gaan.)

Zo had Chaum het cruciale conceptuele fundament gelegd voor wat later
bekend zou worden als \emph{remailers}.

En zijn bijdragen aan het veld van cryptografie zouden daar niet
ophouden\ldots{}

\chapter{Denationalisatie van geld}\label{denationalisatie-van-geld}

Economische depressie en valuta-devaluaties hadden geleid tot fascisme
en, uiteindelijk naar de Tweede Wereldoorlog.

Maar Friedrich Hayek merkte met spijt op dat veel van zijn collega's in
de academia destijds niet herkenden welke schade het monetair beleid had
aangericht. In plaats daarvan schreven vele intellectuelen van die tijd
de opkomst van het fascisme toe aan het falen van de vrije markten.
Terwijl economen, politicologen en andere academici in het Westen
alternatieve manieren overwogen om de samenleving te organiseren, werden
socialistische ideeën steeds populairder, een ontwikkeling die Hayek
ronduit gevaarlijk vond.

Om deze trend tegen te gaan, nam de Oostenrijker het op zich om uit te
leggen waarom socialisme niet het antwoord was, maar deel uitmaakte van
het probleem. Het resulterende boek, \emph{De Weg naar Slavernij}, nam
een meer politieke benadering, hoewel het sterk werd geïnformeerd door
economische inzichten zoals het economisch calculatieprobleem.
Geschreven en gepubliceerd tijdens de oorlogsjaren, was de kernthese dat
collectivistische ideologieën, inclusief zowel fascisme als socialisme,
de neiging hebben om te leiden tot totalitarisme. Het zou het meest
succesvolle boek van Hayek's carrière worden.

Het was wellicht een geschikt moment voor Hayek om zijn aandacht en
energie te verschuiven van de economie naar de politieke sfeer. Na twee
decennia van debatten en rivaliteit, werd het duidelijk dat niet zijn,
maar de ideeën van Keynes over het monetaire beleid mensen en instituten
binnen universiteiten, beleidsinstellingen en overheidsinstellingen voor
zich wonnen. Als de vrije marktideologie al zou overleven, zou het
waarschijnlijk gebaseerd zijn op de Keynesiaanse doctrine.

Dit werd bevestigd toen vertegenwoordigers van de geallieerde landen,
vol vertrouwen in hun overwinning in de oorlog, in 1944 bijeenkwamen in
het Mount Washington Hotel in Bretton Woods, New Hampshire. De
deelnemers waren er om een nieuw monetair systeem te ontwerpen voor de
periode na de oorlog en Keynes was uitgenodigd om het Verenigd
Koninkrijk te vertegenwoordigen. Hayek, die tegen die tijd ook de Britse
nationaliteit had gekregen, werd helemaal niet uitgenodigd.

Het belangrijkste resultaat van de Bretton Woods-conferentie kan worden
beschouwd als een herinvoering van de goudwisselstandaard, maar in
tegenstelling tot de vooroorlogse klassieke goudstandaard was deze sterk
gericht op de Amerikaanse dollar. Internationale handel zou in dollars
worden uitgevoerd, terwijl deze dollars op hun beurt konden worden
ingewisseld voor goud: één troy ounce per \$ 35. Andere nationale
valuta's zouden vaste wisselkoersen vaststellen ten opzichte van de
dollar, en centrale banken werden verwacht de rentetarieven te sturen om
hun nationale valuta's te stabiliseren.

Bretton Woods was ook de geboorteplaats van twee nieuwe internationale
monetaire instellingen, met Keynes als drijvende kracht achter hun
oprichting. De eerste hiervan, het Internationaal Monetair Fonds (IMF),
kreeg de taak om toezicht te houden op internationale wisselkoersen en
kon geld lenen aan landen in financiële problemen. Het tweede, de
Wereldbank, zou ook leningen verstrekken, maar met een sterkere focus op
het heropbouwen van het naoorlogse Europa (en later andere
ontwikkelingslanden).

Het \emph{Bretton Woods-systeem}, zoals het later genoemd werd, werd
kort na de nederlaag van de As-mogendheden in 1945 uitgerold, maar dan
alleen in het Westen. Hoewel de Sovjet-Unie vertegenwoordigers naar
Bretton Woods had gestuurd, weigerde deze communistische staat
uiteindelijk om de overeenkomst te ratificeren, waarbij ze de IMF en de
Wereldbank beschuldigden `takken van Wall Street'\footnote{\hspace{0pt}Edward
  S. Mason and Robert E. Asher, `The World Bank Since Bretton Woods: The
  Origins, Policies, Operations and Impact of the International Bank for
  Reconstruction', 29.} te zijn. In plaats daarvan nam het vele
Oost-Europese landen over na de oorlog. Het markeerde het begin van de
Koude Oorlog, waarin (een Keynesiaanse variant van)
vrijemarktkapitalisme in de Verenigde Staten en West-Europa op vele
manieren zou concurreren met de socialistische doctrine in het Oosten.

Keynes was er echter niet om de effecten van het Bretton Woods-systeem
en de instellingen van deze nieuwe monetaire orde te bestuderen. De
econoom overleed kort na de oorlog, in 1946.

Hoewel Hayek zijn rivaal had overleefd, erkende hij dat de ideeën van
Keynes de voorkeur genoten boven de zijne. Hij had niet het gevoel dat
de nieuwe generatie Keynesiaanse economen echt geïnteresseerd was in een
eerlijk en zinvol debat. De Oostenrijker besloot de London School of
Economics te verlaten en zich aan te sluiten bij de Universiteit van
Chicago. Daar trok hij zich grotendeels terug uit het vakgebied van de
economie en verschoof zijn focus naar de wereld van politiek en
filosofie.

\section{Het tijdperk van Keynes}\label{het-tijdperk-van-keynes}

Halverwege de twintigste eeuw was `het tijdperk van Keynes' onmiskenbaar
aangebroken in het kapitalistische Westen. Overheden, centrale banken en
andere instellingen bepaalden het economisch beleid en economen
trachtten het theoretische raamwerk achter fiscale planning,
stabilisatiebeleid en monetaire stimuli verder uit te werken. Zo
introduceerde bijvoorbeeld Professor William Phillips van de London
School of Economics de Phillips-curve, die aangaf dat er een omgekeerd
verband was tussen inflatie en werkloosheid. Dit was precies zoals de
Keynesiaanse theorie suggereerde: meer overheidsuitgaven resulteerden in
meer banen.

Politici kregen de theoretische rechtvaardiging om met overheidsuitgaven
een economische neergang te bestrijden. Dit leek een tijdje de oplossing
voor alles, aangezien Keynesiaanse beleidslijnen overal in de westerse
wereld werden uitgerold. In het Verenigd Koninkrijk had premier Winston
Churchill in 1944 volledige werkgelegenheid tot een nationaal
beleidsdoel verheven. In de Verenigde Staten ondertekende Franklin D.
Roosevelt's opvolger, president Harry S. Truman, in 1946 de Employment
Act, waardoor de uitvoerende tak van de regering verantwoordelijk werd
voor het beheer van de economie. En in Europa had het naoorlogse
herstelprogramma, informeel bekend als het Marshallplan, het hele
westelijke deel van het continent omgevormd tot een soort Keynesiaans
laboratorium.

Iets later in de Verenigde Staten was Truman's opvolger Dwight D.
Eisenhower de eerste president die het Keynesiaanse economiebeleid echt
ten volle toepaste. Geconfronteerd met enkele korte recessies midden
jaren '50, bestreed de president werkloosheid door grote investeringen
te doen in snelwegen-infrastructuur. En nadat de Sovjet-Unie erin
slaagde de Spoetnik-satelliet in een baan om de aarde te brengen, volgde
Eisenhower dit op met aanzienlijke financiële injecties in het
ruimtevaartprogramma van NASA. Het leek te werken zoals bedoeld: de
Amerikaanse economie bloeide op.

Maar toen Eisenhower tegen het einde van zijn presidentstermijn de
uitgaven moest beperken, resulteerde dit vrijwel onmiddellijk in een
nieuwe depressie. Het Amerikaanse publiek was hier niet blij mee. Ze
waren van mening dat Eisenhower alle middelen voorhanden had om deze
trend te keren, maar dat hij deze keer naliet om in te grijpen.

Tegen de tijd dat de verkiezingen voor de opvolger van Eisenhower
aankwamen, was de economische neergang een groot thema geworden in de
presidentiële campagnes. Terwijl de jonge Democratische kandidaat John
F. Kennedy zijn aanhangers mobiliseerde onder het motto `Laten we dit
land weer in beweging krijgen', worstelde de Republikeinse kandidaat en
zittend vicepresident Richard Nixon om zich te distantiëren van
Eisenhower's conservatieve fiscale benadering aan het einde van zijn
tweede termijn. Toen Kennedy uiteindelijk met een ongelooflijk kleine
marge de race won, bleef een verslagen Nixon ervan overtuigd dat hij de
verkiezingen zou hebben gewonnen als het niet was geweest voor de
bezuinigingen van Eisenhower.\footnote{\hspace{0pt}Andrew F. Brimmer,
  `Remembering William McChesney Martin Jr.', Federal Reserve Bank of
  Minneapolis, 1 september 1998,
  \href{https://www.minneapolisfed.org/article/1998/remembering-william-mcchesney-martin-jr.}{online}}

Gedurende zijn presidentschap bevestigde Kennedy dat hij royaal zou
investeren in de economie op basis van Keynesiaanse principes. Hij blies
Eisenhower's initiële uitgavendrift snel nieuw leven in, door geld te
pompen in het Amerikaanse ruimtevaartprogramma, terwijl hij ook
rijkelijk uitgaf aan het leger. Na de moord op de president slechts
enkele jaren later, in 1963, zette zijn vicepresident en opvolgers
Lyndon B. Johnson dit beleid voort, met de Vietnamoorlog als grote,
nieuwe geldverslindende factor.

In 1968, aan het eind van Johnson's presidentschap, was Nixon klaar voor
een nieuwe gooi naar het ambt. En deze keer won hij. Toen hij het roer
van Johnson overnam, leek Nixon, in zijn hart conservatief op fiscale
vlak, aanvankelijk bereid en gretig om een einde te maken aan deze
Keynesiaanse vrijgevigheid. Hij wees er op dat de regering jarenlang
tientallen miljarden dollars meer had uitgegeven dan het aan belastingen
ontving.

Maar toen de bezuinigingen van Nixon onmiddellijk resulteerden in een
milde economische recessie, besloot de president alsnog mee te spelen
met het spel. Hij verklaarde zichzelf een Keynesiaan, kondigde een plan
voor volledige werkgelegenheid aan en stelde een expansieve begroting
voor om de economie te stimuleren. Nixon had een waardevolle les geleerd
tijdens zijn mislukte poging om een decennium eerder president te
worden, en was klaar om de nieuwe politieke werkelijkheid te
accepteren.\footnote{\hspace{0pt}Wapshott, `Keynes \& Hayek', 242.}

Zoals de president verklaarde in de State of the Union van 1970:

`Ik erken de politieke populariteit van uitgavenprogramma's, en zeker in
een verkiezingsjaar'.\footnote{\hspace{0pt}Wapshott, `Keynes \& Hayek',
  242.}

De Keynesiaanse leer stelde eigenlijk zowel overheidsuitgaven als
bezuinigingen als vereiste: uitgaven om de economie een impuls te geven,
en bezuinigingen wanneer dit niet langer noodzakelijk was, om
versnellende inflatie te voorkomen. Het probleem was echter dat het nu
duidelijk werd dat hoewel verhoogde uitgaven verkiezingen konden winnen,
bezuinigingen dat niet deden. Nixon had op de harde manier geleerd dat
het beter was om vast te houden aan het populairdere stadium van de
Keynesiaanse cyclus van uitgaven, en gemakshalve dat deel van de leer te
negeren dat voorschreef wanneer bezuinigingen nodig waren.

Dit probleem was natuurlijk voorzien. Hayek heeft altijd beweerd dat de
grootste nalatigheid van Keynes wellicht niet op economisch gebied lag,
maar op politiek gebied: het was over het algemeen niet zo dat je van
verkozen vertegenwoordigers kon verwachten dat ze de discipline zouden
aan de dag leggen die een \emph{anti-cyclische} aanpak vereist.

\section{De Nixon-schok}\label{de-nixon-schok}

De overdadige uitgavendrift van Nixon en een aantal van zijn voorgangers
kon in zekere zin worden verklaard als een uitvoering van de
Keynesiaanse doctrine. Maar tegen de jaren '70 werden sommige van de
grootste bezitters van dollars ongemakkelijk. Het werd steeds
duidelijker dat de Verenigde Staten boven hun stand leefden en veel meer
geld drukten en uitgaven dan ze in goud konden verantwoorden.

Sommige landen met grote dollarreserves besloten uiteindelijk om hun
papieren geld geleidelijk aan af te bouwen. Ze wilden dollars omzetten
in goud, wat makkelijker gezegd dan gedaan was, aangezien het
verplaatsen en opslaan van grote hoeveelheden goud een ernstige
logistieke uitdaging vormde. De Franse president Georges Pompidou
stuurde uiteindelijk een compleet oorlogsschip naar New York om het goud
van zijn land op te halen bij de plaatselijke Federal Reserve Bank. Er
waren sterke aanwijzingen dat de Britten hetzelfde plan hadden.

Tot Nixon er op 15 augustus 1971 een einde aan maakte.

Tegenwoordig bekend als de `Nixon-schok', kondigde de president in een
televisietoespraak aan dat de `opschorting' van de omwisselbaarheid van
dollars naar goud `in het belang van monetaire stabiliteit, en in het
beste belang van de Verenigde Staten' was. \footnote{\hspace{0pt}Richard
  Nixon, `Address to the Nation Outlining a New Economic Policy: 'The
  Challenge of Peace'\,', 15 augustus 1971.} Zonder voorafgaande
waarschuwing zorgde Nixon met een Uitvoeringsbevel er eigenhandig voor
dat er een bom terechtkwam onder het Bretton Woods-systeem: landen die
dollars hielden als onderdeel van hun reserves konden het papieren geld
niet langer omzetten in edelmetaal.

Omdat de dollar de enige munteenheid was die onder het Bretton
Woods-systeem omwisselbaar was naar goud, betekende dit onmiddellijk het
einde van de goudstandaard. Wat overbleef waren nationale, ongedekte
valuta, oftwel fiatvaluta. Toen kort daarna zes Europese landen ermee
instemden om de waarde van hun munteenheden aan elkaar te koppelen en ze
te laten fluctueren tegenover de dollar, werd het Bretton Woods-systeem
effectief volledig verlaten.\footnote{\hspace{0pt}Michael J. Graetz and
  Olivia Briffault, `A 'Barbarous Relic': The French, Gold, and the
  Demise of Bretton Woods', in The Bretton Woods Agreements, eds.~Naomi
  Lamoreaux and Ian Shapiro: 121--142.}

Ondertussen had Nixon het voornemen om in tijd voor de verkiezingen van
1972 de economie te laten opbloeien. Volgens hem kon hij dit realiseren
door de Federal Reserve ertoe aan te zetten de rente te verlagen.
Goedkoop geld zou leiden tot inflatie en dat zou op zijn beurt weer
leiden tot minder werkloosheid, zoals de Phillips-curve aangaf. Nixon
wees zijn economische raadgever Arthur Burns aan als voorzitter van de
Federal Reserve. Aangezien de eerdere beperkingen opgelegd door de
gouddekkingsratio nu waren opgeheven, oefende hij onmiddellijk druk uit
op de nieuwe voorzitter.

Nixon kreeg zijn zin. Hij won de herverkiezing. En tegen 1973 waren de
officiële, op de consumentenprijsindex-gebaseerde (CPI), inflatiecijfers
gestegen naar 9,6 procent, om vervolgens in de volgende jaren te stijgen
naar dubbele cijfers.\footnote{Federal Reserve Economic Data, `Consumer
  Price Index for All Urban Consumers', beschikbaar
  \href{https://fred.stlouisfed.org/series/CPIAUCSL}{online}}

\section{Inflatie}\label{inflatie}

Hayek beweerde dat kunstmatig lage rentes zouden leiden tot een
economische groei die niet vol te houden was, gevolgd door een pijnlijke
correctie, waarbij de markt het best met rust kon worden gelaten om een
natuurlijker evenwicht te vinden. Keynesianen daarentegen adviseerden
dat de overheid zou moeten ingrijpen met uitgaven om de economie weer op
het juiste spoor te krijgen na zo'n recessie.

Hayek erkende altijd dat dergelijke Keynesiaanse overheidsuitgaven op de
korte termijn inderdaad kunnen werken. Nieuw geld dat in de economie
gepompt werd door de overheid, zou het kunnen laten lijken alsof de
economie redelijk stabiel blijft. Maar onder die façade van aanhoudende
welvaart, waarschuwde Hayek, zou een donkerdere economische realiteit
schuilgaan: het creëren van geld was slechts een verlenging van de
verkeerde toewijzing van middelen die door de initiële verstoring werd
veroorzaakt.

Dit was misschien wel Hayeks grootste bezwaar tegen deze vorm van
inflatie: het zou na verloop van tijd erg moeilijk worden om het terug
te draaien. Inflatie is verslavend, waarschuwde de econoom. Zodra de
economie gewend raakt aan gemakkelijk geld, zal het alleen maar meer
nodig hebben om het kunstmatig hoge werkgelegenheidsniveau te handhaven.

In een vroeg stadium lijkt inflatie de economie inderdaad te stimuleren.
Als inflatie begint, drijft het de prijzen op naar een hoger niveau dan
bedrijven hadden verwacht. Dit wordt door hen als een aangename
verrassing gezien, omdat ze hun producten voor meer geld kunnen verkopen
dan ze hadden ingecalculeerd, waardoor ze grotere winsten genieten.
Bedrijven die in zwaar weer verkeerden en anders ten onder waren gegaan,
konden hun hoofd boven water houden dankzij deze onverwachte economische
opleving.

Maar als de inflatie langere tijd aanhoudt, legt Hayek uit, zullen
bedrijven uiteindelijk de verwachting van hogere toekomstige prijzen
moeten meewegen. Om concurrerend te blijven, zullen ze hun investeringen
in het productieproces moeten verhogen, wat de prijs van kapitaal en
arbeid omhoogdrijft, tot het punt waarop de algehele winstmarges weer
terug zijn op het punt waar ze begonnen.

Dit betekent ook dat de bedrijven die al kampten met problemen vóórdat
de inflatie toesloeg, weer het risico lopen om failliet te gaan.

Om te voorkomen dat deze bedrijven kopje onder gaan, en om het positieve
effect van het nieuwe geld te verlengen, is nog hogere inflatie nodig.
Deze hogere inflatie zal de prijzen nog meer opdrijven dan verwacht, wat
opnieuw een welkome verrassing betekent voor de bedrijven.

Maar als deze hogere inflatie ook aanhoudt, moet het natuurlijk
uiteindelijk opnieuw worden meegenomen in het productieproces, waardoor
de bedrijven die al worstelden weer het risico lopen om ten onder te
gaan.

Deze bedrijven zouden misschien gered kunnen worden door nog meer
inflatie, maar uiteindelijk, waarschuwde Hayek, kan nieuw geld alleen op
korte termijn stimuleren, terwijl over tijd steeds meer ervan nodig zou
zijn. Het zou onvermijdelijk resulteren in een economie met zowel hoge
inflatie als economische stagnatie.

Hayek:

`En aangezien, als inflatie al enige tijd aan de gang is, veel
activiteiten afhankelijk zullen zijn geworden van het voortduren ervan
in een progressief tempo, krijgen we een situatie waarin, ondanks de
stijgende prijzen, veel bedrijven verlies zullen lijden, en kan er
aanzienlijke werkloosheid zijn. Depressie met stijgende prijzen is een
typisch gevolg van alleen maar het afremmen van de stijging van de
inflatie zodra de economie zich heeft aangepast aan een bepaald
inflatietempo'.\footnote{\hspace{0pt}Friedrich A. Hayek, `Can We Still
  Avoid Inflation?' in `The Austrian Theory of the Trade Cycle and Other
  Essays', ed.~Richard M. Ebeling: 101.}

In economische termen wordt dit resultaat (een economie met zowel
inflatie van de munteenheid als economische stagnatie) stagflatie
genoemd.

\section{Het Cantillon-effect}\label{het-cantillon-effect}

Hayek maakte zich om diverse redenen zeer veel zorgen over inflatie (en
het uiteindelijke vooruitzicht van stagflatie).

Ten eerste kan inflatie het intertemporele prijssysteem verstoren.
Misschien wel het belangrijkste is, dat het de kredietmarkten in de war
kan brengen, aangezien schuldenaren en schuldeisers door inflatie op
zeer verschillende manieren worden beïnvloed: schuldenaren profiteren
als de reële waarde van hun schulden daalt, terwijl schuldeisers verlies
lijden omdat het geld dat ze later terugkrijgen minder waard zal zijn
dan het geld dat ze hebben uitgeleend.

Een tweede aspect van inflatie, en in de ogen van Hayek wellicht nog
bedrieglijker, is dat het de boekhoudkundige praktijken vertekent.
Inflatie kan namelijk winstcijfers oppompen, waardoor het lijkt alsof
het rendement op investeringen hoger is dan in werkelijkheid. Slimme
individuen kunnen misschien de devaluatie van de munteenheid meenemen in
hun winst-en-verliesberekeningen, maar de econoom merkt op dat
belastinginspecteurs nog steeds zullen aandringen op het belasten van de
`schijnwinsten'. Inflatie leidt in wezen tot een impliciete stijging van
de belastingen.

Maar Hayek's grootste bezorgdheid omtrent inflatie was een zorg die al
in de achttiende eeuw beschreven was door de Iers-Franse econoom Richard
Cantillon, en die bekend stond als het Cantillon-effect. \footnote{\hspace{0pt}Mark
  Blaug, `Economic Theory in Retrospect', 4th edition, 20--23.}

Het probleem, kort gezegd, is dat nieuw geld als eerste in omloop wordt
gebracht door het in de economie te spenderen. En wanneer dit nieuwe
geld voor het eerst wordt uitgegeven, wordt het uitgegeven in een
economie die nog steeds de `oude' prijzen hanteert, die nog niet de
extra bijgedrukte geldhoeveelheid weerspiegelen. Het spenderen van dit
geld leidt vervolgens tot het opdrijven van de prijzen van de eerste
goederen of diensten die ermee worden gekocht. De bedrijven die deze
goederen of diensten verkopen, genieten van extra winst.

Deze bedrijven krijgen de kans om het nieuwe geld als eerste opnieuw uit
te geven, terwijl de meeste prijzen in de economie nog niet zijn
aangepast om de toename van het geld aanbod te weerspiegelen. De
bedrijven die \emph{deze} betalingen ontvangen mogen vervolgens het geld
uitgeven in een economie waarin sommige prijzen zijn bijgesteld, maar
nog niet allemaal; toch een voordeel. Dit gaat door totdat het nieuwe
geld volledig door de economie is verspreid: prijsverhogingen vinden
eerst plaats in een deel van de economie, waarna het zich als een
rimpeling door de rest van de samenleving verspreidt.

Dit betekent dat degenen die `dichtbij' de bron van het geld zijn,
hiervan kunnen genieten voordat de prijzen aangepast zijn: ze ervaren
een toename in hun reëel inkomen. Degenen die zich aan de `buitenkant'
bevinden, zien alle prijzen stijgen voordat ze het nieuwe geld te pakken
krijgen. Als ze uiteindelijk een beetje van het nieuwe geld ontvangen,
biedt dit geen comparatief voordeel meer. En terwijl ze wachtten, hebben
ze een daling in hun reëel inkomen ervaren.

Zoals Hayek de bevindingen van Cantillon samenvatte:

`Alleen die personen profiteren van de toename van geld waarvan de
inkomens vroeg stijgen, terwijl voor personen waarvan de inkomens later
stijgen, de toename van de hoeveelheid geld schadelijk is'.\footnote{\hspace{0pt}Hayek,
  `Prices and Production', 203.}

Goudproducenten waren historisch gezien te vinden in het hart van het
Cantillon-effect. Maar naarmate de rol van goud in het financiële
systeem minder belangrijk werd en het Keynesianisme zijn intrede deed,
begonnen overheden steeds vaker hun plaats te claimen in het centrum van
het geldcreatieproces. Telkens wanneer overheden de geldvoorraad zouden
vergroten om de economie te stimuleren, zou dit in onevenredige mate de
overheid zelf ten goede komen. Door het Cantillon-effect zouden ook
overheidsmedewerkers, aannemers van de overheid en de bedrijven die het
dichtst bij de overheid staan qua economische nabijheid, zoals
financiële instellingen, profiteren.

Dit zou resulteren in hogere prijzen in de delen van de economie die
dicht bij de overheid staan, wat op zijn beurt weer meer middelen zou
aantrekken. Belangrijk echter is dat deze middelen niet naar dat deel
van de economie zouden stromen omdat ze daar de meeste waarde voor de
individuen in de samenleving bieden. In plaats daarvan zou het
Cantillon-effect de verdeling van middelen scheef trekken richting de
bron van nieuw geld, in de praktijk richting de overheid en de
financiële sector.

Aangezien het verslavende effect van inflatie vereist dat er steeds meer
van is, zou na verloop van tijd een overmatige hoeveelheid
bedrijfsactiviteit zich gaan richten op de overheid, volledig
afhankelijk van deze creatie van nieuw geld, maar zonder veel waarde aan
de samenleving toe te voegen. Hayek voorspelde een voortdurende
misallocatie van middelen ten gunste van de staat en ten nadele van de
rest van de economie.

En uiteindelijk, zo waarschuwde de econoom, kan voortdurende, overdadige
geldcreatie leiden tot de vernietiging van wat er nog over is van het
vrije-marktsysteem. Hoge inflatie kan de prijzen zo ernstig verstoren en
daarmee de verdeling van middelen, dat er uiteindelijk sterke publieke
druk ontstaat voor de invoering van prijscontroles.

Een rampzalige uitkomst, volgens Hayek:

`Openlijke inflatie is al erg genoeg, maar inflatie die onderdrukt wordt
door {[}prijs{]}controles is nog erger: het is het echte einde van de
markteconomie'.\footnote{\hspace{0pt}Hayek, `Can We Still Avoid
  Inflation?' 108.}

\section{De overheid buitenspel
zetten}\label{de-overheid-buitenspel-zetten}

Keynesianen, die tegen die tijd bijna alle mainstream economen
vertegenwoordigden, deelden Hayek's zorg over inflatie en in het
bijzonder stagflatie niet. Ze geloofden in feite dat stagflatie
onmogelijk was. Zoals de Phillips-curve had aangetoond, waren inflatie
en werkloosheid \emph{omgekeerd} gecorreleerd: een hogere inflatie zou
betekenen dat er minder werkloosheid was.

Maar de beleidsmaatregelen van Nixon begonnen de aannemingen van deze
Keynesianen te weerleggen. Een combinatie van lage rentetarieven en
stijgende olieprijzen door aanhoudende internationale strubbelingen
zorgde in het begin van de jaren 70 al snel voor een opkomende inflatie.
Maar nu verslechterden de economische vooruitzichten wereldwijd
tegelijkertijd. Het decennium zou in het teken komen te staan van
stagflatie. Dit bracht een crisis van geloof in de Keynesiaanse
denkschool met zich mee.

Hayek had zich in de aanloop naar de jaren '70 voornamelijk op politiek
gericht: hij had bijvoorbeeld uitgelegd hoe spontane orde ook gold voor
de ontwikkeling van het rechtsstelsel, waardoor hij gematigder werd
vergeleken met enkele van de meer radicale libertarische Oostenrijkse
economen. Maar halverwege de jaren '70 besloot hij dat hij niet langer
kon zwijgen. Na een lange academische loopbaan die hem van de
Universiteit van Chicago naar de Universiteit van Freiburg en
uiteindelijk de Universiteit van Salzburg bracht, keerde Hayek terug
naar het schrijven over monetair beleid.

Hayek, een Oostenrijker, was ervan overtuigd dat de Keynesiaanse leer en
zijn onjuiste ideeën over geld de wereldwijde economie aan het ruïneren
waren. Hij vond dat mensen hiervan bewust moesten worden gemaakt: dit
was opnieuw een prioriteit voor hem geworden. Hoewel Hayek al ver in de
zeventig was, was hij nog net zo vastberaden en compromisloos als
altijd. Hij beschreef inflatie nu niet alleen als schadelijk, maar ook
als ronduit onethisch, en vergeleek het zelfs met diefstal. Geld was
volgens Hayek al heel lang defect, en hij vond dat het hoog tijd was dat
het gerepareerd werd.

Hoewel decennia van onverantwoorde uitgaven door regeringen hadden
geleid tot stagflatie, vond Hayek het te makkelijk om simpelweg de
verantwoordelijke politici de schuld te geven. Hij geloofde dat het
probleem dieper lag. Overheidsuitgaven waren populair vanwege de
kortetermijnvoordelen die ze konden bieden. Al die geldcreatie gebeurde
in democratische samenlevingen, in feite op algemeen verzoek. Zolang de
overheid en haar instellingen de poortwachters van de munteenheid waren,
zouden invloedrijke belangengroepen politici er uiteindelijk toe
aanzetten om zo'n krachtig instrument in hun voordeel te gebruiken.

Hayek kwam dus tot de conclusie dat overheden helemaal geen
poortwachters van geld zouden moeten zijn:

`Een goede munteenheid, net als een goede wet, moet functioneren zonder
rekening te houden met de effecten die beslissingen van de uitgever
zullen hebben op bekende groepen of individuen', schreef
Hayek.\footnote{Friedrich A. Hayek, `Denationalisation of Money', 89.}
`Zelfs met de beste bedoelingen ter wereld, kan geen enkele regering
deze druk weerstaan {[}van zulke groepen of individuen{]}, tenzij ze
kunnen wijzen op een vaste barrière die ze niet kunnen
overschrijden'.\footnote{Friedrich A. \hspace{0pt}Hayek,
  `Denationalisation of Money', 91.}

Goud zou in theorie een dergelijke sterke barrière kunnen bieden.
Idealiter zou goud dan zelf als munteenheid worden gebruikt, in plaats
van dat het in bankkluizen wordt opgeslagen waar het fractionele
reservebankieren mogelijk maakt. Maar de onpraktische aard van het
direct gebruik van goud bij transacties, en de uitdagingen van een
veilige opslag, betekenen waarschijnlijk dat dit ideaal onhaalbaar is.

Een valuta gedekt door goud --- zoals bij de klassieke goudstandaard ---
was praktischer, maar dit vereist vertrouwen in regeringen om de valuta
niet te devalueren ten opzichte van goud, de dekkingsratio eenzijdig te
veranderen, of zelfs de converteerbaarheid volledig op te heffen. Dit
zijn inderdaad allemaal dingen die Hayek in zijn eigen leven regeringen
heeft zien doen.

Hayek had daarom altijd moeite gehad om de ideale oplossing te vinden
voor het problematische geldstelsel.

Maar nu, in de nadagen van zijn carrière, viel het kwartje eindelijk
voor hem.

`Midden in mijn wanhoop over de hopeloosheid om een politiek haalbare
oplossing te vinden voor wat technisch gezien het eenvoudigst mogelijke
probleem is, namelijk het stoppen van inflatie, heb ik ongeveer een jaar
geleden in een lezing een ietwat schokkende suggestie geopperd. Het
verder uitdiepen hiervan heeft geheel onverwachte nieuwe horizonten
geopend', schreef Hayek halverwege de jaren '70.

`Ik kon niet nalaten het idee verder uit te diepen, omdat de taak om
inflatie te voorkomen mij altijd van het grootste belang heeft geleken.
Niet alleen vanwege de schade en het leed dat grote inflaties
veroorzaken, maar ook omdat ik er al lang van overtuigd ben dat zelfs
milde inflaties uiteindelijk de terugkerende depressies en werkloosheid
veroorzaken. Deze zijn terecht een klacht tegen het systeem van vrij
ondernemerschap en moeten worden voorkomen als een vrije samenleving wil
overleven'.\footnote{\hspace{0pt}Friedrich A. Hayek, `Denationalisation
  of Money: The Argument Refined', 13.}

Het idee was dat geld aan de vrije markt overgelaten moest worden.

\section{Denationaliseren van geld}\label{denationaliseren-van-geld}

Hayek bepleitte zijn zaak voor het eerst in zijn toespraak tot de Goud
en Monetaire Conferentie in Genève in 1975\footnote{Opnieuw gepubliceerd
  in Friedrich A. Hayek, `Choice in Currency: A Way to Stop Inflation'.}.
In 1976 presenteerde Hayek dit onconventionele voorstel uitvoerig in
zijn boek \emph{Denationalisatie van Geld}. In essentie betoogde de
econoom dat banken volledig gedereguleerd zouden moeten worden, zodat ze
elk geld dat ze geschikt achten konden uitgeven: gedekt of ongedekt,
inflatoir of deflatoir, al dan niet in bestaan gebracht door leningen,
en tegen elk renteniveau dat ze wilden vragen.

Hayek stelde voor om het feitelijke staatsmonopolie op geld te
beëindigen.

In dit systeem (een behoorlijk radicale vorm van \emph{vrij bankieren})
zouden banken concurreren om klanten hun geld te laten gebruiken. Hayek
geloofde dat juist deze concurrentie nodig was om de beste vorm van geld
te ontwikkelen: zoals in elke vrije markt, zouden banken hun klanten een
product moeten leveren -- in dit geval, geld zelf -- met betere
eigenschappen dan de producten die hun concurrenten aanbieden. Hayek
redeneerde dat deze concurrentie tot verbetering zou leiden, omdat de
markt het beste beschikbare geld zou selecteren.

Een bijzonder belangrijke eigenschap van elke munteenheid is
ongetwijfeld de hoeveelheid geldeenheden, of specifieker, de
groeisnelheid van de geldvoorraad --- de grootste oorzaak van inflatie.
Gebruikers van een munteenheid zullen waarschijnlijk niet willen dat de
voorraad te snel uitgebreid wordt, omdat dit de koopkracht van de
eenheden die ze bezitten, zou schaden. Hayek geloofde dat concurrentie
in dit domein de uitgevers van valuta eerlijk zou houden: als één bank
te veel van zijn geld zou creëren, zouden klanten snel naar een
alternatief overstappen. In feite zou ongewenste inflatie in essentie
onmogelijk zijn, omdat mensen gewoonweg kunnen kiezen om niet mee te
doen.

Terwijl het vrijwel onmogelijk was gebleken om overheden te stoppen met
het opblazen van hun nationale munteenheden om hun uitgaven te
financieren, redeneerde Hayek dat de discipline van de vrije markt dit
probleem vanzelf zou oplossen.

Als een hypothese over hoe geld eruit zou zien in een wereld met vrij
bankieren, stelde de econoom zich voor dat banken waarschijnlijk zouden
streven naar een specifiek niveau van koopkracht, mogelijk gebaseerd op
een bepaalde prijsindex. Ze kunnen dit doen door meer valuta uit te
geven als de koopkracht van het geld boven dat doelniveau stijgt en door
valuta uit de circulatie te halen als de koopkracht daaronder zakt. Een
groot deel van het werk van de banken zou zijn om uit te zoeken wat
mensen als een wenselijk niveau van koopkracht voor hun valuta
beschouwen, zodat ze het soort geld kunnen selecteren dat het beste aan
hun behoeften voldoet.

`Elke uitgever van een aparte valuta zou in staat moeten zijn om de
hoeveelheid ervan te reguleren om het zo voor het publiek het meest
acceptabel te maken, en concurrentie zou hem daartoe dwingen', schreef
Hayek. `Sterker nog, hij zou weten dat de straf voor het niet waarmaken
van de opgewekte verwachtingen zou leiden tot het onmiddellijk verlies
van klanten'.

Hij voegde eraan toe: `{[}\ldots{]} de uitgevende banken, enkel geleid
door hun streven naar winst, zouden zo het publieke belang beter dienen
dan welke instelling dan ook die dat zogenaamd nastreeft'.\footnote{Friedrich
  A. Hayek, `The Argument Refined', 78.}

In `The Argument Refined', de herziene en uitgebreidere versie van het
boek dat twee jaar later werd gepubliceerd, stelde Hayek voor dat het
meest aannemelijke resultaat van een vrij bankensysteem niet een breed
scala aan valuta's met verschillende doelen voor koopkracht zou zijn. In
plaats daarvan speculeerde de econoom dat de overgrote meerderheid van
de mensen zich zou richten op één type geld dat een algemeen acceptabel
stabiliteitsdoel heeft. Dit stabiliteitsdoel zou op zijn beurt ook door
andere valuta's worden aangenomen, verwachtte Hayek, in wezen het
stabiliteitsdoel zelf veranderend in iets van een meta-valuta waarop
andere valuta's zouden worden gebaseerd.

Op het eerste gezicht zou dit als een toegeving aan het gebruik van een
consumentenprijsindex om stabiliteit te bepalen kunnen worden beschouwd,
en tot op zekere hoogte was dat misschien ook zo. Maar het is belangrijk
om te benadrukken dat Hayek's versie van dit idee niet verplicht of
bindend was, noch dat het een door de overheid aangewezen centraal
orgaan omvatte om te bepalen hoe zo'n index opgezet zou moeten worden.
Hij geloofde dat dit aan de vrije markt overgelaten moet worden, zodat
mensen vrij zijn om te kiezen voor welk doel zij het meest stabiel (of
anderszins wenselijk) beschouwen.

Dit zou inderdaad een valuta met een vaste voorraad kunnen omvatten, of
wat Hayek eerder in zijn carrière had omschreven als neutraal geld.

\section{Ongedekt geld}\label{ongedekt-geld}

Het voorstel van Hayek kreeg niet meteen de steun van al zijn collega's
binnen de Oostenrijkse school van economie. Veel Oostenrijkers waren van
mening dat de markt al een lange tijd geleden de beste vorm van geld had
gekozen. In een strijd die duizenden jaren heeft geduurd, had goud
gewonnen. Zij hielden vol dat goud nog steeds de beste vorm van geld
was, en dat nieuwe valuta's op z'n minst \emph{gedekt} zouden moeten
worden door het kostbare metaal.

Hoewel het in een vrij bankensysteem uiteraard mogelijk zou zijn om een
door goud gedekte valuta te bieden, verwachtte Hayek niet dat de meeste
mensen hiervoor zouden kiezen. Hij erkende dat de markt oorspronkelijk
goud had gekozen als de beste vorm van geld, maar Hayek geloofde dat
regeringen sindsdien de verdere ontwikkeling van geld hadden verhinderd
door monopolisatie en strikte regulatie. Hayek verwachtte dat geld
drastisch zou verbeteren als het aan de markt werd overgelaten.

`Ik geloof dat we het veel beter kunnen doen dan goud ooit mogelijk
heeft gemaakt. Overheden kunnen het niet beter doen. Vrij
ondernemerschap, oftewel de instellingen die zouden voortkomen uit een
proces van concurrentie om goed geld te bieden, zouden dat ongetwijfeld
wel kunnen', schreef hij. `De mogelijkheid tot omzetting {[}naar goud{]}
is een noodzakelijke waarborg die men aan een monopolist moet opleggen,
maar is onnodig bij concurrerende aanbieders die zich niet in de
concurrentiestrijd kunnen handhaven tenzij ze geld bieden dat minstens
zo voordelig is voor de gebruiker als elk ander'.\footnote{Friedrich A.
  Hayek, `The Argument Refined', 83.}

Hayek overwoog echter wel dat nieuwe munten van de vrije markt mogelijk
in eerste instantie door fiatgeld gedekt zouden moeten worden. Dit zou
vergelijkbaar zijn met hoe fiatgeld in eerste instantie door goud werd
gedekt tot mensen het nieuwe geld leerden vertrouwen. Dit kon een tijd
duren, gaf de econoom toe: `Het bijgeloof sterft maar
langzaam'.\footnote{Friedrich A. Hayek, `The Argument Refined', 109.}
Maar uiteindelijk zouden mensen gaan begrijpen dat inflatoir fiatgeld
continu waarde verliest tegenover het privégeld. Ze zouden door
economische prikkels tot het inzicht komen dat wat ze echt van geld
verwachten, schaarste is.

Echter, in een vrij bankensysteem zou er geen regulering zijn die
garandeert dat gedekte valuta's inderdaad uitwisselbaar zouden zijn voor
datgene waardoor ze worden gedekt. Hoewel specifieke afspraken onder
regulier contractrecht bindend kunnen zijn, zouden er geen bijzondere
wetten zijn over de dekkingsratio en zou er ook geen kredietverstrekker
in uiterste nood zijn. Banken zouden de kosten van hun eigen risico's
dragen, en dat geldt ook voor de klanten die ervoor kiezen om deze
banken te vertrouwen met hun geld. Maar dit kan misschien worden gezien
als een goede zaak: vrij bankieren zou het morele risico wegnemen.

En door vrije banken volledig onafhankelijk van het bestaande financiële
systeem te laten functioneren, geloofde Hayek dat het plan ook
daadwerkelijk haalbaar was.

`Niet het minste voordeel van het voorgestelde afschaffen van het
overheidsmonopolie op de uitgifte van geld, is dat het ons de kans zou
geven om ons te bevrijden uit de impasse waarin deze ontwikkeling {[}van
het bankwezen{]} ons heeft geleid', schreef Hayek. `Het zou voorwaarden
creëren waarbij de verantwoordelijkheid voor het beheren van de
hoeveelheid valuta wordt gelegd op agentschappen wiens eigenbelang hen
ertoe zou aanzetten het op zo'n manier te beheren dat het voor de meeste
gebruikers het meest acceptabel is'.\footnote{Friedrich A. Hayek, `The
  Argument Refined', 77.}

In een wereld waar banken en andere financiële instellingen in de loop
der tijd afhankelijk waren geworden van een sterk gereguleerd en nauw
verweven financieel systeem, en waar het buitengewoon moeilijk was om
zinvolle veranderingen aan te brengen, betekende Hayek's oplossing een
nieuwe start. Hij stelde voor om een marktgebaseerd monetair systeem uit
te rollen naast de gevestigde financiële sector, om zo een volledig
nieuw valutasysteem te laten ontstaan als mensen het vrijwillig zouden
adopteren.

Hayek zag geld als een spontane orde.

\section{Het realiseren van vrij
bankieren}\label{het-realiseren-van-vrij-bankieren}

Hoewel de meeste landen technisch gezien alternatieve valuta's toestaan
(er zijn doorgaans geen wetten die ze expliciet illegaal maken) wordt
dit vaak niet weerspiegeld in de praktijk. Bankregulaties, vergunningen
voor geldtransfers, antiwitwasregels, wetten tegen vervalsing evenals
belastingen (zoals vermogenswinstbelasting op alternatieve valuta's)
bieden wetshandhavingsinstanties meer dan voldoende middelen om
bedrijven die valuta uitgeven compleet te sluiten, en hun operators
veroordeeld te krijgen, of op zijn minst maken ze het opereren van
dergelijke bedrijven praktisch onmogelijk.

Hayek wist dat het volledig elimineren van al deze wettelijke en fiscale
obstakels -- ware deregulatie -- op enorme tegenstand zou stuiten.

Hij verwachtte dat het merendeel van deze weerstand afkomstig zou zijn
van overheden: precies dezelfde partijen die deze deregulering
uiteindelijk zouden moeten doorvoeren en juridisch de ruimte voor
valutaconcurrentie zouden moeten faciliteren. Hayek was van mening dat
overheden elke reden hadden om dit te voorkomen: zij waren de grootste
begunstigden van het fiatgeldsysteem en hadden in feite het monopolie op
geld. Hayek geloofde dat als overheidsvaluta zouden moeten concurreren
met vrije marktvaluta's, deze geen schijn van kans zouden hebben, en
overheden dus waarschijnlijk niet tevreden zijn met vrije banken.

Om het nog moeilijker te maken, zou het merendeel van de economen
waarschijnlijk ook tegen zijn, voorzag Hayek. Geld van de vrije markt
zou immers waarschijnlijk elke kans op het manipuleren van rentetarieven
om de koopkracht van de munteenheid te beïnvloeden en deflatie te
voorkomen, elimineren. De meeste economen van de jaren '70 waren van
mening dat overheidsinstellingen zoals de Federal Reserve een
belangrijke rol speelden bij het beheer van de geldvoorraad.

`Ik vrees dat 'Keynesiaanse' propaganda doorgedrongen is tot de massa,
{[}en{]} inflatie respectabel heeft gemaakt en onruststokers heeft
voorzien van argumenten die de professionele politici niet kunnen
weerleggen', schreef Hayek gefrustreerd.\footnote{Friedrich A. Hayek,
  `The Argument Refined', 133.}

Hayek, de Oostenrijkse econoom, voegde nog toe dat hij verwachtte dat de
meeste van de huidige banken ook tegen de verandering zouden zijn. De
`oude bankiers', zoals hij ze noemde, zouden niet in staat zijn om de
nieuwe uitdagingen aan te kunnen die een vrije bankensector op hen zou
zetten.

`Vooral in landen waar de concurrentie tussen banken al generaties lang
wordt beperkt door kartelafspraken, die meestal worden getolereerd en
zelfs aangemoedigd door overheden, zou de oudere generatie bankiers
waarschijnlijk totaal niet in staat zijn om zich voor te stellen hoe het
nieuwe systeem zou werken en daarom in de praktijk unaniem zijn in het
afwijzen ervan', schreef Hayek.\footnote{Friedrich A. Hayek, `The
  Argument Refined', 93.}

Met overheden, economen en bankiers als voorziene tegenstanders,
verwachtte Hayek zeker dat het realiseren van vrij bankieren een
uitdagende strijd zou worden, maar toch geloofde hij sterk dat het
desondanks gedaan moest worden.

Hayek schreef in de tweede, verfijnde versie van zijn boek:
`{[}Denationalisatie van geld is{]} de enige manier waarop we nog kunnen
hopen om de voortdurende vooruitgang van alle overheidssturing richting
totalitarisme te stoppen, wat voor veel scherpzinnige waarnemers
onvermijdelijk lijkt. Maar de tijd dringt. Wat nu dringend nodig is, is
niet de bouw van een nieuw systeem, maar het snel verwijderen van alle
wettelijke obstakels die al tweeduizend jaar de weg hebben geblokkeerd
naar een evolutie die ongetwijfeld gunstige resultaten zal opleveren die
we niet kunnen overzien'.\footnote{Friedrich A. Hayek, `The Argument
  Refined', 134.}

En hij geloofde dat het mogelijk was. De sleutel lag bij het winnen van
de steun van de algemene bevolking. Hayek maakte een vergelijking met de
vrijhandelsbewegingen uit de negentiende eeuw en stelde dat een nieuwe
burgerbeweging, een `vrij geld-beweging', mensen zou kunnen informeren
over de schade die inflatie en valutamanipulatie veroorzaken. Een
bredere publieke bewustwording van deze kwesties zou een solide basis
kunnen vormen voor de zaak. De daadwerkelijke politieke verandering (de
deregulering van de bankensector) werd verondersteld in een later
stadium te volgen.

Opgericht door onderzoekers George Selgin, Lawrence White, en Kevin Dowd
een paar jaar later, in de jaren '80, kwam de moderne vrije-bankenschool
waarschijnlijk het dichtst in de buurt van het opzetten van zo'n
beweging. De door Hayek geïnspireerde lobbygroep deed onderzoek naar de
geschiedenis en het potentieel van vrije banken en publiceerde hun
bevindingen in verschillende boeken en artikelen.

Maar hoewel de moderne vrije-bankenschool een kleine aanhang kreeg van
gelijkgestemde, meestal Oostenrijkse, economen, slaagde het er niet in
om de harten en de geesten van het grote publiek te winnen. Hoewel Hayek
nog lang genoeg leefde om een nieuwe generatie politici en economen zijn
werk over vrije markten en het prijssysteem te zien herontdekken en
revitaliseren, tot het hem zelfs de Nobelprijs voor economie opleverde
in 1976, bleef Hayek's oproep voor monetaire hervorming onbeantwoord.
fiatvaluta bleef oppermachtig tot, in 1992, Hayek op tweeënnegentig
jarige leeftijd stierf.

Toch waren zijn ideeën niet volledig vergeten.

Het werk van Hayek op het gebied van geld diende kort na zijn overlijden
als inspiratie voor een groep hackers en cryptografen uit Californië.
Maar deze hackers en cryptografen waren niet van plan om politici,
economen en bankiers te overtuigen om de wet te veranderen.

Ze gingen een toekomst bouwen zonder hen\ldots{}

\chapter{eCash (en vertrouwensloze
tijdstempels)}\label{ecash-en-vertrouwensloze-tijdstempels}

Sinds Whitfield Diffie en Martin Hellman de Diffie-Hellman
sleuteluitwisseling introduceerden in 1976, begrepen cryptografen hoe
twee mensen die elkaar nooit eerder ontmoet hadden de inhoud van hun
onderlinge berichten konden versleutelen voor iedereen behalve voor
zichzelf. Ondertussen hadden de mixnetwerken van David Chaum al in de
vroege jaren '80 de basis gelegd voor remailers: een digitale
infrastructuur ontworpen om metadata te verbergen. Gecombineerd konden
deze instrumenten een lange weg afleggen naar het bieden van privacy
voor de meeste soorten van elektronische communicatie.

Maar het was ook Chaum die erkende dat het nog geen privacy kon bieden
voor een zeer specifiek soort communicatie: communicatie van
\emph{waarde}.

Gedurende de jaren 70 en begin jaren 80 begon het bankwezen steeds meer
geautomatiseerd te raken. Papieren bankbiljetten en metalen munten, die
in die tijd niet langer door goud werden gedekt, begonnen steeds meer te
worden verdrongen door betaalkaarten, terwijl banken onderling
elektronisch schulden begonnen te vereffenen. Met de komst van de PC, en
in zijn kielzog het internet, verwachtte Chaum dat de digitalisering van
geld alleen maar zou versnellen. Dit zou financiële instellingen in
staat stellen om kosten te besparen en hun beveiliging te verbeteren,
terwijl het gemak voor de consument wordt verhoogd.

Maar Chaum realiseerde zich dat ook deze trend tot duistere situaties
kon leiden. Als betalingsverkeer digitaal zou worden, zouden de banken
die dit mogelijk maken door financiële regelgeving verplicht kunnen
worden om gebruikers zichzelf te laten identificeren voordat ze toegang
krijgen tot deze kanalen. Banken, en bij uitbreiding de
overheidsinstellingen die op hen toezicht houden, zouden dan precies
kunnen weten wie hoeveel geld naar wie stuurt, waar en wanneer.

Chaum zag de mogelijkheid van massasurveillance van betalingen als even
zorgwekkend als de massasurveillance van elke andere vorm van
communicatie. En niet zonder reden: iemands transactieverleden onthult
mogelijk net zoveel persoonlijke informatie als hun tekstcommunicatie,
zo niet meer.

`Er wordt een basis gelegd voor een dossiermaatschappij, waarin
computers gebruikt kunnen worden om op basis van gegevens uit alledaagse
consumententransacties de leefstijlen, gewoonten, locaties en relaties
van individuen af te leiden', waarschuwde Chaum. `Onzekerheid over of
gegevens veilig blijven tegen misbruik door degenen die ze onderhouden
of ervan gebruik maken kan een 'afschrikwekkend effect' hebben, wat
mensen ertoe kan brengen hun zichtbare activiteiten te veranderen.
Naarmate de computerisering meer verspreid raakt, zal het potentieel
voor deze problemen aanzienlijk toenemen'. \footnote{David Chaum,
  `Security Without Identification: Transaction Systems to Make Big
  Brother Obsolete', Communications of the ACM 28, no. 10: 1030--1044.}

Maar, zo stelde hij voor, er was een alternatieve toekomst mogelijk.

David Chaum legde uit: `Elke keer dat een regering of bedrijf besluit om
nog een set transacties te automatiseren, wordt de keuze gemaakt om
informatie in handen van individuen of van organisaties te houden. Aan
de ene kant ligt een ongekende controle en inspectie van mensenlevens,
aan de andere kant een veilige gelijkwaardigheid tussen individuen en
organisaties. De vorm van de samenleving in de volgende eeuw kan
afhangen van welke aanpak de overhand heeft'.\footnote{David Chaum,
  `Achieving Electronic Privacy', Scientific American 267, 2: 96--101.}

Volgens Chaum zou een maatschappij waarin anonieme transacties wel of
niet mogelijk zijn uiteindelijk het verschil maken tussen democratie en
dictatuur. Hij concludeerde daarom dat er een soort digitaal geld nodig
was dat gebruikers een vergelijkbaar niveau van privacy bood als fysiek
contant geld.

De wereld had behoefte aan \emph{elektronisch geld}.

\section{Het
dubbele-uitgavenprobleem}\label{het-dubbele-uitgavenprobleem}

Toen Chaum begon na te denken over ontwerpen voor een elektronisch
geldsysteem, kwam hij al snel de eerste uitdaging tegen die iedereen die
een digitale vorm van geld wil creëren, tegenkomt: \emph{het probleem
van dubbele uitgaven}.\footnote{David Chaum, `Blind Signatures for
  Untraceable Payments', Advances in Cryptology: Proceedings of Crypto
  82: 199--203.}

Eenvoudig gezegd, als valuta uitsluitend bestaat uit digitale
informatie, enen en nullen, is het een fluitje van een cent om te
kopiëren. Een enkele `digitale dollar' kan worden gerepliceerd en naar
twee verschillende ontvangers worden gestuurd\ldots{} of zelfs naar een
miljoen verschillende ontvangers, tot het punt dat het geld last heeft
van hyperinflatie. Het spreekt voor zich dat dit soort vervalsing de
integriteit van de valuta fundamenteel zou schaden en het waardeloos zou
maken.

Traditionele digitale geldsystemen lossen dit probleem van dubbele
uitgaven op door middel van een vertrouwde partij, zoals een bank of
betalingsverwerker.

De meest eenvoudige oplossing is dan ook het gebruik van een
rekeningensysteem. In zo'n systeem hebben alle klanten van de bank een
elektronische rekening bij de bank. Wanneer één van hen een andere wil
betalen, sturen ze simpelweg een bericht naar de bank met de
betalingsdetails. Ervan uitgaande dat de betaler voldoende saldo heeft,
trekt de bank het bedrag af van zijn of haar rekeningsaldo en voegt het
toe aan de rekening van de begunstigde.

Als de betaler niet genoeg geld heeft om de betaling uit te voeren,
wordt de transactie geweigerd. Dus als iemand probeert zijn saldo dubbel
te besteden door twee tegenstrijdige betalingsverzoeken naar de bank te
sturen (terwijl hij niet genoeg geld heeft om beide betalingen te doen),
zal de bank simpelweg kiezen welke transactie doorgaat (waarschijnlijk
het eerste verzoek dat het ontving).

Dit lost het probleem van dubbele uitgaven op\ldots{} maar het
illustreert ook het privacyprobleem waar Chaum zich zorgen over maakte:
de bank zou precies weten wie aan wie betaalt, hoeveel en wanneer.
Bovendien zou de bank totale controle hebben over ieders saldi, en zou
het mogelijk betalingen kunnen blokkeren of terugdraaien, en zelfs geld
kunnen confisceren of verwijderen.

Daarom begon Chaum aan een zoektocht naar een manier waarop zo'n derde
partij (de bank) dubbele uitgaven kon detecteren, zonder de mogelijkheid
om te traceren hoe elke `digitale dollar' zich door de economie beweegt.

\section{Blinde handtekeningen}\label{blinde-handtekeningen}

De bekwaamde cryptoloog loste het probleem op in 1983. Nauwelijks nadat
hij zijn doctoraat in informatica aan de Berkeley-universiteit had
verdiend, werd Chaum aangesteld als professor aan diezelfde
universiteit. Hij publiceerde zijn ontwerp voor een elektronisch
geldsysteem in zijn paper getiteld `Blinde handtekeningen voor
ontraceerbare betalingen'.

Zoals de titel van het artikel al aangeeft, was zijn uitvinding van
\emph{blinde handtekeningen} de sleutel tot zijn ontwerp voor een
privacy-respecterend betaalsysteem.

Chaum's blinde handtekeningen waren een uitbreiding van de publieke
sleutel-cryptografie en meer specifiek van het RSA cryptografische
handtekening algoritme. Om even te herhalen, een cryptografische
handtekening is in feite een stukje data (zoals een bericht) dat
versleuteld is met een \emph{privésleutel} en ontsleuteld kan worden met
een \emph{publieke sleutel}. Wanneer Alice een bericht en een
bijbehorende cryptografische handtekening naar Bob stuurt, moet Bob in
staat zijn het bericht te ontcijferen met Alice's publieke sleutel.
Hierdoor wordt het omgezet in hetzelfde bericht, waardoor wiskundig
wordt bewezen dat de handtekening inderdaad is gemaakt met haar
privésleutel.

Een blinde handtekening voegt dus één laag van encryptie toe aan de mix.

Om Alice een blinde handtekening te laten maken, zou Bob eerst een
speciaal soort encryptiesleutel genereren, de zogenaamde
\emph{verblindende sleutel}, waarmee hij een bericht zou coderen
(versleutelen). Bob zou het versleutelde bericht vervolgens aan Alice
geven, die het met haar privésleutel cryptografisch zou ondertekenen.
Wanneer ze het gecodeerde bericht ondertekent, weet ze niet wat het
originele bericht eigenlijk is: ze \emph{ondertekent blind}.

De resulterende handtekening is wiskundig verbonden aan Alice's publieke
sleutel, zoals elke handtekening zou zijn. Dat wil zeggen, haar publieke
sleutel kan worden gebruikt om het exacte versleutelde bericht dat zij
ondertekende te reproduceren (het bericht zelf zou nog steeds
versleuteld zijn; het zou hetzelfde versleutelde `blob' reproduceren dat
zij van Bob ontving.).

Maar Bob kan ook \emph{als eerste} de verblindende sleutel gebruiken om
de laag encryptie die hiermee gecreëerd is, te verwijderen. In principe
zou dit moeten resulteren in een nieuwe, geldige handtekening van Alice,
die dit keer overeenkomt met het originele bericht. Deze handtekening
wordt de blinde handtekening genoemd. Met de eerste laag encryptie
verwijderd door Bob, kan nu iedereen de publieke sleutel van Alice
gebruiken om het originele bericht te reproduceren vanuit de blinde
handtekening.

Met andere woorden, iedereen die het originele bericht heeft, kan op dat
moment Alice's publieke sleutel gebruiken om te verifiëren dat de blinde
handtekening overeenkomt met het bericht. Dit geldt uiteraard ook voor
Alice zelf. Als Bob haar het originele bericht en de blinde handtekening
geeft, kan zij haar eigen publieke sleutel gebruiken om te verifiëren
dat zij inderdaad een versleutelde versie van dat originele bericht
blind ondertekende.

Als een realistische analogie die Chaum in zijn paper gebruikte, zou het
zijn alsof Bob een brief in een envelop met carbonpapier stopt en deze
envelop aan Alice overhandigt, die de buitenkant van de envelop zou
ondertekenen en deze terug zou geven aan Bob. Als Bob dan de envelop
verwijdert en Alice de brief toont met een carbonkopie van haar
handtekening, zou ze weten dat de brief inderdaad in de envelop zat die
ze had ondertekend.

\section{Anonieme betalingen}\label{anonieme-betalingen}

Om het blinde-handtekeningenschema te gebruiken voor een elektronisch
geldsysteem, zou Alice uit het bovenstaande voorbeeld eigenlijk een bank
zijn: laten we deze bank Alice Bank noemen. Alice Bank is een reguliere
bank, waar klanten bankrekeningen hebben met dollardeposito's. En laten
we zeggen dat Alice Bank vier klanten heeft: Bob, Carol, Dan en Erin.

Nu wil Bob iets kopen van Carol met elektronisch geld.

Ten eerste zou Bob elektronisch geld nodig hebben. Hiervoor zou hij een
`opname' bij Alice Bank aanvragen (idealiter had hij deze opname al
gedaan voordat hij Carol wilde betalen, maar dat is een detail). Vreemd
genoeg, om de opname te doen, creëert Bob eigenlijk zelf de `digitale
dollars' in de vorm van unieke serienummers. Vervolgens versleutelt hij
deze digitale dollars (serienummers) met een verblindende sleutel en
stuurt ze naar Alice Bank.

Alice Bank zou dan elke versleutelde dollar blindelings ondertekenen, en
de handtekeningen terug sturen naar Bob. Voor elke versleutelde dollar
die Alice Bank, blind ondertekend, terug naar Bob stuurt, trekt Alice
Bank een reguliere dollar af van Bob's bankrekening.

Bob zou vervolgens een laag van de encryptie verwijderen met behulp van
zijn verblindende sleutel, zodat Alice's handtekeningen worden omgezet
in blinde handtekeningen. Om Carol te betalen, zou hij simpelweg de
digitale dollars en de bijbehorende blinde handtekeningen naar haar
sturen. Carol gebruikt dan de publieke sleutel van Alice Bank om deze
handtekeningen te verifiëren en als deze kloppen, stuurt ze onmiddellijk
de digitale dollars en blinde handtekeningen door naar Alice Bank.

Alice Bank zou deze digitale dollars nog nooit eerder gezien hebben,
aangezien ze de eerste keer dat ze deze ontving nog versleuteld waren.
Maar, belangrijk is dat Alice Bank in staat zou zijn te bevestigen dat
deze met haar eigen privésleutel waren ondertekend. Alice Bank zou
vervolgens de serienummers controleren met een lokale database om er
zeker van te zijn dat dezelfde digitale dollars niet al door iemand
anders waren gedeponeerd, om zo dubbele uitgaven te
voorkomen.\footnote{Chaum zou later ook een oplossing voorstellen
  waarbij dubbel uitgeven de anonimiteit van de dader zou kunnen
  opheffen, waardoor de noodzaak om elke binnenkomende betaling direct
  te controleren aan de hand van de bankgegevens enigszins wordt
  beperkt, aangezien de dader van een dubbele-uitgavenaanval kan worden
  geïdentificeerd.}

Als de digitale dollars een geldige handtekening bevatten en voorheen
ongebruikt zijn, passeren ze beide controles. Vervolgens maakt Alice
Bank aantekening van deze digitale dollars in haar lokale database om
dubbele uitgaven in de toekomst te voorkomen. Daarna voegt ze het
equivalent van normale dollars toe aan Carol's bankrekening en bevestigt
dit aan haar. Met deze bevestiging weet Carol dat ze een geldige
digitale dollar van Bob heeft ontvangen en geeft ze hem wat hij bij haar
gekocht heeft.\footnote{Als een handig extra detail bevat het systeem
  ook een soort fraudepreventiecontrole, zij het een die ten koste gaat
  van privacy als en wanneer gebruikers ervoor zouden kiezen deze te
  gebruiken. Als Carol ten onrechte zou beweren dat ze nooit betaald is,
  zou Bob kunnen kiezen om de nonce aan Alice Bank te onthullen. Hiermee
  kan hij bewijzen dat hij de digitale dollars heeft gemaakt die Carol
  heeft gestort en dat hij ze aan haar heeft betaald.}

Omdat Alice Bank de ondertekende bankbiljetten pas voor het eerst zou
hebben gezien toen Carol ze stortte, zou Alice Bank geen enkele manier
hebben om te weten dat ze oorspronkelijk van Bob kwamen. Ze hadden net
zo goed van Dan of Erin kunnen komen. Bij uitbreiding, nadat Alice Bank
de `digitale dollars' aan Bob had uitgegeven, kon ze hem niet hebben
gestopt ze uit te geven, aangezien ze geen idee zou hebben welke
digitale dollars ongeldig te verklaren.\footnote{Dat gezegd zijnde, zijn
  er nog enkele andere, mogelijk meer drastische maatregelen die Alice
  Bank had kunnen nemen. Naast het weigeren om digitale dollars aan Bob
  uit te geven, had ze ook alle elektronische contante betalingen kunnen
  blokkeren. Evenzo zou ze bepaalde gebruikers kunnen blokkeren van het
  accepteren van betalingen; zelfs als betalingen niet kunnen worden
  getraceerd, kunnen sommige gebruikers nog steeds worden uitgesloten
  van deelname aan het systeem.}

Inderdaad, Chaum had een vorm van elektronisch geld
ontworpen.\footnote{Er kan worden gesteld dat Chaum een tamelijk losse
  definitie van `contant geld' gebruikte, aangezien contant geld meestal
  meer onderscheidende eigenschappen heeft. Chaums vorm van digitaal
  contant geld bood bijvoorbeeld beperkte overdraagbaarheid van persoon
  tot persoon --- een kenmerk dat fysiek contant geld wel heeft, omdat
  het vrij kan worden doorgegeven. Niettemin heeft Chaum een vorm van
  digitaal geld uitgevonden die ten minste een vergelijkbaar niveau van
  privacy bood als fysiek contant geld, wat zijn hoofddoel was.}

In de jaren na de publicatie van zijn eerste werk over ontraceerbare
betalingen, breidde Chaum de mogelijkheden van elektronisch geld verder
uit in presentaties tijdens de Crypto-conferenties en in verschillende
andere papers. Deze vervolgartikelen werkten precies uit hoe een
elektronische geldregeling te implementeren, waarbij het beste
gedetailleerde voorbeeld hiervan zijn paper uit 1985 was met de
beschrijvende titel `Veiligheid zonder identificatie: transactiesystemen
die Big Brother overbodig maken'.\footnote{David Chaum, `Security
  Without Identification'.}

`De grootschalige geautomatiseerde transactiesystemen van de nabije
toekomst kunnen zo worden ontworpen dat de privacy en de veiligheid van
zowel individuen als organisaties beschermd blijven', zo verklaarde de
introductie van één zin triomfantelijk.

\section{DigiCash}\label{digicash}

Een paar jaar later, tegen 1989, had Chaum zijn intrek genomen in
Amsterdam. Tijdens een van zijn eerdere bezoeken aan Nederland hadden
lokale academici hem een baan aangeboden als hoofdcryptograaf bij het
Centrum voor Wiskunde en Informatica (CWI), welke hij dankbaar had
geaccepteerd. Het stelde hem in staat dicht bij zijn Nederlandse
vriendin te wonen.

Ongeveer in deze periode overwoog de regering van Nederland een nieuw
toltarievenproject. Het concept was dat auto's zouden betalen voor het
privilege om op bepaalde hoofdwegen te rijden door middel van een
smartcard aan hun voorruit, die gescand zou worden door snelle
kaartlezers op verschillende plekken langs de wegen. Maar het idee was
controversieel: de Nederlanders waren niet enthousiast over het idee dat
hun auto's gevolgd zouden worden.

Toen de overheid naar het CWI-onderzoekscentrum kwam om te vragen of ze
wisten van enige privacybeschermende oplossingen om dit soort tolsysteem
te realiseren, zag Chaum de kans waar hij op had gewacht. Hij had zijn
technologie van blinde handtekeningen gepatenteerd, maar tot zijn eigen
verbazing was de interesse in het ontwikkelen van digitale geldschema's
sinds de publicatie van zijn papers beperkt geweest. Hij zag nu in dat
er een unieke kans was om zelf te helpen de technologie in gebruik te
nemen.

Chaum wist een groep studenten van de nabijgelegen Technische
Universiteit Eindhoven te mobiliseren. Hij beloofde hen een reis naar de
International Collegiate Programming Contest (ICPC) in Washington DC, op
zijn kosten, en zelfs een vakantie naar Disney World in Florida, mits ze
hielpen bij het omzetten van zijn blinde handtekening-technologie naar
een werkbaar concept. Tijd was cruciaal, aangezien de Nederlandse
overheid al een ontwikkelteam in gedachten had voor het project en niet
echt zin had om het proces uit te stellen. Chaum en de studenten werkten
dag en nacht, vanuit een van hun woonkamers.

Ze slaagden binnen tien dagen en hun \emph{proof-of-concept} leverde
Chaum het contract op.

Met deze initiële klus op zak, besloot de cryptograaf om DigiCash op te
richten, een start-up gevestigd in Amsterdam die zich zou specialiseren
in digitaal geld en betalingssystemen. Deze betalingssystemen omvatten
uiteraard het overheidsproject voor tolheffing,\footnote{Uiteindelijk
  werd het tolproject niet aangenomen: het idee bleek te controversieel
  in Nederland. De technologie zou echter later worden gelicentieerd
  onder de naam `DyniCash' aan een onderneming in Dallas, Texas die
  gespecialiseerd was in communicatie op microgolf-frequenties voor
  treinen.} maar Chaum, die nu een eigen bedrijf leidde, wilde ook zijn
grotere visie realiseren.

Het was begin jaren 90 steeds duidelijker geworden dat het internet
mainstream zou worden, en Chaum was ervan overtuigd dat elektronische
betalingen uiteindelijk een essentieel onderdeel van deze opkomende
digitale wereld zouden zijn. Net als veel internetexperts in die tijd,
verwachtte hij dat micro-betalingen alomtegenwoordig zouden worden:
webdiensten moesten op de een of andere manier geld verdienen, en de
voor de hand liggende oplossing was inderdaad om mensen kleine bedragen
te laten betalen om ze te gebruiken.

`Naarmate betalingen op het netwerk volwassen worden, ga je voor
allerlei kleine dingen betalen, meer betalingen dan je vandaag de dag
doet', voorspelde Chaum. `Elk artikel dat je leest, elke vraag die je
hebt, je zult ervoor moeten betalen'.\footnote{Peter H. Lewis,
  `Attention Internet Shoppers: E-Cash Is Here', The New York Times, 19
  oktober 1994,
  \href{https://www.nytimes.com/1994/10/19/business/attention-internet-shoppers-e-cash-is-here.html}{online}}

Het prestigeproject van DigiCash was een digitaal betalingssysteem dat
mensen in staat zou stellen om dergelijke betalingen privé te maken, met
behulp van elektronisch geld --- \emph{eCash}.

\section{CyberBucks}\label{cyberbucks}

DigiCash begon snel internationaal onder de aandacht te komen. In een
tijd waarin bedrijven als Netscape en Yahoo! bevestigden dat het grote
geld zich verplaatste naar beginnende internetstart-ups, werd Chaum's
start-up door veel techondernemers in de vroege jaren 90 gezien als een
rijzende ster in deze snelgroeiende industrie.

Het zou Chaum en zijn team, dat enkele van de studenten omvatte waarmee
hij het project begon, meerdere jaren kosten om hun eerste
proof-of-concept om te zetten in een volwaardig betalingssysteem. Met
als uiteindelijk doel hun eCash-technologie te verkopen aan banken, was
een niveau van beveiliging nodig vereist door banken.

In de tussentijd hebben ze wel een vroege versie van hun technologie
uitgebracht. De eerste implementatie van Chaum's elektronische
geldontwerp door DigiCash werd uitgerold vanuit het eigen
bedrijfskantoor, maar in plaats van Amerikaanse dollars, Nederlandse
guldens of een andere fiatvaluta, gebruikte dit vroege elektronische
geldsysteem \emph{CyberBucks}.

CyberBucks was een `verzinsel', niet gedekt door echte waarde, denk aan
speelgeld, als je wilt. Maar het bedrijf beloofde wel nooit meer dan een
miljoen eenheden in omloop te brengen. De virtuele munten werden meestal
gratis weggegeven, en iedereen kon het digitale geld op zijn computer
opslaan, of het laden op een smartcard om een frisdrank of wat beltegoed
te kopen in het DigiCash-gebouw zelf. Deze smartcards waren in feite
onkraakbare, creditcard-achtige computers die speciaal waren ontworpen
om dit soort betalingen te doen, en ze werden een belangrijke focus voor
DigiCash: Chaum geloofde dat de smartcards essentieel waren voor de
privacy van betalingen, omdat betalingen in persoon met een creditcard
zelfs nog grotere privacyproblemen met zich meebrachten dan online
betalingen.

Het idee achter CyberBucks was dat DigiCash-medewerkers konden
experimenteren met de eCash-technologie, terwijl bezoekers konden
proeven van de toekomst. Toen Chaum en zijn collega's besloten om 100
CyberBucks te schenken aan elke handelaar die bereid was om de
internetmunteenheid als betaalmiddel te accepteren, begon een kleine
groep enthousiastelingen CyberBucks ook buiten de kantoren van DigiCash
te gebruiken. Hoewel het doorgaans alleen kon worden uitgegeven aan
gadgetachtige producten zoals digitale afbeeldingen of kleine
puzzelspellen voor Apple's Macintosh-computer, genoot de
bedrijfsmunteenheid toch enige bredere acceptatie.

Daarnaast zijn de CyberBucks uiteindelijk op een niet-officiële
CyberBucks-beurs verhandeld. Gebruikers konden de digitale geld-eenheden
omzetten in daadwerkelijke fiatvaluta en vice versa. Na enige tijd
kregen de CyberBucks een echte marktprijs, terwijl sommige gebruikers
zelfs een klein beetje van de digitale valuta begonnen te verzamelen als
een vorm van sparen of speculeren.

Dit vereiste wel veel vertrouwen in DigiCash. Hoewel het bedrijf
beloofde de CyberBucks-voorraad te beperken tot één miljoen, was er geen
ingebouwde methode om dit plafond af te dwingen. Theoretisch gezien
konden Chaum en zijn collega's veel meer dan een miljoen CyberBucks
uitgeven als ze dat wilden, en door de sterke privacyfuncties van het
systeem zou er voor buitenstaanders geen manier zijn om te controleren
of dit al dan niet was gebeurd.

Bovendien was het valutasysteem volledig afhankelijk van de voortdurende
steun van DigiCash: de bedrijfsserver voorkwam dubbele uitgaven. Hoewel
CyberBucks een leuk en interessant experiment was, was de uiteindelijke
topprioriteit van DigiCash om hun belangrijkste elektronische
geldproduct klaar te stomen voor het grote publiek\ldots{}

\section{eCash}\label{ecash}

Na vier jaar ontwikkeling bereikte eCash een standaard van beveiliging
gelijk aan die van banken. In 1994 begonnen de proeven met echt geld,
waarna financiële instellingen een licentie konden aanvragen bij de
start-up van Chaum om de nieuwe technologie te gebruiken.

De eerste bank die deze kans greep, was de Mark Twain Bank in St.~Louis,
niet bepaald een internationale grootmacht, maar het was een begin.
Klanten en bedrijven met een rekening bij Mark Twain konden genieten van
privacy in hun elektronische transacties door betalingen te doen en te
ontvangen in eCash.

Kort na de Mark Twain Bank volgden andere banken. Banken in
verschillende landen, waaronder de Norske Bank en Bank
Austria\footnote{DigiCash, `Bank Austria and Den norske Bank to Issue
  ecash: the Electronic Cash for the Internet', DigiCash, 14 april 1997,
  geraadpleegd
  \href{https://web.archive.org/web/19970605025912/http://www.digicash.com:80/publish/ec_pres8.html}{online}},
destijds de grootste banken in respectievelijk Noorwegen en Oostenrijk,
alsook de Australian Advance Bank, \footnote{DigiCash, `Advance Bank
  First to Provide DigiCash's ecash System in Australia', DigiCash,
  October, 1996, geraadpleegd
  \href{https://web.archive.org/web/19961102121407/https://www.digicash.com/publish/ec_pres6.html}{online}}
begonnen kort daarna met eCash-proefprojecten. En begin 1996 stapte ook
een van de grootste financiële instellingen ter wereld aan boord:
Deutsche Bank begon de technologie van DigiCash te gebruiken.
\footnote{DigiCash, `DigiCash's Ecash to be Issued by Deutsche Bank',
  DigiCash, 7 mei 1996, geraadpleegd
  \href{https://web.archive.org/web/19961102121355/https://www.digicash.com/publish/ec_pres5.html}{online}}
Credit Suisse, nog een grote internationale speler, sloot zich ook aan
bij de proeven. \footnote{Jeffrey Kutler, `Credit Suisse, Digicash in
  E-Commerce Test', American Banker, 16 juni 1998,
  \href{https://www.americanbanker.com/news/credit-suisse-digicash-in-e-commerce-test}{online}}

Echter, misschien nog opmerkelijker dan de samenwerkingsverbanden die
Chaum tot stand bracht, waren de zakendeals die hij niet wist af te
ronden. En hier begint het verhaal over wat er precies gebeurde bij
DigiCash te variëren, afhankelijk van wie je het vraagt.

Volgens diverse medewerkers van DigiCash hadden grote spelers in de
tech- en financiële industrie grote interesse getoond.\footnote{Next!
  Magazine `Hoe DigiCash alles verknalde', Next!, January 1999,
  geraadpleegd
  \href{https://web.archive.org/web/19990427142412/https://www.nextmagazine.nl/ecash.htm}{online}}
Twee van de drie meest prominente Nederlandse banken, ING en ABN Amro,
zouden naar verluidt aanbiedingen voor samenwerkingen gemaakt hebben aan
DigiCash ter waarde van tientallen miljoenen dollars. Ook betalingsreus
Visa zou Chaum een deal van in de miljoenen hebben voorgelegd. Er wordt
verder gezegd dat ook Netscape geïnteresseerd was: eCash had een
mogelijke toevoeging kunnen zijn aan de populairste webbrowser van die
tijd, maar deze samenwerking kwam niet tot stand.

Het grootste aanbod van allemaal zou echter zijn gekomen van niemand
minder dan Microsoft. Zo luidt het verhaal, Bill Gates wilde eCash
integreren in het Windows 95-besturingssysteem en bood DigiCash ongeveer
100 miljoen dollar om dit mogelijk te maken. In plaats daarvan zou Chaum
twee dollar hebben gevraagd voor elke verkochte versie van Windows 95.
Dit was te hoog gegrepen voor de Amerikaanse softwaregigant en daarmee
was de deal van de baan.

Medewerkers van DigiCash stonden in eerste instantie achter de aanpak
van Chaum, maar elke keer dat ze hoorden over het uiteenvallen van weer
een miljoenendeal, groeide hun twijfel over zijn zakelijk inzicht. Een
medewerker suggereerde later tegenover een Nederlandse reporter dat
dezelfde eigenschap van wantrouwen, die Chaum tot een uitstekend
cryptograaf maakte, hem in de weg stond als zakenman. Zijn `paranoïde'
karakter zou hem ongeschikt maken om zakenrelaties op te bouwen, wat
ertoe leidde dat hij op het laatste moment uit zakenovereenkomsten
stapte.

Bij de DigiCash kantoren nam de irritatie steeds verder toe. Naast het
besef dat hun eigen baan op het spel stond als het bedrijf niet snel
winstgevend zou worden, frustreerde het hen ook op ideologisch gebied
dat Chaum er niet in slaagde om eCash bij meer mensen te introduceren.
DigiCash wist ontwikkelaars aan te trekken die zich inzetten voor
digitale privacy, en in een tijd waarin e-commerce populair begon te
worden bij het grote publiek, waren ze bezorgd dat DigiCash niet mee
ging doen.

Chaum zelf verwerpt deze beweringen echter krachtig als kwaadwillige
laster. Hij beweert dat de verschillende aanbiedingen van meerdere
miljoenen dollars niet zo concreet waren als deze werknemers leken te
denken. In plaats van zijn persoonlijke tekortkomingen als zakenman,
stelt hij dat er simpelweg geen grote markt was voor digitaal geld, een
interpretatie die sommige van de meer commercieel ingestelde
DigiCash-medewerkers ook hebben bevestigd.

Het maakt niet uit welke versie van het verhaal de waarheid dichter
benadert, het is duidelijk dat eind 1996 het geduld in het DigiCash
kantoor op was. De medewerkers van Chaum eisten een verandering in het
bedrijfsbeleid.

\section{Faillissement}\label{faillissement}

Uiteindelijk kwam de verandering vanuit Chaum's thuisland.

Op zoek naar nieuwe fondsen richtte zijn bedrijf zich tot Amerikaanse
durfkapitalisten, aangezien de investeringscultuur in de VS immers beter
bekend was, en een grotere honger naar dit soort hightech-start-ups met
een hoog risico. DigiCash kreeg een financiële injectie, terwijl
MIT-hoogleraar Nicholas Negroponte tot voorzitter van de raad van
bestuur werd benoemd en Chaum als CEO vervangen werd door Michael Nash,
een veteraan van Visa.\footnote{American Banker, `Digicash Sends Signal
  by Hiring Visa Veteran', American Banker, 6 mei 1997,
  \href{https://www.americanbanker.com/news/digicash-sends-signal-by-hiring-visa-veteran}{online}}
Illustratief voor de nieuwe koers van het bedrijf verhuisde het
hoofdkantoor van Amsterdam naar Palo Alto, Californië, in het hart van
Silicon Valley, waar de waarderingen van technologie-start-ups door het
dak gingen. Chaum bleef wel deel uitmaken van DigiCash, maar dan nu als
CTO.

Dit was echter niet precies de soort verandering waarop de meeste
Nederlandse werknemers van DigiCash hadden gehoopt. Verschillende van
hen besloten op dit moment om het bedrijf te verlaten.

En misschien nog belangrijker, het bleek uiteindelijk niet veel verschil
te maken.

eCash sloeg gewoonweg niet aan bij het publiek. De banken die de
technologie uitprobeerden, promootten het niet echt bij hun klanten, en
het hielp waarschijnlijk ook niet dat eCash relatief duur in gebruik
was, waarbij het meestal enkele procenten van de transactiewaarde aan
kosten in rekening bracht. Mark Twain bank had in een paar jaar tijd
slechts 300 handelaren en 5.000 gebruikers ingeschreven, terwijl andere
banken het niet veel beter deden.

En, wellicht Chaum's lezing van DigiCash's geschiedenis kracht
bijzettend, was het nieuwe leiderschap van het bedrijf ook niet in staat
om grote deals te sluiten. Hoewel een samenwerking met CitiBank bijna
rond was, wat DigiCash wellicht het broodnodige momentum had kunnen
geven, liep het uiteindelijk spaak en de grote Amerikaanse financiële
instelling trok zich terug.

Tegen eind 1997 had DigiCash het merendeel van zijn fondsen opgebrand.
Na een laatste herschikking van de organisatie en leiderschap van het
bedrijf, vroeg de start-up van Chaum in 1998 het faillissement aan.

Na acht jaar van bedrijvigheid, had DigiCash het niet kunnen waarmaken
om aan de hype te voldoen die het had gegenereerd onder de eerste
generatie van internetondernemers. Misschien is de onkunde van Chaum om
zakelijke relaties op te bouwen wel de oorzaak voor de mislukking, zoals
sommige voormalige werknemers concludeerden. Of misschien was de vraag
naar anoniem digitaal geld, hoewel een verleidelijk verkooppunt in de
vroege jaren '90, gewoon niet zo hoog als de baanbrekende cryptograaf
aanvankelijk had verwacht. In plaats van micro-betalingen, werd een
groot deel van het web uiteindelijk gefinancierd door advertenties, en
privacy leek niet erg hoog op de prioriteitenlijst van de gemiddelde
consument te staan.

Bovendien worstelde DigiCash met een kip-en-ei-probleem. eCash was
alleen nuttig als mensen het ergens konden uitgeven: zonder plekken om
het digitale geld te besteden, was er geen reden om het in de eerste
plaats te bemachtigen. Tegelijkertijd was het voor handelaren alleen
zinvol om eCash te accepteren, als er genoeg mensen waren die het wilden
uitgeven.

`Het was moeilijk om voldoende handelaren te vinden die het wilden
accepteren, zodat je genoeg consumenten kon vinden die het wilden
gebruiken, of vice versa', herinnert Chaum zich in 1999. Ook zei hij:
`Naarmate het web groeide, nam de sofisticatie van de gebruikers af. Het
was moeilijk om het belang van privacy aan hen uit te
leggen'.\footnote{Julie Pitta, `Requiem for a Bright Idea', Forbes, 1
  november 1999,
  \href{https://www.forbes.com/forbes/1999/1101/6411390a.html}{online}}

De elektronisch geld-start-up van Chaum ging ten onder. De
CyberBucks-server voor dubbele uitgaven ging ook offline. Zonder deze
server was er geen manier om te weten welke valutaeenheden nog geldig
waren. Het betekende, simpel gezegd, het einde van het experiment.
Degenen die nog steeds CyberBucks bezaten, bleven achter met niets
anders dan een stel waardeloze nummers op hun computer.

Hieruit hebben alle betrokkenen bij het nicheproject voor digitaal geld
een waardevolle les getrokken. Hoewel blinde handtekeningen een zekere
mate van privacy garandeerden, bleek de afhankelijkheid van CyberBucks
van een vertrouwde partij in de vorm van DigiCash de fatale fout van het
project te zijn.

Rond deze tijd probeerde iemand anders, toevallig en om een heel andere
reden, een zeer vergelijkbaar soort fout te herstellen\ldots{}

\section{Scott Stornetta}\label{scott-stornetta}

Vers van de Stanford Universiteit met een PhD in natuurkunde, was Scott
Stornetta enthousiast om zijn nieuwe baan te beginnen bij wat in 1989
het epicentrum van computerwetenschappelijke innovatie was: het in New
Jersey-gevestigde telecom onderzoekscentrum Bellcore.

Bellcore had in feite de leiding over de architectuur van een groot deel
van de Amerikaanse telecommunicatiesystemen in een periode waarin de
informatietechnologie zich in een razend tempo ontwikkelde en het
internet iedere dag groter werd. Bovendien maakte de cryptografie een
ware renaissance door. Ze bevonden zich, zoals Stornetta later
omschreef, in een `gouden tijdperk van onderzoek'. Nieuwe medewerkers
kregen zelfs geen specifieke taken toegewezen. De dertigjarige
natuurkundige kreeg de instructie om \emph{zelf} te ontdekken wat van
belang was en vervolgens zijn aandacht hieraan te geven en hieraan te
werken.

Het bleek zo te zijn dat Stornetta al iets belangrijks in gedachten had
voordat hij überhaupt een voet in zijn nieuwe werkomgeving had gezet.

Voordat hij naar de oostkust verhuisde, bracht Stornetta enkele jaren
door op Stanford, waar hij werkte vanuit het Xerox
PARC-onderzoekscentrum in Palo Alto. De divisie van Xerox was een
revolutionaire omgeving die baanbrekende innovaties zoals de personal
computer, Ethernet en laserprinten mogelijk had gemaakt, maar de
afgelopen jaren werd Stornetta ook geconfronteerd met een nieuw en
lelijk probleem in het sterk gedigitaliseerde onderzoekscentrum:
vervalsingen.

Vervalsing is natuurlijk geen nieuw fenomeen. Mensen hebben in feite
geprobeerd om documenten te vervalsen sinds de uitvinding van het
schrift. Maar digitale vervalsingen waren een relatief nieuw concept, en
Stornetta was gaan geloven dat ze een nog uitdagender probleem
vertegenwoordigden. Terwijl fysieke vervalsing vaak sporen achterlaat,
kan een digitaal document, of het nu een arbeidscontract is,
verzekeringspapieren, of een universitair diploma, smetteloos worden
aangepast.

Digitale authenticatie loste inderdaad een deel van dat probleem op:
cryptografische handtekeningen konden bewijzen dat een elektronisch
document gecontroleerd (ondertekend) was door de juiste persoon. Maar
dit zou niet voorkomen dat dezelfde persoon later gewijzigde documenten
creëert en ondertekent. Je kan geen onderscheid maken tussen een oude
bit en een nieuwe bit, dus hoe kan iemand ooit zeker zijn dat ze kijken
naar een origineel document in plaats van een latere vervalsing?

Stornetta voorzag een crisis in geloofwaardigheid en besloot zijn eerste
periode bij Bellcore te besteden aan het oplossen van dit probleem.

Hij had ook al een mogelijke oplossing in gedachten. Stornetta wilde een
tijdregistratiesysteem voor digitale documenten ontwerpen. Het is veel
moeilijker om met een vervalsing weg te komen als mensen kunnen bewijzen
dat het originele document op een eerder moment in de tijd bestond.

Stornetta had nog niet precies uitgevogeld hoe zo'n
tijdregistratiesysteem zou werken, maar hij vermoedde dat cryptografie
wel eens een belangrijk deel van de oplossing kon zijn. Hoewel hij zelf
geen cryptograaf was, had hij het geluk dat de cryptograaf van Bellcore,
Stuart Haber, tevens degene die Stornetta bij Bellcore in dienst nam,
erin toestemde om met hem aan dit project te werken.

In de daaropvolgende weken brainstormden Stornetta en Haber over ideeën,
speculerend over mogelijke strategieën om de uitdaging waarvoor zij
stonden op te lossen.

\section{Hash-ketens}\label{hash-ketens}

Eén van hun meest belovende ideeën maakte gebruik van een
hash-functie,\footnote{Hash-functies werden voor het eerst voorgesteld
  door de wiskundige George B. Purdy van de University of Illinois at
  Urbana-Champaign in zijn artikel `A High Security Log-in Procedure',
  Communications of the ACM 17, no. 8: 442--445.} een soort
eenrichtingsfunctie die data omzet in een unieke en ogenschijnlijk
willekeurige reeks cijfers van een vaste lengte. Alle digitale data kan
worden gehasht, of het nu een enkele letter is, een heel boek, een
muziekbestand of de broncode van een programma. Het is van belang dat
dezelfde data altijd hetzelfde gehashte resultaat oplevert, maar als de
oorspronkelijke data ook maar een beetje wordt veranderd, zou de
resulterende hash onherkenbaar anders zijn. Als een enkele komma uit een
boek wordt verwijderd, zou de resulterende hash totaal niet meer lijken
op de hash van het originele boek.

Stornetta en Haber opperden dat documenten konden worden voorzien van
een tijdstempel door een speciale tijdstempeldienst. Een document zou
samen met een tijdcode gehashed worden die aangaf wanneer de
tijdstempeldienst het ontvangen had, en deze hash zou cryptografisch
ondertekend worden door de tijdstempelserver als een soort bewijs. Om
aan te tonen dat een document op een bepaald moment bestond, kon de
eigenaar van het document het originele document en de tijdstempel
verstrekken, en iedereen kon die vervolgens invoeren in een hashfunctie
en verifiëren dat er daadwerkelijk een identieke hash door de
tijdstempelserver was ondertekend.

Daarnaast speculeerden Stornetta en Haber dat verschillende documenten
chronologisch gekoppeld konden worden in een hash-keten. Dat wil zeggen,
elk nieuw document dat de tijdstempeldienst ontving, zou niet alleen
samen met een tijdscode gehasht worden, maar ook met de hash van het
vorige document. Deze nieuwe hash zou op zijn beurt weer samen met het
volgende document worden gehasht om de volgende hash te creëren,
enzovoort. De resulterende `keten' van hashes kon exact bewijzen welke
documenten in welke volgorde waren getijdstempeld, en zou op die manier
een chronologisch `ruggengraatrecord' vormen van alle verwerkte
documenten.

Dit betekende echter wel dat de tijdstempeldienst zelf vertrouwd zou
moeten worden om niet te rommelen met het centrale archief. Theoretisch
gezien, zou deze dienst vervalsingen kunnen maken door dezelfde
documenten te hashen en te ondertekenen met verschillende tijdstempels,
en het zou een nieuwe hash-keten kunnen creëren om de chronologische
volgorde van de documenten te veranderen, of zelfs documenten volledig
uit het archief kunnen verwijderen.

In de wereld van de informatica was het vrij gewoon om op dergelijke
diensten te vertrouwen: publieke sleutels werden bijvoorbeeld meestal
verstrekt door een certificaatautoriteit die deze sleutels koppelde aan
specifieke identiteiten. Maar dit was voor Stornetta en Haber geen
ideale oplossing. Zij waren van mening dat veiligheid in de digitale
ruimte niet afhankelijk zou moeten zijn van vertrouwen in een specifiek
individu of entiteit, net zoals de cryptografische hulpmiddelen die ze
tot hun beschikking hadden, zou tijdstempeling idealiter op zichzelf
moeten kunnen staan.

Dit bleek het meest uitdagende deel van het probleem te zijn.

Zolang er maar één entiteit dienst deed als een tijdstempelservice,
vereiste dat vertrouwen in die entiteit. Het oplossen van dit probleem
door meer entiteiten toe te voegen om een systeem van
veiligheidscontroles te creëren, leek ook een doodlopende weg. Zelfs als
iemand de taak zou krijgen om toezicht te houden op de eerlijkheid van
de tijdstempeldienst, zou iedereen nog steeds moeten vertrouwen dat deze
persoon geen verbond sluit met de tijdstempeldienst om de records te
wijzigen. Om dezelfde reden zou het toevoegen van een derde, vierde of
vijfde persoon om hen beiden eerlijk te houden, niet echt helpen. Dit
zou op zijn best de omvang van de samenzwering die nodig is om
historische records te vervalsen vergroten, maar het zou de mogelijkheid
daarvan niet volledig uitsluiten.

Stornetta en Haber leken tegen een fundamenteel probleem te zijn
aangelopen dat cryptografie niet kon oplossen. Na weken van vruchteloze
brainstormsessies zagen de collega's van Bellcore uiteindelijk geen
andere optie dan te concluderen dat wat ze echt wilden bereiken niet
mogelijk was.

Als een soort van troost, besloten ze hun bevindingen te publiceren. Ook
al hadden Stornetta en Haber het vertrouwensprobleem niet opgelost, ze
konden nu tenminste aantonen dat dit probleem onoplosbaar was\ldots{}

\section{Het verdelen van vertrouwen}\label{het-verdelen-van-vertrouwen}

Pas toen Stornetta beargumenteerde dat het probleem onoplosbaar was,
besefte hij dat ze daadwerkelijk fout zaten --- althans technisch
gezien.

Stornetta en Haber hadden geconcludeerd dat het toevoegen van meer
entiteiten om controles uit te voeren op de tijdstempeldienst het
probleem van vertrouwen niet oplost, maar alleen verandert hoe groot de
samenzwering zou moeten zijn. Inderdaad, dit lijkt logischerwijs waar te
zijn, en in de meeste gevallen is het inderdaad waar.

Maar Stornetta kwam nu tot het inzicht dat er een uitzondering op deze
regel is: als iedereen de tijdstempeldienst controleert, kan niemand
samenspannen: er zou niemand meer over zijn om tegen te samenzweren.
Zolang iedereen iedereen in de gaten houdt, is er helemaal geen
vertrouwde partij meer nodig!

Stornetta stelde: `Als we in essentie een samenzwering kunnen creëren
die zo groot is dat het de hele wereld omvat, dan zouden we in feite het
probleem hebben omgekeerd en een systeem zonder vertrouwen hebben
gecreëerd'.\footnote{Scott Stornetta, `The Missing Link between Satoshi
  \& Bitcoin: Cypherpunk Scott Stornetta', interview by Naomi Brockwell,
  NBTV, with Naomi Brockwell, YouTube, 6 september 2018,
  \href{https://www.youtube.com/watch?v=fYr-keVOQ18}{online}}

Het is uiteraard onwaarschijnlijk dat de groep die op vervalsingen
controleert zo groot wordt dat letterlijk iedereen op aarde erbij hoort.
Desalniettemin, deze nieuwe inzichten betekenden een echte doorbraak in
het denken van Stornetta en Haber.

Dit resulteerde uiteindelijk in de publicatie van hun onderzoek uit
1990: `Hoe plaatst men een tijdstempel op een digitaal
document?'.\footnote{Stuart Haber and Scott W. Stornetta, `How to
  Time-Stamp a Digital Document', Journal of Cryptology 3: 99--111.} De
paper stelde nieuwe normen in het domein van digitale tijdstempels en
presenteerde twee enigszins verschillende benaderingen.

Het eerste voorstel leek sterk op hun idee van een hash-keten
ruggengraatrecord, waarbij de tijdstempeldienst elk update
cryptografisch zou ondertekenen en chronologisch aan het record zou
linken. Dit zou bewijzen dat het daadwerkelijk de tijdstempeldienst was
die het nieuwe document toevoegde, en in welke volgorde. Maar belangrijk
was dat in plaats van enkel de tijdstempeldienst te vertrouwen met het
ruggengraatrecord, dit record nu met alle deelnemers gedeeld zou worden.

Het geniale van deze oplossing was dat als de tijdstempeldienst ooit zou
proberen om te antidateren, te deleten of een eerder getijdstempeld
document op een of andere manier te veranderen, elke gebruiker die een
kopie van de ruggegraatrecord behield, de verandering zou opmerken. Als
de inhoud of de tijdcodes van een document zelfs maar een klein beetje
gewijzigd werd, zou dit de bijbehorende hash volledig veranderen, wat op
zijn beurt weer elke volgende hash zou veranderen, waardoor het volledig
onverenigbaar zou worden met de wijdverbreide record: de
tijdstempeldienst zou nooit wegkomen met zijn vervalsingspoging.

De tweede oplossing die Stornetta en Haber beschreven, schafte zelfs de
tijdstempeldienst volledig af. In deze variant, zou een groep
deelnemende gebruikers om de beurt een nieuw document aan de hash-keten
toevoegen. Wanneer iemand een document wilde voorzien van een
tijdstempel, zou de willekeur van de hash van dit document worden
gebruikt om te bepalen welke deelnemer het moest ondertekenen, als een
soort hash-loterij.

Met niet één, maar twee briljante voorstellen, waarvan er één nog minder
vertrouwen vereiste dan de andere, vormde de paper van Stornetta en
Haber een grote sprong voorwaarts voor het digitaal tijdstempelen.

Dat gezegd hebbende, brachten hash-kettingen wel een nieuw probleem met
zich mee: ze waren niet bijzonder schaalbaar, vooral als je rekening
hield met de bescheiden rekenkracht die een gemiddelde computergebruiker
begin jaren '90 ter beschikking had. Voor elk document dat aan het
hoofdrecord werd toegevoegd, was er een nieuwe hash nodig, dus na
verloop van tijd zouden de deelnemende gebruikers heel wat data moeten
opslaan als deze systemen echt populair zouden worden.

En aangezien deze gebruikers steeds meer data moesten opslaan om deel te
kunnen nemen, zouden waarschijnlijk meer van hen ervoor kiezen om niet
meer deel te nemen en gewoon het record te vertrouwen dat door de
tijdstempeldienst en andere gebruikers wordt bijgehouden. Dit zou op
zijn beurt weer (de nood aan) vertrouwen in deze systemen introduceren:
om echt veilig te zijn, was het tijdstempelschema uitdrukkelijk
afhankelijk van brede deelname.

Het was wiskundige Dave Bayer die dit raadsel hielp oplossen, met behulp
van \emph{Merkle Trees}.

\section{Een boom van hashes}\label{een-boom-van-hashes}

In de jaren na het afronden van zijn stage bij Martin Hellman, had Ralph
Merkle een naam voor zichzelf gemaakt als een van de vooraanstaande
cryptografen van zijn generatie. Onder zijn vele innovaties had hij een
nieuwe eenrichtingsfunctie ontworpen, een sneller versleutelingsprotocol
geïntroduceerd en zijn eigen handtekeningsalgoritme voorgesteld. Hoewel
hij technisch gezien niet medeauteur was van het onderzoek, zagen veel
cryptografen Merkle als de derde uitvinder van de Diffie-Hellman
sleuteluitwisseling.

Het meest opvallende is misschien wel dat Merkle in 1979 de Merkle Tree
had uitgevonden.\footnote{Ralph C. Merkle, `A Certified Signature',
  Advances in Cryptology --- CRYPTO '89: Proceedings: 218--238.}
Oorspronkelijk ontworpen als onderdeel van een systeem voor het
produceren van authenticatiecertificaten voor een raadpleegbare lijst
van publieke sleutels, bieden Merkle-bomen een compacte en veilige
controle op de inhoud van allerlei soorten gegevenssets door hashes op
een slimme wiskundige manier te combineren.

Concreet genomen, aggregeert een Merkle-boom verschillende stukken data
cryptografisch via een aantal eenvoudige stappen. Allereerst worden de
verschillende stukken individueel gehasht, zodat elk stuk data zijn
eigen unieke hash heeft. Vervolgens worden al deze hashes in paren van
twee gegroepeerd. Elk paar hashes wordt dan samen gehasht, wat één
nieuwe hash per paar produceert. Al de nieuwe hashes worden dan opnieuw
gepaard, en deze paren worden weer samen gehasht. Dit proces herhaalt
zich totdat er nog maar één hash over blijft, de zogenaamde
\emph{Merkle-wortel} (gevisualiseerd, lijkt de resulterende
datastructuur op een soort stamboom, maar dan voor grote getallen in
plaats van personen).

Merkle-bomen vergemakkelijken controles om te zien of de hash van een
specifiek stuk data in de boom is opgenomen. Belangrijk is dat dit
mogelijk is zonder dat men de overige data die gecodeerd werd, noch
zelfs de meeste andere hashes, hoeft te zien. Alles wat nodig is, is een
\emph{Merkle-bewijs}, dat bestaat uit de relevante `takken' van de boom.
Dit dient in wezen als een compacte set van `aanwijzingen' om het pad te
vinden van de Merkle-wortel naar de hash van het specifieke stuk data.

Ondertussen is het strikt onmogelijk om iets in een boom te bewerken of
te verwijderen zonder de hele boom te veranderen of, nauwkeuriger,
zonder de wortel te veranderen. Als een stukje data wordt gewijzigd of
verwijderd, zou de overeenkomstige hash ook veranderen, wat op zijn
beurt noodzakelijkerwijs de weergave van zijn `kind'-hash beïnvloedt,
wat natuurlijk de volgende hash beïnvloedt, en zo verder, helemaal tot
aan de wortel van de boom. Niet ongelijk aan hash-ketens tonen
Merkle-bomen ondubbelzinnig aan of data is gewijzigd, maar dan in een
veel compacter formaat.

Het zou een waardevol middel blijken te zijn in de strijd tegen digitale
fraude.

\section{Een keten van wortels}\label{een-keten-van-wortels}

Goed bekend met de vele cryptografische voorstellen die in de anderhalf
decennium vóór de publicatie van Stornetta en Haber's eerste paper zijn
geïntroduceerd, stelde Bayer aan de onderzoekers van Bellcore voor dat
ze Merkle's hash-structuur konden gebruiken voor tijdstempels, wat het
duo met plezier accepteerde. Hun tweede paper, `Het verbeteren van de
efficiëntie en betrouwbaarheid van digitale tijdstempels', werd
gepubliceerd in 1993, en Bayer werd opgenomen als derde auteur.
\footnote{Dave Bayer, Stuart Haber and Scott W. Stornetta, `Improving
  the Efficiency and Reliability of Digital Time-Stamping', Conference
  Paper, Sequences II: Methods in Communication, Security, and Computer
  Science: 329--34.}

Stornetta, Haber en Bayer stelden voor om talrijke documenten
tegelijkertijd te tijdstempelen door ze allemaal te bundelen in één
grote Merkle-boom, waarvan er dagelijks één wordt aangemaakt. Gebruikers
zouden dan niet voor elk getijdstempeld document een hash hoeven bij te
houden. Integendeel, ze zouden alleen de dagelijkse Merkle-wortel als
basisregistratie nodig hebben en enkel hun eigen Merkle-bewijzen om de
hash van hun getijdstempelde documenten in de betreffende boom te
vinden.

Het resulteerde in een opmerkelijke verbetering van de efficiëntie,
waardoor meer gebruikers deel konden nemen aan het tijdstempelproces.
Het zou zelfs mogelijk zijn om de dagelijkse Merkle-wortel te publiceren
in een krant, waar het zeer openbaar zou zijn en bewaard zou worden in
fysieke krantenarchieven (Stornetta en Haber richtten later de
tijdstempel-start-up Surety op, die inderdaad Merkle-wortels opnam in de
geclassificeerde advertenties van de New York Times).

Daarnaast, en zoals uitgediept in Stornetta en Haber's derde paper
`Secure Names for Bit-Strings'\footnote{Stuart Haber and Scott W.
  Stornetta, `Secure Names for Bit-strings', CCS '97: Proceedings of the
  4th ACM Conference on Computer and Communications Security,: 28--35.},
kan elke nieuwe Merkle-boom ook de vorige Merkle-wortel bevatten. Zo
ontstaat een reeks van Merkle-wortels die zelf een chronologische
hash-keten vormen. Als er per dag één Merkle-boom wordt gecreëerd, dan
wordt de Merkle-wortel van gisteren opgenomen in de Merkle-boom van
vandaag, terwijl de Merkle-wortel van vandaag opgenomen zou worden in de
Merkle-boom van morgen, en zo verder.

Op deze manier zou ook de volgorde van de Merkle-bomen cryptografisch
met elkaar verbonden zijn. Dit zou vervalsing nog moeilijker maken:
zelfs als iemand erin zou slagen om bijvoorbeeld één editie van de
\emph{New York Times} in het fysieke archief te vervalsen om op de een
of andere manier een andere Merkle-wortel op te nemen, zou dit niet
overeenkomen met de Merkle-wortels die in alle kranten zijn gepubliceerd
sinds die tijd, en natuurlijk ook niet met de persoonlijke archieven van
mensen. Deze methode zou dus vervalsing van de gegevens bijna onmogelijk
maken.

Inderdaad, als iemand zou proberen een document te antidateren, zou het
niet alleen de Merkle-wortel van die specifieke dag veranderen, maar zou
het ook incompatibel zijn met elke tijdstempelregistratie die sindsdien
is gepubliceerd. In praktische zin zou vervalsing onmogelijk zijn. In
feite ontwierpen Scott Stornetta en Stuart Haber een vorm van
historische gegevensverificatie.

De sleutel tot hun succes, was de verdeling van vertrouwen.

\chapter{De Extropianen}\label{de-extropianen}

Friedrich Hayek wilde geld denationaliseren en David Chaum wilde het
anoniem maken. Hoewel zowel de econoom als de cryptograaf een
revolutionair idee hadden, hadden ze niet helemaal hetzelfde doel voor
ogen.

Ze inspireerden echter wel dezelfde man.

Max O'Connor is opgegroeid in het bescheiden Britse stadje Bristol
tijdens de jaren '60 en '70. Al op jonge leeftijd werd zijn fantasie
geprikkeld door echte gebeurtenissen, zoals de maanlanding die hij op
vijfjarige leeftijd op televisie gadesloeg, alsook door de fictieve
verhalen uit de stripboeken die hij verslond. Hij droomde van een
toekomst waarin de mensheid haar mogelijkheden op sciencefiction-achtige
wijze zou uitbreiden. Hij fantaseerde over een wereld waarin mensen over
röntgenvisie beschikten, desintegratiepistolen bij zich droegen en in
staat waren dwars door muren te lopen.

In zijn tienerjaren had O'Connor een interesse in het occulte
ontwikkeld. Hij dacht dat de sleutel om bovenmenselijk potentieel te
realiseren misschien te vinden was in hetzelfde veld als astrale
projectie, wichelroedelopen en reïncarnatie. Om deze mogelijkheden te
onderzoeken, richtte hij de club voor Psychische Ontwikkeling en
Onderzoek op bij zijn school, waar hij en zijn mede-junior-occultisten
het bovennatuurlijke bestudeerden.

Maar O'Connor, die rond deze tijd bijzonder geïnteresseerd raakte in
levensverlenging, vond niet precies wat hij zocht. Hij kwam tot het
besef dat er consequent overtuigend bewijs ontbrak dat enige van de
mystieke praktijken daadwerkelijk werkten.

De tiener veranderde uiteindelijk volledig van gedachten over het
occulte en kwam tot de conclusie dat er geen waarde te behalen was uit
deze overtuigingen en praktijken. In plaats van het bovennatuurlijke,
besloot hij dat de vooruitgang van de mensheid het beste gediend was
door wetenschap en logica.

Zelfs zonder bovennatuurlijke krachten kon O'Connor zijn eigen
potentieel ten minste maximaliseren door hard te werken. Op school was
hij een gretige leerling en ook ambitieus, ten minste zolang de
onderwerpen in de klas boeiend waren. Hij was vooral geïnteresseerd in
onderwerpen over sociale organisatie, en hij slaagde uiteindelijk als
beste van zijn economieklas op school.

Al dat harde werk wierp vruchten af toen O'Connor in 1984 werd
toegelaten tot de Universiteit van Oxford. Zijn drive om te presteren en
het beste uit zichzelf te halen, leek alleen maar toe te nemen op deze
prestigieuze universiteit. Hij studeerde gedurende drie aaneengesloten
jaren met grote inzet, waarbij hij cursussen volgde in politiek,
economie en filosofie. Op zijn drieëntwintigste had hij in alle drie de
disciplines een graad behaald.

Op dat moment was het tijd voor een verandering van omgeving. Als jong
volwassene, wilde de verse Oxford-afgestudeerde schrijver worden, maar
de oude universiteitsstad met haar vochtig klimaat, donkere winters en
traditionele Britse waarden, bood hem niet de energie of inspiratie die
hij zocht. Het was tijd om ergens anders heen te gaan, naar een nieuwe
plek\ldots{} een opwindende plek.

In 1987 vond O'Connor zijn nieuwe bestemming toen hij een beurs ontving
om een PhD-programma in filosofie te volgen aan de Universiteit van
Zuid-Californië. Hij verhuisde naar Los Angeles.

Bij aankomst in de Gouden Staat voelde O'Connor zich meteen thuis. Het
zonnige weer van Los Angeles was een duidelijke verbetering ten opzichte
van het sombere Oxford, en in scherp contrast met de conservatieve
mentaliteit die heerste in Groot-Brittannië, stimuleerde het culturele
klimaat aan de westkust van Amerika ambitie en het streven naar succes:
Californiërs vierden prestaties, ze hadden respect voor het nemen van
risico's, en ze prezen degenen die verandering teweegbrachten.

Hier zou O'Connor een nieuw leven beginnen, als een nieuwe man.

Om de nieuwe start te markeren, besloot hij zelfs zijn naam te
veranderen: vanaf dat moment zou Max O'Connor door het leven gaan als
`Max More'.

`Het leek echt de essentie van wat mijn doel is te vatten: altijd
verbeteren, nooit stilstaan', legde hij later uit. `Ik zou beter worden
in alles, slimmer, fitter en gezonder worden. Het zou een constante
herinnering zijn om vooruit te blijven gaan'.\footnote{Ed Regis, `Meet
  the Extropians', Wired, 1 oktober 1994,
  \href{https://www.wired.com/1994/10/extropians/}{online}}

\section{FM-2030}\label{fm-2030}

Het uitbreiden van menselijk potentieel en specifiek levensverlenging
waren nooit echt populaire onderwerpen in Engeland. Maar in Californië
ontdekte Max More dat hij niet de enige was met interesse in deze
thema's.

Een van Max More's collega's aan de Universiteit van Zuid-Californië
(USC) was een in België geboren Iraans-Amerikaanse auteur en leraar.
Deze persoon werd geboren als Fereidoun M. Esfandiary, maar stond beter
bekend onder de naam `FM-2030'. Gedurende de jaren '70 en '80 was hij
druk bezig met het populair maken van een radicaal futuristische visie
voor de mensheid.

Geïnspireerd door wereldwijde protestbewegingen in de jaren '60, waar
hij mensen uit alle hoeken van de wereld zag opstaan tegen
overheidsfraude en sociale onrechtvaardigheid, begon FM-2030 zich een
toekomst voor te stellen waarin de mensheid grenzen zou overstijgen om
een universele dialoog te vestigen, vrij van nationaliteit, politiek en
erfdeel. Om dit te verwezenlijken, begon hij hightech woningen en
levenswijzen te bevorderen, zodat mensen in een onderling verbonden
mondiale gemeenschap kunnen leven.

Bovendien, zo voorspelde FM-2030, zou de ontwikkeling van nieuwe
technologieën ingenieurs de instrumenten geven om de wereld dramatisch
ten goede te veranderen. Hij geloofde dat eventuele risico's verbonden
aan technologische innovatie gecompenseerd zouden worden door de
voordelen van vooruitgang: zonne- en atoomenergie zouden zorgen voor een
overvloed aan energie, mensen zouden Mars koloniseren, robotwerkers
zouden onze vrije tijd verhogen, en mensen zouden in staat zijn om
vanuit het comfort van hun eigen huizen hun brood te verdienen dankzij
de komst van telewerken.

Nog interessanter is dat FM-2030 voorspelde dat technologie binnenkort
het punt zou bereiken dat het mensen zelf drastisch zou kunnen
verbeteren. Gezondheidszorg zouden aanzienlijk verbeteren aangezien meer
ziekten genezen, en genetische fouten gecorrigeerd konden worden:
toekomstige farmaceutica zou het menselijk potentieel kunnen verhogen
door bijvoorbeeld de hersenactiviteit te verbeteren.

Uiteindelijk verwachtte hij dat de medische wetenschap zelfs in staat
zou zijn het ouder worden te `genezen', en dus dat eindige menselijke
levensduur geen probleem meer zou zijn. Volgens FM-2030 zou de mensheid
rond zijn honderdste verjaardag in het jaar 2030 de dood overwinnen. Het
getal in zijn naam verwijst naar dit idee (`FM' stond dan weer voor
verschillende benamingen zoals \emph{Future Man}, \emph{Future Marvel},
of \emph{Future Modular} --- en soms iets anders, afhankelijk van zijn
stemming of wie het hem vroeg).

Omdat traditionele beperkingen op het menselijk potentieel, zoals
eindige levensduur, zouden worden weggenomen terwijl bionische
lichaamsdelen en andere kunstmatige verbeteringen steeds meer nieuwe
mogelijkheden zouden ontgrendelen, voorspelde FM-2030 dat mensen
uiteindelijk het meest radicaal zouden transformeren en zichzelf zouden
omvormen tot synthetische, post-biologische organismen.

`Het is slechts een kwestie van tijd voordat we onze lichamen
herschikken tot iets totaal anders, iets dat beter is aangepast aan de
ruimte, iets dat levensvatbaar zal zijn in ons zonnestelsel en zelfs
daarbuiten'.\footnote{Douglas Martin, `Futurist Known as FM-2030 Is Dead
  at 69', The New York Times, 11 juli 2000,
  \href{https://www.nytimes.com/2000/07/11/us/futurist-known-as-fm-2030-is-dead-at-69.html}{online}}

\section{Transhumanisme}\label{transhumanisme}

FM-2030 was overtuigd dat vooruitgang op het gebied van technologie
uiteindelijk de menselijke conditie zou veranderen: de mensheid zou
zichzelf upgraden om een nieuwe, verbeterde versie van de soort te
creëren. Technologie zou een post-menselijke toestand tot stand brengen.

Voor velen klonken deze voorspellingen eerder fantastisch. Maar toen K.
Eric Drexler, een onderzoeksmedewerker bij het \emph{MIT Space Systems
Laboratory}, rond dezelfde tijd een techniek beschreef voor het
vervaardigen van machines op moleculair niveau, begon het fantastische
al wat minder onwaarschijnlijk te klinken. Drexler was van mening dat
nanotechnologie industrieën zoals computergebruik, ruimtevaart en
productie, evenals oorlogvoering, fundamenteel kon veranderen.

Inderdaad, Drexler geloofde dat nanotechnologie ook de gezondheidszorg
kon revolutioneren. Hij legde uit dat fysieke afwijkingen meestal
veroorzaakt worden door verkeerd geordende atomen en stelde een toekomst
voor waarin nanobots het menselijk lichaam binnen konden gaan om deze
schade met ongeëvenaarde precisie te herstellen, waardoor het lichaam in
wezen van binnenuit volledig kon genezen.

Als zodanig zou nanotechnologie in staat zijn om zo ongeveer elke ziekte
te genezen, en uiteindelijk het leven zelf te verlengen, speculeerde
Drexler.

`Veroudering is in wezen niet anders dan elke andere fysieke
aandoening', schreef Drexler in zijn boek `Motoren van Creatie' uit
1986, `het is geen magisch effect van kalenderdatums op een mysterieuze
levenskracht. Broze botten, gerimpelde huid, lage enzymactiviteiten,
trage wondgenezing, slecht geheugen, en de rest zijn allemaal het gevolg
van beschadigde moleculaire machines, chemische onevenwichtigheden en
verkeerd geordende structuren. Door alle cellen en weefsels van het
lichaam weer een jeugdige structuur te geven, zullen reparatiemachines
de jeugdige gezondheid herstellen'.\footnote{K. Eric Drexler, `Engines
  of Creation', 146.}

Dit waren precies het soort ideeën die Max More als geen ander wisten te
boeien.

Bovendien waren deze ideeën voor More niet enkel grappige speculaties.
Hij was ervan overtuigd dat de soort voorspellingen die FM-2030 en
Drexler deden, beschouwd verdienden te worden als iets fundamentelers.
Hij was ervan overtuigd dat ze een nieuw perspectief boden op het
menselijk bestaan, en zelfs op de realiteit zelf. Terwijl More de
concepten van de futuristen verzamelde, bestudeerde en overdacht,
formuleerde de PhD-kandidaat ze uiteindelijk tot een nieuw en
onderscheidend filosofisch kader: \emph{transhumanisme}.

Het algemene idee en de term `transhumanisme' werden al in de jaren '50
gebruikt door evolutionair bioloog Julian Huxley, maar het was More die
het nu echt vestigde als een bijgewerkte versie van de humanistische
filosofie. Net als het humanisme, respecteert het transhumanisme rede en
wetenschap, terwijl het geloof, aanbidding en bovennatuurlijke concepten
als een hiernamaals verwerpt. Maar waar humanisten waarde en betekenis
halen uit de menselijke natuur en het bestaande menselijke potentieel,
zouden transhumanisten vooruit kijken en pleiten voor het overstijgen
van de natuurlijke beperkingen van de mensheid.

`Transhumanisme', vatte More kort samen, `verschilt van humanisme
doordat het de radicale veranderingen erkent en voorziet in de aard en
mogelijkheden van ons leven als gevolg van diverse wetenschappen en
technologieën zoals neurowetenschap en neurofarmacologie,
levensverlenging, nanotechnologie, kunstmatige ultra-intelligentie, en
ruimtebewoning, gecombineerd met een rationele filosofie en
waardesysteem'.\footnote{Max More, `Transhumanism: Towards a Futurist
  Philosophy', maxmore.com, geraadpleegd
  \href{https://web.archive.org/web/20051029125153/http://www.maxmore.com/transhum.htm}{online}}

\section{Extropianisme}\label{extropianisme}

Alle transhumanisten willen het menselijk potentieel verbeteren. Echter,
ondanks dat ze hetzelfde doel hebben, realiseerde Max More zich dat
verschillende transhumanisten zeer uiteenlopende benaderingen kunnen
voorstaan om dit doel te bereiken.

More geloofde zelf in een positieve, levendige en dynamische benadering
van transhumanisme. Hij gaf de voorkeur aan een boodschap van hoop,
optimisme en vooruitgang. Maar hij geloofde niet dat deze vooruitgang
afgedwongen of zelfs gepland kon worden. Hij verwierp de Star
Trek-achtige versie van de toekomst waarin de mensheid onder een enkele,
alwetende, wereldregering valt die de soort vooruit moet leiden.

More was van mening dat transhumanisten konden profiteren van de
inzichten van Friedrich Hayek.

Technologische innovatie vergt kennis en middelen, en Hayek had
uitgelegd dat die kennis van nature verspreid is over de hele
samenleving, terwijl middelen het best over de economie worden verdeeld
door vrije marktprocessen. Als mensen gewoon de vrijheid krijgen om te
experimenteren, te innoveren en samen te werken op hun eigen
voorwaarden, bedacht More, dan zou technologische vooruitgang vanzelf
ontstaan.

Met andere woorden, een welvarender \emph{morgen} werd het best
gerealiseerd als de samenleving zichzelf \emph{vandaag} als een spontane
orde kon organiseren.

More vond een vroege bondgenoot in mede USC-student Tom W. Bell. Net als
More, omarmde Bell de transhumanistische filosofie en gaf hij de
voorkeur aan More's vreugdevolle en vrije benadering om het te bereiken.
Hij besloot dat hij zou bijdragen aan de verspreiding van deze relatief
nieuwe ideeën door erover te schrijven onder zijn nieuwe pseudoniem: Tom
Morrow.

En om hun visie te concretiseren, introduceerde Morrow de term
\emph{extropie}. Als tegenhanger van `entropie', het proces van afbreuk
en van verval, stond extropie voor verbetering en groei, zelfs oneindige
groei. Degenen die, zoals Max More en Tom Morrow, deze
transhumanistische visie onderschreven, zouden als Extropianen worden
beschouwd.

Vervolgens schreef More de fundamentele beginselen van de Extropiaanse
beweging uit in enkele pagina's tekst, getiteld `De Extropiaanse
Principes: Een Transhumanistische Verklaring'. Het bevatte vijf
hoofdrichtlijnen of, inderdaad, principes: \emph{grenzeloze uitbreiding,
zelftransformatie, dynamisch optimisme, intelligente technologie}, en
--- als een expliciete knipoog naar Hayek --- \emph{spontane orde}. In
het kort, de principes (in het engels) vormden het acroniem B.E.S.T.
D.O. I.T. S.O.

`Voortdurende verbeteringen betekenen dat we de natuurlijke en
traditionele beperkingen van menselijke mogelijkheden uitdagen', vatte
More de doelen van de beweging samen in De Extropiaanse Principes.
`Wetenschap en technologie zijn essentieel om beperkingen op levensduur,
intelligentie, persoonlijke vitaliteit en vrijheid uit de weg te ruimen.
Het is absurd om slaafs de 'natuurlijke' grenzen aan onze levensduur te
accepteren. Het leven zal waarschijnlijk verder gaan dan de grenzen van
de aarde - de geboorteplaats van biologische intelligentie - om het
heelal te bewonen'.\footnote{Max More, `The Extropian Principles: A
  Transhumanist Declaration', maxmore.com, geraadpleegd
  \href{https://web.archive.org/web/20090130143449/https://www.maxmore.com/extprn3.htm}{online}}

Net als de transhumanistische visie die eraan ten grondslag lag, was de
Extropiaanse toekomst ambitieus en spectaculair. Naast levensverlenging,
wat mogelijk de centrale pijler van de beweging vertegenwoordigde,
omvatten de vooruitzichten van Extropianen een breed scala aan
futuristische technologieën. Deze variëerden van kunstmatige
intelligentie tot ruimtekolonisatie en \emph{mind uploading}, tot
menselijk klonen, fusie-energie, en nog veel meer.

Het is belangrijk om te overwegen dat Extropianisme echter geworteld
moest blijven in wetenschap en technologie, zelfs als het vaak over zeer
speculatieve versies daarvan ging. In plaats van weg te drijven naar het
domein van sciencefiction, moesten Extropianen nadenken over hoe ze een
betere toekomst kunnen realiseren door middel van kritisch en creatief
denken, proactief zijn, en continu leren.

Deze oproep ging over `rationeel individualisme' of `cognitieve
onafhankelijkheid', schreef More. Extropianen moesten leven volgens hun
`eigen oordeel, weloverwogen en geïnformeerde keuzes maken, profiteren
van zowel succes als tekortkoming', legde hij uit. Dat vereiste op zijn
beurt vrije en open samenlevingen waarin diverse informatiestromen en
verschillende zienswijzen de kans krijgen om te bloeien.

Anders gezien, geloven Extropianen dat staten en hun regeringen
voornamelijk een obstakel voor vooruitgang vormen. Belastingen beroven
mensen van middelen om te produceren en te bouwen, grenzen en andere
reisbeperkingen kunnen verhinderen dat mensen zich op de plekken kunnen
bevinden waar ze het meeste waarde toevoegen aan de globale samenleving,
en overheidsregulaties beperken alleen maar het menselijke vermogen om
te experimenteren en innoveren.\footnote{De Amerikaanse Food and Drug
  Administration was een bijzonder restrictief voorbeeld: de federale
  instelling maakte het knutselen aan en uitproberen van nieuwe soorten
  medicijnen en geneeskunde zo goed als onmogelijk.}

`Centraal bevel over gedrag belemmert ontdekking, diversiteit en
afwijkende meningen', concludeerde More.\footnote{More, `Extropian
  Principles'.}

\section[De subcultuur]{\texorpdfstring{De
subcultuur\footnote{Een groot deel van deze sectie is gebaseerd op Finn
  Brunton's `Digital Cash: The Unknown History of the Anarchists,
  Utopians, and Technologists Who Built Cryptocurrency', 118--134.}}{De subcultuur}}\label{de-subcultuur131}

In de herfst van 1988 publiceerden Max More en Tom Morrow de eerste
editie van een nieuw tijdschrift genaamd \emph{Extropy}. Dit markeerde
de officiële start van de Extropiaanse beweging.

Hoewel More en Morrow slechts vijftig exemplaren van deze eerste editie
hadden gedrukt, wist het tijdschrift al snel wetenschappers, ingenieurs,
onderzoekers en andere toekomstgerichte Californiërs uit diverse
vakgebieden met elkaar te verbinden. Onder de abonnees waren
computertechnici, raket-ingenieurs, neurochirurgen, chemici en nog veel
meer. Bovendien bevonden zich onder hen ook enkele opvallende namen,
zoals de baanbrekende cryptograaf Ralph Merkle of de Nobelprijs winnende
theoretisch fysicus Richard Feynman.

Wat ze gemeen hadden, was een wetenschappelijk geïnspireerd optimisme
over de toekomst. Met \emph{Extropy} hadden ze eindelijk een tijdschrift
om kennis op te doen over de meest radicale futuristische ideeën die er
waren, of ze konden hun eigen radicale ideeën delen als gastauteurs.

Bovendien bood het Extropianisme hen een uniek inzicht in het leven
zelf.

Gewapend met een superieure gereedschapskist om het voorheen
onverklaarde te verklaren, had de wetenschap in de afgelopen paar eeuwen
een groot deel van de greep van religie op de samenleving weggenomen.
Maar volgens More zou wetenschap alleen niet voldoende zijn om de
religie volledig uit te roeien, omdat het nog een andere belangrijke
functie vervult: het biedt zingeving. Mensen houden, ondanks al het
bewijs, vast aan religie, betoogde More, voornamelijk omdat ze
vertrouwen op hun geloof om door moeilijke tijden heen te komen en
zichzelf een gevoel van doel te geven. En hoewel hij ontdekte dat de
meeste religies feitelijk niet zo'n groot gevoel van doel bieden (ze
hebben de neiging om mensen onder een aantal krachtige wezens te
plaatsen of dit leven te bagatelliseren ten gunste van een hiernamaals)
vertegenwoordigde dit toch een vorm van doel of zingeving.

Om religie volledig weg te werken, moesten mensen een alternatieve bron
van zingeving krijgen, geloofde More:

`De extropiaanse filosofie kijkt niet buiten ons naar een superieure
buitenaardse kracht voor inspiratie. In plaats daarvan kijkt het in ons
en voorbij ons, en projecteert het vooruit naar een schitterende visie
van onze toekomst. Ons doel is niet God, het is de voortzetting van het
proces van verbetering en transformatie van onszelf naar steeds hogere
vormen. We zullen onze huidige interesses, lichamen, geesten en vormen
van sociale organisatie ontgroeien. Dit proces van expansie en
transcendentie is de bron van betekenisvolheid'.\footnote{Max More,
  `Transhumanism'.}

Dit Extropiaanse perspectief op het leven zou zich in de komende jaren
ontwikkelen tot een kleine, lokale subcultuur in Californië, met unieke
gewoonten en rituelen. De Extropianen hadden hun eigen logo (vijf pijlen
die vanuit het midden spiraalsgewijs naar buiten waren gericht, wat
groei in elke richting suggereert) en ze kwamen samen in een officieus
clubhuis (of `nerd house') dat Nextropia heette. Ze ontwikkelden hun
eigen handruk (het omhoog schieten van hun handen met verstrengelde
vingers, net zo lang tot hun armen helemaal uitgestrekt waren: \emph{The
sky's the limit}!), ze organiseerden evenementen (waar sommigen van hen
Extropiaanse kostuums droegen, bijvoorbeeld door zich als
ruimtekolonisten te verkleden), en, onder leiding van Max More en Tom
Morrow, veranderden diverse Extropianen hun namen: er was een
MP-Infinity, een Skye D'Aureous en iemand die zichzelf R.U. Sirius
noemde.

Toen de Extropiaanse gemeenschap uitgroeide van enkele tientallen tot
honderden mensen, richtten More en Morrow in 1990 ook het Extropie
Instituut op, waarbij FM-2030 zich aansloot als derde oprichtend lid. De
non-profit onderwijsorganisatie zou een tweemaandelijkse nieuwsbrief
produceren, Extropiaanse conferenties organiseren en (wat voor die tijd
vooruitstrevend was) een e-maillijst hosten om online discussie te
faciliteren. Hoewel e-mail in die tijd nog een niche technologie was,
wisten de technisch onderlegde en toekomstgerichte Extropianen over het
algemeen goed hoe ze het ontluikende internet moesten benutten.

Sommigen van hen werkten zelfs aan een bijzonder ambitieus
internetnavigatieproject\ldots{}

\section{\texorpdfstring{De \emph{high-tech}
Hayekianen}{De high-tech Hayekianen}}\label{de-high-tech-hayekianen}

K. Eric Drexler, wiens werk een belangrijke inspiratiebron was voor Max
More, werd niet lang na oprichting lid van de Extropiaanse gemeenschap,
net als verschillende van zijn vrienden. Deze vrienden waren
technologie-enthousiastelingen die graag aan enkele van de meest
innovatieve en uitdagende projecten van die tijd werkten.

Een van hen was Mark S. Miller, die destijds de hoofdarchitect was van
Xanadu, het allereerste hypertext project ter wereld (Hypertext is de
tekst waarop je kunt klikken om je naar verschillende delen van het
internet te brengen). Het ambitieuze Xanadu-project, dat al in 1960 werd
opgericht, was dertig jaar later nog steeds in ontwikkeling.

In het kader van de projectontwikkeling publiceerden Drexler en Miller
gedurende de jaren `80 diverse artikelen over het toewijzen van
rekenkracht over computernetwerken. Kort gezegd, stelden ze voor dat
computers in wezen hun overbodige CPU-cycli konden 'verhuren' aan de
hoogte bieder. Zelfzuchtige computers zouden hun middelen over het
netwerk verdelen via virtuele markten om de efficiëntie te
maximaliseren, en dat alles zonder de noodzaak van een centrale
beheerder. Dit zou het gebruik van rekenkracht waar het het meest
gewaardeerd werd mogelijk maken, terwijl het investeringen in meer
hardware zou aanmoedigen als de vraag hiertoe voldoende was.

Inderdaad, Drexler en Miller gebruikten Hayeks inzichten over de vrije
markt om computernetwerken te ontwerpen.

Drexler en Miller hadden het werk van Hayek bestudeerd op advies van een
andere bijdrager aan Xanadu en wederzijdse vriend, Phil Salin. Salin,
een futurist met economiediploma's van UCLA en Stanford, vond het leuk
om inzichten van de vrije markt te vermengen met de laatste
technologische ontwikkelingen. Het meest opmerkelijk was dat hij midden
jaren '80 concludeerde dat de tijd rijp was voor het oprichten van een
privé ruimtevaartindustrie, en lanceerde één van de meest ambitieuze
start-ups van dat decennium in de vorm van het privé ruimtevaartbedrijf
Starstruck.

De drie mannen, Drexler, Miller en Salin, werden door het economisch
tijdschrift \emph{Market Process} uitgeroepen tot de `high-tech
Hayekianen', een bijnaam die het trio met trots aanvaardde.\footnote{Don
  Lavoie, Howard Baetjer, and William Tulloh, `High-Tech Hayekians: Some
  Possible Research Topics in the Economics of Computation', Market
  Process 8: 119--146.}

\section{AMIX}\label{amix}

Ondanks de succesvolle lancering van een raket in de ruimte, eindigde
Starstruck als een commerciële mislukking: Salin ontdekte dat de
Amerikaanse overheid het praktisch onmogelijk maakte om een
ruimtevaartbedrijf te runnen omdat de door de belastingbetaler
gesubsidieerde Space Shuttle de markt continu ondermijnde.

Gelukkig was dit niet Salins enige project. Naast het adviseren van
Drexler en Miller, publiceerde hij ook artikelen en essays over de
economische effecten van de computerrevolutie die hij persoonlijk
ervaarde,\footnote{E.g., Phil Salin, `Costs and Computers', Release 1.0:
  5--18; Phil Salin, `The Ecology of Decisions, or 'An Inquiry into the
  Nature and Causes of the Wealth of Kitchens'\,', Market Process 8:
  91--114.} en deze dienden als basis voor nog een ambitieus streven:
Salin zou een online marktplaats voor het kopen en verkopen van
informatie creëren. Hoewel misschien niet zo spectaculair als het
lanceren van raketten in een baan rond de aarde, geloofde hij dat dit
project de wereld op nog grotere schaal kon veranderen.

De \emph{American Information Exchange}, kortweg AMIX, was een
marktplaats waarbij in principe elk soort informatie verkocht kon worden
waar mensen voor wilden betalen. Dat kon advies zijn van een monteur
over het weer aan de praat krijgen van een oude auto, computercode voor
het automatiseren van de boekhouding van een tandartspraktijk, of
misschien wel een ontwerptekening voor een nieuw vakantiehuis in de
Florida Keys. Als het informatie was, kon het op AMIX worden verkocht.

Salin was van mening dat het grootste voordeel van AMIX zou bestaan uit
een aanzienlijke vermindering van de transactiekosten in de ruimste
betekenis van het woord. Dat wil zeggen, alle kosten die verbonden zijn
aan het maken van een aankoop, inclusief de zogenaamde
opportuniteitskosten (de `kosten' van het mislopen van andere dingen).
Zo kan een transactiekost bijvoorbeeld de opportuniteitskost zijn van
het doen van marktonderzoek om uit te vinden welke verzekeraar de beste
deal biedt, of de kosten van het bellen naar verschillende slijterijen
om te weten te komen welke van hen een specifiek merk wijn verkoopt.

Op AMIX konden mensen ervoor kiezen om iemand anders te betalen om de
beste verzekeringsmogelijkheid voor hen te vinden, of informatie aan te
schaffen over slijterijen en hun voorraden. Als iemand op de
informatiehandelsmarkt deze diensten aanbood voor minder geld dan het de
potentiële kopers effectief zou kosten om de informatie zelf te vinden,
zou de handel via AMIX de transactiekosten van de daadwerkelijke
aankopen verminderen. AMIX kon het kopen van verzekeringen, wijn, en
vele andere goederen en diensten goedkoper maken door de
transactiekosten te verminderen.

Salin was van mening dat de maatschappij enorm baat zou hebben van zo'n
efficiëntieverbetering, omdat lagere transactiekosten bepaalde ruilen
mogelijk zouden maken die anders niet rendabel zouden zijn geweest. Als
iemand bijvoorbeeld niet de tijd heeft om een dozijn slijterijen te
bellen, zou de mogelijkheid om in plaats daarvan iemand anders een
kleine vergoeding te betalen om dit voor hen te doen, kunnen resulteren
in de verkoop van nog een fles wijn. Dit zou de wijnkenner, de slijterij
en ook de AMIX-onderzoeker beter af maken.

Kortom, meer handel betekent een betere verdeling van hulpbronnen over
de economie --- spontane orde.

\section{Cryogenica}\label{cryogenica}

AMIX was een vooruitstrevend concept. Maar het was ook zijn tijd ver
vooruit. Toen AMIX live ging in 1984, hadden Salin en zijn kleine team
de marktplaats vanuit het niets opgebouwd. Het reputatiesysteem dat ze
ontwikkelden was het eerste in zijn soort, net als hun tool voor de
oplossing van geschillen, en aangezien er nog geen online
betalingsverwerkers operationeel waren, moesten ze dit ook zelf
implementeren. Zelfs websites bestonden op dat moment nog niet, wat
betekende dat AMIX-gebruikers hun eigen netwerk moesten opzetten, waarop
ze moesten inbellen via \emph{dial-up}-modems, omdat breedbandinternet
nog niet bestond. Het is dan ook niet verwonderlijk dat het project
traag op gang kwam.

Helaas kreeg Salin de kans niet om AMIX verder te ontwikkelen: kort na
de lancering van het project werd bij hem maagkanker vastgesteld. Salin
verkocht AMIX uiteindelijk in 1988 aan het softwarebedrijf Autodesk, dat
het project, in 1992, net na de dood van de innovatieve Hayekiaan op een
leeftijd van eenenveertig jaar, stopzette.

Toch is er voor Extropianen altijd hoop\ldots{} zelfs in de dood.

Als oneindige levensduur voor de mensheid werkelijk binnen handbereik
is, zoals de Extropianen geloven, betekent net voor deze transhumane
doorbraak sterven een extra bittere toevoeging aan de tragedie. Met de
finishlijn in zicht struikelen, misschien slechts enkele decennia te
vroeg, kan het verschil betekenen tussen sterven zoals alle mensen tot
nu toe doorheen de geschiedenis hebben gedaan, en eeuwig leven door het
ondergaan van de transformatie van de menselijke conditie. Een
twintigtal jaren te vroeg sterven, kan betekenen dat je de eeuwigheid
misloopt.

Dit is waarom de Extropianen een noodplan ontwikkelden --- een
ontsnappingsroute om de kloof te overbruggen. De Extropianen omarmden
cryogenica.

Vandaag de dag bewaren vijf faciliteiten verspreid over de VS, China en
Europa\footnote{Faciliteiten omvatten Alcor en het Cryonics Institute,
  Kriorus, Tomorrow Bio en Yinfeng Biological Group.} enkele honderden
lichamen en hoofden van overleden individuen door ze te cryopreserveren.
Voordat ze stierven, meldden deze mensen zich aan om hun lichamen (of
alleen hun hoofden) zo snel mogelijk na klinisch overlijden in te
vriezen en op te slaan bij temperaturen onder nul. Meer dan duizend
andere mensen hebben zich ook aangemeld om hun lichamen of hoofden te
laten conserveren na hun overlijden.

De Extropiaanse voorspelling stelt dat deze individuen misschien op een
bepaald moment in de toekomst weer tot leven gebracht kunnen worden.
Hoewel ze klinisch dood zijn, wachten de mensen die in biostase worden
gehouden in wezen op voortgang van wetenschap en technologie tot op een
punt waar ze kunnen worden ontdooid, opgewekt en genezen van welke
kwalen dan ook hen hadden verslagen. Ze zouden enkele decennia in de
toekomst ontwaken in goede gezondheid, klaar om deel te nemen aan de
transhumane toekomst die hen wacht\ldots{} zo gaat de theorie in ieder
geval. Er is, natuurlijk, geen daadwerkelijke garantie dat dergelijke
opwekkingen ooit mogelijk zullen zijn. Met de technologie van vandaag is
het zeker niet haalbaar. Maar met de technologie van morgen, wie weet?

Zelfs als men inschat dat de kans op succes (zeer) klein is, zou je
redelijkerwijs kunnen schatten dat de kans op uiteindelijke herleving
groter is dan nul. Dat is een gok die Salin en andere Extropianen bereid
waren te nemen.

De overgebleven, levende, Extropianen zullen in de tussentijd gewoon de
vlam van het transhumanisme brandend moeten houden.

\section{Digitale valuta}\label{digitale-valuta}

De Extropiaanse beweging was, net zoals Max More zelf, van nature thuis
in Californië. Silicon Valley werd begin jaren '90 steeds meer erkend
als de wereldwijde hotspot voor innovatie. Dit trok enkele van de meest
ambitieuze technologen, wetenschappers en ondernemers naar de
Amerikaanse westkust.

Maar er was een opmerkelijke uitzondering. In de vroege jaren '90
raakten sommige van de Extropianen ervan overtuigd dat een bijzonder
interessante en belangrijke technologie daadwerkelijk werd ontwikkeld
door een kleine start-up aan de overkant van de Oceaan. Ze waren van
mening dat de realisatie van elektronisch geld cruciaal was, en David
Chaum leek alle kaarten in handen te hebben.

Voor ten minste één Extropiaan, een computerwetenschapper genaamd Nick
Szabo, was dit een reden om naar Amsterdam te verhuizen en zelf voor
DigiCash te gaan werken. Tegelijkertijd begon game-ontwikkelaar Hal
Finney het belang van digitaal geld aan zijn mede-Extropianen te
propageren, in de hoop dat meer van hen zich erbij zouden aansluiten.
Verspreid over zeven pagina's in de tiende editie van Extropy,
gepubliceerd begin 1993, beschreef Finney de interne werking van Chaum's
digitale geldsysteem en legde uit waarom Extropianen met hun libertaire
ethos dit ter harte zouden moeten nemen.

`Vandaag zijn we op een pad dat, als er niets verandert, zal leiden tot
een wereld met potentieel meer regeringsmacht, bemoeienis, en controle',
waarschuwde Finney. `We kunnen dit veranderen; deze technologieën {[}van
digitaal geld{]} kunnen de relatie tussen individuen en organisaties
revolutioneren, door ze voor het eerst op gelijke voet te zetten.
Cryptografie kan een wereld mogelijk maken waarin mensen controle hebben
over informatie over zichzelf, niet omdat de overheid hen die controle
heeft gegeven, maar omdat zij zelf de enigen zijn die de cryptografische
sleutels bezitten om die informatie te onthullen'.\footnote{Hal Finney,
  `Protecting Privacy with Electronic Cash', Extropy 10: 14.}

Finney kreeg gelijk: de community deelde over het algemeen zijn zorgen,
en zij begrepen waarom elektronisch geld een belangrijk deel van de
oplossing vormde. Toen zij meer te weten kwamen over cryptografisch
beveiligd geld, begonnen sommige Extropianen bovendien met het idee te
spelen dat de potentiële voordelen van elektronisch geld zelfs groter
konden zijn dan enkel privacy.

Waar Chaum voornamelijk bezig was geweest met de anonieme functies van
digitale valuta, begonnen deze Extropianen ook het potentieel van
digitale valuta in het kader van monetaire hervorming in overweging te
nemen.

Tegen 1995 bereikte de hernieuwde interesse van de Extropianen een
hoogtepunt in een speciale \emph{Extropy}-editie: het vijftiende nummer
van het tijdschrift was volledig gewijd aan digitaal geld. De omslag van
het tijdschrift bevatte opvallend een blauw-rood achtig ontwerp van een
bankbiljet waarop niet een staatshoofd, maar het portret van Hayek te
zien was. `Vijftien Hayeks', luidde de denominatie, en het zou naar
verluidt uitgegeven zijn door de `Virtuele Bank van Extropolis'.

In het tijdschrift bespraken ongeveer de helft van alle artikelen het
potentieel van elektronisch geld, met verschillende auteurs die
uiteenlopende mening uitdrukten met betrekking tot de digitalisering van
geld. Natuurlijk omvatten deze ideeën de bekende privacyfuncties die het
ontwerp van Chaum bood. Maar de meeste auteurs gingen ook op verkenning
naar aanvullende ideeën.

In zijn `Introductie tot Digitaal Geld', speculeerde software-ingenieur
Mark Grant bijvoorbeeld dat digitaal geld gebruikt kon worden om lokale
munteenheden op te zetten. Hij stelde ook een bijzonder pittig
alternatief voor om Chaumiaans geld te ondersteunen.

`Net zoals de \emph{personal computer} en de laserprinter het voor
iedereen mogelijk hebben gemaakt om een uitgever te worden, maakt
digitaal geld het voor iedereen mogelijk om een bank te worden, of ze nu
een groot bedrijf zijn of een straathoekdrugsdealer met een laptop en
een mobiele telefoon', legde Grant uit. `Sterker nog, naarmate de
nationale schulden blijven toenemen, zouden veel mensen wellicht
voordelen zien in het gebruik van contant geld dat wordt gedekt door,
laten we zeggen, cocaïne in plaats van contant geld dat louter wordt
gedekt door het vermogen van een regering om belastingen te
innen'.\footnote{Mark Grant, `Introduction to Digital Cash', Extropy 15:
  15.}

Een andere bijdrager, web-ontwikkelaar Eric Watt Forste, schreef een
lyrische recensie over het werk van George Selgin, een moderne
onderzoeker van de vrije bankenschool, genaamd `The Theory of Free
Banking'. In zijn boek doet Selgin nauwgezet verslag van hoe de
bankeninfrastructuur zou kunnen evolueren in een omgeving van vrije
banken. Watt Forste suggereerde dat dit boek ook als blauwdruk kon
dienen voor de digitale wereld.

`Terwijl crypto-experts druk bezig zijn uit te leggen hoe deze banken
technologisch zouden kunnen functioneren, legt de theorie van vrij
bankieren uit hoe ze economisch zouden kunnen functioneren',
concludeerde Watt Forste zijn recensie.\footnote{Eric Watt Forste, `The
  Theory of Free Banking', Extropy 15: 53.}

Lawrence White, de naaste ideologische bondgenoot van Selgin in de vrije
bankbeweging, had zelfs een artikel bijgedragen aan het tijdschrift.
Hoewel zijn bijdrage voornamelijk een meer technische vergelijking was
tussen elektronische geldschema's en bestaande betaaloplossingen, hintte
White ook aan hoe digitale valuta de internationale bankdynamieken
drastisch konden veranderen: `Een belangrijk potentieel voordeel van
elektronische geldoverdracht via de persoonlijke computer is dat het
gewone consumenten betaalbare toegang tot \emph{offshore}-bankieren kan
geven'.\footnote{Lawrence H. White, `Thoughts on the Economics of
  'Digital Currency'\,'. Extropy 15: 18.}

Maar misschien wel het meest opmerkelijke van alles, was het artikel van
Max More waarin hij het op zich nam om Hayek's baanbrekende boek over
concurrerende valuta's samen te vatten en te presenteren.

\section{Geld verder
denationaliseren}\label{geld-verder-denationaliseren}

Het werk van Hayek had het Extropianisme gevormd. Het inzicht van de
Oostenrijker in gedistribueerde kennis, vrije markten en spontane orde
was een centrale inspiratiebron voor Max More toen hij de
organisatorische principes van de beweging formuleerde. Nu vroeg More
zijn mede-Extropianen om ook een van Hayek's veel recentere voorstellen
te overwegen: een radicaal idee dat tot dan toe beperkte aandacht had
gekregen.

De oprichter van de Extropiaanse beweging betoogde voor de
denationalisatie van geld.

In zijn artikel toonde More zichzelf een goed onderlegde student van
Hayek's werk en een effectieve communicator van diens ideeën. Hij
presenteerde een beknopte samenvatting van Hayek's bijdragen aan het
bredere debat over monetair beleid en legde uit hoe het fiatgeldsysteem
verantwoordelijk was voor vier `economische kwalen': inflatie,
instabiliteit, ongedisciplineerde staatsuitgaven en economisch
nationalisme.

Inflatie wordt veroorzaakt door de uitbreiding van de geldvoorraad door
de overheid in een Keynesiaanse poging om de werkloosheid te verlagen,
legde More uit, maar in werkelijkheid verstoort dit de economie,
verhoogt het de effectieve belastingen en heeft het bovendien een
verslavend effect.

Ondertussen wordt onstabiliteit veroorzaakt door manipulatie van de
rente door de centrale bank (More vat Hayek's conjunctuurcyclustheorie
samen), niet door een inherente instabiliteit in de markt (zoals hij
benadrukte dat zowel Keynesianen als Marxisten beweren).

`Economisch nationalisme (of wat Hayek eigenlijk monetair nationalisme
noemde), tast bovendien op onnodige wijze verschillende delen van de
economie aan op onvoorspelbare en nadelige manieren', schreef More.

En tenslotte, legde More uit, maakte het monetaire systeem
ongedisciplineerde staatsuitgaven mogelijk: fiatgeld helpt bij het
vergroten van de reikwijdte van de overheid.

`De staat breidt zijn macht grotendeels uit door meer welvaart van
productieve individuen te nemen', schreef de Extropiaan. `Belastingen
bieden een manier om nieuwe agentschappen, programma's en macht te
financieren. Het verhogen van belastingen wekt weinig enthousiasme op,
daarom wenden regeringen zich vaak tot een ander middel van
financiering: lenen en uitbreiden van de geldvoorraad'.\footnote{Max
  More, `Hayek's Denationalisation of Money', Extropy 15: 20.}

Elk van deze kwalen belemmerde economische groei, wat vervolgens de
menselijke vooruitgang beperkte. More vat het probleem bondig samen:
fiatgeld frustreerde de missie van de Extropians.

More beweerde echter dat de problemen konden worden opgelost. Zoals
Hayek in \emph{Denationalisation of Money} had beschreven, was de
oplossing om geld over te laten aan de vrije markt. Als het (de facto)
staatsmonopolie op geld zou worden afgeschaft, zou concurrentie tussen
valuta particuliere valuta-uitgevers stimuleren om werkelijk de meest
wenselijke vorm van geld aan te bieden. Inflatie, instabiliteit,
ongedisciplineerde staatsuitgaven en monetair nationalisme zouden
verleden tijd zijn.

Dat gezegd hebbende, was More zich er ook van bewust dat dit niet
gemakkelijk zou zijn. Terwijl Hayek altijd had geloofd dat het
overtuigen van regeringen om zijn voorstel over te nemen een zware kluif
zou zijn, was de Extropiaan waarschijnlijk nog pessimistischer over dit
vooruitzicht dan de Oostenrijkse econoom in zijn boek was geweest.
Aangezien regeringen het meeste profijt hebben van hun monopolie, hadden
ze geen enkele stimulans om het af te schaffen en alle reden om het
juist niet te doen.

Maar nu zag More een nieuwe kans. Hij was van mening dat Hayeks visie
kon worden gerealiseerd door gebruik te maken van de recente interesse
en innovatie rond elektronisch geld. Het was voor overheden kinderspel
om een geldmonopolie te handhaven wanneer banken makkelijk te
lokaliseren, te reguleren, te belasten, te bestraffen en te sluiten
waren, maar wanneer banken \emph{gehost} kunnen worden op persoonlijke
computers aan de andere kant van de wereld en kunnen opereren met
anonieme digitale valuta, zou de situatie drastisch veranderen.

More dacht dat overheden formeel geen afscheid zouden nemen van het
geldmonopolie. Maar, zo redeneerde hij, De juiste combinatie van
technologieën kon dit monopolie wel veel lastiger afdwingbaar maken.

Via zijn artikel in het tijdschrift, riep de grondlegger van de beweging
op om transactieprivacy en concurrentie in valuta gezamenlijk te
overwegen.

`Concurrerende valuta's zullen het huidige systeem aftroeven door
inflatie te beheersen, de stabiliteit van dynamische markteconomieën te
maximaliseren, de omvang van de overheid te beperken en door de
absurditeit van de natiestaat te erkennen', schreef More. `Deze
hervorming combineren met de introductie van anoniem digitaal geld zou
een mokerslag zijn voor de bestaande orde --- digitaal geld maakt het
moeilijker voor overheden om transacties te controleren en te belasten'.

Concluderend stelde hij: `Ik betreur het recente overlijden van Hayek
ten zeerste. {[}\ldots{]} Omdat hij niet in biostase is geplaatst, zal
Hayek nooit de dagen van elektronisch geld en concurrerende
privévaluta's meemaken die zijn denken mogelijk zal helpen realiseren.
Als we het voortouw willen blijven nemen in de toekomst, laten we dan
kijken wat we kunnen doen om deze cruciale ontwikkelingen te versnellen.
Wie weet zien we ooit nog een privévaluta die zijn naam
draagt'.\footnote{More, `Hayek's Denationalisation of Money', 20.}

\part{Deel II: Cypherpunks}

\chapter{De Cypherpunk-beweging}\label{de-cypherpunk-beweging}

Tim May kon een glimp van de toekomst zien. Hij had een gave om het
potentieel van nieuwe technologieën te herkennen en kon voorspellen hoe
ze de samenleving zouden beïnvloeden voordat bijna iemand anders dat
kon.

Het meest opvallend was dat May al vroeg inzag hoe belangrijk
persoonlijke computers en het internet zouden worden. Zo vroeg als 1973
bemachtigde hij een primitieve DARPA-account op de campus van de UC
Santa Barbara, waar hij natuurkunde studeerde. Een jaar later, op
22-jarige leeftijd, kreeg hij een baan bij Intel, waar hij zou werken in
de Memory Products Division.

De jonge natuurkundige leverde een belangrijke bijdrage in de vroege
geschiedenis van het bedrijf door het alfadeeltjesprobleem op te lossen:
May ontdekte dat de geïntegreerde schakelingen van Intel onbetrouwbaar
waren vanwege licht radioactief verpakkingsmateriaal. Dit zette hem op
weg naar een geweldige carrière bij de snelgroeiende fabrikant van
halfgeleiderchips.

Omdat hij een deel van zijn salaris in de vorm van aandelenopties
ontving, had de natuurkundige van Intel ongeveer een decennium later,
tegen midden jaren 80, een klein fortuin vergaard: op slechts
vierendertigjarige leeftijd concludeerde May dat hij genoeg rijkdom had
verzameld om de rest van zijn leven te onderhouden. Hij besloot
vroegtijdig met pensioen te gaan en verhuisde naar Santa Cruz, een
kustplaats zo'n dertig mijl ten zuiden van San Jose, Californië. Het
grootste deel van het daaropvolgende jaar bracht hij door in een
comfortabele strandstoel om boeken over economie, technische papers en
cyberpunkromans te lezen.

Cyberpunkverhalen, een relatief nieuw genre in die tijd, speelden zich
meestal af in high-tech dystopieën. De boeken schilderden over het
algemeen een grimmige versie van de toekomst, maar eentje waarin het
internet (of een geëvolueerde versie ervan) een toevluchtsoord bood voor
hun vrijheidsgezinde hoofdpersonen. In \emph{True Names} van Vernor
Vinge, verbergen een groep hackers zich voor de sterke mannen van de
regering door hun pseudonieme avatars vrijelijk te laten rondzwerven
door een kleurrijke en driedimensionale representatie van het internet
zelf. In \emph{Snow Crash} van Neil Stephenson verloren naties
grotendeels hun macht aan grote bedrijven en de maffia, terwijl mensen
aan hun miezerig bestaan ontsnapten door alternatieve levens te leiden
in een virtuele wereld. En \emph{Neuromancer} van William Gibson
presenteert evenzo een wereldwijd verbonden, virtuele realiteitsomgeving
als een kleurrijk alternatief voor een vijandige
onderwereldmaatschappij.

May was van plan om uiteindelijk zelf een cyberpunkroman te schrijven,
gemodelleerd naar Ayn Rand's \emph{Atlas Shrugged}. In Rand's verhaal,
dat oorspronkelijk in 1957 werd gepubliceerd, omarmt een Amerika in
verval socialistische doctrines, terwijl het Amerikaanse volk zich
afkeert van de meest succesvolle ondernemers van het land. Sommige van
deze ondernemers besluiten uiteindelijk om `in staking' te gaan: een
kleine gemeenschap van doorgewinterde vernieuwers vestigt zich in een
afgelegen bergketen genaamd `Galt's Gulch', en gebruiken een scherm van
warmtestralingsschermen en reflectoren om zich te verbergen voor de
buitenwereld. De subtiele boodschap van het boek is dat Amerika's meest
ijverige ondernemers niet gedemoniseerd, maar gekoesterd en gevierd
zouden moeten worden.

Na het lezen van Rand's meesterwerk `Atlas Shrugged' als tiener, had dit
boek May op het spoor gezet om meer te leren over vrije markten en
libertarisme. Uiteindelijk zou hij zich toeleggen op de studie van de
Oostenrijkse economie, en in het bijzonder, op de theorieën van
Friedrich Hayek.

Dit alles maakte dat de aspirant-schrijver zich als vanzelfsprekend
thuis voelde in een niche subcultuur in Californië die in de jaren '80
opkwam. Via enkele van zijn lokale vrienden leerde May de Extropianen
kennen en vond hij een ideologische thuis. Hoewel hij niet volledig mee
was met enkele van de meer buitensporige toekomstvisies van de
transhumanisten (ideeën zoals eeuwig leven, het uploaden van de hersenen
of een AI singulariteit) was hij wel toegewijd aan vrijheid en
technologische vooruitgang.

In deze context raakte Tim May bevriend met Phil Salin, de `high-tech
Hayekiaan' die tevergeefs had geprobeerd om een privé
ruimtetransportindustrie op te zetten met zijn start-up Starstruck. May
en Salin deelden een passie voor zowel de Oostenrijkse economie als
technologie. Beiden geloofden dat de voormalige verder kon worden
ontwikkeld door de laatste te benutten.

\section{BlackNet}\label{blacknet}

May had ongeveer een jaar op het strand doorgebracht met boeken toen
Salin hem vertelde over AMIX, het ambitieuze internetproject waar hij
aan werkte.

AMIX, legde Salin aan zijn vriend uit, zou een online marktplaats zijn
voor het kopen en verkopen van informatie. Hij vertelde May hoe dit de
transactiekosten sterk zou kunnen verlagen, wat enorme voordelen zou
opleveren voor de vrije markt. Hij vroeg May wat hij van het idee vond;
Salin wilde graag de feedback van zijn vriend horen.

Het concept leek May inderdaad interessant. Maar na er even over na te
denken, kwam hij tot de conclusie dat zijn interesse voor heel andere
redenen werd gewekt dan die van Salin. May vertelde zijn vriend dat hij
dacht dat deskundig advies of winkeltips, de soorten informatie waar
Salin aan had gedacht, waarschijnlijk niet echt waardevol zouden zijn.
Maar hij geloofde wel dat er een hoge vraag zou zijn naar een heel
andere categorie van informatie.

Geheime informatie.

Mensen zouden bereid zijn om veel geld te betalen voor bedrijfsgeheimen,
geclassificeerde overheidsdocumenten, militaire inlichtingen,
kredietgegevens, medische dossiers, verboden religieus materiaal of
illegale pornografie, stelde May voor.\footnote{Tim May, `Untraceable
  Digital Cash, Information Markets, and BlackNet', The Computers
  Freedom \& Privacy Conference , geraadpleegd
  \href{https://web.archive.org/web/20130501134401/https://osaka.law.miami.edu//~froomkin/articles/tcmay.htm}{online}}
En belangrijk, sommige mensen die toegang hebben tot dit soort
informatie zouden bijna zeker bereid zijn om het te verkopen voor de
juiste prijs, als ze dat anoniem kunnen doen.

Natuurlijk wist May dat het kopen en verkopen van dit soort informatie
in veel gevallen illegaal zou zijn. Als het op AMIX zou worden gezet,
zou Salin naar alle waarschijnlijkheid gedwongen worden om de handel
ervan te verbieden. Maar May voorzag dat dit uiteindelijk geen echt
verschil zou maken. De ontwikkelingen op het gebied van cryptografie die
hij in academische tijdschriften had gelezen, zouden uiteindelijk in
handen van mensen komen, legde hij aan Salin uit, dus het was slechts
een kwestie van tijd totdat er een volledig anonieme variant van AMIX
zou ontstaan waar gebruikers alleen bekend zijn bij hun pseudoniemen, en
aankopen werden gedaan met anoniem digitaal geld.

May gaf deze vorm van informatiemarkt de naam \emph{BlackNet}.

In de maanden na zijn eerste gesprek met Salin, bleef May nadenken over
de bredere implicaties die een dienst als BlackNet met zich mee zou
brengen. Door zijn eigen ideeën verder uit te werken, kwam hij tot de
conclusie dat anonieme informatiemarkten uiteindelijk het fundamenteel
onveilig konden maken voor grote bedrijven om hun medewerkers überhaupt
met gevoelige informatie te laten omgaan. Deze medewerkers zouden immers
altijd in de verleiding kunnen komen om een extra zakcentje te verdienen
door de data online te verkopen.

Volgens May zou dit een soort catch-22 situatie kunnen introduceren.
Bedrijfsgeheimen zouden bedrijven vermoedelijk een voorsprong op hun
concurrenten geven als ze op grote schaal binnen het bedrijf worden
gebruikt, maar in dat geval zou het waarschijnlijk slechts een
tijdelijke voorsprong zijn voordat de informatie naar de concurrenten
lekte. Of, de bedrijfsgeheimen zouden op een zeer beperkte schaal
gebruikt kunnen worden om lekken te voorkomen, in welk geval de
voorsprong op de concurrenten ook niet zo groot zou zijn.

Mogelijk, zo stelde May voor, zou het simpele bestaan van een BlackNet
de economische prikkels die grote bedrijven in de eerste plaats
levensvatbaar maken, fundamenteel kunnen dooreen schudden. In plaats van
miljardenbedrijven zouden we door radicale transparantie een meer
verspreide en levendige economie kunnen zien, gekenmerkt door een veel
diverser aanbod aan kleinere bedrijven.

En May realiseerde zich uiteindelijk dat deze dynamiek niet alleen grote
bedrijven zou beïnvloeden. Het zou ook net zo goed regeringen en hun
strijdkrachten kunnen beïnvloeden, evenals andere openbare instellingen
die vertrouwelijke informatie verwerken. Een enkele corrupte
overheidsmedewerker met financieel gewin als motief zou voldoende zijn
om allerlei geclassificeerde dossiers te verspreiden naar de best
betalende kopers op het internet. Niet in staat om gevoelige gegevens te
beveiligen, zou de macht van de overheid aanzienlijk kunnen afzwakken.
May was dol op dit idee.

En ook een andere lokale vriend van hem\ldots{}

\section{Eric Hughes}\label{eric-hughes}

Toen Eric Hughes in de late jaren 80, toen hij halfweg de twintig was,
wiskunde studeerde aan Berkeley, had de cryptografische revolutie zijn
weg al in het curriculum gevonden. Tegen de tijd dat hij afstudeerde,
was hij goed op de hoogte van recente innovaties van mensen zoals
Whitfield Diffie, Martin Hellman, Ralph Merkle en David Chaum. En net
zoals zij, begreep Hughes intuïtief het veelbelovende potentieel van hun
doorbraken in de context van een steeds meer gedigitaliseerde
samenleving.

Hughes ontdekte dat fundamentele mensenrechten, zoals het recht op
privacy, constant onder druk stonden van overheden. Zelfs al waren
sommige van deze rechten juridisch gewaarborgd, leek het erop dat
overheden altijd een manier vonden om inbreuk te maken als ze daartoe de
mogelijkheid kregen.

Voor Hughes bood moderne cryptografie een methode om individuele privacy
te beschermen, zonder te moeten vertrouwen op wetten, de interpretatie
ervan door politici of rechters. Het recht op privécommunicatie kon in
plaats daarvan gewaarborgd worden door technologieën zoals publieke
sleutel-cryptografie en mixnetwerken.

Hughes besefte dat het niveau van privacy dat bereikt kon worden met
sterke cryptografie, uiteindelijk volledige immuniteit tegen fysieke
bedreigingen en dwang kon bieden. Zolang anonieme internetgebruikers hun
werkelijke identiteit geheim konden houden, kon niets dat ze online
zouden doen of zeggen hen mogelijk in fysiek gevaar brengen.

Toen de jonge wiskundige hoorde dat Chaum een bedrijf had opgericht in
Nederland om een elektronisch geldsysteem te implementeren, besloot de
afgestudeerde van Berkeley om er te solliciteren. Net als Chaum geloofde
hij dat geld hoe dan ook digitaal zou worden, en een privacybehoudende
vorm van valuta kon het verschil betekenen tussen een vrije samenleving
en een totalitaire dystopie. Bovendien geloofde Hughes dat de
cryptografische protocollen van Chaum het potentieel hadden om dat
verschil te maken. Chaum, op zijn beurt, geloofde dat Hughes een goede
aanvulling zou zijn voor zijn bedrijf; hij werd aangenomen.

Toen Hughes in 1991 in Amsterdam aankwam om zijn nieuwe avontuur te
beginnen, raakte hij vrij snel gedesillusioneerd door wat hij in de
kantoren van DigiCash aantrof. Hij ontdekte tot zijn ontsteltenis dat
Chaum smartcards, de fraudebestendige creditcard-achtige computers
speciaal ontworpen voor betalingen, tot hoeksteen van zijn ontwerp had
gemaakt. In plaats van zich puur te richten op de kracht van wiskunde en
het perfectioneren van de cryptografische protocollen die nodig waren om
elektronisch geld voor het internet te implementeren, zag hij dat
DigiCash zich concentreerde op dure en niet-controleerbare
hardwareproducten om offline betalingen mogelijk te maken.

Hughes was van mening dat Chaum een ernstige strategische fout maakte
door pragmatiek en kostenefficiëntie te onderschatten ten gunste van
experimentele features. Uiteindelijk kwam de jonge wiskundige tot de
conclusie dat DigiCash toch niet de plek voor hem was. Na slechts zes
weken in Amsterdam vertrok hij bij de start-up.

Eenmaal terug in Californië, overwoog Hughes op zoek te gaan naar een
woning die iets dichter bij de zee lag. Hij besloot een paar dagen in
Santa Cruz door te brengen om een huis te zoeken. Hij kon er verblijven
bij een oude vriend die daar een paar jaar eerder naartoe was verhuisd:
Tim May.

Toen Eric Hughes in 1991 in Santa Cruz aankwam, deelde May zijn visie
voor anonieme informatie markten met hem. Hij legde uit hoe BlackNets
gebruik zouden maken van het soort privacy tools die Hughes had
bestudeerd en wilde bouwen, en hoe deze de macht van grote bedrijven en
overheidsinstellingen konden verminderen, of zelfs volledig beperken.

Hoewel Hughes zichzelf niet zozeer als een libertariër van de vrije
markt beschouwde zoals Tim May dat deed, intrigeerde het concept van
anonieme informatiemarkten hem net zo goed. De komende paar dagen konden
ze het alleen maar hebben over het enorme potentieel van moderne
cryptografie. Terwijl ze filosofeerden over de implicaties van anonieme
netwerken, de levensvatbaarheid van pseudonieme reputatiesystemen, en de
vooruitzichten van grenzeloze betalingen, moest de huizenjacht even
wachten.

Maar na enkele dagen te hebben gediscussieerd over mogelijk baanbrekende
toepassingen voor publieke sleutel-encryptie, remailers en digitaal
geld, leidden hun gesprekken steeds weer terug naar dezelfde knagende
vraag.

\emph{Waarom was er nog steeds geen software die deze protocollen
implementeerden?}

Geen van de baanbrekende crypto-innovaties die sinds de jaren 70 werden
voorgesteld, werden in de praktijk gebracht door echte mensen, omdat er
geen computerprogramma's beschikbaar waren die deze protocollen
implementeerden. Terwijl academische papers in detail uitlegden hoe
Alice en Bob privé konden communiceren dankzij de Diffie-Hellman
sleuteluitwisseling of RSA-encryptie, was dit aan het eind van de dag
volledig nutteloos zolang er geen software bestond die deze taken voor
Alice en Bob uitvoerde.

Toegegeven, er waren wel een paar projecten in ontwikkeling. In feite
werkte Chaum aan een elektronisch betaalsysteem, hoewel hij dat niet
helemaal ontwikkelde op de manier die Hughes graag zou zien. Daarnaast
werkte een van May's collega-Extropianen, de lokale Bay Area
computerwetenschapper en cryptograaf Phil Zimmermann, aan een op
RSA-gebaseerde publieke sleutelencryptiesoftware genaamd \emph{Pretty
Good Privacy} (PGP).

Toch leken dit erg magere resultaten, als je bedenkt hoe groot de
doorbraken waren die May en Hughes zo enthousiast maakten voor de
toekomst. Hoewel de nieuwe golf van crypto zich ongeveer vijftien jaar
door het academia had verspreid, en een serie succesvolle
Crypto-conferenties een reeks baanbrekende concepten hadden voorgesteld,
bleef de daadwerkelijke softwareontwikkeling ver achter.

\section{De vergadering}\label{de-vergadering}

Eric Hughes verhuisde uiteindelijk niet naar Santa Cruz. Maar de reis
was zeker niet verspild. Gedurende zijn bezoek aan het stadje aan het
strand kwamen Hughes en May overeen dat het tijd was om het gat tussen
de academische wereld en de echte wereld te dichten, en zij
concludeerden dat zij zelf het initiatief moesten nemen om dit te laten
gebeuren. May en Hughes zetten zich in voor het bijeenbrengen van een
groep van enkele van de slimste en meest bekwame cryptografen en hackers
uit de Bay Area, en gingen aan het werk.

De eerste persoon die ze bij hun plan betrokken was John Gilmore, een
vroege medewerker van Sun Microsystems en medeoprichter van de digitale
rechtenorganisatie \emph{Electronic Frontier Foundation} (EFF). Hij had
binnen lokale hackerskringen al geruime tijd gesproken over het brengen
van cryptografie naar het grote publiek. De drie van hen (May, Hughes en
Gilmore) begonnen vervolgens meer gelijkgestemde individuen uit te
nodigen die volgens hen niet zouden terugdeinzen voor enig
\emph{hands-on} technologie-activisme.

Een paar maanden verstreken, tot op een zaterdag in september een groep
van ongeveer twee dozijn gelijkgestemde individuen zich verzamelde in
Hughes' nieuwe en op dat moment nog ongemeubileerde appartement in
Oakland. De meeste aanwezigen waren afkomstig uit de hacker-gemeenschap
in de Bay Area, terwijl May ook een kleine Extropian-delegatie had
geregeld. Dit zorgde voor een bijzonder technologiebewuste groep mensen.

May opende de bijeenkomst met een inleiding over cryptografie. Hij
bracht de aanwezigen op de hoogte van het veelbelovende potentieel van
publieke sleutel-cryptografie evenals enige van de innovatieve schema's
die waren voorgesteld sinds de doorbraak van Diffie en Hellman. Hierna
deelde hij een door hemzelf samengesteld boekje uit waarin de
basisprincipes werden uitgelegd en belangrijke termen werden
gedefinieerd. De groep begon vervolgens te discussiëren over de
mogelijke gevolgen voor de samenleving als cryptografische hulpmiddelen
op grote schaal beschikbaar zouden worden, maar daar tegenover ook de
onrustbarende implicaties van een toekomst zonder dergelijke
hulpmiddelen.

In de namiddag begon de hele bende, vanwege een gebrek aan meubels
zittend op de vloer, op speelse wijze te experimenteren met analoge
representaties van cryptoprotocollen. Ze creëerden het op papier
gebaseerde `crypto-anarchie-spel', waarbij enveloppen fungeerden als een
protocol voor het anonimiseren van berichten, een prikbord fungeerde als
een informatiebeurs en Monopolie-geld werd rondgegeven alsof het
digitale contanten waren. Op een leuke manier kreeg iedereen een gevoel
van hoe deze systemen zouden functioneren.

De bijeenkomst was een succes. Aan het einde van een dag vol
mini-seminars, brainstormsessies en spelletjes, was iedereen die het
appartement van Hughes had bereikt voor deze speciale gelegenheid,
aangestoken met hetzelfde gevoel van enthousiasme dat May, Hughes en
Gilmore had aangezet om deze unieke groep mensen bij elkaar te brengen.
Belangrijker nog, ze deelden nu de visie van de organisatoren dat de
crypto-protocollen waarover ze hadden geleerd, geïmplementeerd moesten
worden als werkende software en zo ver mogelijk verspreid moesten
worden. Om de zaak verder te bevorderen, gingen ze ermee akkoord om de
bijeenkomst een maandelijks evenement te maken.

Op dit moment concludeerde de groep ook dat ze een pakkende naam nodig
hadden om zichzelf mee te omschrijven. Hughes had het tot dan toe de
Cryptologie Amateurs voor Sociale Onverantwoordelijkheid (of kortweg
CASO) genoemd, maar nu overwogen ze suggesties zoals `De Crypto Vrijheid
Liga', `Privacy Hackers', en `De Crypto Cabal'. Middenin al dit gepraat
riep Hughes' vriendin van dat moment, hacker Jude Milhon, gekscherend
uit: `Jullie zijn gewoon een stel cypherpunks!'\footnote{Tim May, `The
  Cyphernomicon', oorspronkelijk verspreid via de
  Cypherpunk-mailinglijst, 10 september 1994, beschikbaar
  \href{https://cdn.nakamotoinstitute.org/docs/cyphernomicon.txt}{online}}

Met de slimme samentrekking van `cipher' en `cyberpunk', had de groep
hun naam te pakken.

\section{De Cypherpunks}\label{de-cypherpunks}

De bijeenkomsten die volgden, werden op wisselende locaties gehouden,
vaak bij iemand thuis of in iemands werkruimte, en vormden een centraal
punt voor het delen van informatie, discussie en projectcoördinatie.
Natuurlijk bood het iedereen ook de kans om elkaar wat beter te leren
kennen, terwijl nieuwe mensen welkom waren om zich aan te sluiten en
meer te leren over het initiatief en hoe ze konden deelnemen.

De Cypherpunks schetsten tijdens deze vroege bijeenkomsten toekomstige
doelen en werkten hun strategieën uit om deze doelen te bereiken.

Allereerst hadden de Cypherpunks zich tot dusverre tot doel gesteld om
een dystopische toekomst te voorkomen, een toekomst waarin digitale
communicatie kan worden gemonitord, geanalyseerd en uiteindelijk
misbruikt. Net zoals de cryptografen die hen inspireerden, waren ze
bezorgd dat zo'n verlies aan privacy despoten en tirannen zou kunnen
versterken, ten koste van de individuele vrijheden: May kondigde op een
gegeven moment half-grappend aan dat George Orwell's \emph{1984}
verplichte lectuur was voor iedereen binnen de groep.

Maar de Cypherpunks waren niet alleen van plan om privacy te promoten of
te eisen. Ze zouden zich niet beperken tot het lobbyen bij verkozen
ambtenaren, of werken via het politieke en juridische proces, zoals
sommige bestaande belangenorganisaties (zoals de EFF) al deden.

Een belangrijk onderdeel van hun strategie was dat de Cypherpunks zelf
de behoeder van hun privacy gingen worden.\footnote{Hal Finney, `Chaum
  on the wrong foot?' oorspronkelijk via de Cypherpunk-mailinglijst, 22
  augustus 1993, beschikbaar
  \href{https://cypherpunks.venona.com/date/1993/08/msg00652.html}{online}}

`We moeten onze eigen privacy verdedigen als we verwachten er nog enige
te hebben', schreef Hughes in `Het Manifest van de Cypherpunk', dat de
mede-oprichter van de groep hardop voorlas tijdens een
Cypherpunk-bijeenkomst begin 1993. `We moeten samenkomen en systemen
creëren die anonieme transacties mogelijk maken. Mensen hebben
eeuwenlang hun eigen privacy verdedigd met gefluister, duisternis,
enveloppen, gesloten deuren, geheime handdrukken en koeriers. De
technologieën van het verleden boden geen sterke privacy, maar
elektronische technologieën doen dat wel.'

Ze hadden plannen om deze elektronische technologieën te ontwikkelen en
deze als gratis software te verspreiden. In overeenstemming met de
hacker-ethiek, hadden ze niet de intentie om iemand om toestemming te
vragen om dit te doen.

`Cypherpunks schrijven code', verklaarde Hughes. `We weten dat iemand
software moet schrijven om privacy te verdedigen, en wij gaan dat
doen.'\footnote{Eric Hughes, `A Cypherpunk's Manifesto', oorspronkelijk
  via de Cypherpunk-mailinglijst, 17 maart 1993, beschikbaar
  \href{https://cypherpunks.venona.com/date/1993/03/msg00392.html}{online}}

`Cypherpunks schrijven code'. Dit werd de informele strijdkreet van de
groep.

\section{De mailinglijst}\label{de-mailinglijst}

De maandelijkse bijeenkomsten van de Cypherpunks waren vrij en open van
aard. Naast de vaste kern van reguliere deelnemers, kwamen er ook
nieuwsgierige nieuwkomers om een indruk te krijgen van wat er gaande
was. Om de coördinatie hiervan te vergemakkelijken, richtte Hughes een
e-maillijst op, gehost op de computer van Gilmore, waar hij aankomende
evenementen aankondigde. Elke abonnee ontving handig een bericht in hun
inbox met daarin de datum en locatie.

Maar de mailinglijst van de Cypherpunks zou al snel een groter doel gaan
dienen. Het duurde niet lang voordat de lijst werd gebruikt om
discussies van de fysieke vergaderingen voort te zetten. Niet veel later
werden er volledig nieuwe onderwerpen op de mailinglijst geïntroduceerd,
die niets te maken hadden met wat er besproken was tijdens de
persoonlijke bijeenkomsten. Toen het aantal berichten toenam, begon de
mailinglijst van de Cypherpunks een eigen leven te leiden.

E-mail had natuurlijk als extra voordeel dat iedereen kon deelnemen,
ongeacht de geografische afstand, en vanuit het comfort van hun eigen
huis. Vrij voorspelbaar, groeide de Cypherpunks mailinglijst snel en
overtrof die de aantallen van de feitelijke Cypherpunk evenementen.
Slechts weken na de lancering had de lijst al 100 abonnees, beduidend
meer dan de paar dozijn hackers en cryptografen die de maandelijkse
bijeenkomsten bijwoonden.

En de populariteit van de mailinglijst explodeerde pas echt toen het
technologietijdschrift Wired in mei 1993 zijn cover-verhaal wijdde aan
de Cypherpunks. `Rebellen met een doel (jouw privacy)' stond er op de
cover, net boven een foto van drie gemaskerde mannen die de Amerikaanse
vlag vasthielden (De uitdrukkingsloze witte maskers met computercodecode
erop gekrabbeld verborgen de gezichten van May, Hughes en Gilmore).
Dankzij dit cover-verhaal had nu vrijwel iedereen met interesse in
computers gehoord van de groep privacy-activisten en door hun
wereldwijde bereik, stroomden honderden mensen van over heel de
Verenigde Staten en de rest van de wereld toe om zich aan te melden voor
hun e-maillijst.

In de daaropvolgende jaren werd de Cypherpunks mailinglijst een klein
fenomeen op het vroege internet. Met tot wel 2.000 abonnees en soms
bijna evenveel e-mails per maand, bespraken de Cypherpunks een breed
scala aan onderwerpen: van cryptoprotocollen, tot overheidsbeleid, tot
implementaties van software, en tips voor boeken of films, evenals
periodieke klaagzangen en verhitte discussies. De lijst bood een
platform voor publieke discussie tussen enkele van de meest
getalenteerde hackers op de planeet, terwijl Silicon Valley's CEO's en
mainstream journalisten ook graag meelazen.

Tim May onderscheidde zich op de mailinglijst dankzij zijn vele mails:
niemand was actiever dan hij, en niemand leverde een groter aanbod aan
bijdragen. Hij schetste toekomstscenario's, deelde ideeën, nam deel aan
discussies, gaf technische uiteenzettingen, stelde strategieën voor, gaf
commentaar op actuele gebeurtenissen, linkte naar relevante artikelen en
deelde regelmatig `snel geschreven' essays, soms zelfs verscheidene op
een dag.

Maar hij onderscheidde zich ook door zijn unieke gevoel voor humor. Soms
voerde hij sarcastisch het woord tegen de Cypherpunk-agenda vanuit het
perspectief van een Ingsociaanse overheid en kafferde hij andere
deelnemers aan de lijst uit als `burgereenheden'. Andere keren maakte
hij opzettelijk politiek incorrecte grappen om de grenzen uit te dagen
van wat als sociaal acceptabel werd beschouwd. Zo veranderde hij op een
gegeven moment bijvoorbeeld zijn e-mailhandtekening in `een bijbeltekst
in afwachting van beoordeling onder de Communications Decency Act',
waarbij de begeleidende tekst een incestueuze orgie
beschrijft.\footnote{Tim May, `Degrees of Freedom', oorspronkelijk via
  de Cypherpunk-mailinglijst, 8 februari 1996, beschikbaar
  \href{https://cypherpunks.venona.com/date/1996/02/msg00637.html}{online}}
Of misschien overschreed hij de sociaal aanvaardbare grenzen ronduit,
afhankelijk van wie je het zou vragen. May leek er hoe dan ook niet veel
om te geven.

En zijn aanwezigheid op de mailinglijst diende ook nog een ander
waardevolle doel. Hoewel de lijst volledig ongemodereerd was, speelde
May vaak de rol van een onofficiële moderator, en begeleidde gesprekken
waar nodig. Als een discussie dreigde te ontsporen, had hij de gewoonte
om zijn kenmerkende analytische perspectief in te brengen, waarbij hij
uitlegde waarom hij geloofde dat bepaalde opmerkingen of onderwerpen al
dan niet geschikt waren voor de lijst, maar zonder daadwerkelijk iets te
verbieden; hij wilde niet dat iemand die macht had, zelfs hijzelf niet.
In plaats van anderen op te leggen hoe ze zich moeten gedragen, had May
een manier om het voortouw te nemen door voorbeeld, en wanneer andere
abonnees klaagden over de inhoud op de mailinglijst, moedigde hij hen
ook aan om het voortouw te nemen door voorbeeld.

Voor veel van zijn abonnees was May in de maanden en jaren dat de lijst
actief was waarschijnlijk de belichaming van wat de Cypherpunk-filosofie
voorstelde. Door zijn sterke aanwezigheid, zowel qua inhoud als in zijn
sturende rol, was hij een leidende en kenmerkende stem van de beweging
geworden.

May zelf, daarentegen, benadrukte vaak dat hij niet de hele
Cypherpunk-gemeenschap vertegenwoordigde. Net zoals hij geen enkele
moderator in controle wilde hebben over de mailinglijst, verwierp hij
ten stelligste het idee dat hij, of wie dan ook, als een formele leider
of woordvoerder van de beweging beschouwd moest worden. Hij stond erop
dat hij slechts één stem was van de velen.

`Hoewel ieder van ons wellicht zijn of haar persoonlijke (hiërarchische)
rangorde van anderen heeft, is het belangrijk dat we nooit hebben
geprobeerd deze rangordes te formaliseren of erover te 'stemmen'. Of te
stemmen om een Grote Leider te kiezen', betoogde May op een gegeven
moment. `Onze kracht zit in onze aantallen en in onze ideeën, niet in de
man die we op een kantoor in Washington hebben geïnstalleerd zodat hij
persconferenties kan geven en \emph{oneliners} kan leveren voor
journalisten. Onze kracht zit in ons meerkoppige (durf ik 'Medusa' te
zeggen?), multinationale, informeel gebrek aan structuur.'\footnote{Tim
  May, `Who shall speak for us?', oorspronkelijk via de
  Cypherpunk-mailinglijst, beschikbaar
  \href{https://cypherpunks.venona.com/date/1995/09/msg02189.html}{online}}

Inderdaad, de Cypherpunks waren geen organisatie in de traditionele zin
van het woord. Het was een bewust informele, ongestructureerde en open
groep. De Cypherpunks hadden geen stemprocedure, geen
vertegenwoordigers, en ze gaven zelfs geen gezamenlijke verklaringen.
Iedereen kon een Cypherpunk worden, maar alle Cypherpunks zetten zich
uiteindelijk in als individuen. Ze hadden geen specifieke taken of
regels, noch kon iemand anderen verantwoordelijk houden voor hun eigen
acties.

Toch werd actie aangemoedigd. Als iemand vond dat de Cypherpunks een
specifieke technologie moesten ontwikkelen, aan een bepaald evenement
moesten deelnemen of op een andere manier tot de zaak konden bijdragen,
was het aan die persoon om het initiatief te nemen en te kijken of
anderen ook wilden helpen.

In feite werkte de Cypherpunk-beweging niet alleen aan een ander type
toekomst. Voor May vertegenwoordigde het die toekomst al.~Hij zag de
open, toestemmingloze en non-hiërarchische manier waarop de Cypherpunks
en hun e-maillijst opereerden als een model voor een opkomende
\emph{crypto-anarchistische} samenleving.

\section{Crypto-anarchie}\label{crypto-anarchie}

De Extropians speculeerden soms over het creëren van vrije gebieden om
te ondermijnen, zich te verbergen, of te ontsnappen aan staatscontrole
over hun levens. Sommigen van hen stelden voor om steden te bouwen op
grote drijvende eilanden in de zee -- het zogenaamde \emph{seasteading}
-- als de weg voorwaarts. Anderen geloofden dat het misschien mogelijk
zou zijn om een klein eiland op te kopen om er een libertaire
samenleving te stichten. Weer anderen suggereerden dat ze allemaal naar
een specifieke jurisdictie moesten verhuizen en proberen lokale
politieke structuren te beïnvloeden om zoveel mogelijk wetten en
regulaties aan de kant te zetten.

Maar Tim May had eigenlijk niet echt zin om te verhuizen. Hij had een
beter idee.

Sinds May de stukjes bij elkaar had gevoegd om het ontwrichtende
potentieel van anonieme informatiemarkten te zien, was hij een toekomst
gaan visualiseren die leek op de werelden uit zijn cyberpunk boeken,
terwijl hij tegelijkertijd analogieën maakte met \emph{Atlas Shrugged}.
Hij besefte dat het soort samenleving dat hij wilde beschrijven in zijn
aankomende roman, werkelijkheid kon worden.

In de roman van Rand maken de productieve ondernemers hun ontsnapping
mogelijk met behulp van futuristische technologie. Hoewel May erkende
dat warmtestralingsschermen nog steeds tot het domein van de
sciencefiction behoorden, was de medeoprichter van de
Cypherpunk-beweging gaan inzien dat het internet en sterke cryptografie
uiteindelijk een vergelijkbare ontsnapping uit de macht van de staat
konden faciliteren, net als in de verhalen van de cyberpunk-genre.

May schetste deze visie in `Het crypto-anarchistisch
manifest'.\footnote{De titel en tekst waren een soort parodie op het
  Communistisch Manifest van Karl Marx en Friedrich Engels, terwijl de
  term crypto-anarchie een soort woordspeling is die verwijst naar
  crypto-fascisme, de geheime steun voor fascisme.} Hij schreef het
oorspronkelijk voor de editie van de Crypto-conferentie in 1988, las het
korte manifest voor tijdens de allereerste bijeenkomst van de
Cypherpunks in Hughes's appartement en deelde het later ook via hun
mailinglijst.

`Een spook waart rond in de moderne wereld, het spook van de
crypto-anarchie', begon het manifest met een knipoog naar het
Communistisch Manifest van Karl Marx en Friedrich Engels, voordat het
voorspelde dat computertechnologie en cryptografische protocollen `de
aard van de overheidsregulering volledig zullen veranderen, het vermogen
om de economische interacties te belasten en te controlen, het vermogen
om informatie geheim te houden, en zelfs de aard van vertrouwen en
reputatie zullen veranderen.'

Om een paar paragrafen verder te concluderen:

`Net zoals de technologie van drukken de macht van de middeleeuwse
gilden en de sociale machtsstructuur heeft veranderd en verminderd, zo
zullen cryptologische methoden de aard van bedrijven en de inmenging van
de overheid in economische transacties fundamenteel
veranderen.'\footnote{Tim May, `The Crypto Anarchist Manifesto',
  oorspronkelijk via de Cypherpunk-mailinglijst, 22 november 1992,
  beschikbaar
  \href{https://cypherpunks.venona.com/date/1992/11/msg00204.html}{online}}

Het internet was nog niet getransformeerd naar de kleurrijke 3D-wereld
zoals die in de romans van Vinge, Stephenson en Gibson werd voorgesteld.
Maar als je deze verhalen meer als metaforische representaties van
online domeinen beschouwde, had May ze toch als visionair leren
waarderen. Terwijl het internet verder zijn onvermijdelijke pad naar
massa-adoptie vervolgde, en mensen stapje voor stapje zouden leren om
zichzelf online te organiseren, geloofde May dat de instellingen van de
reële wereld uiteindelijk zouden worden vervangen door hun
cyber-equivalenten. Het internet zou in toenemende mate dienen als
facilitator voor een parallelle, digitale samenleving, met eigen
gemeenschappen, ondernemingen en uiteindelijk ook eigen economieën.

`Dit maakt snelle experimentatie, zelfselectie en evolutie mogelijk',
stelde May voor op de Cypherpunks mailinglijst. `Als mensen een bepaalde
virtuele gemeenschap beu raken, kunnen ze deze verlaten. De
cryptografische aspecten zorgen ervoor dat hun lidmaatschap van een
bepaalde gemeenschap onbekend blijft voor anderen (vis-a-vis de fysieke
of buitenwereld, oftewel, hun 'echte namen') en fysieke dwang
vermindert.'

Verdergaand: `De elektronische wereld is geenszins volledig, aangezien
we nog steeds een groot deel van ons leven in de fysieke wereld zullen
doorbrengen. Maar de economische activiteit in het domein van het Net
neemt sterk toe en deze ideeën van 'crypto-anarchie' zullen de macht van
fysieke staten om inwoners te belasten en te dwingen verder
ondermijnen.'\footnote{Tim May, `Libertaria in Cyberspace',
  oorspronkelijk via de Cypherpunk-mailinglijst, 9 augustus 1993,
  \href{https://cypherpunks.venona.com/date/1993/08/msg00168.html}{online}}

Dit alles werd mogelijk gemaakt door de kracht van cryptografie. Niet
alleen zouden crypto-middelen gebruikers helpen hun echte identiteit te
beschermen, waardoor ze beschermd werden tegen fysiek geweld, maar
zouden dezelfde middelen ook toestaan dat twee personen zaken konden
doen zonder dat een van hen ooit wist met wie ze te maken hadden.

`{[}\ldots{]} sterke cryptografie is het 'bouwmateriaal' van
cyberspace', schreef May aan zijn mede-Cypherpunks, `de mortel, de
bakstenen, de steunbalken, de muren. Niets anders kan de 'permanentie'
bieden\ldots{} zonder crypto zijn de muren bij de eerste aanraking door
een kwaadwillend persoon of organisatie onderhevig aan instorting. Met
crypto kan zelfs een 100 megaton waterstombom de muren niet
doorbreken.'\footnote{Tim May, `Cyberspace, Crypto Anarchy, and Pushing
  Limits', oorspronkelijk via de Cypherpunk-mailinglijst, 3 april 1994,
  \href{https://cypherpunks.venona.com/date/1994/04/msg00096.html}{online}}

May voorzag dat Cypherpunk-tools mensen zouden helpen om hun economische
activiteit voor de staat te verbergen en zo een `Galt's Gulch in
cyberspace' zouden creëren. Hij keek uit naar een toekomst waarin dit
uiteindelijk zou leiden tot de volledige ineenstorting van regeringen.
Zonder gedwongen herverdeling van welvaart, zou deze toekomstige
economie zichzelf organiseren rondom vrijwillige interactie en vrije
markten. Er zou een \emph{spontane orde} ontstaan via het internet.

`In feite heeft Hayek héél véél met de Cypherpunks te maken!' schreef
May op de Cypherpunk-mailinglijst. `Van 'The Road to Serfdom' tot `Law,
Legislation, and Liberty', zijn werken hebben diepgaande invloed op mij
gehad en op vele anderen. {[}\ldots{]} Ik zou zelfs zeggen dat Hayek een
kandidaat zou zijn geweest om op de cover van `Wired' te staan\ldots{}
natuurlijk ervan uitgaande dat hij 60 jaar jonger was, sommige van zijn
lichaamsdelen had gepiercet, en beter nog, een Netchick was.'\footnote{Tim
  May, `Hayek', oorspronkelijk via de Cypherpunk-mailinglijst, 27
  augustus 1996,
  \href{https://cypherpunks.venona.com/date/1996/08/msg02102.html}{online}
  Als een grappige noot sloot Tim May een andere e-mail af met een
  compliment aan een van zijn mede-Cypherpunks: `Ik wil gewoon eindigen
  op een positieve noot voordat ik vertrek voor de feestdagen.' Zie: Tim
  May, `Re: The War on Some Money {[}long{]}', oorspronkelijk via de
  Cypherpunk-mailinglijst, 21 december 1995,
  \href{https://cypherpunks.venona.com/date/1995/12/msg01044.html}{online}}

\section{Code}\label{code}

Tim May was er zich goed van bewust dat de crypto-anarchistische ideeën
die hij verspreidde niet precies aantrekkelijk waren voor een breed
publiek; niet iedereen hoopt op een ineenstorting van regeringen.
Waarschijnlijk zou het nog erger worden als men erachter zou komen dat
onkraakbare encryptie en anonieme mixnetwerken grootschalige
verspreiding van kinderpornografie mogelijk maakten en veilige
communicatie voor terroristische cellen faciliteerden. Veel mensen
zouden deze nieuwe instrumenten waarschijnlijk met angst en woede
bekijken.

Maar May weigerde de risico's die deze nieuwe technologieën
introduceerden te bagatelliseren.

`Privacy heeft zijn prijs', betoogde hij simpelweg. `Het vermogen van
mensen om achter gesloten deuren misdaden te beramen en te plegen is
evident, en toch eisen we geen verborgen camera's in woningen,
appartementen en hotelkamers!'\footnote{Tim May, `'Stopping Crime'
  Necessarily Means Invasiveness', oorspronkelijk via de
  Cypherpunk-mailinglijst, 17 oktober 1996,
  \href{https://cypherpunks.venona.com/date/1996/10/msg01269.html}{online}}

Voor May was crypto-anarchie bovendien geen verre utopie die brede steun
voor zijn ideeën vereiste: integendeel, hij beschouwde het bijna als een
voldongen feit. Hij begreep dat de weg ernaartoe bezaaid kon zijn met
tegenslagen en onderdrukkende wetten, gerechtvaardigd door beleidsmakers
die angst inboezemden over de `Vier Ruiters van de Infocalyps':
terroristen, pedofielen, witwassers en pornografen. Maar cypherpunk
tools waren goedkoop te verspreiden, makkelijk te gebruiken en konden
niet on-uitgevonden worden. Op de lange termijn, geloofde May, was
succes vrijwel gegarandeerd.

Dit betekende niet dat elke Cypherpunk Mays nogal radicale visie deelde,
noch hoefden ze dat te doen. Vaker wel dan niet waren Cypherpunks
libertariërs, maar velen onderschreven ook een gematigder wereldbeeld
dan May. Sommige Cypherpunks waren zelfs helemaal geen libertariërs, met
meerdere van hen die zichzelf identificeerden als socialist.

`Ik ben geen libertariër, en het is onwaarschijnlijk dat ik dat ooit zal
zijn', schreef Hughes aan de mailinglijst. `Het streven naar privacy
staat los van de meeste partijpolitieke standpunten. Zo sterk als de
libertarische aanwezigheid op deze lijst is, is het geenszins de enige
visie. Juist omdat cypherpunk kwesties dwars door het politieke spectrum
snijden, zijn ze zo krachtig.'\footnote{Eric Hughes, `No digital coins',
  oorspronkelijk via de Cypherpunk-mailinglijst, 24 augustus 1993,
  \href{https://cypherpunks.venona.com/date/1993/08/msg00690.html}{online}}

Wat van belang was, stemden May en Hughes mee in, was dat de Cypherpunks
ongeacht hun politieke overtuiging konden werken aan een gezamenlijk
doel: het ontwikkelen en verspreiden van tools voor privacy.
\emph{Cypherpunks schrijven code}.

Inderdaad, de groep was nauwelijks gestart toen Hughes een vroege versie
ontwikkelde van de eerste remailer ooit, gebaseerd op het
mixnetwerkvoorstel van David Chaum. De eerste uitvoering van deze
remailer zou emails accepteren, details die naar de afzender verwezen
verwijderen en vervolgens doorsturen naar, of de volgende remailer, of
naar de bedoelde ontvanger van de email. Deze uitvoering van Hughes
omvatte echter nog geen encryptietools; het was nog altijd in
ontwikkeling.

Gelukkig kreeg Hughes al snel hulp van een andere Cypherpunk. Hal
Finney, een van de Extropianen die zich bij Tim May hadden aangesloten
voor de Cypherpunk-bijeenkomsten, was net begonnen met het bijdragen aan
de implementatie van Phil Zimmermann's PGP. Met zijn nieuw verworven
kennis van publieke sleutel-cryptografie duurde het niet lang voordat
Finney PGP integreerde in de remailer-code van Hughes.

Slechts enkele weken na de eerste samenkomst hadden de Cypherpunks al
een volledig werkende remailer ontwikkeld. En die gingen ze ook zelf
gebruiken. Finney en een aantal andere Cypherpunks namen het initiatief
om de remailer-programma's uit te voeren, en ze verspreidden software en
beginnershandleidingen zodat anderen zich bij hen konden aansluiten. Op
deze manier werden remailers operationeel. Tegen het einde van 1992 kon
iedereen met een computer en internetverbinding e-mails versturen zonder
hun metadata te onthullen.

Zo rond diezelfde tijd was Zimmermann van plan om PGP 2.0 te lanceren.
De bijdragen van Finney aan deze nieuwe versie hadden geleid tot
aanzienlijke verbeteringen ten opzichte van de eerste versie, en de
software beschikte nu over een \emph{web-of-trust} systeem. Dit stelt
gebruikers in staat om cryptografisch te garanderen dat een publieke
sleutel werkelijk aan een bepaald persoon toebehoort.

Met de gloednieuwe remailer-software van Hughes en de verbeterde
encryptietool van Zimmermann, gingen de Cypherpunks goed van start.

Maar het zou niet lang duren voordat ze in het defensief gedwongen
werden\ldots{}

\section{De crypto-oorlogen}\label{de-crypto-oorlogen}

Toen Bill Clinton in januari 1993 het ambt van de 42e President van de
Verenigde Staten op zich nam, waren er door zijn nieuwe administratie
snel zorgen geuit omtrent het mogelijk onheilspellende gebruik van
persoonlijke computers en het internet. Ze kondigden aan dat
wetshandhavingsinstanties nieuwe tools nodig zouden hebben om mee te
kunnen gaan met het recente tempo van technologische vernieuwing.

Dit bleek een voorbode te zijn van de `crypto-oorlogen' van de jaren
negentig: de Amerikaanse overheid zou proberen het gebruik van
cryptografie te beperken.

De eerste klap viel toen Zimmermann het onderwerp werd van een
strafrechtelijk onderzoek. In deze periode werden cryptografische
protocollen die sleutels gebruikten die groter waren dan 40 bits in de
Verenigde Staten geclassificeerd als munitie onder de definitie van de
\emph{Arms Export Control Act}. Het verzenden of meenemen van een sterk
cryptosysteem naar het buitenland vereiste een vergunning, vergelijkbaar
met de vergunning die nodig is voor internationaal vervoer van
vuurwapens, munitie of explosieven.

Zimmermann bezat geen dergelijke licentie, maar hij verspreidde zijn
gratis software wel via het internet. Omdat het internet geen
landsgrenzen kent, insinueerde de overheid dat Zimmermann zijn software
en de daarin opgenomen cryptoprotocollen illegaal geëxporteerd had.

En toen was er de Clipper-chip, een chipset ontwikkeld door de NSA en
ondersteund door de regering Clinton. De Clipper-chip gebruikte
cryptografie met publieke sleutels om data te versleutelen, maar de NSA
had een speciale ontcijfersleutel toegevoegd aan het protocol. Het plan
was dat telecombedrijven zoals AT\&T de chipset zouden adopteren, zodat
gebruikers hun telefoongesprekken konden versleutelen. De
telecombedrijven zouden echter een ontcijfersleutel bewaren, die op
verzoek aan de regering kon worden overhandigd. Dergelijke
\emph{sleutelbewaring} was noodzakelijk voor nationale
veiligheidsredenen, stond de regering Clinton erop: de autoriteiten
moesten in staat zijn om de telefoongesprekken van potentiële en
verdachte terroristen te beluisteren.

De Clipper-chip werd tegengewerkt door voorstanders van privacy en
belangenorganisaties voor burgerlijke vrijheden in het hele land,
waaronder de Electronic Frontier Foundation (EFF), het Electronic
Privacy Information Center (EPIC) en de American Civil Liberties Union
(ACLU). Ook technologietijdschriften zoals \emph{Wired},
cryptografie-start-ups zoals RSA van Ron Rivest, Adi Shamir en Leonard
Adleman, en enkele politici (zowel Democraten als Republikeinen) voerden
oppositie. Critici waren van mening dat de zogenaamde sleutelbewaring de
kans vergrootte dat burgers onderworpen zouden worden aan verhoogde en
mogelijk onwettige overheidssurveillance.

Uiteraard wezen de Cypherpunks de Clipper-chip ook af. De weerstand
tegen de NSA-chip werd een van de eerste regelmatig terugkerende
onderwerpen op de mailinglijst. Het was duidelijk dat het overhandigen
van decryptiesleutels aan telecombedrijven (en, in verlengde daarvan,
aan overheidsdiensten die de telecoms reguleren) deze cryptografische
protocollen bijna nutteloos zou maken.

Voor de Cypherpunks betekende privacy in grote mate vrijheid van
overheidsinmenging. Aalen zo konden ze een Orwelliaanse toekomst
voorkomen. Uiteraard zou de crypto-anarchistische visie waarschijnlijk
niet ver komen met hun eigen `Galt's Gulch' in cyberspace als de
cryptografische bouwstenen die hen beschermden tegen de staat poreus
bleken te zijn.

Tegelijkertijd stelden de Cypherpunks dat het verplichten van het
gebruik van gecompromitteerde of zwakke encryptieprotocollen de wereld
in geen enkele zin betekenisvol veiliger zou maken. Kwaadwillende
actoren zouden nog steeds degelijke encryptie kunnen gebruiken onder de
zwakke laag van Clipper-chip encryptie om de inhoud van hun communicatie
te verbergen. Dus zelfs als overheidsagenten hun toegewezen
decryptiesleutel zouden gebruiken, zouden ze enkel meer versleutelde
tekst tegenkomen.

Tim May was van mening dat de Cypherpunks een cruciale rol konden spelen
in de naderende cryptografische conflicten. In tegenstelling tot
gevestigde belangengroepen die beter georganiseerd, beter gefinancierd
en beter in contact stonden met beleidsmakers en toezichthouders, zag
May in het anarchistische organisatiemodel van de Cypherpunks juist een
sterkte. Verspreid over de Verenigde Staten en daarbuiten, en zonder
formeel leiderschap of organisatiestructuur, waren de Cypherpunks niet
te coöpteren. Terwijl organisaties zoals EFF, EPIC en ACLU bereid waren
tot zachtaardige onderhandelingen, weigerden de Cypherpunks elke vorm
van gematigdheid te adopteren.

`Op een bepaalde manier vervullen de Cypherpunks een belangrijke
ecologische niche door het voeren van de radicale, buitensporige
oppositie\ldots{} misschien een beetje zoals de rol die de Zwarte
Panters, Yippies en Weather Underground een generatie geleden speelden',
schreef May. \footnote{Tim May, `Crypto Activism and Respectability',
  oorspronkelijk via de Cypherpunk-mailinglijst, 21 april 1993,
  \href{https://cypherpunks.venona.com/date/1993/04/msg00400.html}{online}}

Inderdaad, het verzet van de Cypherpunks nam allerlei vormen aan: juist
omdat ze geen formele organisatie hadden, handelden de Cypherpunks
uiteindelijk uit eigen beweging. Sommigen van hen verspreidden flyers
met informatie over de Clipper-chip in lokale winkelcentra. Anderen
analyseerden hoe de chip werkte om te proberen fouten in het ontwerp te
vinden. En natuurlijk was het schrijven van code de voornaamste
strategie van de Cypherpunks.

`Potentiële inbreuk in de echte wereld zou het volledig verbieden van
sterke crypto kunnen zijn, waarbij ciphers goedgekeurd moeten worden
door de overheid', opperde May in zijn denken over verdere escalatie van
de crypto-oorlogen. `Zo'n verbod zal een zwaar en verwoestend effect
hebben op onze privacy, op ons vermogen om cyberspace-werelden op te
bouwen die ik heb beschreven, en over het algemeen op
computer-gemedieerde markten.'

Hij concludeerde: `Ons onmiddellijke doel moet zijn om ervoor te zorgen
dat de 'geest uit de fles is', dat voldoende crypto-tools en kennis
wijdverspreid zijn zodat zo'n overheidsverbod zinloos wordt.'\footnote{Tim
  May, `Opportunities in Cyberspace', oorspronkelijk via de
  Cypherpunk-mailinglijst, 8 september 1993,
  \href{https://cypherpunks.venona.com/date/1993/09/msg00140.html}{online}}

\section{Sucessen}\label{sucessen}

De Cypherpunks waren gemotiveerd en gedreven, maar stonden tegenover de
volle kracht van de Amerikaanse overheid. Weinigen hadden verwacht dat
ze als overwinnaars uit de crypto-oorlogen zouden komen.

Toch boekten ze het ene succes na het andere.

Een van de meest opmerkelijke overwinningen kan worden toegeschreven aan
Matt Blaze, een beveiligingsonderzoeker bij Bell Labs en een frequente
deelnemer aan de Cypherpunks mailgroep. In 1994, deelde Blaze een
pijnlijke klap uit aan de Clipper-chip door een paper te publiceren die
een fout aantoonde in het ontwerp van de chip, die gebruikers toestond
om de speciale decryptiesleutel uit te schakelen. Terwijl de
reputatieschade van de Clipper-chip al groeide door de hevige tegenstand
die het te verduren had, blijkt nu dat de Clipper-chip zelfs niet deed
waarvoor hij verondersteld werd ontworpen te zijn.

Het was voldoende om telecombedrijven ervan te overtuigen de NSA
technologie niet te adopteren. Het project zou geheel ter ziele gaan
tegen 1996.

Zo wisten Cypherpunks Ian Goldberg en David Wagner in 1995 ook de
gesloten broncode, en \emph{export-grade} encryptiestandaard van
Netscape te kraken. Dit deden ze in het kader van een wedstrijd,
ontworpen door Hal Finney, en ze hadden hier slechts enkele uren voor
nodig. Het schaadde het imago van een van de vooraanstaande bedrijven
van Silicon Valley.

Als reactie op de inbreuk, beweerde Netscape dat ze wettelijk verhinderd
waren om sterkere encryptiestandaarden aan te bieden in het buitenland.
Hoewel dit maar een deel van het probleem was, bood het weinig
geruststelling aan potentiële klanten buiten de VS. Het incident
benadrukte ook een ander probleem als gevolg van de classificatie van
encryptie als munitie: Amerikaanse technologiebedrijven liepen het
risico marktaandeel te verliezen aan buitenlandse
concurrenten.\footnote{Michelle Quinn, `The Cypherpunks Who Cracked
  Netscape', San Francisco Chronicle, 20 september 1995,
  \href{https://people.eecs.berkeley.edu//~daw/press/iang/ian1.html}{online}}

Weer een andere Cypherpunk, Brad Huntting, kwam in 1994 met een slim
idee om de exportrestricties op dergelijke crypto-protocollen uit te
dagen: hij publiceerde code in fysieke vorm om aan te tonen dat verboden
op software-distributies in strijd zijn met fundamentele rechten.

`Het recht op vrije meningsuiting wordt beschermd door de Amerikaanse
grondwet. We hoeven alleen maar aan te tonen dat encryptiesoftware
gelijk staat aan spraak', schreef hij aan de mailinglijst. `Dit zou niet
te moeilijk moeten zijn (misschien een beetje pijnlijk, maar niet
moeilijk). De daad zou een gepubliceerd werk moeten betreffen (bij
voorkeur in de gedrukte zin).'\footnote{Tim May, `Re: Stalling the
  crypto legislation for 2-3 more years', oorspronkelijk via de
  Cypherpunk-mailinglijst, 23 juli 1994,
  \href{https://cypherpunks.venona.com/date/1994/07/msg01245.html}{online}}

Ongeveer een jaar later publiceerde Zimmermann \emph{PGP: Broncode en
Interne Werking}: hij had de volledige PGP-broncode in een boek
afgedrukt. Zoals Huntting inderdaad had aangegeven, vallen boeken
(inclusief de export ervan) in de VS onder de bescherming van het Eerste
Amendement. Door dezelfde informatie die hem onderwerp van een
strafrechtelijk onderzoek had gemaakt, vrij en legaal in harde kaft te
verspreiden, legde Zimmermann de absurditeit van de regelgeving rondom
crypto-export bloot.

Hoewel het nooit werd bevestigd dat het iets te maken had met de
publicatie van zijn boek, liet de Amerikaanse overheid Zimmermann's zaak
vroeg in 1996 vallen.

Bovendien zouden de exportverboden tegen het einde van dat jaar volledig
worden opgeheven. Een combinatie van juridische uitdagingen, economisch
schadelijke beperkingen op de Amerikaanse technologie-industrie, en de
onomkeerbaar wijdverbreide verspreiding van crypto-protocollen buiten de
Verenigde Staten bewoog de Clinton administratie om commerciële
encryptie volledig van de Munitielijst te schrappen.

De Cypherpunks hadden dit niet alleen gedaan: de beweging om
cryptografie tijdens de crypto-oorlogen te verdedigen was breder dan
enkel zij. Maar, zij hadden onmiskenbaar een belangrijke rol gespeeld.
Het losse collectief van hackers en cryptografen, verenigd door weinig
meer dan een mailinglijst op het internet, was de strijd aangegaan met
de Amerikaanse overheid, en had gewonnen.

\chapter{Cypherpunk-valuta}\label{cypherpunk-valuta}

De Cypherpunks hadden zich tot doel gesteld privacy in het digitale
tijdperk te verdedigen. Ze begrepen dat de privacy die fysiek geld
biedt, in gevaar was. Als elektronische betalingen papiergeld en metalen
munten zouden vervangen, konden banken en andere transactieverwerkers
(en bij uitbreiding de regeringen die hen reguleren) alle economische
activiteit monitoren.

De cryptograaf David Chaum en zijn eCash-systeem dienden als grote
inspiratiebron en zijn waarschuwingen echoënden na. De Cypherpunks waren
van mening dat dit uiteindelijk het einde van de menselijke vrijheid zou
kunnen betekenen.

`{[}\ldots{]} als de overheid deze cashloze samenleving creëert, dan zal
de overheid ongekende controle hebben over vrijwel elk aspect van ons
leven', zo zou Tim May stellen op de Cypherpunk-mailinglijst.

`Elke transactie, hoe onbeduidend ook, zal worden vastgelegd, bewaard en
geanalyseerd. Er zal een volledige audittrail bestaan van alle aankopen,
voedselvoorkeuren, entertainmentkeuzes, contacten met anderen,
enzovoort', schreef hij. `Bovendien kunnen transacties die als politiek
incorrect worden beschouwd, en er zijn tientallen duidelijke voorbeelden
om uit te kiezen, simpelweg worden \ldots verboden\ldots{} door het
typen van een paar regels instructies in de relevante
databanken.'\footnote{Tim May, `Scenario for a Ban on Cash
  Transactions', oorspronkelijk verstuurd naar de
  Cypherpunk-mailinglijst, November 24, 1992,
  \href{https://cypherpunks.venona.com/date/1992/11/msg00211.html}{online}}

May gaf in zijn bericht vrij alledaagse voorbeelden om dit punt
duidelijk te maken. Iemand die ooit was gearresteerd voor rijden onder
invloed zou zo bijvoorbeeld kunnen worden verboden om bier te kopen bij
een slijterij, of zwangere vrouwen (`en onder Clintons geautomatiseerde
zorgsysteem zal dit allemaal bekend zijn') zouden kunnen worden
verhinderd sigaretten te kopen. Overheidscontrole over transacties zou
niet alleen invloed hebben op gevaarlijke criminelen of extreme
politieke dissidenten, benadrukte de Cypherpunk, het zou uiteindelijk
leiden tot totale controle over alle burgers: `Vergis je niet, een door
de overheid gerunde maatschappij zonder contant geld zal erger zijn
{[}dan{]} het allerergste van Orwell.'\footnote{Tim May, `Scenario for a
  Ban.'}

Dit is waarom May tijdens de allereerste bijeenkomst van de Cypherpunks
het concept van elektronisch geld presenteerde, en waarom de losse
verzameling van hackers en cryptografen die samenkwamen in het
ongemeubileerde appartement van Eric Hughes het idee direct omarmden. De
creatie van een digitaal betalingssysteem met sterke privacygaranties,
zo geloofden de Cypherpunks, kon helpen zo'n dystopische toekomst te
voorkomen.

Maar dit was niet de enige reden waarom de Cypherpunks digitaal geld
wilden creëren; ze waren van mening dat internetgeld hun beweging op
meer dan één manier kon bevorderen.

Behalve privacy, waren sommige Cypherpunks ook zeer geïnteresseerd in
andere functies die elektronisch geld mogelijk kon bieden, zoals snelle
transactieafhandeling, onomkeerbaarheid, of kostenefficiëntie. Ze waren
enthousiast over de nieuwe mogelijkheden die dergelijke functies
mogelijk konden maken: internetgeld kon online diensten en spellen ten
goede komen, speculeerden ze, of \emph{machine-to-machine} betalingen
mogelijk maken. Dit zou op zijn beurt misschien innovatieve nieuwe
soorten markten kunnen faciliteren, zoals de markten voor het toewijzen
van computerrekenkracht voorgesteld door de hightech Hayekianen.

Dichter bij huis zou digitaal geld voordelen kunnen bieden voor
Cypherpunk-projecten, zoals remailers. De door Eric Hughes en Hal Finney
ontwikkelde remailers werden aanvankelijk gratis aangeboden door
Cypherpunks, maar het was niet duidelijk of deze regeling stand kon
houden. Naarmate deze diensten in de loop van de tijd populairder
werden, zou het beheren van een remailer uiteindelijk een te grote
belasting kunnen worden voor vrijwillige hobbyisten; Hughes verwachtte
dat beheerders op een dag kosten zouden moeten gaan rekenen. Om wille
van voor de hand liggende redenen zouden dergelijke betalingen anoniem
moeten zijn: gebruikers zouden hun identiteit niet moeten onthullen om
remailers te gebruiken.

Evenzo was de ontwikkeling van anonieme informatiemarkten, BlackNets,
afhankelijk van het bestaan van een privacybeschermende vorm van
digitaal geld. Mensen zouden pas bereid zijn geclassificeerde documenten
of geheime rapporten via internet te verkopen, als ze er zeker van waren
dat superieuren bij hun overheidsinstantie of bedrijf niet konden
ontdekken dat zij het waren die deze gegevens voor een beetje extra geld
verkochten: dit betekende dat dit beetje extra geld vrij van elke
identificerende eigenschap moest zijn.

En elektronisch geld was uiteindelijk een cruciale bouwsteen in Tim
May's crypto-anarchistische visie voor de toekomst. Het opzetten van een
parallelle samenleving in de digitale ruimte --- een `Galt's Gulch in
cyberspace' --- vereiste dat mensen hun inkomen en rijkdom verborgen
konden houden voor hun overheid. Een anonieme digitale valuta zou mensen
in staat stellen belastingheffing te ontlopen.

May begreep goed dat elektronisch geld op zichzelf niet meteen de
belastingambtenaar overbodig zou maken. Mensen die zichtbaar deelnemen
aan de economie (`de man die werkt bij Lockheed of achter de toonbank
bij Safeway') zouden nog steeds de rekening betalen voor diensten van de
overheid. Maar volgens May zou, als een significant deel van de economie
succesvol en consequent belastingen kon ontduiken, dit uiteindelijk een
verandering in het publieke sentiment veroorzaken dat dan weer een veel
grotere sociale verandering zou teweegbrengen.

`Wanneer het nieuws zich verspreidt dat veel consultants, schrijvers,
informatieverkopers en dergelijke een groot deel van hun inkomen
afschermen door gebruik te maken van netwerken en sterke crypto, zal de
impact een ondermijning van de steun voor belastingen zijn', schreef
May. `Het belastingstelsel is al wankel, een nationale schuld van \$5
biljoen, die elk jaar groeit, en het zou niet veel van een duw nodig
kunnen hebben om een 'faseverandering' te veroorzaken; een
belastingopstand.'\footnote{Tim May, `Crypto Anarchy, the Government,
  and the National Information Infrastructure', oorspronkelijk verstuurd
  naar de Cypherpunk-mailinglijst, November 29, 1993,
  \href{https://cypherpunks.venona.com/date/1993/11/msg01106.html}{online}}

En tot slot was er nog de mogelijkheid van monetaire hervorming.

In het bijzonder zagen sommige van de Cypherpunks, zoals Tim May, die
ook deel uitmaakte van de Extropiaanse gemeenschap, in elektronisch geld
een instrument voor de verwezenlijking van Friedrich Hayek's vrije
bankeneconomie. Ze beseften dat aangezien elektronisch geld in alles wat
de markt wil kan worden uitgedrukt, dit concurrentie tussen valuta's
veel praktischer kon maken en veel moeilijker te stoppen. Vrije banken
konden overal ter wereld gevestigd zijn, terwijl iedereen met een
internetverbinding hun valuta, anoniem, zou kunnen gebruiken.

`De sterke encryptie die wordt gebruikt, biedt meer flexibiliteit in het
omzeilen van normale valutaregels en kan gebruikers in staat stellen om
onderling overeen te komen welke valuta ze willen gebruiken', schreef
May op de Cypherpunk-mailinglijst.\footnote{Tim May, `DigiCash can use
  whatever currencies are valued', oorspronkelijk verstuurd naar de
  Cypherpunk-mailinglijst, May 4, 1994,
  \href{https://cypherpunks.venona.com/date/1994/05/msg00243.html}{online}}
En: `een van de potentiële voordelen van sterke encryptie is de vaak
besproken 'denationalisatie van geld.'\,'\footnote{Tim May, `Re:
  Hettinga's e\$yllogism', oorspronkelijk verstuurd naar de
  Cypherpunk-mailinglijst, June 28, 1997,
  \href{https://cypherpunks.venona.com/date/1997/06/msg01637.html}{online}}

In zijn geheel stelde elektronisch geld iets voor dat vergelijkbaar is
met de heilige graal van de Cypherpunks.

\section{Het compromis van Chaum}\label{het-compromis-van-chaum}

Toen de Cypherpunks in de vroege jaren '90 net op het toneel verschenen,
was David Chaum's eCash het enige elektronische geldproject dat een
sterke privacy garandeerde. In hun discussies over digitaal geld tijdens
bijeenkomsten of op de Cypherpunks mailinglijst, maakten Tim May en de
andere Cypherpunks vaak expliciet of impliciet verwijzingen naar Chaum's
ontwerp. Hij had met zijn methode van blinde handtekeningen een cruciaal
onderdeel van de privacy-puzzel opgelost.

Verschillende van de Cypherpunks gingen zelfs persoonlijk naar
Amsterdam, om een tijdje te werken bij Chaum's start-up voor digitale
valuta. Naast mede-oprichter van de Cypherpunks Eric Hughes en
computerwetenschapper (en Extropiaan) Nick Szabo, waren dit bijvoorbeeld
ook beveiligingsspecialist Bryce `Zooko' Wilcox-O'Hearn en de vroege
Cypherpunk Lucky Green.

Chaum daarentegen, was niet bijzonder gecharmeerd door de meer radicale
crypto-anarchistische aspiraties die May en sommige van de andere
Cypherpunks voorstonden. Deze cryptograaf was niet bezig met een
digitaal cashsysteem om digitale zwarte markten te faciliteren en hij
had geen verlangen om mensen te helpen regeringen omver te werpen door
middel van massale belastingontduiking. Chaum vond privacy noodzakelijk
om de democratie te redden, niet om ervan af te komen. Hoewel niet alle
Cypherpunks May's meer radicale visie deelden, zou Chaum geen enkele
associatie met hun beweging willen en hij heeft zich nooit aangesloten
bij hun mailinglijst.

Intussen waren ook niet alle Cypherpunks zonder meer tevreden met Chaum
en zijn werk.

Hughes had natuurlijk na slechts enkele weken bij het bedrijf besloten
om DigiCash te verlaten; zijn teleurstelling in Chaum's zakelijke
strategie diende uiteindelijk als motivatie om de Cypherpunk-beweging op
te richten. In de daaropvolgende jaren toonde Hughes zich op de
mailinglijst van de Cypherpunks als een consequente en soms harde
criticus van zijn voormalige werkgever: hij bekritiseerde regelmatig de
voortdurende focus van de start-up op hardwareproducten.

De hoop van de Cypherpunks dat Chaum de belofte van digitaal geld zou
waarmaken, brokkelde verder af toen men ontdekte dat eCash werd
ontworpen zonder sterke privacygaranties voor verkopers (ontvangers van
eCash transacties). Hoewel het elektronische geldsysteem van DigiCash
robuuste privacy bood voor kopers (zenders van transacties), kon de
echte identiteit van een eCash-ontvanger worden onthuld als de zender en
de bank samenwerkten. Kortom, de zender zou de onbeschermde digitale
contanten met de bank moeten delen, zodat wanneer de ontvanger de
eCash-fondsen stortte, de bank dit kon koppelen aan de echte naam die
bij de rekening van de ontvanger hoorde.

Tot verbazing van veel Cypherpunks, beschouwde Chaum dit als een
aantrekkelijke eigenschap. Chaum, de CEO van DigiCash, redeneerde dat
eCash, met de optie om ontvangers te deanoniemiseren, minder snel
gebruikt zou kunnen worden voor afpersing, ontvoering of andere
verontrustende criminele activiteiten. Daarnaast zou het zijn digitale
geldsysteem waarschijnlijk aantrekkelijker maken voor banken en andere
financiële instellingen, en (vooral) voor toezichthouders op deze
instellingen.

\section{Privacy zonder compromissen}\label{privacy-zonder-compromissen}

Voor de meeste Cypherpunks was het compromis over de anonimiteit van
verkopers in eCash een grote teleurstelling.

Hun voornaamste doel was om de bestaande privacy die contant geld al
bood te behouden. Dit omvatte anonimiteit aan beide zijden van een
transactie: bij het wisselen van een dollarbiljet onthullen noch kopers,
noch verkopers hun persoonlijke gegevens. Bovendien \emph{vereiste}
May's crypto-anarchistische visie volledige anonimiteit van zowel kopers
als verkopers: niemand zou gestolen legerdocumenten te koop aanbieden op
een BlackNet als het voor de koper heel eenvoudig was om hun echte naam
te achterhalen.

De meeste Cypherpunks hadden daarom geen enkele wens of intentie om
compromissen te sluiten op het gebied van privacy om elektronische
geldtechnologie minder aantrekkelijk te maken voor criminelen. Ze waren
ook niet geïnteresseerd in het ontwerpen van softwaretools om digitaal
geld acceptabeler te maken voor regelgevers: overheden werden
\emph{zelf} beschouwd als de grootste mogelijke bedreiging voor hun
privacy, zeker in de dystopische toekomst die ze probeerden te
voorkomen. Ze waren ervan overtuigd dat elk zwak punt in
privacy-systemen misbruikt zou worden.

`Ieder systeem dat de overheid in staat stelt om een transactie te
traceren, of om een bericht te traceren, of om toegang te krijgen tot
sleutels, gooit in feite de vrijheidsbevorderende voordelen van
cryptografie volledig weg', betoogde May op de mailinglijst. `Vraag
jezelf eens af of de regering van Myanmar, bekend als SLORC, haar
'Overheidstoegang tot Sleutels' niet zou gebruiken om dissidenten in de
jungle op te pakken. Zouden Hitler en Himmler `sleutelherstel' hebben
gebruikt om te achterhalen met wie de Joden communiceerden zodat ze
allemaal konden worden opgepakt en vermoord? Zou de Oost-Duitse Stasi
eCash transacties hebben getraceerd? Voor elke regering op de planeet
{[}\ldots{]} kan men makkelijk tientallen voorbeelden bedenken waar
toegang tot sleutels, toegang tot dagboeken, toegang tot
uitgavenrecords, etc., misbruikt zou worden.'\footnote{Tim May,
  `'Stopping Crime' Necessarily Means Invasiveness', oorspronkelijk
  verstuurd naar de Cypherpunk-mailinglijst, October 17, 1996,
  \href{https://cypherpunks.venona.com/date/1996/10/msg01269.html}{online}}

En hoewel het waar is dat afpersers, ontvoerders en andere criminelen
ook zouden profiteren van een volledig anoniem betalingssysteem,
geloofde May niet dat de soorten privacycompromissen die Chaum
nastreefde ook maar iets konden oplossen in de realiteit. Hij merkte op
dat de situatie waarin een ontvoerder wordt gepakt omdat zijn identiteit
wordt onthuld door samenwerking tussen de betaler en de bank
waarschijnlijk nooit echt zou voorkomen, omdat de ontvoerder in eerste
instantie dit systeem met aangetaste privacyfuncties niet zou gebruiken.

Eerder zou, zolang er ergens in de wereld \emph{een}
privé-betaaloplossing bestond, de kans groot zijn dat ontvoerders eisten
dat het geld gewoon via dat systeem naar hen wordt verzonden\ldots{} en
er zou waarschijnlijk altijd wel \emph{ergens} een privé-betaaloplossing
bestaan.

`Het kan een fysieke bank zijn, zoals de Bank van Albanië, of het kan
een ondergronds betalingssysteem zijn, zoals de maffia, de Tongs, de
Triades, of wat dan ook', suggereerde May. `Helaas voor {[}financiële
toezichthouders{]}, en jammer genoeg voor de slachtoffers van dergelijke
misdaden, lijkt een wereldwijde stopzetting van dergelijke systemen niet
mogelijk, zelfs met draconische politiestaatsmaatregelen. Er zijn gewoon
te veel tussenruimtes waarin de bits zich kunnen verstoppen. En er is te
veel economische prikkel voor sommige personen of banken om dergelijke
fondsoverdrachtmethoden aan te bieden.'\footnote{Tim May, `Untraceable
  Payments, Extortion, and Other Bad Things', oorspronkelijk verstuurd
  naar de Cypherpunk-mailinglijst, December 21, 1996,
  \href{https://cypherpunks.venona.com/date/1996/12/msg01468.html}{online}}

Chaum had, door middel van eCash, mede vorm gegeven aan hoe May en zijn
collega privacy-activisten dachten over online privacy en elektronische
betalingen. Maar een paar jaar na het starten van zijn digitale
geldproject, maakte deze baanbrekende cryptograaf fundamenteel andere
afwegingen dan de meeste van de Cypherpunks.

Teleurgesteld schreef May naar de Cypherpunks mailinglijst:

`Als het gaat om volledig ontraceerbaar digitaal geld, het echte e-geld,
dan zijn wij misschien wel de vaandeldragers hiervan.'\footnote{Tim May,
  `'Stopping Crime'.'}

\section{Octrooien}\label{octrooien}

Sommige van de Cypherpunks kwamen wel met oplossingen om een
eCash-achtig systeem te creëren met volledige privacy voor zowel kopers
als verkopers.

Zooko stelde bijvoorbeeld dat het probleem makkelijk opgelost kon worden
als gebruikers pseudonieme bankrekeningen mochten hebben: banken kunnen
een eCash-betaling niet aan een echte identiteit koppelen als ze de
echte identiteit van hun rekeninghouders überhaupt niet kennen (de
Cypherpunks spraken soms over offshore bankieren in deze context).

Nick Szabo stelde een andere aanpak voor. Verkopers zouden misschien de
eCash die ze hebben ontvangen kunnen ruilen met de eCash van een ander
persoon, en dan deze `nieuwe' eCash naar de bank sturen. Op deze manier
zou de oorspronkelijke betaling niet gelinkt kunnen worden aan de echte
identiteit van de ontvanger. Natuurlijk zou dit wel vertrouwen vergen in
de persoon die de eCash uitwisselt. De uitwisselaar zou immers nog
steeds kunnen samenwerken met de bank om de identiteit van de
eCash-ontvanger te onthullen, of zelfs de eCash kunnen stelen door
halverwege de transactie af te haken.

Een derde optie werd voorgesteld door een anonieme deelnemer aan de
mailinglijst die een plan bedacht waarbij de verkoper betrokken was bij
de aanvankelijke creatie van de eCash. In het originele eCash-protocol
zijn alleen de koper en de bank betrokken bij de generatie van blinde
sleutels en blinde handtekeningen. Echter, in dit protocol zou ook de
uiteindelijke ontvanger van de elektronische valuta een laag van
versleuteling toevoegen.

Maar voor al deze oplossingen was er één groot probleem: David Chaum
bezat het patent op het algoritme voor de blinde handtekening. En de
Cypherpunks waren van mening dat hij zijn product alleen zou licenseren
aan `respectabele' organisaties die zijn technologie, naar zijn mening,
niet in diskrediet zouden brengen. Het spreekt voor zich dat de
Cypherpunks concludeerden dat hij het niet aan hun gevarieerde groep van
crypto-anarchisten en privacyfundamentalisten zou licenseren.\footnote{Douglas
  Barnes, `Re: cypherpunks digicash bank?', oorspronkelijk verstuurd
  naar de Cypherpunk-mailinglijst, October 8, 1995,
  \href{https://cypherpunks.venona.com/date/1995/10/msg00731.html}{online}}

Deze realiteit was een bron van frustratie voor de Cypherpunks, wellicht
het best verwoord in een uitgebreide post op een mailinglijst door Tim
May. Door zowel de hacker-ethiek als de Extropian-filosofie te
belichamen, betoogde de medeoprichter van de groep dat technologische
vooruitgang het best werd bereikt door experimenteren en concurrentie.
Hij bepleitte dat softwarepatenten beide belemmerden.

Fysieke producten, zelfs als ze gepatenteerd zijn, kunnen tenminste naar
wens van de koper gebruikt worden na aankoop, merkte May op.
Bijvoorbeeld, gepatenteerde microchips werden in de jaren '80 vrijelijk
gebruikt door onafhankelijke technologen om hun eigen computers te
bouwen, wat de revolutie van de persoonlijke computer in gang zette. In
vergelijking daarmee leggen softwarelicentieregelingen allerlei
beperkingen op hoe de software gebruikt kan worden, zelfs nadat deze is
aangeschaft, wat effectief verdere innovatie doodt.

`{[}\ldots{]} het heikele punt is dat de patenten op de software-ideeën
en -concepten betekenen dat experimenteerders, ontwikkelaars, en hackers
geen licentie kunnen kopen voor digicash op dezelfde manier als ze
zouden doen voor sommige IC's; en vervolgens experimenteren, ontwikkelen
en hacken', legde May uit. `De man in zijn garage die een 'digitale
postzegel' probeert te ontwikkelen, kan bijvoorbeeld de Chaumiaanse
verblindingsprotocollen niet gebruiken zonder advocaten in te huren,
Chaum zijn voorafgaande vergoeding te betalen, en zijn ontwerpen en
bedrijfsplannen (die hij waarschijnlijk zelfs niet heeft!) te
openbaren.'\footnote{Tim May, `Software Patents are Freezing Evolution
  of Products', oorspronkelijk verstuurd naar de
  Cypherpunk-mailinglijst, October 7, 1995,
  \href{https://cypherpunks.venona.com/date/1995/10/msg00685.html}{online}}

Hierdoor verwachtte May dat de ontwikkeling van digitaal geld tot
stilstand zou komen.

Hoewel David Chaum ongetwijfeld een pionier was op dit gebied, begonnen
de Cypherpunks de cryptograaf steeds meer te zien als iemand die de
toekomst van het elektronisch geld eerder hinderde dan bevorderde.

\section{Alternatieven}\label{alternatieven}

Als de Cypherpunks het eCash protocol wilden verbeteren, moesten ze
daarvoor nog een decennium wachten tot de patenten van Chaum verlopen
waren, en dat leek een eeuwigheid. Tim May, Eric Hughes en John Gilmore
hadden de Cypherpunk beweging juist opgezet omdat ze al gefrustreerd
waren door de gebrek aan voortgang in de cryptografie, en dat al ruim
tien jaar na de publicatie van Chaum's eerste digitale geld papers.

Het was tijd voor een elektronische vorm van geld, en de
\emph{Cypherpunks schrijven code}; ze zouden een manier moeten vinden om
de patenten te omzeilen.

Een manier om dit te doen was door naar alternatieven te zoeken. Chaum's
voorstellen uit de jaren 80 hadden het begin gemarkeerd van een reeks
publicaties in de jaren 90: binnen enkele jaren werden ongeveer 200
papers gepubliceerd over het onderwerp elektronisch geld en werden
enkele tientallen nieuwe patenten ingediend. \footnote{Eduard de Jong,
  `Electronic Money: From Cryptography and Smart Cards to Bitcoin and
  Beyond', Fraunhofer SmartCard Workshop 2017 .} Als een van de meer
opvallende voorbeelden ontwierpen Ron Rivest en Adi Shamir (bekend van
RSA) in 1996 twee \emph{micropayment}-schema's genaamd PayWord en
MicroMint, terwijl in datzelfde jaar de NSA ook een elektronisch
geldschema zou presenteren.\footnote{Laurie Law, Susan Sabett, and Jerry
  Solinas, `How To Make a Mint: The Cryptography of Anonymous Electronic
  Cash', National Security Agency Office of Information Security
  Research and Technology, Cryptology Division, June 18,
  1996,~\href{https://groups.csail.mit.edu/mac/classes/6.805/articles/money/nsamint/nsamint.htm}{online}}

Van tijd tot tijd stelden diverse Cypherpunks ook alternatieve ontwerpen
voor elektronisch geld voor op de mailinglijst. Als een vroeg, en nogal
origineel, voorbeeld heeft de informatica student Hadon Nash een
digitaal valutaschema gepresenteerd dat hij `Digitaal Goud' noemde. In
zijn bericht beschreef Nash een systeem waarbij gebruikers `eigenaar'
zouden zijn van getallen; om het even welk getal, maar lagere getallen
zouden een lagere waarde hebben. Een getal claimen was net zo simpel als
het produceren van een cryptografische handtekening met dat getal, en de
eerste persoon die het claimde zou het `bezitten'. Het overdragen van
het eigendom van een getal werd gedaan met een bericht dat de nieuwe
eigenaar bevatte, alleen geïdentificeerd met een publieke sleutel, en
een cryptografische handtekening die aantoonde dat de oorspronkelijke
eigenaar de overdracht goedkeurde. De eigendomsgeschiedenis van elk
getal kon dan worden gevolgd door een reeks cryptografische
handtekeningen.\footnote{Hadon Nash, `Digital Gold', oorspronkelijk
  verstuurd naar de Cypherpunk-mailinglijst, August 24, 1993,
  \href{https://cypherpunks.venona.com/date/1993/08/msg00698.html}{online}}

De meeste ontwerpen voor elektronisch geld introduceerden echter niets
baanbrekends. Als het over privacyfuncties gaat, waren er veel,
waaronder het door de NSA voorgestelde schema, die slechts variaties
waren op of iteraties van het ontwerp van Chaum. En degenen die wel een
vernieuwende aanpak introduceerden, hadden vaak andere fundamentele
beperkingen. Zo konden de elektronische geldschema's van Rivest en
Shamir bijvoorbeeld alleen echt nuttig zijn voor tamelijk specifieke
soorten betalingen van een lage waarde, omdat de geldigheid van de
valuta-eenheden snel zou verlopen.

Andere ideeën waren op een fundamenteler niveau gebrekkig, zoals Nash's
`Digital Gold'-oplossing, die geen rekening hield met het probleem van
dubbel uitgeven. Zoals Hal Finney op de Cypherpunks mailinglijst aangaf,
als hetzelfde nummer werd overgedragen aan meerdere publieke sleutels,
zouden verschillende gebruikers kunnen eindigen met uiteenlopende
eigendomsregistraties (Nash stelde wel een systeem voor waarin
gebruikers bij `agentschappen' zouden moeten controleren of een munt
door hen zou worden geaccepteerd voordat ze deze zelf accepteerden, wat
zou bewijzen dat het nummer niet dubbel werd uitgegeven, maar dit deel
van het voorstel was niet erg goed uitgewerkt).

`Ik moet echter wel zeggen', schreef Finney, `dat ondanks het feit dat
ik niet echt denk dat het voorstel voor digitaal goud technisch haalbaar
is, het idee om eigenaar te zijn van getallen enorm brutaal en
behoorlijk creatief is.'\footnote{Hal Finney, `Digital Gold, a bearer
  instrument?', oorspronkelijk verstuurd naar de
  Cypherpunk-mailinglijst, August 26, 1993,
  \href{https://cypherpunks.venona.com/date/1993/08/msg00788.html}{online}}

\section{De experimenten}\label{de-experimenten}

Omdat er niet veel veelbelovende alternatieven leken te zijn, was een
populaire methode om Chaum's patenten te omzeilen dus om onder de radar
te blijven.

Verschillende Cypherpunks zouden uiteindelijk hun eigen versies van
eCash implementeren, maar om rechtszaken te voorkomen, waren deze
digitale geldsystemen alleen bedoeld voor testdoeleinden. Ze gingen
ervan uit dat hun experimentele projecten getolereerd zouden worden,
zolang ze geen commerciële intenties hadden en niet gebruikt werden om
echte waarde over te dragen.

Midden 1994 vertegenwoordigden de speelgeldschema's een kleine trend
onder de Cypherpunks. Zo lanceerde Matt Thomlinson, een bijdrager aan de
mailinglijst, een implementatie van eCash die hij Ghostmarks noemde.
Tegelijkertijd lanceerde Pr0duct Cypher, een pseudonieme individu die
ook bijdroeg aan PGP, Magic Money. En Mike Duvos, een andere
terugkerende gast op de mailinglijst, beheerde een Magic Money
implementatie die hij Tacky Tokens noemde.

Bovendien presenteerde iemand die zichzelf Black Unicorn noemt, zelfs
het `volledig gedekte' DigiFrancs systeem:

`DigiFrancs worden gedekt door 10 kratten Light Cola, die zich bevinden
in de 'kluis' reserves van UniBank in Washington, DC. DigiFrancs kunnen
worden ingeruild voor hun equivalent in ongekoelde, 16 oz Light Cola
blikjes op aanvraag, in Washington, DC. Deze regeling impliceert geen
overeenkomst tussen enige van de partijen en de Coca-Cola
maatschappij.'\footnote{Black Unicorn, `DigiCash Announcement',
  oorspronkelijk verstuurd naar de Cypherpunk-mailinglijst, May 10,
  1994,
  \href{https://cypherpunks.venona.com/date/1994/05/msg00616.html}{online}}

Hoewel dit uiteraard bedoeld was als grap, zelfs de door Light Cola
gedekte DigiFrancs waren niet daadwerkelijk bedoeld om als geld te
worden gebruikt, kregen de verschillende soorten elektronisch geld toch
enige bekendheid als ruilmiddel. Sommige Cypherpunks accepteerden deze
speelvaluta's in ruil voor digitale snuisterijen zoals GIFs.

Het waren niet bepaald grote waardeoverdrachten, maar het riep wel een
interessante vraag op.

`Nu, als je nog wakker bent, komt het leuke deel', kondigde Pr0duct
Cypher aan, nadat hij de technische details van Magic Money op de
mailinglijst had uitgelegd. `Hoe introduceer je echte waarde in je
digicash systeem? Hoe krijg je mensen eigenlijk zover dat ze het willen
proberen?'

Chaum wilde zijn eCash-systeem implementeren via bestaande financiële
instellingen, waar digitale geld-eenheden inwisselbaar zouden zijn voor
echte dollars (of andere fiatvaluta). eCash zou in feite gedekt worden
door geld op de bank, wat --- hopelijk --- mensen gerust genoeg zou
stellen om te beginnen met het accepteren van dit nieuwe type geld als
betaling.

Maar Pr0duct Cypher stelde nu voor dat deze inlosbaarheid voor een
dollar misschien niet echt nodig was. Misschien, speculeerde hij, zou
digitaal geld helemaal geen dekking nodig hebben.

`Wat maakt goud waardevol?' vroeg hij retorisch. `Het heeft enkele
nuttige eigenschappen: het is een goede geleider, is bestand tegen
corrosie en chemicaliën, enzovoorts. Maar die zijn pas recent van belang
geworden. Waarom is goud al duizenden jaren waardevol? Het is mooi, het
glanst, en het allerbelangrijkste, het is schaars.'

Pr0duct Cypher concludeerde:

`Jouw digitaal geld moet schaars zijn.'\footnote{Pr0duct Cypher, `Magic
  Money Digicash System', oorspronkelijk verstuurd naar de
  Cypherpunk-mailinglijst, February 4, 1994,
  \href{https://cypherpunks.venona.com/date/1994/02/msg00247.html}{online}}

\section{Gedekt digitaal geld}\label{gedekt-digitaal-geld}

Gedurende de volgende paar jaar, de midden jaren negentig, kwamen
onderwerpen als de aard van geld, waarde en dekking van valuta
regelmatig ter sprake op de mailinglijst van de Cypherpunks. Het was
logisch dat de Cypherpunks deze kwesties bespraken, want om elektronisch
geld te creëren, moesten ze begrijpen wat het eigenlijk waarde zou
geven. Waarom zou iemand echte producten of diensten opgeven in ruil
voor een nummer op een scherm?

Het bleek dat diverse leden van de lijst hierover zeer uiteenlopende
ideeën hadden.

Sommigen beschouwden digitaal contant geld echt niet als iets waardevols
en dachten zelfs dat het een beetje ongepast was om er in monetaire
termen over te praten. Software-ontwikkelaar Perry Metzger, een van de
actievere leden in discussies over digitaal geld, beweerde bijvoorbeeld
dat een term zoals `anonieme digitale bankcheques' in feite een
nauwkeurigere beschrijving zou zijn.

`Hetzelfde geldt voor een cheque die kan worden GEBRUIKT voor geld, maar
in feite een manier is om geld OVER TE DRAGEN, digicash is van zichzelf
geen bron van waarde, het is een boekhoudsysteem voor iets dat waarde
heeft', schreef hij. `Dat iets kan dollars zijn, goud, cocainefutures
gecontracteerd op de Bogota Commodity Exchange, koekjes, of alles wat
mensen besluiten een goed ruilmiddel te vinden.'\footnote{Perry E.
  Metzger, `Re: Virtual Cash', oorspronkelijk verstuurd naar de
  Cypherpunk-mailinglijst, May 3, 1994,
  \href{https://cypherpunks.venona.com/date/1994/05/msg00131.html}{online}}

Anderen, zoals Eric Hughes, hadden er geen probleem mee om digitaal
contant geld als een vorm van geld te beschouwen, maar waren het ermee
eens dat de bits en bytes gedekt moesten worden door \emph{iets} van
waarde; in de meeste gevallen zou een conventionele fiatvaluta zoals de
dollar of het pond de voor de hand liggende keuze zijn. Hij geloofde dat
voor de popularisatie van elektronisch geld, gebruikers de garantie
moesten krijgen dat ze hun digitale geld kunnen inruilen voor een
vastgesteld bedrag aan `echt' geld.

Een andere visie kwam van Robert Hettinga, een Cypherpunk die (in zijn
eigen woorden) ooit in `de kooi van Morgan Stanley in Chicago'
werkte.\footnote{Robert Hettinga, `Re: digital cc transactions, digital
  checks vs real digital cash', oorspronkelijk verstuurd naar de
  Cypherpunk-mailinglijst, May 3, 1997,
  \href{https://cypherpunks.venona.com/date/1997/05/msg00147.html}{online}}
Als een andere zeer actieve deelnemer in discussies over geld en
elektronisch geld op de Cypherpunk-mailinglijst, betoogde Hettinga dat
de behoefte aan inwisselbaarheid voor `echt' geld na verloop van tijd
zou verdwijnen, aangezien mensen langzaam vertrouwen zouden krijgen in
de nieuwe digitale munteenheid.

Hij maakte een analogie met de evolutionaire biologie, en schreef:

`Vergeet niet dat de eerste vleugels geëvolueerd zijn door op vijvers
scherende insecten zodat ze sneller over vijvers konden glijden. Na
verloop van tijd, toen die beginnende vleugels uiteindelijk volledige
vleugels werden, hadden de vliegende insecten geen vijvers meer nodig.
Met dat idee in gedachten, zullen digitale waardepapieren voorlopig nog
moeten interfacen met de wereld van fysieke boekhouding, om te kunnen
worden omgezet in andere activa.'

Maar, zo voorspelde Hettinga: `Uiteindelijk zullen die activa op een
gegeven moment niet langer boekvermeldingen zijn.'\footnote{Robert
  Hettinga, `Re: Bypassing the Digicash Patents', oorspronkelijk
  verstuurd naar de Cypherpunk-mailinglijst, April 29, 1997,
  \href{https://cypherpunks.venona.com/date/1997/04/msg00811.html}{online}}

Ondertussen geloofde Tim May niet dat geld in de eerste plaats gedekt
hoefde te zijn. In lijn met het vooruitziende principe van Hayek en zijn
vrije banken, stelde hij dat het uiteindelijk allemaal neerkwam op
toekomstige verwachtingen: mensen hadden slechts vertrouwen nodig dat
geld zijn koopkracht zou behouden.

`Ik geloof dat alle vormen van geld, of het nu harde valuta of fiat is,
voortkomen uit de verwachting dat het geld in de toekomst ongeveer
dezelfde waarde zal hebben', schreef hij. `Noem het de 'de theorie van
de grotere dwaas van geld.' Alles wat je belangrijk vindt, is dat een
grotere dwaas het geld zal aanvaarden.'\footnote{Tim May, `Re:
  alternative b-money creation', oorspronkelijk verstuurd naar de
  Cypherpunk-mailinglijst, December 11, 1998,
  \href{https://cypherpunks.venona.com/date/1998/12/msg00455.html}{online}}

Maar toch zag May elektronische valuta niet echt als geld op zich. Hij
beschouwde het eerder als iets nieuws, een manier om geld over te maken
van de ene bankrekening of opslagplaats naar de andere, eerder
vergelijkbaar met cheques of bankoverschrijvingen. Hoewel hij dus niet
geloofde dat digitale valuta op de traditionele manier `gedekt' hoefde
te worden, ging May er wel van uit dat banken of bankachtige instanties
die valuta uitgeven, uiteindelijk controle zouden hebben over de
daadwerkelijke munteenheid die het elektronisch geld dekt: `Wat digitale
valuta echt ondersteunt, is de reputatie van de instanties.'\footnote{Tim
  May, `What backs up digital money?', oorspronkelijk verstuurd naar de
  Cypherpunk-mailinglijst, March 27, 1996,
  \href{https://cypherpunks.venona.com/date/1996/03/msg01576.html}{online}}

En toen was er een Britse Cypherpunk die dacht dat elektronisch geld
helemaal niet gedekt hoefde te zijn\ldots{}

\section{Adam Back}\label{adam-back}

Adam Back, een post-doctoraatsstudent in de informatica aan de
Universiteit van Exeter en midden in de twintig, had nog nooit in
persoon een Cypherpunk bijeenkomst bijgewoond. De mailinglijst had hij
wel online gevonden. Hij raakte bijzonder geïnteresseerd in elektronisch
geld en werd snel een van de meest actieve deelnemers in de discussies
over dit onderwerp.

Back had zelf een beetje geëxperimenteerd met CyberBucks en had uit
eerste hand ervaren dat mensen waarde konden toekennen aan puur
digitale, ongedekte munten. CyberBucks, gebaseerd op niet veel meer dan
een belofte over het totale aanbod, werden verhandeld voor echt geld en
gebruikt in echte handel, zelfs als het slechts in een kleine niche in
een obscure hoek van het internet was.

Hoewel CyberBucks op het moment dat de server van DigiCash offline ging
direct waardeloos werd, zag Back geen enkele reden waarom een nieuw en
beter elektronisch geldsysteem op dezelfde manier geen waarde kon
krijgen, en het zelfs beter kon doen.

`Waarom koppel je je eCash systeem niet los van US dollars door
kredietkaarten / debitkaarten / cheques / cash, en zet je een volledig
op zichzelf staand systeem op?' stelde hij voor op de Cypherpunks
mailinglijst in april 1997.\footnote{Adam Back, `Re: Bypassing the
  Digicash Patents', oorspronkelijk verstuurd naar de
  Cypherpunk-mailinglijst, April 30, 1997,
  \href{https://cypherpunks.venona.com/date/1997/04/msg00822.html}{online}}

`Wat we willen, is volledig anonieme eenheden van uitwisseling, met
ultra lage transactiekosten en overdraagbaarheid. Als we dat voor elkaar
krijgen {[}\ldots{]} dan zullen de banken verworden tot de verouderde
{[}dinosaurussen{]} die ze verdienen te zijn', voegde Back een paar
dagen later toe in een vervolg-email. `Ik denk dat dit een goede
uitkomst zou zijn, en ik zie dit liever gebeuren dan dat iemand veel
moeite doet om de banken erbij te betrekken.'\footnote{Adam Back,
  `digital cc transactions, digital checks vs real digital cash',
  oorspronkelijk verstuurd naar de Cypherpunk-mailinglijst, May 2, 1997,
  \href{https://cypherpunks.venona.com/date/1997/05/msg00104.html}{online}}

Back geloofde dat, net als elk ander product dat op de vrije markt wordt
verhandeld, de krachten van vraag en aanbod de waarde van digitale
valuta zouden bepalen. Gebruikers zouden het kopen en verkopen voor
traditionele valuta en het accepteren in ruil voor goederen en diensten.
Zodra er een marktwaarde werd vastgesteld, kon deze elektronische valuta
worden gebruikt om met een simpele muisklik waarde over het internet te
verzenden. En wat vooral belangrijk was, je hoefde niet te maken te
hebben met financiële instellingen.

Back voorspelde dat de waarde die mensen zouden hechten aan een
ongedekte digitale valuta uiteindelijk afhing van de eigenschappen
ervan. Een digitale valuta die, net als contant geld, anoniem gebruikt
kon worden, zou naar verwachting meer gewaardeerd worden dan een
vergelijkbare digitale valuta die sporen van de uitgavenpatronen van
gebruikers naliet, bijvoorbeeld. Een digitale valuta die kan blijven
bestaan, ongeacht het faillissement van een specifiek bedrijf, zou
waarschijnlijk meer waard zijn dan een valuta die het risico loopt 's
nachts nutteloos te worden door het offline gaan van een enkele server.
En zo verder.

Back presenteerde uiteindelijk een lijst van zes wenselijke
eigenschappen voor elektronisch geld\footnote{Adam Back, `Re:
  Bypassing.'}:

\begin{enumerate}
\def\labelenumi{\arabic{enumi}.}
\tightlist
\item
  Anoniem (privacy behoudend, zowel ontvanger als betaler blijven
  onbekend)
\item
  Gedistribueerd (om het moeilijk te maken om het af te sluiten)
\item
  Hebben enige ingebouwde schaarste
\item
  Vereist geen vertrouwen in een individu
\item
  Bij voorkeur offline (moeilijk te realiseren met alleen software)
\item
  Herbruikbaar.
\end{enumerate}

Elk systeem dat deze zes eigenschappen kon bieden, zou echte waarde
aantrekken, voorzag Back, en zou zo een echte, ongedekte, digitale
munteenheid worden.

Het klonk misschien simpel, maar de post-doctorale onderzoeker in de
informatica wist dat het dat zeker niet was. Sommige eigenschappen waren
zelfs in de meest algemene informatica context lastig te realiseren,
terwijl het integreren daarvan in een digitaal valutasysteem
hoogstwaarschijnlijk aanzienlijk moeilijker zou zijn. Het succesvol
combineren van alle zes in één systeem zou nog veel moeilijker zijn.

Desalniettemin, in echte Cypherpunk-stijl, praatte Adam Back niet alleen
over elektronisch geld.

Hij schreef computercode\ldots{}

\chapter{Hashcash}\label{hashcash}

Adam Back groeide op in de East Midlands van Engeland en had op
dertienjarige leeftijd genoeg gespaard om de populaire Sinclair ZX81
thuiscomputer te kopen. Dit apparaat, dat voor het eerst uitkwam in
januari 1981, was ongeveer de omvang van een faxmachine, had een
ingebouwd toetsenbord en kon worden aangesloten op de televisie van het
gezin. Hierdoor kon de jongen de 8-bits wonderen van dit apparaat
ontdekken, gewoon in het comfort van zijn eigen woonkamer.

Toen hij na schooltijd met de computer speelde, naast daadwerkelijke
videogames soms gewoon door berekeningen te maken om een bepaald effect
te bereiken dat hij interessant vond, raakte de jonge Adam Back
gefascineerd door de mogelijkheden van de computerwereld.

En hij had er een talent voor, ook. Omdat er geen \emph{native
assembler} beschikbaar was om mens-leesbare broncode om te zetten naar
machine-leesbare binaire code (enen en nullen), bedacht de jongen hoe
hij zelf een assembler kon coderen voor de ZX81. Dit vereiste van hem
dat hij de meest basale interne werking van de computer
\emph{reverse-engineerde}, waarbij hij zelfs binaire codering moest
terugvertalen naar broncode om te begrijpen hoe andere programma's
werkten.

Toen Back op ongeveer zestienjarige leeftijd naar de universiteit ging,
had hij al een geavanceerd begrip van de werking van computers en wist
hij dat hij deze discipline wilde blijven verkennen. Hoewel informatica
op dat moment geen officieel vak was op zijn school, was er wel een
computerlab op de campus. Back maakte dankbaar gebruik van deze
faciliteit en besteedde vaak lange dagen in het lab om zelfstandig meer
geavanceerde programmeertalen te leren.

Na het college ging Back computerwetenschappen studeren aan de
Universiteit van Exeter, gelegen in het zuidwesten van Engeland. Tijdens
zijn studies raakte hij bijzonder geïnteresseerd in gedistribueerd
computergebruik, waarbij meerdere computers via een netwerk samenwerken
aan hetzelfde probleem, meestal door het probleem op te delen in
kleinere delen (\emph{parallell computing}).

Gedreven door zijn verlangen om het curriculum te beheersen, scoorde
Back gedurende zijn drie jaar als bachelorstudent regelmatig het hoogst
in zijn klas. Dit zette hem begin jaren '90 in een positie waardoor hij
zijn interesse diepgaand kon nastreven. Op eenentwintigjarige leeftijd
stond de Universiteit van Exeter Back toe om zijn Master over te slaan,
en accepteerde ze hem direct in een PhD-programma gefocust op
gedistribueerd computergebruik. Gedurende de volgende vier jaar zou hij
verschillende coördinatiemethoden voor netwerkcomputers bestuderen, en
de mogelijke foutmodi ervan.

Hier werd Back geconfronteerd met langlopende uitdagingen in
gedistribueerd computergebruik, zoals het Byzantijnse Generaalsprobleem,
waarmee computerwetenschappers al worstelden sinds de late jaren '70. De
kern van dit probleem was dat coördinatie over meerdere computers lastig
kon zijn op open netwerken, waar kwaadwillende (of zelfs per ongeluk
onbetrouwbare) knooppunten zich konden aansluiten en de pogingen van
eerlijke deelnemers om een consensus te bereiken konden frustreren.

Het klassieke metaforische verhaal, voor het eerst beschreven door
computergoeroe Leslie Lamport, ging over een groep (inderdaad)
Byzantijnse generaals die een stad omsingelden.\footnote{Leslie Lamport,
  Robert Shostak, and Marshall Pease, `The Byzantine Generals Problem',
  ACM Transactions on Programming Languages and Systems: 382--401.}
Terwijl elk van hen de situatie inschatte, zouden ze via boodschappers
communiceren om te coördineren of ze zouden aanvallen of zich
terugtrekken. Hoewel beide opties acceptabel waren, zouden ze allemaal
moeten overeenkomen met de gekozen optie; het zou een ramp zijn als
sommige generaals zouden aanvallen, terwijl anderen besluiten zich terug
te trekken.

Het probleem is echter dat er ook verraderlijke generaals (en/of
boodschappers, afhankelijk van de versie van de allegorie) zijn, die
tegenstrijdige boodschappen (`aanval' of `terugtrekken') aan
verschillende generaals konden sturen, wat tot onenigheid kon leiden.
Omdat directe communicatie tussen verschillende partijen onmogelijk is,
zouden sommigen eerst de ene instructie (`aanval') te zien krijgen,
terwijl anderen aanvankelijk de andere instructie (`terugtrekken')
zouden zien.

De uitdaging was dus om een protocol te ontwerpen dat elke eerlijke
deelnemer onafhankelijk zou kunnen volgen, zodat ze allemaal tot
dezelfde conclusie kwamen. Of dit nu de beslissing was om een stad wel
of niet aan te vallen, of om consensus te bereiken over hoe een gedeelde
computertaak te coördineren.

Dit waren precies het soort uitdagingen waar Adam Back graag zijn
hersenen over kraakte.

\section{De Cypherpunk}\label{de-cypherpunk}

Als PhD-kandidaat kwam Back aan dezelfde universiteit in contact met een
masterstudent die probeerde het RSA-encryptieprotocol te optimaliseren
voor parallele berekening. Dit leek Back een interessant project, en het
lag dicht bij zijn eigen onderzoeksgebied. Hij besloot dus om te helpen.
Dit betekende dat hij zich moest verdiepen in het RSA-algoritme, wat hem
op zijn beurt ook leidde naar het bestuderen van Phil Zimmermann's
nieuwe PGP-project.

Inmiddels had Back ook een interesse in vrije markten ontwikkeld, en had
hij enige sympathie voor anarcho-kapitalisme. De hyperlibertaire
ideologie waarin alle functies van de staat volledig worden vervangen
door op de markt gebaseerde oplossingen leek op de toekomstige
samenleving beschreven in een van zijn favoriete boeken: de
cyberpunk-klassieker \emph{Snow Crash}.

Zo vond Back de cryptografische hulpmiddelen waar hij over leerde op
sociologisch vlak zeer interessant.

`Minder privacy is negatief voor de vrije markteconomie, omdat een
afname van privacy leidt tot een toename van overheidsinterventies,
grotere overheidsapparaten, fascisme, etc. waardoor de vrije
markteconomie naar de knoppen gaat', zou hij later verklaren. `Hierdoor
zullen armoede en voedseltekorten ontstaan, vergelijkbaar met wat er
gebeurde in de voormalige USSR, die nog steeds langzaam herstelt van de
ondergang die veroorzaakt werd door fascistisch
overheidsbeleid.'\footnote{Adam Back, `Re: 'why privacy' revisited',
  oorspronkelijk verstuurd naar de Cypherpunk-mailinglijst, 22 maart
  1997,
  \href{https://cypherpunks.venona.com/date/1997/03/msg00586.html}{online}}

Daarnaast realiseerde de student zich snel dat dit soort technologieën
individuen de middelen konden bieden om fundamentele rechten uit te
oefenen zoals vrijheid van meningsuiting en vrijheid van vergadering. In
wezen ontdekte Back dat cryptografie aanzienlijke implicaties kon hebben
voor het evenwicht van macht tussen individuen en de staat.

Toen hij op het internet naar plaatsen zocht om deze onderwerpen te
bespreken, ontdekte de jonge PhD-kandidaat uit Exeter dat hij niet de
enige was die tot dit inzicht was gekomen.

Halverwege de wereld hadden de Cypherpunks net hun reguliere
bijeenkomsten georganiseerd en, nog belangrijker, de
Cypherpunk-mailinglijst gelanceerd. Back, al programmeur van in zijn
kindertijd, was erg geïnspireerd door hun doel om software te schrijven
die een positieve sociale impact had en hij meldde zich snel aan voor de
mailinglijst.

`Dat was een interessante uitlaatklep', herinnerde Back zich later aan
zijn eerste interacties met de Cypherpunks-mailinglijst, `want mensen
waren bezig met andere dingen dan PGP, andere dingen die je kon bouwen
met encryptie en cryptografie. Ik heb een groot deel van mijn doctoraat
eigenlijk niet besteed aan werken aan gedistribueerde systemen, maar aan
leren over cryptografische protocollen, voornamelijk met als toegepaste
interesse om te denken over wat een bepaald cryptografie-artikel je zou
toestaan om te bouwen.'\footnote{Adam Back, `The Bitcoin Game \#59:
  Dr.~Adam Back', interview door Rob Mitchell, The Bitcoin Game,
  YouTube, 25 oktober 2018,
  \href{https://www.youtube.com/watch?v=xxYsRjanphA&t=647s}{online}
  10:47--11:19.}

In de loop der jaren werd Back een van de meest actieve deelnemers aan
deze lijst. Soms droeg hij zelfs tientallen e-mails bij in een maand.
Hij had een sterke interesse in filosofische onderwerpen zoals privacy,
vrije meningsuiting en libertarisme, en nam deel aan diepgaande
technische discussies over onderwerpen zoals anonieme remailers of
versleutelde bestandssystemen, technologieën waaraan hij ook heeft
bijgedragen.

Back raakte ook betrokken bij de `crypto-oorlogen' die niet al te lang
na zijn toetreding tot de mailinglijst uitbraken, en werd op veel
manieren zeer direct beïnvloed door deze strijd. Toen de Amerikaanse
overheid cryptografie reguleerde onder de US Munitions List, was het
Amerikanen wettelijk niet langer toegestaan om, bijvoorbeeld, het
RSA-algoritme te delen met Back, of met een van zijn landgenoten (in het
geval van RSA kende Back natuurlijk al het algoritme).

Het verbod raakte een gevoelige snaar bij de jonge Cypherpunk. Hij was
van mening dat de betwiste cryptografische protocollen in werkelijkheid
individuen alleen maar toestonden om rechten uit te oefenen die zij
juridisch gezien al moesten hebben: als privégesprekken zijn toegestaan,
waarom zou publieke sleutel-cryptografie dan niet toegestaan zijn? En
wellicht nog belangrijker, cryptografie was in feite gewoon wiskunde.
Back vond het zowel absurd als zorgwekkend dat de VS in feite het delen
van bepaalde getallen en vergelijkingen illegaal maakten.

Het leidde de Britse Cypherpunk ertoe om op een unieke manier zijn punt
te bewijzen. Volgens de activistische ethos van de groep, maakte Back
`munitie'-shirts: zwarte t-shirts met in witte letters het RSA-protocol
erop gedrukt. Volgens de wet was iedereen die Back's kleding droeg bij
het verlaten van de Verenigde Staten, technisch gezien, een
wapenexporteur. Hij zou de shirts verkopen aan zijn mede-Cypherpunks en,
passend, DigiCash's proefvaluta CyberBucks als betaling accepteren.

Passend genoeg, want mogelijk meer dan wat dan ook, was Adam Back
bijzonder geïnteresseerd in elektronisch geld.

\section{In actie komen}\label{in-actie-komen}

Toen Adam Back zich bij de mailinglijst van de Cypherpunks aansloot, was
DigiCash werkelijk onbetwistbaar op gebied van anonieme digitale valuta.
Maar de voortgang van eCash verliep minder snel dan hij en vele anderen
hadden gewenst.

Back deelde de beoordeling van de Cypherpunks dat dit grotendeels kwam
omdat David Chaum zijn patenten gebruikte om zijn technologie exclusief
te houden. Hij vond dat het beleid van DigiCash, met zijn ingewikkelde
licentieschema's en beperkingen op wat gebruikers wel en niet met de
eCash-technologie konden doen, ervoor zorgde dat hackers en knutselaars
geen kans kregen om met de technologie te experimenteren en deze te
verbeteren. Door dit beleid was de technische vooruitgang vrijwel tot
stilstand gekomen.

Back verpersoonlijkte de Cypherpunk-filosofie en was niet het type dat
rustig zou afwachten tot de zaken veranderden.

In de hoop dat het de zaken zou versnellen, schreef Back initieel
software-\emph{libraries} (broncode die door andere ontwikkelaars kon
worden gebruikt) voor zowel eCash als Brands Cash. Laatstgenoemde is een
op eCash geïnspireerd elektronisch geldsysteem ontworpen door voormalig
DigiCash-medewerker Stefan Brands. Terwijl hij hiermee bezig was,
ontdekte Back ook hoe hij Brands's systeem kon uitbreiden om offline
transacties te vergemakkelijken, zonder de noodzaak om bij elke betaling
bij de bank te controleren op dubbele uitgaven (hoewel toen Back zijn
oplossing met Brands besprak, bleek dat iemand anders dat probleem al
eerder had opgelost).

Toen eCash te midden de jaren 90 nog steeds niet echt van de grond kwam,
begon Adam Back pas echt ongeduldig te worden.

`Technologie voor blinde handtekeningen bestaat al geruime tijd, maar er
is niet één voorbeeld van een praktisch, echt wereldwijd gebruik van
deze technologie', schreef Back gefrustreerd naar de
Cypherpunk-mailinglijst in oktober 1995.\footnote{Adam Back, `Re:
  cypherpunks digicash bank?', oorspronkelijk verstuurd naar de
  Cypherpunk-mailinglijst, 8 oktober 1995,
  \href{https://cypherpunks.venona.com/date/1995/10/msg00734.html}{online}}

Hij merkte op dat niet-anonieme internetbetaalsystemen snel marktaandeel
veroverden, wat betekende dat de toekomst van digitale transacties een
gevaarlijke richting insloeg. Als deze privacy-schendende alternatieven
zich eenmaal zouden verankeren in de gewoonten van mensen als hun
standaard online betalingsoplossingen, geloofde hij dat het aanzienlijk
moeilijker zou zijn om internetgebruikers over te laten stappen naar
anoniem digitaal geld.

Het was tijd om actie te ondernemen. Aangezien Back niet veel vertrouwen
had in de voortgang van DigiCash, en het leek alsof niemand anders aan
haalbare alternatieven voor eCash werkte, concludeerde hij dat de
snelste weg naar succes misschien wel was om samen te werken met Chaum.
Hij stelde voor dat een groep Cypherpunks een start-up bank zou
oprichten en eigenlijk de technologie van DigiCash zou licentiëren om
zelf door fiatgeld gedekte eCash uit te geven.

`Ik meen het serieus en zou erin willen investeren', maakte Back
duidelijk op de mailinglijst. `Wat denken jullie ervan? De eerste
DigiCash-bank, 'gerund' en eigendom van een groep
cypherpunks?'\footnote{Adam Back, `cypherpunks digicash bank?',
  oorspronkelijk verstuurd naar de Cypherpunk-mailinglijst, 7 oktober
  1995,
  \href{https://cypherpunks.venona.com/date/1995/10/msg00690.html}{online}}

De enige persoon die reageerde op zijn voorstel wees erop dat dit plan
waarschijnlijk ook niet zou werken: tenslotte wilde Chaum zijn
technologie niet in licentie geven aan een bende Cypherpunks.\footnote{Barnes,
  `cypherpunks digicash.'} Het idee stierf een stille dood.

Twee jaar later, in de zomer van 1997, keerde Back terug naar de
mailinglijst om het idee van een gedistribueerde bank voor te stellen.
Nu hij met succes zijn doctoraatsprogramma had afgerond en als
postdoc-onderzoeker aan de universiteit werkte, maakte hij deze keer
echt gebruik van zijn expertisegebied. Hij legde uit dat de operaties
van een bank over een netwerk van verschillende mensen of entiteiten
verspreid konden worden. Ze zouden in staat zijn om een virtuele
computer te simuleren door berichten uit te wisselen en gecodeerde
functies te berekenen, opperde Back, waarbij de virtuele computer in
feite zou functioneren als een reguliere bank.

Door de bank op te splitsen in meerdere, onderling afhankelijke,
partijen, zou er geen enkele entiteit volledig vertrouwd hoeven te
worden met de bankoperaties. En hoewel het Byzantijnse Generaalsprobleem
nog steeds niet volledig was opgelost, kon het systeem alleen worden
misleid als een bepaald aantal deelnemers samenspanden.

`De bank zou bestaan binnen het netwerk, in deze virtuele CPU', schreef
Back. `Individuele deelnemers kunnen komen en gaan, maar de beveiligde
software-entiteit, die de bank en de accountinformatie is, zou blijven
voortbestaan.'\footnote{Adam Back, `distributed virtual bank',
  oorspronkelijk verstuurd naar de Cypherpunk-mailinglijst, 27 augustus
  1997,
  \href{https://cypherpunks.venona.com/date/1997/08/msg01289.html}{online}}

Hij kreeg geen antwoord op zijn e-mail.

\section{Spam}\label{spam}

In de tweede helft van de jaren '90 moest elke e-mailservice op het
internet omgaan met ogenschijnlijk steeds toenemende hoeveelheden
ongewenste mail, of \emph{spam}: ongevraagde berichten die in bulk
werden verzonden, doorgaans door adverteerders. De Cypherpunks werden
hierbij niet gespaard; promoties voor afslankpillen, producten voor
penisvergroting en aanbiedingen om snel rijk mee te worden veroorzaakten
alsmaar meer vervuiling op de mailinglijst.

Het probleem was vooral ernstig voor de remailers. De anoniem makende
diensten die door verschillende Cypherpunks werden beheerd, werden
gemakkelijk en vaak misbruikt om spam te versturen. Sommige van de
beheerders vermoedden zelfs dat hun remailers specifiek het doelwit
waren van zogenoemde \emph{denial-of-service} (DOS) aanvallen; het doel
zou zijn geweest om de remailers te overladen met nutteloze e-mails om
hun dienst onbruikbaar te maken voor legitiem gebruik.

Adam Back, die zelf ook een remailer beheerde, was lid van een groep
Cypherpunks die zich toelegde op het oplossen van dit probleem. En nog
belangrijker, ze probeerden dit probleem op te lossen zonder terug te
vallen op oplossingen die internetgebruikers verplichtten zich te
identificeren. Want voor remailers was privacy het belangrijkste punt.

Ze wilden ook niet afhankelijk zijn van wetten en regelgeving. Hoewel er
stemmen opgingen om eenvoudigweg spam e-mails illegaal te maken, was het
niet de manier van de cypherpunks om de overheid in te schakelen om hun
problemen op te lossen.

De overheid het probleem niet laten oplossen, was ook belangrijk omdat
het niet altijd duidelijk was wat precies als spam beschouwd kon worden.
Als de staat de autoriteit krijgt om dat onderscheid te maken, zou dat
effectief regeringen toestaan om te bepalen welke vormen van
communicatie tussen internetgebruikers acceptabel zijn en welke niet.
Dit zou de deur kunnen openen voor politiek gemotiveerde censuur,
waarschuwden Cypherpunks zoals Tim May.\footnote{Tim May, `Re: More on
  digital postage', oorspronkelijk verstuurd naar de
  Cypherpunk-mailinglijst, 15 februari 1997,
  \href{https://cypherpunks.venona.com/date/1997/02/msg02295.h}{online}}

Adam Back benadrukte dat om spammers wettelijk verantwoordelijk te
kunnen houden, ze ook geïdentificeerd dienen te worden. Als een
overheidsdienst tegen spam de taak krijgt om de daders te vangen,
waarschuwde Back zijn mede-Cypherpunks, dan zouden remailers
waarschijnlijk een groot doelwit worden, met mogelijk ernstige gevolgen
voor de online privacy in het algemeen.\footnote{Adam Back, `Re: bulk
  postage fine', oorspronkelijk verstuurd naar de
  Cypherpunk-mailinglijst, 3 augustus 1997,
  \href{https://cypherpunks.venona.com/date/1997/08/msg00070.html}{online}}

`De gevaren het inzetten van de overheid om spammers aan te vallen, is
dat dit het internet is, en we willen geen regulering van de inhoud door
overheden, noch pogingen tot afdwingen van 'escrow van identiteit',
`internetrijbewijzen' of iets anders', schreef Back aan de
Cypherpunk-mailinglijst. `We lossen het zelf wel op zonder de behoefte
aan overheidsinterventie, hartelijk dank.'\footnote{Adam Back, `no
  government regulation of the net', oorspronkelijk verstuurd naar de
  Cypherpunk-mailinglijst, 3 augustus 1997,
  \href{https://cypherpunks.venona.com/date/1997/08/msg00087.html}{online}}

De cypherpunks waren het er in het algemeen over eens dat een betere
oplossing zou zijn om een digitaal equivalent van postzegels te creëren.
Als het verzenden van een email kosten met zich mee zou brengen, zou dit
spammers ontmoedigen, vooral omdat spammers meestal tienduizenden, zo
niet miljoenen, ongewenste e-mails moeten versturen om winst te maken
uit hun activiteiten. Op dezelfde manier vereiste het bombarderen van
remailers massa's ongewenste e-mails.

Digitale postzegels konden verschillende manieren functioneren.
Remailers konden bijvoorbeeld postzegels in rekening brengen voor het
doorsturen van e-mails. Dit zou niet alleen spam ontmoedigen die via
deze diensten worden verzonden, maar het zou ook, als leuk extraatje,
een financiële prikkel introduceren om remailers te runnen. Als
alternatief, of daarnaast, zou een vergoeding kunnen worden toegekend
aan de ontvanger wiens e-mailsoftware geprogrammeerd kan worden om
berichten met onvoldoende vergoeding af te wijzen.

De eigenlijke postzegel kon op verschillende manieren worden ontworpen,
maar de meeste ideeën hielden een uitgever van elektronische zegels in.
Deze uitgever zou bijvoorbeeld unieke grote getallen kunnen genereren en
deze kunnen verkopen (wellicht voor digitaal geld). Het unieke nummer
zou dan in een e-mail moeten worden opgenomen, en zodra een remailer of
het e-mailprogramma van de ontvanger het bericht ontvangt, zou het bij
de uitgever controleren of het nummer inderdaad een geldige postzegel
is. Als dat zo is, kan de remailer of ontvanger misschien wat geld
(digitaal geld?) terugkrijgen van de uitgever, afhankelijk van de
specifieke ontwerpkenmerken van het systeem. Als de postzegel niet
geldig was, zou de e-mail simpelweg geweigerd worden.

Een andere optie zou zijn om gewoon digitale contanten direct als
postzegel te gebruiken. Een e-mail kan in dit geval een heel klein
beetje digitale valuta bevatten in een speciaal veld, dat de beoogde
ontvanger er dan kan uithalen. Er was een tijd waarin het leek alsof dit
een goede toepassing voor eCash zou kunnen zijn.

Helaas bleek het in de praktijk ingewikkelder te zijn. Ten eerste waren
de vroege eCash versies nog niet in staat om zeer specifieke soorten
overboekingen die nodig waren voor postzegels te verwerken. Bovendien
zou het gebrek aan anonimiteit van de ontvanger in het elektronische
contantsysteem van Chaum betekenen dat de verzender van een bericht kon
worden gedwongen om samen te werken met de bank om de echte identiteit
van een remailer-operator en/of de e-mailontvanger te onthullen.

Het maakte DigiCash's eCash grotendeels ongeschikt voor postzegels.

\section{De eerste elektronische
postzegel}\label{de-eerste-elektronische-postzegel}

Wat Adam Back niet wist was dat het postzegelprobleem al een paar jaar
eerder was opgelost, en op een heel andere manier dan de Cypherpunks
hadden overwogen.

In de vroege jaren '90 realiseerden Cynthia Dwork en Moni Naor, twee
computerwetenschappers bij IBM, zich dat een elektronisch mailsysteem
veel voordelen had ten opzichte van de traditionele post: e-mail was
veel sneller, veel goedkoper, en bood veel meer mogelijkheden dan de
postdienst ooit zou kunnen bieden. Maar ze realiseerden zich ook dat
e-mail zijn eigen uitdagingen met zich meebracht. Ze voorzagen dat met
de toenemende populariteit van e-mail, ook spam zou toenemen.

`Met name de eenvoud en lage kosten om elektronische post te versturen,
en dan vooral het gemak om eenzelfde bericht naar veel verschillende
partijen te sturen, nodigen in principe uit tot misbruik', legden ze uit
in hun paper uit 1992, getiteld `Pricing via Processing of Combatting
Junk Mail.'\footnote{Cynthia Dwork and Moni Naor, `Pricing via
  Processing or Combatting Junk Mail', Advances in Cryptology ---
  \emph{Crypto} '92: 139--147.}.

Er was een oplossing nodig, stelden ze vast, en dat was precies wat het
artikel leverde. Dwork en Naor stelden een systeem voor waarbij
afzenders een beetje extra data aan hun e-mail moeten toevoegen. Deze
data zou de oplossing zijn voor een wiskundig probleem, afgeleid van de
eigenschappen van de e-mail zelf, en dus uniek voor die e-mail. Zonder
de correcte oplossing toegevoegd, zouden e-mail\emph{clients} de e-mail
volledig moeten afwijzen.

In hun artikel stelden Dwork en Naor drie mogelijke puzzels voor, allen
gebaseerd op cryptografische algoritmen zoals handtekeningsschema's. In
alle gevallen zou het toevoegen van de juiste oplossing aan een e-mail
niet al te moeilijk zijn (voor een computer): het zou wellicht een paar
seconden aan rekenkracht vereisen. Toch zou dit een kleine kostenpost
betekenen. Controleren of de oplossing correct is, zou daarentegen zeer
makkelijk zijn en nauwelijks enige rekenkracht kosten.

De kerngedachte achter dit systeem was dat het oplossen van een juiste
oplossing voor een puzzel niet veel moeite zou kosten voor individuele
gebruikers die af en toe een e-mail willen sturen naar collega's,
familie of vrienden. Maar voor spammers zou het snel oplopen. Om
duizenden of zelfs miljoenen berichten op een enkele dag te sturen,
zouden zij evenveel puzzels moeten oplossen, wat in totaal aanzienlijke
hoeveelheden rekenkracht zou vereisen.

`Het idee is om een gebruiker een matig moeilijke, maar niet
onoplosbare, functie te laten berekenen om toegang te krijgen tot de
hulpbron, en zo lichtzinnig gebruik te voorkomen', legden Dwork en Naor
uit. Ze stelden voor om spammen te veranderen in een dure
aangelegenheid.

Hoewel Dwork en Naor de term zelf niet hebben voorgesteld, zou de soort
oplossing die zij introduceerden later bekend komen te staan als een
`proof-of-work'-systeem. Het oplossen van het wiskundeprobleem zou
bewijzen dat er daadwerkelijk \emph{werk} is verricht.

Het was een handige oplossing. Maar of het nu kwam omdat het idee iets
te vooruitstrevend was voor die tijd, of omdat het simpelweg niet breed
genoeg werd geadverteerd, kreeg het voorstel uit de vroege jaren 90
buiten een vrij kleine groep academici niet veel aandacht. Het
postzegelsysteem van Dwork en Naor is nooit in de praktijk gebracht,
laat staan gebruikt, en veel van de Cypherpunks waren waarschijnlijk
onbekend met dit idee.

Gelukkig zou het concept spoedig opnieuw worden uitgevonden.

\section{Hashcash}\label{hashcash-1}

\emph{{[}MEDEDELING{]} implementatie van hash cash porto}

Op 28 maart 1997 kregen de inmiddels ongeveer 2000 abonnees van de
Cypherpunk-mailinglijst een bericht in hun inbox.\footnote{Adam Back,
  `ANNOUNCE hash cash postage implementation', oorspronkelijk verstuurd
  naar de Cypherpunk-mailinglijst, 28 maart 1997,
  \href{https://cypherpunks.venona.com/date/1997/03/msg00774.html}{online}}
Het bericht werd verstuurd door Adam Back en bevatte een beschrijving en
vroege implementatie van wat hij `hashcash' noemde: een
`postzegelsysteem gebaseerd op gedeeltelijke hash-botsingen'. Hij had
met succes een postzegel-oplossing voor e-mail ontworpen.

Net als het schema van Dwork en Naor, zou de Hashcash porto niet betaald
worden aan operators van remailers, of aan de ontvangers van een e-mail,
of aan wie dan ook. In plaats daarvan zou het alleen extra kosten met
zich meebrengen voor de verzenders.

Een week eerder had Back al nagedacht over het tegengaan van spam door
de kosten van massa-mail te verhogen, hoewel nog vrij oppervlakkig: `Een
nevenvoordeel van het gebruik van PGP is dat PGP-versleuteling enige
meerkost zou betekenen voor de spammer: hij kan waarschijnlijk minder
berichten per seconde versleutelen dan hij kan spammen via een T3-link',
merkte hij op in een discussie over het toevoegen van extra privacy aan
remailers.\footnote{Adam Back, `Re: Remailer problem solution?' 23 maart
  1997,
  \href{https://cypherpunks.venona.com/date/1997/03/msg00631.html}{online}}

Back's nieuwe voorstel maakte dit algemene idee nog explicieter: het zou
vereisen dat afzenders een bewijs van geleverd werk aan hun e-mails
toevoegen. Inderdaad, hashcash was op verschillende manieren
vergelijkbaar met het postzegelschema van Dwork en Naor: het
\emph{proof-of-work} zou uniek zijn voor de e-mail, en het zou een
beetje rekenkracht vereisen om te produceren.

Zoals de naam al doet vermoeden, was het voorstel van Back echter
gebaseerd op hashing.

Hashing -- de cryptografische truc die alle data omzet in een unieke en
schijnbaar willekeurige reeks getallen van een specifieke lengte -- is
een volkomen onvoorspelbaar proces. Hoewel dezelfde data altijd
hetzelfde hashresultaat oplevert, is de enige manier om te ontdekken hoe
de hash van een stuk data er in de eerste plaats uit zal zien, om het
daadwerkelijk te hashen. Het is deze onvoorspelbaarheid waar hashcash op
slimme wijze gebruik van maakte.

Om `hashcash' te genereren, moest een gebruiker een hash creëren uit de
metadata van een e-mail (zoals het adres van de verzender, het adres van
de ontvanger, de tijd enz.) en een willekeurig nummer, een zogenaamde
`nonce'. Maar hier zat een addertje onder het gras: niet elke
resulterende hash werd als `geldig' beschouwd. In plaats daarvan moest
de binaire versie van een geldige hash beginnen met een vastgesteld
aantal nullen. En er was maar één manier om een hash te genereren die
met voldoende nullen begon: de gebruiker moest verschillende nonces
proberen en nieuwe hashes maken totdat hij er één vond die toevallig aan
de norm voldeed. Simpel \emph{trial and error}.

Het aantal voorafgaande nullen dat vereist is, bepaalt hoe moeilijk het
zou zijn om een geldige hash te vinden. Meer nullen zouden het
moeilijker maken, omdat computers gemiddeld meer pogingen zouden moeten
doen.

`Het idee van het gebruik van partiële hashes is dat ze naar willekeur
duur in berekening kunnen worden gemaakt', legde Back het voordeel van
het gebruik van hashes uit, `en toch kunnen ze onmiddellijk geverifieerd
worden.'

Net als bij de oplossing van Dwork en Naor was het idee dat gewone
gebruikers vrij snel, binnen enkele seconden, een geldige hash moeten
kunnen vinden om met een e-mail te versturen. Tegelijkertijd zouden
spammers niet in staat moeten zijn om dit duizenden of miljoenen keren
te doen en nog steeds winstgevend te blijven.

`{[}\ldots{]} als het geen 20 bit hash heeft {[}\ldots{]} heb je een
programma dat het terugstuurt met een bericht waarin de vereiste porto
wordt uitgelegd, en waar de software te vinden is', legde Back uit op de
Cypherpunk-mailinglijst. `Dit zou spammers van de ene op de andere dag
wegwerken, aangezien 1.000.000 x 20 = 100 MIP jaren, wat meer
rekenkracht is dan ze hebben.'

Een subtiele verandering ten opzichte van Dwork en Naor's oplossing was
dat hun `proof-of-work' systeem niet onderhevig was aan toeval. Hun
postzegelschema vereist in principe het oplossen van een vrij eenvoudige
puzzel, wat betekent dat een krachtigere computer de puzzel altijd
sneller zou oplossen dan een zwakkere computer.

Het genereren van een geldige hash, daarentegen, is een kwestie van
gokken. Hoewel een krachtigere computer meer gokjes per seconde zou
kunnen doen, zou een zwakkere computer nog steeds af en toe geluk kunnen
hebben en sneller een geldige hash vinden (hoewel spammers hoe dan ook
duizenden of miljoenen geldige hashwaarden zouden moeten genereren per
spamsessie, zou het kleine beetje variatie bij het genereren van een
enkele hashcash proof-of-work niet echt een verschil maken, althans niet
binnen de context van het stoppen van ongewenste mail).

`Hashcash kan een tussentijdse oplossing bieden totdat digicash breder
wordt toegepast', concludeerde Back in zijn aankondiging. `Hashcash is
gratis, het enige dat je hoeft te doen is wat processorkracht van je PC
gebruiken. Het sluit aan bij de online cultuur van vrije communicatie,
waarin de financieel minder bedeelden op gelijke voet kunnen concurreren
met miljonairs, gepensioneerde overheidsfunctionarissen en dergelijke.'

En: `Hashcash kan ons een alternatieve methode bieden voor het beheersen
van spam als digicash misloopt (verboden wordt of vereist dat
gebruikersidentiteiten in escrow worden gehouden).'

\section{Digitale schaarste}\label{digitale-schaarste}

Hashcash werd in de jaren na de aankondiging van Adam Back enigszins
geadopteerd. Apache's open source SpamAssassin platform implementeerde
de oplossing, terwijl Microsoft het idee in zijn incompatibele Postmark
systeem uitprobeerde. Adam Back, samen met enkele andere academici,
kwamen ondertussen op alternatieve toepassingen voor het proof-of-work
systeem, waaronder oplossingen tegen DoS-aanvallen.

Maar de hashcash postzegel is nooit echt mainstream gegaan. De
innovatieve aard van de oplossing was waarschijnlijk niet voldoende om
de opstarthindernis te overwinnen: iemand kon niet echt beginnen met het
eisen van hashcash voor binnenkomende e-mails als niemand anders het
gebruikte, omdat dit zou leiden tot afwijzing van alle binnenkomende
berichten door hun e-mailclient. Tegelijkertijd was er geen aanmoediging
om hashcash te gebruiken voor uitgaande e-mails als niemand het eiste.
Net als David Chaum's eCash, leed ook het elektronische porto-systeem
van Back aan een kip-en-ei-probleem, waarvoor geen gemakkelijke
oplossing leek te zijn.

Adam Back maakte zich hier echter niet zoveel zorgen om. Back, die
inmiddels onderzoeker was bij de Universiteit van Exeter en aan een
gecodeerd berichtensysteem voor medische data werkte, dacht vrijwel
onmiddellijk na de publicatie van zijn voorstel al verder dan enkel
hashcash. Het oplossen van het postzegelprobleem was een goede
uitdaging, maar de computeringenieur was vooral gefascineerd door de
gedachte om digitaal geld voor algemene doeleinden te creëren.

En hoewel veel Cypherpunks nog steeds aannamen dat een elektronisch
geldsysteem zou moeten worden geïntegreerd in de bestaande financiële
infrastructuur, zoals eCash, had Back een andere visie. Hij geloofde dat
hashcash een geheel nieuwe richting van onderzoek kon bieden om die
visie te verwezenlijken.

De sleutelinnovatie van hashcash was dat het puur digitale data (in
wezen getallen) aan echte hulpbronnen in de fysieke realiteit koppelde.
Het creëren van een proof-of-work vereiste rekenkracht, en deze
rekenkracht zelf verbruikt elektriciteit, die op zijn beurt energie kost
om te produceren. Terwijl de meeste digitale dingen kosteloos en bijna
eindeloos gekopieerd kunnen worden, kon het proof-of-work op een
bepaalde manier de fundamentele schaarste aan energie in de fysieke
realiteit naar de digitale wereld overbruggen.

Inderdaad, hashcash was digitaal, maar toch schaars. Het totale aantal
hashcash `valuta-eenheden' (bij gebrek aan een beter woord) was in
zekere mate beperkt: er zou nooit meer hashcash zijn dan er zou kunnen
worden geproduceerd met de hoeveelheid energie die mensen bereid en in
staat zouden zijn om eraan te spenderen.

Dit was een cruciaal inzicht, omdat ingebouwde schaarste een van de zes
eigenschappen was die Adam Back op zijn lijstje voor een ideaal
elektronisch geldsysteem had gezet. Tot voor kort kon digitale schaarste
slechts gecreëerd worden als een belofte, zoals DigiCash's belofte om
een harde limiet te stellen aan het totale aantal CyberBucks dat ze
zouden creëren. Maar beloftes kunnen natuurlijk gebroken worden. Back
geloofde dat proof-of-work de mogelijkheid bood om schaarste te
garanderen op een veel fundamenteler niveau.

Tegelijkertijd wist hij dat hashcash niet kon functioneren als
volwaardig digitaal geld. Hoewel het anoniem kon worden gebruikt,
moeilijk te stoppen was, geen vertrouwen in een individu vereiste en ook
enige mate van schaarste had, voldeed dit eigenlijk slechts aan drie van
de zes criteria op de shortlist van Back.

Het grootste probleem was dat hashcash niet herbruikbaar was. Elke
munteenheid was op maat gemaakt om bij een specifieke email te passen,
en kon dus niet elders opnieuw worden uitgegeven en leverde geen extra
voordeel op voor de ontvangers.

Back overwoog daarom dat hashcash, of meer algemeen het proof-of-work
principe, de basis zou kunnen vormen voor een ander soort elektronisch
geldsysteem.

Eén van zijn eerste suggesties was een \emph{Chaumiaans} systeem. Hierin
zou de bank elektronisch geld uitgeven bij ontvangst van hashcash:
gebruikers zouden proof-of-work creëren en daarvoor ongedekt digitaal
geld terugkrijgen. Dit geld zou anoniem, herbruikbaar en enigszins
schaars zijn --- hoewel de schaarste in de praktijk eerlijk gezegd vrij
zwak zou zijn, omdat mensen altijd meer bewijzen zouden kunnen creëren
als ze dat zouden willen. En met computerprocessors die elk jaar
krachtiger worden, zou het produceren van geldig proof-of-work in de
loop van de tijd steeds goedkoper worden.\footnote{Back, `Re:
  Bypassing.'}

Inderdaad, de fundamentele schaarste van hashcash was meer technisch van
aard. Als het echt de basis zou vormen voor een elektronische valuta,
zouden nieuwere en krachtigere computers uiteindelijk de markt
overspoelen met verse valuta-eenheden en zou de valuta hyperinflateren.

Bovendien zouden gebruikers de bank moeten vertrouwen dat deze geen geld
uit het niets creëert. Net als elk elektronisch geldsysteem dat Chaum's
ontwerp volgt, zou de entiteit die de digitale valuta uitgeeft en
dubbele uitgaven voorkomt, \emph{zelf} ook vertrouwd moeten worden om
niet onterecht zichzelf te verrijken.

Back geloofde, echter, dat vrije marktcompetitie mogelijk kon helpen dit
probleem op te lossen.

`Wellicht zou je dus meerdere banken kunnen hebben en reputatie het werk
laten doen, als je de protocollen zo kunt regelen dat het duidelijk zou
zijn of een bank meer geld aan het drukken was dan het aan
hash-botsingen had ontvangen. {[}\ldots{]} Maar als je meerdere banken
hebt dan moet je een uitwisselingsmechanisme hebben. De markt zou hier
waarschijnlijk zorg voor kunnen dragen, door wisselkoersen te bepalen op
basis van de reputaties van banken', suggereerde hij, in wat nu heel erg
leek op een vrijbankiersysteem zoals beschreven door Friedrich Hayek.

Desalniettemin geloofde de jonge Cypherpunk uit Exeter dat ze het
misschien zelfs nog beter konden doen dan dat:

`Het zou beter zijn om iets te hebben dat geen vertrouwen vereist en dat
geen mogelijkheid heeft voor bedrog, in plaats van te vertrouwen op
reputatie om ze te sorteren.'\footnote{Back, `Re: Bypassing.'}

\chapter{Bit Gold}\label{bit-gold}

De vader van Nick Szabo was een van de 200.000 Hongaren die hun land
verlieten nadat Sovjettroepen de anti-communistische opstand van 1956
hadden neergeslagen. Terwijl tienduizenden van zijn
mede-vrijheidsstrijders werden opgesloten of geïnterneerd, en in sommige
gevallen zelfs geëxecuteerd, besloot hij alles achter te laten.
Uiteindelijk vond hij een nieuwe thuis aan de andere kant van de
Atlantische Oceaan, in \emph{het land van de vrijheid}.

Hoewel dit bijna een decennium voor zijn geboorte gebeurde, zou de
ervaring van zijn vader Nick vormen. Als zoon van een vluchteling,
opgroeiend ver weg van het onderdrukkende Sovjetgezag over Oost-Europa,
werd bij de Amerikaanse jongen van de tweede generatie al vroeg een diep
wantrouwen tegen alles wat ook maar enigszins leek op communisme of
overheidsinmenging ingeplant.\footnote{Nick Szabo, `Why Cryptocurrency?
  Governments Abuse Their Power --- Nick Szabo Interview Part 1',
  interview by Zulu Republic, Zulu Republic, YouTube, 25 oktober 2018,
  \href{https://www.youtube.com/watch?v=LZw4LNLYUgc}{online}}

Uiteindelijk vond Szabo een grote bron van inspiratie in de werken van
Friedrich Hayek. Het boek \emph{The Road to Serfdom} leek de
transformatie van de Sovjet-Unie naar een totalitaire staat nauwkeurig
te hebben beschreven, met de onderdrukking van de Hongaarse revolutie
als een van de eerste grote voorbeelden hiervan buiten het Russische
thuisland. Szabo zou later Hayek's boek uit 1988, \emph{The Fatal
Conceit} --- een andere, meer politieke, weerlegging van het socialisme,
dat de historische belangrijkheid van privé-eigendom benadrukte ---
noemen als een van de belangrijkste boeken die hij ooit heeft
gelezen.\footnote{Nick Szabo , `Some of the most important books I've
  read', \emph{X}, 31 januari 2016,
  \href{https://x.com/NickSzabo4/status/693682157525401601}{online}}

Ondertussen raakte de jonge Szabo geïnteresseerd in computers, een
nieuwe technologie die in die tijd snel evolueerde in de VS en de rest
van de Westerse wereld, terwijl het communistische Oosten ver
achterbleef. Toen in de late jaren '70 en vroege jaren '80 de eerste
computers hun weg vonden naar Amerikaanse huizen en kantoren (zijn
moeder nam regelmatig haar Apple II mee naar huis van haar werk) zag hij
al snel de mogelijkheden van deze machines. Tegen het midden van de
jaren '80 bracht dit hem ertoe om informatica te gaan studeren aan de
Universiteit van Washington in Seattle.

Tijdens zijn studie, deed Szabo een jaar stage bij het \emph{Jet
Propulsion Laboratory} (JPL), een onderzoekscentrum van NASA, voordat
hij in 1989 zijn diploma in de informatica behaalde. Nu midden in zijn
twintiger jaren, besloot hij zo'n 1300 kilometer naar het zuiden te
verhuizen, naar de San Francisco Bay Area, waar zijn vaardigheden
bijzonder gewild waren. Hij vond een programmeerbaan bij IBM, dat
gedurende de jaren '80 met zijn relatief goedkope microcomputers de
standaard had gezet voor thuiscomputers.

Szabo's persoonlijke interesses waren echter altijd breder dan alleen
informatica. Als een echte veelweter, genoot hij ervan om in zijn vrije
tijd een scala aan onderwerpen te bestuderen: van politiek tot biologie
en van geschiedenis tot economie, met een speciale focus op vrije
markteconomie. Net als Hayek, die vooral in deze latere fase van zijn
carrière economische concepten gebruikte om hardnekkige politieke
realiteiten uit te leggen, waardeerde Szabo hoe het combineren van
multi-disciplinaire kennis hem kon helpen nieuwe inzichten op te doen.
Hij hield ervan deze inzichten te benutten om te speculeren over de
toekomst van technologie, samenleving en mensheid.

Dit maakte hem perfect passen bij de Extropianen.

De techno-utopische, futuristische beweging kreeg tijdens zijn eerste
jaren in de Bay Area steeds meer aanhangers, met name in en rond Silicon
Valley, en Szabo zou zichzelf profileren als een vooraanstaand lid van
de gemeenschap. De informaticus publiceerde essays en brieven in het
\emph{Extropy} magazine over onderwerpen als ruimtekolonisatie
(geïnspireerd door zijn ervaring bij JPL), kunstmatige intelligentie,
nanotechnologie en meer.

Het was ook via de Extropiaanse gemeenschap dat Szabo kennis maakte met
Tim May.

\section{De Crypto-anarchist}\label{de-crypto-anarchist}

Op een dag nodigde May Szabo uit om deel te nemen aan een bijeenkomst
van de Cypherpunks. Szabo ging graag op de uitnodiging in; hij was zelf
ook ongerust over de schijnbare erosie van privacy in het opkomende
digitale tijdperk.

Toen Szabo een paar weken later de groep privacy-activisten ontmoette,
wist hij dat hij op de juiste plek was.

Gedurende de volgende paar jaar hielp Szabo, waar hij kon, aan de zaak
van de Cypherpunks. Het meeste opvallend was zijn voortrekkersrol in het
verzet tegen de Clipper Chip van de NSA, zowel op de mailinglijst van de
Cypherpunks als in het echte leven: hij gaf tijdens verschillende
evenementen lezingen over het onderwerp en deelde soms flyers uit om
mensen te informeren over de risico's verbonden aan dit soort
surveillancetechnologie. Hij had een manier om de verontrustende
implicaties van de opdoemende inbreuken op de privacy over te brengen
naar een niet-technisch publiek, en was blij om zijn aanpak te delen met
de andere Cypherpunks, zodat zij hetzelfde konden leren doen.

Maar Szabo's interesse in digitale privacy was uiteindelijk slechts een
deel van een veel groter geheel. Hij omarmde het idee van een `Galt's
Gulch in cyberspace.' Zoals uitvoerig beschreven in de vele posts op de
mailinglijst door zijn vriend Tim May, had Szabo zich gerealiseerd dat
de benodigde instrumenten om individuen te beschermen tegen staatsmacht
niet langer alleen voorbehouden waren aan fictieliteratuur. De
computerwetenschapper met Hongaarse roots was het eens met het feit dat
cryptografie het ideaal van echte vrije markten, ongevoelig voor het
gebruik van fysieke macht of staatsdwang, dichter bij kon brengen.

`Als we een stap terug nemen en kijken naar wat veel cypherpunks
proberen te bereiken, dan is er een hoofdthema van een Ghandiaanse
cyberspace waar geweld slechts een illusie kan zijn, of het nu in Mortal
{[}Kombat{]} of 'flame wars' is', schreef hij in de
Cypherpunk-mailinglijst. `Cypherpunks willen dat onze telefoonnetwerken,
internetbedrijven, etc., zowel beschermd zijn tegen geweld en niet
afhankelijk zijn van geweld voor hun bestaan. Onze informatiesystemen
uit de 20e eeuw, van uitgeverijen tot kredietkaarten, zijn vaak sterk
afhankelijk geweest van de dreiging van geweld, meestal via
wetshandhaving, om intellectuele eigendomsrechten te beschermen, fraude
te voorkomen, schulden in te vorderen, enzovoort.'

Zijn voorstel was:

`Er is geen utopie in zicht waar dergelijke bedreigingen volledig kunnen
worden geëlimineerd, maar we kunnen erkennen dat ze bestaan en
zorgvuldig werken aan het verminderen van onze afhankelijkheid
ervan.'\footnote{Nick Szabo, `Re: Crypto + Economics + AI = Digital
  Money Economies', oorspronkelijk verstuurd naar de
  Cypherpunk-mailinglijst, 19 september 1995,
  \href{https://cypherpunks.venona.com/date/1995/09/msg01303.html}{online}}

\section{Slimme contracten}\label{slimme-contracten}

Misschien wel meer dan wie dan ook, was Szabo zich er sterk van bewust
dat voor het realiseren van May's visie op crypto-anarchie, private
communicatie slechts een eerste stap was.

Als leerling van Hayek's werk (inclusief diens latere, meer politieke
geschriften) was de Cypherpunk tot het inzicht gekomen dat welvarende
samenlevingen zo goed presteerden omdat ze bepaalde regels hadden
aangenomen, in de vorm van wetten, die menselijk gedrag reguleerden. Hij
vergeleek het met het gelaagde ontwerp van computerprotocollen, en
erkende dat de wetten die de basis van de samenleving vormden (zoals
grondwetten) de meest basale, maar ook de belangrijkste regels van
allemaal waren. Alle andere regels zijn gebaseerd op deze fundamenten.

Szabo stelde dat twee van de fundamentele bouwstenen in de `basislaag'
van elke welvarende samenleving inderdaad eigendomsrechten en
contractrecht waren. Eigendomsrechten vergemakkelijken en stimuleren
investeringen, productie en uitbreiding, terwijl contractrecht handel en
specialisatie mogelijk maakt. Hij geloofde dat een samenleving die
eigendomsrechten en contractrecht handhaaft, uiteindelijk kan evolueren
tot een moderne kapitalistische samenleving.

Maar Szabo was zich ervan bewust dat eigendomsrechten en contractrecht
in moderne kapitalistische samenlevingen worden gehandhaafd door de
staat. Eigendom en de overdracht van eigendom worden uiteindelijk
afgedwongen door rechtbanken en de politie, die gebruikmaken van het
staatsmonopolie op geweld. Dit kan impliciet gebeuren door (de dreiging
van) boetes of gevangenisstraf, of expliciet door iemands eigendom met
geweld te beschermen of terug te geven.

Om een staatloze en geweldloze cyberspace-economie te creëren,
concludeerde Szabo dat crypto-anarchisten eerst een nieuwe basis moesten
leggen.

Het eerste deel van deze basis, de eigendomsrechten, kon met behulp van
publieke sleutel-cryptografie worden gerealiseerd, zoals uitvoerig door
Tim May is uitgelegd; persoonlijke gegevens zouden worden beveiligd met
wiskunde in plaats van met fysieke kracht.

Maar het was niet meteen duidelijk hoe twee mensen deze gegevens konden
uitwisselen zonder risico op wanprestatie van de tegenpartij, oftwel
\emph{tegenpartijrisico}. Een partij moest altijd eerst hun gegevens
verzenden of hun deel van de afspraak afronden. Op dat moment kon de
tegenpartij verdwijnen en zijn verplichting niet nakomen. In cyberspace
waren er geen rechtbanken of politie om contractrecht te handhaven.

Een mogelijke oplossing was het gebruik van arbitrage om geschillen te
beslechten, waarbij een scheidsrechter de middelen zou krijgen om de
betreffende data te verplaatsen van de ene handelspartner naar de andere
(op eigen discretionaire bevoegdheid, of wellicht in samenwerking met
een van de handelspartners). Dit zou echter vereisen dat beide partijen
de arbiter vertrouwen om niet te stelen of samen te zweren met hun
tegenpartij. Dit vertrouwen zou misschien na verloop van tijd kunnen
worden opgebouwd door de reputatie van de arbiter: een onberispelijke
staat van dienst van de arbiter zou op een bepaalde manier kunnen dienen
als een haalbaar alternatief voor de handhaving door de staat, hoewel
het nooit volledig risicovrij zou zijn.

Maar halverwege de jaren 90 meende Szabo dat hij een beter idee had. Hij
bedacht een oplossing die niet door een van de handelspartijen kon
worden bedrogen en geen betrouwbare scheidsrechter nodig had. Szabo
stelde \emph{slimme contracten} voor; digitale contracten die uit
zichzelf hun eigen voorwaarden zouden afdwingen, zoals vastgelegd in
computercodes.\footnote{Nick Szabo, `Smart Contracts: Building Blocks
  for Digital Free Markets', \emph{Extropy} 16 : 50--64.}

Stel dat Alice, bij wijze van vereenvoudigd voorbeeld, een geheime code
van Bob wil kopen voor € 10. Een slim contract kan dan zo worden
geprogrammeerd dat zodra Bob de geheime code invoert op een vooraf
afgesproken plek waar Alice hem kan vinden (misschien door het toe te
voegen aan het contract zelf, niet ongelijk aan een digitale
handtekening), de code van het contract dit herkent en verifieert, om
vervolgens \emph{automatisch} € 10 van Alice's account af te trekken en
over te maken naar Bob (Szabo gebruikte een frisdrankautomaat als een
primitieve analogie: het `contract' tussen een consument en een
frisdrankautomaat stelt dat als de consument een muntje in de automaat
stopt, de machine --- door een geautomatiseerde respons --- een blikje
frisdrank of een reep chocolade teruggeeft, zonder verdere menselijke
tussenkomst).

In theorie zouden \emph{smart contracts} (slimme contracten)
buitengewoon complex kunnen worden opgesteld, en kunnen ze in principe
alle soorten contractuele clausules bevatten, waaronder pandrechten,
obligaties, of omschrijving van eigendomsrechten. Dit zou zelfs volledig
nieuwe zakelijke regelingen mogelijk maken, opperde Szabo, waarbij hij
het voorbeeld gaf van een smart contract voor een autolening die, bij
het niet terugbetalen van de lening, automatisch de controle over de
digitale autosleutels terug zou geven aan de bank.

Nick Szabo geloofde dat de meeste eigenschappen en kenmerken die een
slim contract zou moeten bevatten, konden worden geïmplementeerd door
gebruik te maken van de steeds groter wordende gereedschapskist van de
Cypherpunks: `Protocollen gebaseerd op wiskunde, genaamd
\emph{cryptografische protocollen}, zijn de fundamentele bouwstenen die
de verbeterde afwegingen tussen waarneembaarheid, verifieerbaarheid,
privaatrechtelijkheid, en afdwingbaarheid in slimme contracten
implementeren', schreef hij. \footnote{Szabo, `Smart Contracts', 51.}

De grootste uitdaging was echter om ervoor te zorgen dat het contract
automatisch, betrouwbaar en onvoorwaardelijk uitgevoerd zou worden. Dit
hield vooral in dat geen van de partijen bij het contract
verantwoordelijk zou moeten zijn voor de uitvoering ervan.

Als algemeen vertrekpunt vermoedde Szabo dat de beste oplossing te
vinden was in het domein van gedistribueerd berekening, waar deelnemende
computers strikte protocollen moesten volgen om met elkaar overeen te
stemmen. Hoewel het Byzantijnse Generaalsprobleem (het
coördinatieprobleem dat Adam Back op de universiteit leerde kennen)
inderdaad nog niet volledig was opgelost (er bleven enkele risico's als
er te veel onbetrouwbare deelnemers waren) geloofde hij dat er robuuste
protocollen waren ontworpen voor de meeste scenario's.

`De modaliteiten van contractuele relaties kunnen vaak worden
geformaliseerd en gestandaardiseerd, en vervolgens worden uitgevoerd via
netwerkgebaseerde protocollen', legde Szabo uit op de
Cypherpunk-mailinglijst. `Deze protocollen, samen met economische
prikkels, beschermen de uitvoering van het contract tegen zowel fraude
door de hoofdpartijen als aanvallen van derden.' \footnote{Nick Szabo,
  `Re: Crypto + Economics + AI = Digital Money Economies',
  oorspronkelijk verstuurd naar de Cypherpunk-mailinglijst, 19 september
  1995,
  \href{https://cypherpunks.venona.com/date/1995/09/msg01303.html}{online}}

Dat gezegd zijnde, zouden de meeste slimme contracten vermoedelijk ook
een vorm van geld nodig hebben: meestal moet minstens een van de
partijen een betaling doen als onderdeel van de deal. Bovendien moest
dit een soort valuta zijn die over het internet kon worden overgedragen,
en die autonoom door een computerprogramma kon worden verstuurd\ldots{}
bij voorkeur anoniem.

Voor slimme contracten is elektronisch geld noodzakelijk.

\section{Vertrouwde derde partijen}\label{vertrouwde-derde-partijen}

Als digitale equivalenten van eigendomsrechten en contractrecht twee van
de fundamentele bouwstenen waren die nodig zijn om een `Galt's Gulch in
cyberspace' te realiseren, was een digitaal equivalent voor geld de
derde.

Nick Szabo was één van de eersten onder Extropianen en Cypherpunks die
deze inzichten herkende. Toen Hal Finney voor het eerst de voordelen van
elektronisch geld begon te bepleiten in het \emph{Extropy}-magazine, was
Szabo de Extropiaan die naar Amsterdam verhuisde om voor David Chaum te
werken. Rond dezelfde tijd dat de eerste webshops hun virtuele deuren
openden, trad Szabo toe tot DigiCash als internetprogrammeur.

Tijdens de vroege jaren '90 werkte hij een tijdje op het kantoor van het
bedrijf, en kreeg Szabo de kans om uit eerste hand te ervaren hoe het
eraan toe ging bij de meest beloftevolle start-up voor elektronisch geld
ter wereld. Hij kon zien hoe DigiCash het eerste Chaumiaanse digitale
geldproduct op de markt bracht in de vorm van CyberBucks, het ongedekte
prototype van eCash, waarvoor de start-up beloofde nooit meer dan een
miljoen eenheden uit te geven. Hij was ook getuige hoe CyberBucks een
beetje koopkracht verwierven, aangezien sommige Cypherpunks en andere
geïnteresseerde technofielen bereid waren om kleine bedragen van het
digitale geld te kopen met reguliere valuta, of het te accepteren in
ruil voor goederen of diensten met een lage waarde.

Het systeem van CyberBucks was volledig afhankelijk van het bedrijf
DigiCash: Szabo realiseerde zich dat de start-up van Chaum simpelweg
meer dan een miljoen eenheden van het elektronische geld kon uitgeven
als ze van gedachten zouden veranderen, of als ze vanaf het begin
gelogen hadden, of als een oneerlijke medewerker het systeem zou
proberen te bedriegen, enzovoorts. Bovendien zou er geen enkele manier
zijn om het te detecteren als ze het deden (om nog maar te zwijgen over
het feit dat CyberBucks onmiddellijk waardeloos zou worden als de
servers van DigiCash ooit uitvielen, wat uiteindelijk ook gebeurde).

Szabo beschouwde dit als aanzienlijke problemen.

`Een van de dingen die ik daar leerde, was hoe gemakkelijk het was om
met mensen hun saldo's te knoeien in een centraal gereguleerde valuta',
herinnerde de pionier van de digitale valuta zich later. `Wie zou zijn
rijkdom toevertrouwen aan een stel slordige Frank Zappa-fans in het
verre Amsterdam?'\footnote{Nick Szabo, e-mail aan auteur, 10 juli 2018.}
(ze hielden ervan om platen van Frank Zappa op het kantoor van DigiCash
te draaien).

Net als Scott Stornetta en Stuart Haber een aantal jaar eerder, wakkerde
dit bij Szabo de neiging aan om dieper na te denken over de rol van
vertrouwde entiteiten in het groeiende online ecosysteem. Hoewel
cryptografie privacy en beveiliging kon bieden op een niveau dat gelijk
of zelfs beter was dan wat mogelijk is in de fysieke wereld, merkte
Szabo op dat protocolontwikkelaars doorgaans een bepaald type risico
leken te negeren. Het was het risico dat inherent was verbonden aan de
afhankelijkheid van een \emph{vertrouwde derde partij} (TTP).

Het typische voorbeeld is een certificaatautoriteit die een register
bijhoudt van de werkelijke identiteiten gekoppeld aan hun publieke
sleutels. Hoewel dit een handige manier is om iemands publieke sleutel
te vinden, zijn deze publieke sleutels eigenlijk maar zo veilig te
gebruiken als de certificaatautoriteit zelf. Als een publieke sleutel
die door een certificaatautoriteit wordt vermeld, eigenlijk toebehoort
aan een aanvaller, zou Szabo benadrukken, kan deze aanvaller elk bericht
ontcijferen dat bedoeld is voor wie dan ook die geassocieerd is met de
publieke sleutel in het register.

In zijn essay over dit onderwerp, toepasselijk genaamd `Trusted Third
Parties are Security Holes' \footnote{Nick Szabo, `Trusted Third Parties
  are Security Holes', \emph{Satoshi Nakamoto Institute} ,
  \href{https://nakamotoinstitute.org/library/trusted-third-parties/}{online}},
legde Szabo later uit dat vertrouwde derde partijen vaak niet werden
opgenomen in de kosten van een ontwerp, en waarom hij geloofde dat dit
een vergissing was. Vertrouwde derde partijen zijn veiligheidsgaten,
beargumenteerde de Cypherpunk, \emph{zelfs} als ze zelf daadwerkelijk
eerlijk zijn: ze kunnen aantrekkelijke doelwitten worden voor
kwaadwillige hackers, of misschien zelfs voor naties en hun regelgevende
instanties tijdens periodes van politieke instabiliteit of
onderdrukking.

Een gecompromitteerde vertrouwde derde partij kan dus enorm duur
blijken. Szabo beargumenteerde dat de kosten van een mogelijke inbreuk
moeten worden meegenomen bij het ontwerpen van protocollen, wat volgens
hem zou leiden tot het feit dat veel protocollen opnieuw ontworpen
moeten worden om de vertrouwde derde partijen volledig te elimineren.

Szabo stelde nogmaals dat mogelijke oplossingen misschien wel te vinden
zijn in de wereld van gedistribueerde berekening.

`De beste 'TTP' van allemaal is er een die niet bestaat, maar waarvan de
noodzaak {[}ertoe{]} is geëlimineerd door het protocolontwerp, of die
geautomatiseerd en verdeeld is over de deelnemers van een protocol',
concludeert hij in zijn paper.

\section{Vrij bankieren}\label{vrij-bankieren}

Net als Chaum waren de meeste Cypherpunks vooral geïnteresseerd in
elektronisch geld vanwege de mogelijkheden tot privacy die het kon
bieden. Maar Szabo, May, en enkele andere Cypherpunks die zich bij de
beweging hadden aangesloten vanuit de \emph{Extropian}-community, waren
ook gedreven door monetaire hervorming. Ze waren voornamelijk
geïnteresseerd in de ideeën van Hayek over vrij bankieren, zoals ook
beschreven door Max More in de digitale geldeditie van \emph{Extropy}.

Bovendien had Szabo het werk van George Selgin bestudeerd, wiens boek
`The Theory of Free Banking' in datzelfde tijdschrift was besproken, en
dat van Selgin's collega Lawrence H. White, die zelf een artikel aan het
tijdschrift had bijgedragen. Aangezien Hayek net voor de oprichting van
de Cypherpunkbeweging overleed, dacht Szabo dat de medeoprichters van de
door Hayek geïnspireerde moderne school voor vrij bankieren wellicht
konden helpen bij het ontwerp van een elektronisch geldsysteem.

Ergens midden jaren '90 besloot hij een nieuwe, meer themagerichte
mailinglijst te maken: de Libtech-lijst. Hier zouden vrije bankiers als
Selgin en White, evenals geïnteresseerde Extropianen en Cypherpunks,
gerichte discussies voeren over bankieren, monetaire economie en het
allerbelangrijkste: de ontwerpen voor digitale valuta.\footnote{Nick
  Szabo, `Nick Szabo on Cypherpunks, Money and Bitcoin', interview door
  Peter McCormack, \emph{What Bitcoin Did}, 1 november 2019,
  \href{https://www.whatbitcoindid.com/podcast/nick-szabo-on-cypherpunks-money-and-bitcoin}{online}}

Toen twee zeer verschillende werelden elkaar ontmoetten op de Libtech
lijst, keek Szabo met een frisse blik naar de inzichten uit de
Oostenrijkse economie, door zijn eigen ervaring als
computerwetenschapper. Dit stelde hem in staat nog duidelijker dan
voorheen de gebreken van fiatgeld te zien. Waar Hayek uitvoerig had
gewaarschuwd voor de mankementen van centrale banken, zag Szabo in dat
dit uiteindelijk te wijten was aan een fundamentele ontwerpfout van het
huidige monetaire systeem.

\emph{Centrale banken waren vertrouwde derde partijen.}

`Het probleem, kort gezegd, is dat ons geld voor zijn waarde momenteel
afhankelijk is van het vertrouwen in een derde partij', betoogde Szabo
later. `Zoals vele inflatoire en hyperinflatoire episodes in de 20e eeuw
hebben aangetoond, is dit geen ideale situatie.'\footnote{Nick Szabo,
  `Bit Gold', \emph{Unenumerated}, 27 december 2008,
  \href{https://unenumerated.blogspot.com/2005/12/bit-gold.html}{online}}

Door deze fundamentele zwakte centraal te stellen, bevestigde Szabo op
een bepaalde manier Hayek's analyse, althans voor zichzelf.

Maar tegelijkertijd betwistte hij enigszins het monetaire ecosysteem dat
Selgin voorspelde dat zou ontstaan in een vrije bankomgeving. Als het
probleem van het moderne monetaire systeem was dat iedereen een centrale
bank moest vertrouwen, dan zou vrij bankieren gewoon particuliere banken
tot de nieuwe vertrouwde derde partijen maken. Zelfs als concurrentie
hen eerlijk kon houden, vertegenwoordigden deze TTP's (Trusted Third
Parties) een bredere reeks veiligheidsrisico's, variërend van
losgeslagen werknemers tot draconische overheden, die marktdynamieken
niet noodzakelijkerwijs volledig zouden oplossen.

Als en wanneer deze nieuwe vertrouwde derde partijen -- de banken -- het
vertrouwen van hun klanten zouden schenden, zouden mensen niet simpelweg
overstappen naar hun concurrenten, zo voorspelde Szabo. In plaats
daarvan zouden mensen opnieuw aandringen op waarborgen van de centrale
bank. Zo was het moderne monetaire systeem immers in de eerste plaats
ontstaan.

Voordat vrij bankieren een haalbaar alternatief kon zijn, concludeerde
Szabo, moest er volledig nieuwe vorm van elektronisch geld worden
ontworpen dat het vertrouwen minimaliseerde. Net zoals eigendomsrechten
en contractrecht voor cyberspace vanaf nul opnieuw moesten worden
uitgevonden, zou ook een digitale munteenheid moeten worden gecreëerd
door te starten vanuit basisprincipes.

\section{De oorsprong van geld}\label{de-oorsprong-van-geld}

Om geld te kunnen creëren, moest Szabo het eerst begrijpen.

De Cypherpunk was natuurlijk niet de eerste die de natuur van geld
bestudeerde, en hij dacht ook niet dat hij dat was. Hij was goed op de
hoogte van de theorieën van Carl Menger en Ludwig von Mises over de
oorsprong van geld, en deelde grotendeels hun inschatting dat geld
voortkwam uit ruilhandel. Maar waar Menger en Mises hun stelling hadden
ontwikkeld door logica en redeneren, ging Szabo op zoek naar
daadwerkelijke historische documenten, en zelfs archeologische resten.

In zijn zoektocht graafde Szabo in de prehistorie van de mensheid en
deed hij onderzoek naar pre-industriële samenlevingen zoals die van de
\emph{Native Americans}. Hij ontdekte dat geld zelfs ouder was dan
tekst: vroege vormen van geld werden al gebruikt door
jager-verzamelaarstammen. Dit leidde uiteindelijk tot zijn hypothese dat
geld letterlijk zo oud zou kunnen zijn als de mensheid zelf. Zoals eerst
gesuggereerd door evolutiebioloog Richard Dawkins in zijn baanbrekende
werk \emph{The Selfish Gene},\footnote{Richard Dawkins, `The Selfish
  Gene', 244.} zou het vermogen om geld te gebruiken diep ingebed kunnen
zijn in het menselijk DNA en zou het de overleving van de soort ten
voordele gekomen kunnen zijn.

In zijn latere essay `Shelling Out: The Origins of Money',\footnote{Nick
  Szabo, `Shelling Out: The Origins of Money', \emph{Satoshi Nakamoto
  Institute},
  \href{https://nakamotoinstitute.org/library/shelling-out/}{online}}
legt Szabo uit hoe.

Een van de grote voordelen van de mensheid in de meedogenloze
overlevingsstrijd van moeder natuur, ontdekte Szabo, is dat de meeste
leden van de soort bereid en in staat zijn om samen te werken om hun
krachten te bundelen of om arbeid te verdelen en zich te specialiseren
om vervolgens de opbrengst te delen. Dit is waarschijnlijk altijd zo
geweest: prehistorische jagers die succesvol een wild zwijn doodden,
zouden het vlees delen met hun stamleden. Hun medestamleden zouden een
andere keer de gunst retourneren, misschien door de blauwe bessen of
eetbare paddenstoelen te delen die ze de volgende dag verzamelden.

Deze vorm van altruïstisch gedrag zou iedereen kunnen ten goede komen,
zolang alle stamleden elkaar kenden en min of meer konden bijhouden wat
ieders bijdragen waren. Alle stamleden hadden een publieke reputatie.
Omdat profiteurs, degenen die nooit iets bijdroegen aan de stam,
uiteindelijk konden worden uitgesloten van de deelrondes of zelfs uit de
stam konden worden verbannen, was iedereen sterk gemotiveerd om bij te
dragen en hun deel te doen.

Maar dit model zou onhoudbaar worden als een stam (of vaker: een groep
stammen) te groot werd. Menselijke hersenen kunnen slechts een beperkt
aantal sociale relaties onderhouden (populair bekend als `Dunbar's
nummer', wat 150\footnote{Het is echter belangrijk op te merken dat dit
  getal---vernoemd naar primatoloog Robin Dunbar---bekritiseerd is en
  tegenwoordig in de wetenschappelijke gemeenschap niet met de
  specifieke nauwkeurigheid wordt gehanteerd die het suggereert. Hoewel
  er een limiet is aan het aantal stabiele sociale relaties dat mensen
  kunnen onderhouden, is het werkelijke aantal niet per se 150 voor
  iedereen. Zie bijvoorbeeld Patrick Lindenfors, Andreas Wartel, and
  Johan Lind, `'Dunbar's Number' Deconstructed', Biology Letters, 5 mei
  2021:
  \href{https://royalsocietypublishing.org/doi/10.1098/rsbl.2021.0158}{online}}
is), dus het wordt moeilijk om iedereens reputatie bij te houden als er
meer mensen dan dit nummer zijn. Als niemand zich kan herinneren wie met
de anderen heeft gedeeld en wie niet, brokkelt het `publieke
reputatiesysteem' af, en krijgen profiteurs vrij spel op kosten van
iedereen.

Om te voorkomen dat er misbruik van ze wordt gemaakt, zou het zelfs
rationeel worden voor elk individu om zelf ook te stoppen met delen en
een profiteur te worden, ook al zou iedereen beter af zijn als iedereen
zou delen. Met andere woorden, er zou iets van een groot
\emph{gevangenendilemma}\footnote{Strikt genomen is een `public goods
  game' de meer precieze speltheoretische analogie.} ontstaan.

Maar Szabo legde uit dat genen strategieën kunnen coderen om oplossingen
te vinden voor uitdagingen in de speltheroie van de echte wereld.
Gedurende lange perioden en via natuurlijke selectie zouden de beste
eigenschappen, die de overleving van een soort bevorderen, dominant
worden.

Het gebruik van geld werd door de Cypherpunk als zo'n eigenschap
beschouwd.

Voor het grootste deel van de menselijke geschiedenis was dit echter een
heel ander soort geld dan wat moderne samenlevingen gebruiken.

Doorheen de eeuwen en over verschillende culturen heen, droegen mensen
juwelen, zoals kettingen, iets wat geen enkel ander dier doet. De
oppervlakkige uitleg hiervoor is dat mensen simpelweg plezier beleven
aan het dragen van dergelijke sieraden. Maar Szabo begreep dat er een
fundamenteelere vraag schuilging achter deze simpele verklaring, het
soort vraag dat een evolutionair bioloog zou stellen. \emph{Waarom}
hebben mensen zich zo ontwikkeld dat ze plezier beleven aan het dragen
van sieraden?

Net als Dawkins vermoedde Szabo dat mensen in de loop van de eeuwen een
voorliefde voor ornamenten hadden ontwikkeld omdat dit een evolutionair
voordeel bood: het stelde hen in staat om te coöpereren en middelen te
`delen' op een grotere schaal dan alleen binnen hun eigen stam.

Szabo ontdekte bijvoorbeeld dat halskettingen en andere verzamelobjecten
verhandeld werden tussen stammen in ruil voor voedsel, wapens of
bruiden. De ornamenten konden later teruggeruild worden of met een
andere stam geruild worden voor andere hulpbronnen. In plaats van te
onthouden wie wat gedeeld had, dienden de sieraden als een soort
proto-geld en bevorderden ze wat evolutionaire psychologen
\emph{wederzijds altruïsme} noemen.

Het stelde stammen in staat een mate van samenwerking en specialisatie
tussen hen te bevorderen. Verschillende stammen jaagden bijvoorbeeld op
verschillende soorten dieren in verschillende delen van het jaar, wat
hen uiteindelijk allemaal ten goede kwam.

\section{Proto-geld}\label{proto-geld}

Niet zomaar elk sieraad voldeed echter.

Toen hij de resten van pre-civiele samenlevingen analyseerde, ontdekte
Szabo dat mensen uit alle culturen de neiging hadden om zich te richten
op verzamelobjecten met enkele zeer specifieke eigenschappen. Hoewel
archeologen voorbeelden van proto-geld hadden gevonden die zo
uiteenlopend waren als schelpen van een zeldzaam type slak, tot
struisvogelei-scherven, tot mammoet-tanden, deelden ze allemaal drie
algemene kenmerken.

Allereerst waren de verzamelobjecten relatief makkelijk te beschermen
tegen onopzettelijk verlies en diefstal. Kettingen zijn waarschijnlijk
het beste voorbeeld in dit opzicht: ze zijn bijna onmogelijk te
verliezen als ze om de nek worden gedragen. Alternatief gezien, waren
ornamenten die niet op iemands lichaam gedragen konden worden, doorgaans
minstens makkelijk te verbergen.

Ten tweede, de verzamelobjecten vertegenwoordigden een niet-te-vervalsen
schaarste. Dat wil zeggen, ze zouden kostenlijk zijn om te maken of
moeilijk te vinden: een mammoettand was schaars omdat een mammoet doden
niet gemakkelijk is, terwijl struisvogel eieren moeilijk te bemachtigen
zijn.

`Op het eerste gezicht lijkt de productie van een goed uitsluitend omdat
het duur is, volledig verspillend', werkte Szabo later uit in `Shelling
Out'. `De onvervalsbaar dure grondstof voegt echter voortdurend waarde
toe door welvarende overdrachten mogelijk te maken. Steeds meer van de
kosten worden terugverdiend bij elke transactie die mogelijk wordt
gemaakt of goedkoper wordt gemaakt. De kosten, aanvankelijk volledig
verspild, worden omgeslagen over vele transacties.'

En ten derde, het was doorgaans vrij eenvoudig om vast te stellen dat
het proto-geld inderdaad onvervalsbaar zeldzaam was, door simpelweg
observaties of metingen te doen. De zeldzame slakkenhuisjes
bijvoorbeeld, zou iedereen in deze stammen eenvoudig herkend hebben,
terwijl het namaken ervan onmogelijk zou zijn geweest met de
gereedschappen die ze ter beschikking hadden.

Szabo ontdekte dat de oudste vormen van geld meestal makkelijk te
beveiligen waren en aantoonbaar moeilijk te verkrijgen.

Moderne fiatvaluta's bezaten volgens velen in wezen geen van de drie
kwaliteiten van proto-geld. Ze waren niet bijzonder gemakkelijk te
beschermen tegen diefstal, en de meeste mensen deden zelfs geen poging
om hun eigen geld te beveiligen, in plaats daarvan vertrouwden ze op
derde partijen (banken) voor veilige bewaring. Maar wellicht nog
belangrijker: fiatvaluta was niet fundamenteel schaars; regeringen en
centrale banken konden naar believen meer geld drukken, of digitaal
extra geld aanmaken via een druk op de knop.

Szabo benadrukte dat het zwevende fiatvalutasysteem dat al tientallen
jaren de wereldstandaard was, een grote historische uitzondering was.
Rekening houdend met de weinige soortgelijke voorbeelden waarvan hij
wist (sommige in het dynastieke China, twee in het Frankrijk van de
achttiende eeuw en de Confederatie-dollar tijdens de Amerikaanse
burgeroorlog) verwachtte hij niet dat het nieuwe monetaire experiment
zou blijven duren. Hij was ervan overtuigd dat fiatvaluta uiteindelijk
ten onder zou gaan.

Veel Oostenrijkse economen waren uiteraard tot vergelijkbare conclusies
gekomen als Szabo. Degenen die een terugkeer naar de goudstandaard
voorstaan, geloven specifiek dat het edelmetaal de beste vorm van geld
is, grotendeels vanwege de onvervalsbare schaarste ervan.

Szabo was echter niet overtuigd dat goud de beste vervanging was: hoewel
het edelmetaal inderdaad moeilijk te verkrijgen was, was het ook
moeilijk te beveiligen.

`Dure metalen en verzamelobjecten hebben een onvervalsbare schaarste
vanwege de hoge creatiekosten. Dit gaf geld ooit een waarde die
grotendeels onafhankelijk was van een vertrouwde derde partij.
Edelmetalen hebben echter problemen. Het is te kostbaar om metalen
steeds opnieuw te testen voor gewone transacties. Daarom werd een
vertrouwde derde partij (meestal geassocieerd met een belastingontvanger
die de munten accepteerde als betaling) gevraagd om een standaard
hoeveelheid van het metaal in een munt te stempelen. Het vervoeren van
grote hoeveelheden waardevol metaal kan zeer onzeker zijn, zoals de
Britten ontdekten toen ze tijdens de Eerste Wereldoorlog goud naar
Canada vervoerden over een door U-boten geïnfecteerde Atlantische Oceaan
om hun goudstandaard te ondersteunen', schreef Szabo.

Hij voegde toe:

`Erger nog, je kunt online niet betalen met metaal.'\footnote{Szabo,
  `Bit Gold.'}

De Cypherpunk wilde de wenselijke monetaire eigenschappen van goud
reproduceren in een elektronisch geldsysteem --- een digitale valuta met
onvervalsbare kostbaarheid.

Toen Adam Back in 1997 hashcash aankondigde, leek dit eindelijk mogelijk
te zijn.

\section{Bit Gold}\label{bit-gold-1}

Minder dan een jaar later, in 1998, had Szabo zijn eigen voorstel voor
een digitale munteenheid ontworpen: \emph{Bit Gold}. Hoewel hij Bit Gold
nog niet in code had geïmplementeerd (het was tot dan toe slechts een
idee) deelde hij een beschrijving ervan op de Libtech lijst.

Net als Hashcash, was Bit Gold ontworpen rondom het `proof-of-work'
concept. De vereiste rekenkracht om deze `proof-of-work' te genereren
koppelde de creatie van de valuta aan de kosten voor energie, met iets
wat lijkt op digitale schaarste als gevolg. Vanuit Szabo's visie
vertegenwoordigde de `proof-of-work' een onvervalsbare kostbaarheid.

Het proof-of-work systeem van Bit Gold zou beginnen met een
`kandidaat-reeks', wat in feite een willekeurig getal is. De gedachte
hierachter was dat iedereen deze string kon pakken en samenvoegen met
hun eigen nonce om zo iets te creëren dat op een hash lijkt. Gezien de
kenmerken van hashen, zou de daaruit voortkomende hash een nieuwe, op
het eerste gezicht willekeurige, reeks cijfers zijn.\footnote{In Szabo's
  oorspronkelijke voorstel zou Bit Gold eigenlijk gebruikmaken van een
  `secure benchmark function', wat iets anders is dan een hashfunctie,
  maar vergelijkbaar genoeg dat het product een `hash' noemen en het
  proces `hashen' noemen voldoende nauwkeurig is om de werking van Bit
  Gold te begrijpen.}

Het trucje, dat ook door hashcash wordt gebruikt, was dat niet alle
hashes volgens het Bit Gold protocol als geldig werden beschouwd. In
plaats daarvan moest een geldige hash beginnen met een vooraf bepaald
aantal nullen. Door de onvoorspelbare aard van hashing, was de enige
manier om zo'n hash te vinden door middel van \emph{trial and error},
waarbij bij elke poging een nieuwe nonce werd gebruikt.

Wanneer iemand een geldige hash vond, zou deze hash de nieuwe
kandidaat-reeks worden. De volgende geldige hash zou dan gegenereerd
moeten worden vanuit deze nieuwe kandidaat-reeks en een ander nonce.
Zodra een tweede geldige hash gevonden werd, zou deze op zijn beurt de
nieuwe kandidaat-reeks worden, en zo verder.

Het systeem van Bit Gold zou na verloop van tijd een lange reeks hashes
genereren, waarbij de meest recente hash altijd fungeert als nieuwe
kandidaat-reeks.

Het geldgedeelte leek in grote lijnen op dat van de eerdere voorstel tot
het bezit van cijfers van Hadon Nash aan de Cypherpunk-mailinglijst. Wie
een geldige hash produceerde zou letterlijk dit hash `bezitten'.
Eigendom van hashes zou worden vastgelegd in een digitaal
eigendomsregister, waar alle hashes zouden worden toegewezen aan de
publieke sleutels van hun eigenaren.\footnote{Nick Szabo, `Secure
  Property Titles with Owner Authority', \emph{Satoshi Nakamoto
  Institute},
  \href{https://nakamotoinstitute.org/library/secure-property-titles/}{online}}
Aangezien het register alleen publieke sleutels zou gebruiken, geen
namen, kon Bit Gold vrij anoniem worden gebruikt.

Om een hash `uit te geven', zou de eigenaar een bericht moeten
ondertekenen waarin wordt aangegeven wie de nieuwe eigenaar is (opnieuw,
door naar deze persoon te verwijzen alleen door hun publieke sleutel).
Als de digitale handtekening overeenkomt met de publieke sleutel die in
het eigendomsregister staat vermeld, zou de overdracht geldig zijn en
zou het register worden bijgewerkt om de nieuwe eigenaar van de hash te
reflecteren. Zonder een geldige handtekening, zou de overdracht
afgewezen moeten worden en zou de hash in het bezit blijven van de
huidige eigenaar.

Het proof-of-work zou ervoor zorgen dat Szabo's elektronische geld
aantoonbaar moeilijk te verkrijgen is, terwijl publieke
sleutel-cryptografie het veilig zou maken.

Dat is natuurlijk onder voorwaarde dat het register zelf veilig is.

\section{Het register}\label{het-register}

Dus wie zou het register onderhouden?

Szabo begreep dat elke enkele entiteit die het register zou bijhouden,
een vertrouwde derde partij zou zijn. Hoewel zelfs deze derde partij
niet in staat zou zijn om bewijzen te vervalsen, iedereen kon
onmiddellijk herkennen dat de hashes ongeldig zijn, kon het potentieel
nog steeds hashes dubbel uitgeven, of transacties censureren, of
misschien zelfs hashes stelen van andere gebruikers.

In Szabo's voorstel zou het eigendomsregister dus onderhouden worden
door een Bit Gold `eigendomsclub'. Deze club bestond uit `clubleden'
(oftewel internetservers) die het eigendomsregister tussen hen zouden
dupliceren en gezamenlijk bijhouden wie wat bezit. Als een van de
clubleden zou proberen vals te spelen of te stelen, zouden de andere
clubleden dit opmerken en de overdracht afwijzen, waardoor het
vertrouwen tussen hen wordt verdeeld.

Szabo stelde voor om dit te implementeren met behulp van dezelfde
soorten gedistribueerde computerprotocollen die hij voorzag voor slimme
contracten. Zoals uitgelegd in zijn paper `Veilige Eigendomstitels met
Eigenaarsautoriteit', waren deze ontwerpen het best te begrijpen als
geavanceerde stemmingen: zolang de meeste servers eerlijk bleven, zouden
ze consensus bereiken over de toestand van het register. Als slechts een
minderheid van de servers zou falen of uit de pas zou lopen, zou het
systeem als geheel prima moeten blijven functioneren.

Dit was echter niet perfect: het zogeheten Byzantijnse Generaalsprobleem
was niet volledig opgelost. Er konden vooral vervelende problemen
ontstaan als het register het doelwit zou worden van een Sybil-aanval.
In dit soort computer aanval slaagt één kwaadwillig persoon erin zich
voor te doen als meerdere verschillende deelnemers, waardoor de
stemprocedures worden overrompeld. Szabo zou dit later omschrijven als
het `Sokpop probleem'.\footnote{Nick Szabo, `Nick Szabo --- The Quiet
  Master of Cryptocurrency \ldots{} Gehost door Naval Ravikant'
  interview door Tim Ferriss, \emph{The Tim Ferriss Show}, YouTube, 12
  augustus 2017,
  \href{https://www.youtube.com/watch?v=3FA3UjA0igY}{online}}

Toch geloofde Szabo dat dit zichzelf zou kunnen oplossen. Hij opperde
dat zelfs in een scenario waarin een meerderheid van de clubleden zou
proberen vals te spelen, het register openbaar zou zijn. Dus
cryptografische bewijzen zoals (het ontbreken van) geldige
handtekeningen konden worden gebruikt om Bit Gold-gebruikers op de
hoogte te stellen van dit wangedrag. De eerlijke minderheid van
registerbeheerders kon dan afsplitsen om een concurrerend
eigendomsregister te creëren. Wanneer ze de keuze hadden tussen een
aantoonbaar oneerlijk meerderheidsregister of een eerlijk
minderheidsregister, achtte Szabo het waarschijnlijk dat gebruikers de
laatste zouden prefereren.

`Als de regels worden overtreden door de winnende stemmers, kunnen de
correcte verliezers de groep verlaten en een nieuwe groep vormen,
waarbij ze de oude titels erven', legde hij uit. `Gebruikers van de
titels (vertrouwende partijen) die correcte titels willen behouden,
kunnen zelf veilig verifiëren welke afsplitsing de regels correct heeft
gevolgd en overschakelen naar de juiste groep.'\footnote{Szabo, `Secure
  Property Titles.'}

Szabo was zich ervan bewust dat het niet de meest elegante oplossing was
en enkele belangrijke vragen bleven nog onbeantwoord: in het geval van
een dubbele uitgave-aanval, hoe zouden offline gebruikers weten welke
transactie eerst kwam? En in het verlengde daarvan, als deze dubbele
uitgave zou resulteren in het creëren van een concurrerend register
omdat verschillende leden van de eigendomsclub verschillende transacties
eerst zagen, hoe zouden deze gebruikers dan weten welke `correct' is?

Desondanks zag Szabo dit in zijn ogen als een betere oplossing dan
vertrouwen op een derde partij.

\section{Het beheersen van inflatie}\label{het-beheersen-van-inflatie}

Het laatste probleem dat Szabo moest oplossen was inflatie.

Naast de onmogelijkheid om eigendom van het proof-of-work over te
dragen, had het `hashcash'-systeem van Adam Back nog een groot probleem.
Het genereren van geldige hashes zou na verloop van tijd eenvoudiger
worden, aangezien computers elk jaar krachtiger werden. Daarom konden de
hashes niet goed functioneren als geld: hyperinflatie (of zelfs alleen
al de vooruitzichten van hyperinflatie) zou waarschijnlijk betekenen dat
zo'n munteenheid niet van de grond zou komen.

Szabo bedacht ook hiervoor een oplossing.

Omdat alle geldige Bit Gold hashes dienst deden als kandidaat-reeksen
voor de volgende hash, werden ze noodzakelijkerwijs van een tijdstempel
voorzien, in die zin dat de volgorde ervan niet kon worden veranderd.
Bovendien konden nieuwe hashes mogelijk afzonderlijk worden vastgelegd
op daadwerkelijke tijd-stempelservers om een registratie bij te houden
van wanneer ze werden gegenereerd.

Szabo legde uit dat deze tijdstempels een goed idee zouden geven van hoe
moeilijk het moet geweest zijn om een Bit Gold-hash te produceren: een
oudere hash was moeilijker te produceren dan een recentere hash.

Dit verschil moet dan volgens Szabo meegenomen worden in de waarde van
een hash:

`De kosten van de reeks zijn evenredig met de onwaarschijnlijkheid van
de reeks. We hebben empirisch bewijs dat aantoonbaar onwaarschijnlijke
documenten, zeldzame drukfouten op postzegels bijvoorbeeld, behoorlijk
waardevol worden: hoe zeldzamer en hoe verifieerbaarder, hoe
waardevoller.'\footnote{Nick Szabo, `Bit Gold: Towards Trust-Independent
  Digital Money',
  \href{https://web.archive.org/web/20140406003811/http://szabo.best.vwh.net/bitgold.html}{online}}

Met andere woorden, een geldige `1998 hash' zou meer waard moeten zijn
dan een geldige `2008 hash'.

Om de waarde van de hashes vast te stellen, wilde Szabo gebruikmaken van
een beproefde en oude oplossing: de markt. Hij wilde een speciale
marktplaats creëren waar Bit Gold hashes tegen elkaar verhandeld konden
worden. Kopers en verkopers kunnen op die manier een eerlijke relatieve
prijs voor elk van hen vinden. Misschien zou één hash uit 1998 zo'n tien
hashes uit 2008 waard zijn, waarbij de exacte wisselkoers vermoedelijk
zou worden bepaald door de gedaalde kosten van rekenkracht gedurende dat
decennium.

Maar dit zou nog een ander probleem creëren, wist Szabo: `de bits (de
puzzeloplossingen) van de ene periode {[}\ldots{]} zijn niet
inwisselbaar met die van de volgende periode'.\footnote{Nick Szabo, `Bit
  Gold Markets', \emph{Unenumerated}, 27 december 2008,
  \href{https://unenumerated.blogspot.com/2008/04/bit-gold-markets.html}{online}}

De Cypherpunk begreep dat fungibiliteit, waarbij elke eenheid van een
valuta gelijk is in waarde aan elke andere eenheid van dezelfde
denominatie, een cruciale eigenschap is van geld. Een winkelier moet een
betaling kunnen accepteren zonder te hoeven nadenken over de exacte
waarde van het ene bankbiljet ten opzichte van het andere; elk
dollarbiljet zou moeten volstaan. Als hashes verschillend gewaardeerd
zouden worden, zou het de fungibiliteit van Bit Gold in de weg staan.

Nick Szabo, de bedenker van `Bit Gold', had ook hiervoor een oplossing
bedacht: hij zag een op `vrij bankieren' geïnspireerde `tweede laag'
bovenop de `basisslaag' van Bit Gold voor zich.\footnote{Szabo, `Bit
  Gold Markets.'} Deze tweede laag zou bestaan uit een speciaal soort
banken, die veilig controleerbaar zouden moeten zijn vanwege het
openbare karakter van het Bit Gold-register. Deze banken zouden
verschillende hashes uit verschillende tijdperioden verzamelen en, op
basis van hun relatieve marktwaarde, deze bundelen tot pakketten met een
standaardwaarde. Een pakket van bijvoorbeeld `1998' zou slechts één hash
kunnen bevatten, terwijl een pakket van `2008' er tien zou bevatten.

Deze pakketten zouden uiteindelijk opgesplitst worden in een specifiek
aantal eenheden, wellicht uniek per bank. Alice Bank kon bijvoorbeeld
10,000 Alicebucks per bundel uitgeven, of deze bundel nu een 1998 pakket
of een 2008 pakket was. Het zijn deze eenheden die uiteindelijk als het
onderling inwisselbare geld gebruikt zouden worden door reguliere
gebruikers voor dagelijkse uitgaven, idealiter in de vorm van privé,
Chaumiaanse eCash. Tegelijkertijd zouden gebruikers van Alice Bank
altijd in staat moeten zijn om hun Alicebucks in te wisselen voor de
daadwerkelijke hashes die deze onderbouwen.

`Samengevat', concludeerde Szabo zijn voorstel, `is al het geld dat de
mensheid ooit heeft gebruikt op de een of andere manier onzeker geweest.
Deze onzekerheid heeft zich op vele manieren gemanifesteerd, van
vervalsing tot diefstal, maar de meest schadelijke is waarschijnlijk
inflatie geweest. Bit Gold kan ons misschien een vorm van geld geven met
een tot nog toe ongeziene veiligheid tegen deze gevaren.'\footnote{Szabo,
  `Bit Gold.'}

Nick Szabo had gelijk, tenminste in theorie. Waar hashcash het concept
van digitale schaarste had geïntroduceerd, liet het voorstel van Bit
Gold zien hoe dit kan worden omgezet in overdraagbaar elektronisch geld.

\chapter{B-geld (en BitTorrent)}\label{b-geld-en-bittorrent}

Niet lang nadat Bit Gold had laten zien hoe proof-of-work in
overdraagbaar elektronisch geld kon worden omgezet, werd er een
enigszins vergelijkbaar voorstel voor digitale valuta ingediend bij de
Cypherpunk-mailinglijst. De auteur, Wei Dai, was een bekende naam binnen
de Cypherpunk gemeenschap, maar dan alleen bij naam.

Inderdaad, privacy was het oprichtingsbeginsel van de Cypherpunks, maar
het lukte weinigen om deze principes zo effectief uit te voeren als Wei
Dai. Hoewel zijn betrokkenheid bij de Extropians het vermoeden wekte dat
hij in de Bay Area woonde, verscheen Dai nooit bij de persoonlijke
bijeenkomsten van de Cypherpunks. Gedurende enige tijd waren de leden
van de mailinglijst niet eens zeker of ze correspondentie voerden met
een man of een vrouw. Hun onzekerheid ging zo ver dat de Cypherpunks
zich zelfs afvroegen of Dai wel echt bestond, speculerend dat de naam
wellicht een pseudoniem was; sommigen vermoedden dat het eigenlijk een
alter-ego van Nick Szabo was.

In werkelijkheid was Wei Dai, een hij, een jonge informaticus die
toevallig enkele jaren jonger was dan Szabo aan de Universiteit van
Washington, hoewel de twee elkaar nooit op de campus hebben ontmoet. Als
liefhebber van cyberpunk boeken, had Dai als student een interesse
ontwikkeld in cryptografie omdat hij geloofde dat het de mensheid kon
helpen beschermen tegen toekomstige entiteiten zoals de Blight, een
kunstmatige super-intelligentie die diende als de voornaamste antagonist
in Vernor Vinge's roman `A Fire Upon the Deep'.\footnote{Wei Dai, `Work
  on Security Instead of Friendliness?' \emph{GreaterWrong}, 21 juli
  2012,
  \href{https://www.greaterwrong.com/posts/m8FjhuELdg7iv6boW/work-on-security-instead-of-friendliness}{online}}

Zijn interesse in cryptografie leidde Dai uiteindelijk naar de
Cypherpunk-mailinglijst, waar hij kennis maakte met de talrijke
bijdragen van Tim May. Terwijl hij zich verdiepte in Mays visie op de
toekomst van de samenleving, raakte de jonge informaticus nog meer
gefascineerd door het transformatieve potentieel van cryptografie, nu
met een sterke nadruk op privacy en vrijheid van overheidsinmenging.

Dai, wiens naam aangeeft dat hij van Chinese afkomst is, begon zich
uiteindelijk te mengen in de gesprekken op de mailinglijst. In het
midden van de jaren '90 betrok Dai zichzelf bij discussies over allerlei
onderwerpen, variërend van de economie van digitale reputatiesystemen
tot de toepassing van speltheorie op het gebied van cryptografie,
voorstellen om traceerbare betalingssystemen om te zetten in anonieme,
en nog veel meer.

Gedurende die tijd nam Dai de filosofie en missie van de Cypherpunks als
zijn eigen aan.

`Er heeft nooit een regering bestaan die niet vroeg of laat probeerde de
vrijheid van haar onderdanen te verminderen en meer controle over hen te
verkrijgen, en waarschijnlijk zal er nooit een dergelijke regering zijn'
vat Dai op een bepaald moment samen wat hij beschouwt als de bindende
ethos van de beweging. `Daarom zullen we, in plaats van te proberen onze
huidige regering te overtuigen om het niet te proberen, de technologie
(bijvoorbeeld, remailers en elektronisch geld) ontwikkelen die het voor
de regering onmogelijk zal maken om te slagen.' \footnote{Wei Dai, `Law
  vs Technology', oorspronkelijk verstuurd naar de
  Cypherpunk-mailinglijst, 10 februari 1995,
  \href{https://cypherpunks.venona.com/date/1995/02/msg00508.html}{online}}

\emph{Cypherpunks schrijven code.}

Ook Wei Dai ontwikkelde een aantal tools om de Cypherpunk-zaak te
bevorderen. Dit bevatte een versleuteld tunneling-protocol (dat de
overdracht van data van het ene netwerk naar het andere mogelijk
maakte), een veilig systeem om bestanden te delen en de Crypto++
softwarebibliotheek (die vrij beschikbare cryptografische algoritmen
bevat geschreven in de programmeertaal C++). Door zijn werk in codering
en zijn doorgaans intelligente en inzichtelijke e-mails, verdiende Dai's
bijdragen hem een reputatie als een van de meest productieve en
waardevolle deelnemers aan de mailinglijst van de Cypherpunks, ondanks
zijn meer ongrijpbare persoonlijkheid.

Hoewel het op het eerste gezicht vreemd kan lijken, is het niet zo
ongelooflijk dat sommigen vermoeden dat Nick Szabo en Wei Dai eigenlijk
dezelfde persoon zijn: de twee hadden veel gemeen. Naast het volgen van
dezelfde universitaire studies en dat ze beiden deel uitmaakten van
zowel de Cypherpunk als de Extropian gemeenschappen, hadden Szabo en Dai
een bijzondere interesse in elektronisch geld, en om vele van dezelfde
redenen. Ze wilden helpen om Tim May's `Galt's Gulch in cyberspace' te
realiseren, en beiden begrepen het belang van digitale contracten in
deze context.

In november 1998, net na zijn afstuderen aan de universiteit, kondigde
Wei Dai informeel zijn eigen elektronisch geld aan. Dit voorstel werd
bijna terloops aangekondigd in een mail waarin hij ook een geüpdatete
versie van zijn anoniem communicatieprotocol, bekend als PipeNet,
bekendmaakte.\footnote{Wei Dai, `PipeNet 1.1 and b-money',
  oorspronkelijk verstuurd naar de Cypherpunk-mailinglijst, 26 november
  1998,
  \href{https://cypherpunks.venona.com/date/1998/11/msg00941.html}{online}}
Het gebeurde slechts weken nadat Szabo voor het eerst zijn digitale
munteenheid besprak op de Libtech mailinglijst. Dai, die ook actief was
op deze lijst, had b-geld gecreëerd.

`Ik ben gefascineerd door Tim May's crypto-anarchie', legde Dai zijn
motivatie uit in het voorstel. `In tegenstelling tot de gemeenschappen
die traditioneel geassocieerd worden met het woord 'anarchie', wordt in
een crypto-anarchie de regering niet tijdelijk vernietigd, maar
permanent verboden en overbodig. Het is een gemeenschap waar de dreiging
van geweld machteloos is omdat geweld onmogelijk is, en geweld
onmogelijk is omdat de deelnemers niet in verband kunnen worden gebracht
met hun daadwerkelijke namen of fysieke locaties.'

Hij concludeerde:

`Het protocol dat in dit artikel wordt voorgesteld, maakt het voor
ontraceerbare pseudonieme entiteiten mogelijk om efficiënter samen te
werken, door hen te voorzien van een ruilmiddel en een manier om
contracten af te dwingen. {[}\ldots{]} Ik hoop dat dit een stap is in de
richting van het praktisch mogelijk maken van crypto-anarchie, naast de
theoretische mogelijkheid.'\footnote{Wei Dai, untitled b-money
  description, 1998,
  \href{https://web.archive.org/web/20090415130807/https://www.weidai.com/bmoney.txt}{online}}

\section{B-geld}\label{b-geld}

B-geld leek in belangrijke opzichten op Bit Gold, hoewel het op andere
punten verschilde.

Net als Bit Gold, zou b-geld in principe bestaan op een grootboek (wat
Szabo een `register' noemde). Dit grootboek zou publieke sleutels
vermelden, en het aantal toegeschreven munteenheden aan elke publieke
sleutel aangeven. Om het elektronische contant geld te verplaatsen,
ondertekenden gebruikers cryptografisch een bericht dat aangeeft hoeveel
munteenheden er worden uitgegeven van de bijbehorende publieke sleutel,
en naar welke publieke sleutel ze werden uitgegeven. Als de transactie
geldig was (de publieke sleutel had voldoende geld en de handtekening
klopte), zou het grootboek dienovereenkomstig worden bijgewerkt.

Net zoals Szabo, legde Dai een sterke nadruk op het belang van
contracten. B-geld was ontworpen om de uitvoering van contracten te
vergemakkelijken, en een significant deel van het voorstel was gewijd
aan het uitleggen van de taken van bemiddelaars in geschillenoplossing
(hoewel het niet helemaal de autonome slimme contracten waren die Szabo
oorspronkelijk in gedachten had, waren er enige cryptografische
veiligheidsmaatregelen opgesteld om bepaalde vormen van fraude te
voorkomen).

Szabo was er ook in geslaagd om Wei Dai ervan te overtuigen dat het
minimaliseren van vertrouwen essentieel was. \emph{Vertrouwde derde
partijen zijn beveiligingsrisico's}, gaf Dai toe, en hij kwam tot de
conclusie dat een elektronisch geldsysteem niet mocht staan of vallen
met één enkele entiteit om de saldo's van gebruikers bij te houden,
transacties mogelijk te maken of dubbele uitgaven te voorkomen.

In plaats daarvan bedacht Dai twee alternatieve oplossingen.

De eerste variant van b-geld was met name zeer ambitieus. In deze
variant was er geen centrale instantie, maar onderhield \emph{elke}
gebruiker van het systeem zijn eigen exemplaar van het grootboek. Bij
elke nieuwe b-geldtransactie zou iedere gebruiker afzonderlijk de
geldigheid ervan controleren en hun eigen versie van het grootboek
bijwerken als de transactie in orde bleek. Zolang iedereen actueel
bleef, zouden de grootboeken gesynchroniseerd blijven onder alle
gebruikers.

In theorie is het grote voordeel van zo'n gedistribueerd systeem dat
corruptie onmogelijk zou zijn. Als iemand bijvoorbeeld te veel geld
toeschrijft aan zijn eigen publieke sleutels, zou dit geen enkel effect
hebben op iemand anders: alle andere grootboeken zouden onveranderd
blijven. Als de bedrieger probeerde zijn vervalste geld uit te geven,
zou niemand anders die transactie als geldig zien. Net zoals bij de
tijdstempeloplossing van Scott Stornetta en Stuart Haber zou iedereen
ervoor zorgen dat alle anderen eerlijk zou blijven.

Het leek een ideale oplossing om het grootboek over alle gebruikers te
verspreiden --- in theorie.

Helaas wist Wei Dai dat het in de praktijk niet haalbaar zou zijn. Om
dubbele uitgaven te voorkomen, vereiste het systeem een `synchroon en
onverstoorbaar anoniem uitzendkanaal'.\footnote{Dai, untitled b-money
  description.} Alleen als alle gebruikers zeker konden zijn dat zij
allemaal dezelfde transacties in precies dezelfde volgorde ontvingen,
kon iedereen er vertrouwen in hebben dat hun grootboeken
gesynchroniseerd waren en dat een betaling die zij zouden ontvangen ook
door alle anderen zou worden geregistreerd. Dit leek onwerkbaar: dubbele
uitgavetransacties konden tegelijkertijd naar verschillende delen van
het netwerk worden gestuurd, terwijl onbetrouwbare deelnemers
eenvoudigweg konden liegen over de volgorde van de transacties die ze
hadden ontvangen.

O in technische termen: Wei Dai's eerste oplossing negeerde het
Byzantijnse Generaalsprobleem.

Dit is waarom Dai in hetzelfde voorstel met een tweede oplossing kwam.

In deze tweede versie van het b-geld systeem, zou niet iedereen een
versie van het hoofdregister bijhouden, maar zou het systeem bestaan uit
twee soorten deelnemers: reguliere gebruikers en `servers'. Net zoals de
`eigendomsclub' in Szabo's Bit Gold voorstel, zouden alleen deze servers
de b-geld hoofdregisters bijhouden. Om er zeker van te zijn dat een
transactie werd voltooid, moesten reguliere gebruikers checken bij een
willekeurige subset van servers, en een betaling pas als finaal
beschouwen als die servers de transactie erkenden.

Dit introduceerde natuurlijk wel weer wat vertrouwde partijen in het
systeem. De servers konden samenwerken om overdrachten te blokkeren,
transacties dubbel uit te geven, mogelijk fondsen te stelen of zelfs
regelrecht geld voor zichzelf te creëren.

Wei Dai stelde daarom een manier voor om de servers eerlijk te houden.

`Hij stelde voor dat elke server verplicht is om een bepaald bedrag op
een speciale rekening te storten, die zou worden gebruikt als potentiële
boetes of beloningen voor bewijzen van wangedrag', stelde hij voor. `Ook
moet elke server periodiek zijn huidige database van geldcreatie en
geldbezit publiceren en zich daaraan binden. Elke deelnemer moet
controleren of zijn eigen rekeningsaldi kloppen en dat de som van de
rekeningsaldi niet groter is dan het totale bedrag aan gecreëerd geld.'

Echter, het b-geld voorstel ging niet in detail in op een van deze
oplossingen. Wellicht het meest problematisch was dat Dai niet uitlegde
wie zou bepalen of er wangedrag plaatsvond, als dit niet werd bepaald
door de (samenzwerende) servers zelf, of hoe de boetes konden worden
afgedwongen. Net zoals Bit Gold geen oplossing had geboden om conflicten
tussen servers te beslechten, had b-geld dat ook niet gedaan.

\section{Monetaire beleid van b-geld}\label{monetaire-beleid-van-b-geld}

Net als Bit Gold zou b-geld een zuiver digitale munteenheid zijn. Er zou
geen bank of bedrijf zijn die de digitale eenheden dekte met dollars of
goud, en geen garantie dat iemand de munteenheid zou accepteren voor
betaling.

Maar net zoals Szabo, dacht Dai niet dat dit een probleem zou zijn.

`Denk er op deze manier over na', betoogde hij op de mailinglijst van de
Cypherpunks. `In het geval van goederengeld, wordt de waarde deels
bepaald door de industriële/esthetische waarde van het goed en deels
door het nut van het goederengeld als ruilmiddel. In het geval van
fiatgeld en b-geld, komt alle waarde voort uit zijn nut als
ruilmiddel.'\footnote{Wei Dai, `Re: alternative b-money creation',
  oorspronkelijk verstuurd naar de Cypherpunk-mailinglijst, 11 december
  1998,
  \href{https://cypherpunks.venona.com/date/1998/12/msg00448.html}{online}}

In tegenstelling tot Szabo, wilde Dai zijn valuta echter voorzien van
een gericht monetair beleid. Waar de koopkracht van Bit Gold overgelaten
werd aan de markt, met geldige proof-of-work hashes die vrij verhandeld
worden voor wat kopers en verkopers bereid zijn te accepteren, was
b-geld specifiek ontworpen om een voorspelbare koopkracht te bieden.

Net als Irving Fisher en de Vereniging voor Stabiel Geld ongeveer 80
jaar eerder, stelde Dai voor om de koopkracht van zijn munteenheid te
koppelen aan een consumentenprijsindex. Hij wilde dat dezelfde
hoeveelheid b-geld-eenheden op elk moment een gelijk aandeel van deze
index kon kopen. Met andere woorden, de gemiddelde prijs van goederen en
diensten, uitgedrukt in b-geld, moest stabiel blijven.

Als we het breder bekijken, zou het creëren van valuta in b-geld
vergelijkbaar werken met Bit Gold: iedereen zou nieuwe valuta-eenheden
kunnen genereren door middel van een proof-of-work door een geldige hash
te produceren, vermoedelijk ook gebaseerd op een bepaalde
kandidaat-reeks. Wie de hash creëerde, mocht deze houden (of misschien
zouden ze een equivalent in b-geld-eenheden krijgen; een geldig
proof-of-work zou heel goed op de grootboek kunnen worden weerspiegeld
als 100 digitale `munten'.).

Het belangrijkste verschil met Bit Gold, was echter dat de
moeilijkheidsgraad om een geldig proof-of-work te genereren, kon
veranderen.

Alle gebruikers van het systeem (b-geld versie 1) of de servers (b-geld
versie 2) zouden voortdurend moeten bepalen hoeveel een mandje goederen
zou kosten in verhouding tot de productie van een geldige hash. Dat wil
zeggen, als het creëren van een hash goedkoper wordt (door verbeteringen
in computerhardware) in verhouding tot de prijsindex, dan zou de
moeilijkheidsgraad om een geldige hash te produceren naar boven moeten
worden bijgesteld: de hash zou met meer nullen moeten beginnen. Een
nieuwe hash zou dan alleen aan het grootboek worden toegevoegd als aan
de meest recente drempelwaarde was voldaan.

Wei Dai vermeldde in de bijlage van zijn voorstel ook een alternatieve
benadering om een vergelijkbaar resultaat te bereiken. Het aanmaken van
geld kon gebeuren via een veiling. In dit geval zouden alle gebruikers
(b-geld versie 1) of de servers (b-geld versie 2) eerst de optimale
uitbreiding van de geldvoorraad bepalen, waarna deze nieuwe eenheden
b-geld geveild zouden worden aan degene die bereid en in staat was om
het met het meeste proof-of-work te betalen.

Een grote voordeel van deze benaderingen was dat alle b-geld hashes,
ongeacht wanneer ze werden gecreëerd, dezelfde waarde moesten hebben: ze
zouden fungibel zijn. Dit elimineerde de noodzaak om een hele andere
bankenlaag te ontwerpen bovenop de basislaag van de valuta, zoals Szabo
had voorgesteld voor Bit Gold.

Het was een innovatieve aanpak, maar opnieuw bleef er veel
ongespecificeerd. Zowel voor de aanpassingsmethode van de
moeilijkheidsgraad als het veilingmodel, bleef het in het voorstel van
b-geld onduidelijk hoe gebruikers (of servers) zouden beslissen over de
volgende moeilijkheidsgraad voor proof-of-work, of de optimale toename
van de geldvoorraad\ldots{} en hoe geschillen in dit deel van het proces
zouden kunnen worden opgelost (het Byzantijnse Generaalsprobleem bleef
de kop opsteken).

`B-geld was nog geen volledig ontwerp om in de praktijk te brengen',
erkende Dai later.\footnote{Wei Dai, comment in the discussion thread
  `AALWA: Ask any LessWronger anything', \emph{LessWrong}, 2014,
  \href{https://www.lesswrong.com/posts/YdfpDyRpNyypivgdu/aalwa-ask-any-lesswronger-anything}{online}}
Het voorstel bood een ruwe schets van hoe een elektronisch geldsysteem
eruit kon zien, maar er waren nog meerdere problemen die moesten worden
opgelost voordat het als een daadwerkelijke digitale valuta zou kunnen
functioneren.

Dai zelf besloot echter dat hij niet degene zou zijn die deze problemen
zou oplossen.

\section{Ontgoocheling}\label{ontgoocheling}

In zijn voorstel voor b-geld leek Wei Dai nog altijd optimistisch over
de mogelijkheden en het potentieel van Tim May's crypto-anarchistische
visie. Maar in werkelijkheid was de ongrijpbare informaticus de droom
van de Cypherpunks aan het opgeven.

`Ik ben niet verder gegaan met het ontwerpen omdat ik, tegen de tijd dat
ik klaar was met het opschrijven van b-geld, eigenlijk al wat
gedesillusioneerd was geraakt door crypto-anarchie', herinnerde Dai zich
later. `Ik had niet voorzien dat zo'n systeem, eenmaal geïmplementeerd,
zoveel aandacht en gebruik zou aantrekken buiten een kleine groep van
\emph{hardcore} cypherpunks.'\footnote{Dai, comment.}

Dai's ontgoocheling weerspiegelde een groeiend sentiment binnen de
Cypherpunk-gemeenschap tegen het einde van de jaren 90. Het internet was
inmiddels echt mainstream geworden, maar de Cypherpunks ontdekten dat
het grote publiek nogal onverschillig was over online privacy. Het leek
erop dat de meeste mensen er geen probleem mee waren om
betalingsverwerkers volledig inzicht te geven in hun uitgavenpatroon, en
ze leken er ook geen probleem mee te hebben om een
\hspace{0pt}\hspace{0pt}spoor van hun andere online activiteiten achter
te laten. De gemiddelde internetgebruiker dacht zelfs niet eens aan het
versleutelen van hun e-mails.

Sinds de Cypherpunks voor het eerst bijeen kwamen in het ongemeubileerde
appartement van Eric Hughes, hadden ze bijna tien jaar besteed aan het
transformeren van revolutionaire crypto-protocollen in werkende
software. Tot hun grote teleurstelling bleek dat bijna niemand hierin
geïnteresseerd was. Het doorzettingsvermogen in hun activisme voor
privacy had hen weliswaar geholpen de crypto-oorlogen te winnen, maar
dit leek nu een tamelijk nutteloze inspanning te zijn geweest. De meeste
internetgebruikers bleken namelijk perfect op hun gemak met het opgeven
van zowat al hun persoonlijke informatie in ruil voor iets meer gemak.

Hughes had zich inmiddels vrijwel volledig teruggetrokken uit de
gemeenschap en de mailinglijst. Maar niet voordat hij zijn nuchtere
herbeoordeling van de `cypherpunks schrijven code'-filosofie had
aangeboden, kenmerkend voor de recente desillusie van Wei Dai, hemzelf
en andere Cypherpunks.

`Misschien het belangrijkste wat ik van de cypherpunks heb geleerd is
dat alleen code niet voldoende is. Niet alleen code, niet wijdverspreide
code, zelfs niet veelgebruikte code', schreef Hughes zich richtend tot
de Cypherpunk-mailinglijst. `Voor langdurig succes is een zekere mate
van tolerantie in de samenleving nodig voor activiteiten die in privé
worden ondernomen. Niet alleen gemakkelijk of makkelijker, maar
noodzakelijk.'\footnote{Eric Hughes, `Kid Gloves or Megaphones',
  oorspronkelijk verstuurd naar de Cypherpunk-mailinglijst, 14 maart
  1996,
  \href{https://cypherpunks.venona.com/date/1996/03/msg00932.html}{online}}

Hughes was tot het inzicht gekomen dat het essentieel was dat het grote
publiek zou begrijpen waarom privacy belangrijk was. Code was uiteraard
ook nog steeds nodig --- code maakte privacy in eerste instantie
mogelijk. Maar hij geloofde nu dat code uiteindelijk alleen maar nuttig
was als er een brede publieke consensus bestond dat mensen daadwerkelijk
het recht zouden moeten hebben om hun privacy te beschermen. Zonder zo'n
publieke consensus zou het gebruik van cryptografie kunnen worden
gemarginaliseerd en wellicht zelfs verboden, met als risico dat de
overblijvende gebruikers mogelijk zouden worden geviseerd en vervolgd.

`Hetzelfde geldt voor anonieme transacties', schreef Hughes. `Tenzij er
een soortgelijke consensus bestaat, zullen we weer te maken krijgen met
een marginale activiteit. Ik beschouw dit als een verlies.'\footnote{Hughes,
  `Kid Gloves.'}

Het optimisme en de assertiviteit die de beweging in de begindagen
kenmerkten, werden steeds meer overschaduwd door een gevoel van
somberheid en verlatenheid.

In plaats daarvan kwamen sommige van de meer hoopvolle impulsen voor de
Cypherpunk-missie in de late jaren '90 van relatieve buitenstaanders van
de gemeenschap.

\section{Zero-Knowledge Systems}\label{zero-knowledge-systems}

De Canadese broers Austin en Hamnett Hill waren nog maar halverwege de
twintig toen ze TotalNet, de internetprovider die ze hadden opgericht en
die de derde grootste van hun land werd, verkochten. Met wat geld in hun
zak en de tijd om het uit te geven, zochten de twee naar hun volgende
project toen ze op de Cypherpunk-mailinglijst stuitten. Ze raakten
volledig in de ban van het techno-libertarische ethos van de beweging.

In 1997 besloten de twee broers, samen met hun vader Hammie Hill, om hun
middelen, connecties, en zakelijke talenten in te zetten, en richtten ze
\emph{Zero-Knowledge Systems} op. Het nieuwe bedrijf nam zich voor om de
visie van de Cypherpunks werkelijkheid te laten worden, en om er
tegelijkertijd ook wat geld mee te verdienen.

De kernactiviteit van de start-up was een privacy-netwerk dat ze
\emph{Freedom} noemden. Freedom was gebaseerd op Wei Dai's PipeNet, het
anonieme communicatieprotocol waarvan een bijgewerkte versie zou worden
aangekondigd in dezelfde Cypherpunks-mailinglijstpost die b-geld
introduceerde. Net als PipeNet, werden in Freedom's
verhullingstechnieken een geavanceerdere variant van David Chaum's
originele remailer-protocol ingebed, maar waar remailers alleen e-mails
anonimiseerden, paste Freedom de mix-technologie toe om alle soorten
internetgegevens te verhullen: e-mails, surfen, tekst-chat en meer.

Gebruikers van Freedom konden in wezen op het internet `inloggen' onder
verschillende identiteiten: misschien een reguliere identiteit voor
professioneel werk, een pseudonieme identiteit voor politieke
betrokkenheid, en nog een pseudoniem voor seksueel getinte
webactiviteiten. Niemand, zelfs Zero-Knowledge Systems niet, zou in
staat zijn de pseudonieme online identiteiten te koppelen aan een echte
identiteit, of aan andere pseudoniemen.

De start-up wekte behoorlijk wat interesse op binnen de
Cypherpunk-gemeenschap en, belangrijker nog, de oprichters van
Zero-Knowledge Systems wisten de durfkapitalisten uit te leggen waarom
ze de kans niet mochten missen om te investeren in \emph{de toekomst van
privacy}. Binnen een paar jaar slaagde de start-up erin om tientallen
miljoenen dollars op te halen.

Waarschijnlijk konden de Hills beter verkopen dan de meeste Cypherpunks.
Ze wisten ook op een goede manier de ambitieuze doelen van
Zero-Knowledge Systems aan het grote publiek te presenteren. Er
verschenen aantrekkelijke printadvertenties in bekende tijdschriften
zoals \emph{Wired, Forbes en Fortune}, met teksten als `Ik ben geen stuk
van je inventaris', `Ik ben een individu en je zult mijn privacy
respecteren' en `Op het net heb ik de controle'. Bijzonder scherpe
lezers konden ook een verborgen boodschap ontrafelen uit een binaire
code op de pagina's, die zich vertaalde naar `Wie is John Galt?' --- een
beroemde zin uit Ayn Rand's \emph{Atlas Shrugged}.

En wellicht nog het belangrijkst van alles, Zero-Knowledge Systems wist
het beste talent in de privacy-sector aan te trekken. Enkele van de
bekendste cryptografen en computerwetenschappers op de
Cypherpunk-mailinglijst besloten zich bij de start-up te voegen,
waaronder Ian Goldberg -- die tijdens de crypto-oorlogen het SSL
crypto-protocol van Netscape had gebroken en het bedrijf als
`Hoofdwetenschapper en Hoofd Cypherpunk' zou dienen -- en Adam Back.
Stefan Brands, de uitvinder van Brands Cash, werd ook aangenomen en zijn
patenten op elektronisch geld werden eveneens door de start-up gekocht.

De Hills waren zeker niet kort van ambitie. Freedom was het hoofdproject
van het bedrijf, maar Zero-Knowledge Systems wilde uiteindelijk de
idealen van Cypherpunks breed realiseren. Het onderzoeks- en
ontwikkelingsteam van de start-up, ook wel de `Kwaadaardige Genieën'
genoemd, kreeg de taak om aanvullende producten te ontwerpen. Dit
omvatte onder andere een elektronisch geldsysteem gebaseerd op het
ontwerp van Brands, met de codenaam Zorkmid (een verwijzing naar de
munteenheid van een vroeg online spel).

Oondanks alles, leek één probleem echter aan te houden. De meeste
internetgebruikers gaven gewoon niet veel om privacy.

Zero Knowledge Systems had het plan om tegen 2000 zo'n 2,5 miljoen
gebruikers van Freedom aan te kunnen. Maar hoewel de 250 medewerkers
zich flink hebben ingezet om dit mogelijk te maken, waren er rond de
eeuwwisseling slechts iets meer dan twaalfduizend actieve gebruikers op
het netwerk, of minder dan een procent van het oorspronkelijke doel.

Deels kwam dit doordat veel mensen moeite hadden met het installeren van
de software, maar zelfs degenen die erin slaagden Freedom operationeel
te krijgen, ontdekten dat hun internetsnelheid aanzienlijk afnam bij het
gebruik van de service. Buiten een relatief technisch onderlegde kern
van gebruikers (voornamelijk mannen tussen de 25 en 35 jaar) waren
weinig mensen bereid dit voor lief te nemen, en daarnaast nog de
jaarlijkse vergoeding van € 50 aan Zero-Knowledge Systems te betalen.
Aan het einde van de rit zagen mensen gewoonweg geen reden om Freedom te
gebruiken: de voordelen voor de privacy waren voor hen onzichtbaar.

`Of in de woorden van Austin Hill, een paar jaar later: 'Iedereen
beweert te geven om privacy, maar men zou zo een DNA-staal afstaan voor
een 'gratis' Big Mac.'\footnote{Austin Hill, ontwerpdocument van 2005
  gedeeld met de auteur, 30 maart 2022.}'

Om het bedrijf te redden, heeft Zero Knowledge Systems uiteindelijk haar
strategie veranderd. In plaats van op de algemene internetgebruiker te
focussen, zou de start-up vanaf 2001 haar inspanningen richten op
gevestigde bedrijven, zoals financiële instellingen en telecombedrijven.
Ze boden hen beveiligde database- en communicatiesystemen aan. Tot grote
ontsteltenis van zijn kleine, maar toegewijde gebruikersbasis werd
Freedom stopgezet.

Hiermee liet de start-up het Cypherpunk-ethos grotendeels achter zich,
en personen zoals Back en Brands vertrokken kort daarna. Toen het
bedrijf uiteindelijk zijn naam veranderde in Radialpoint, was Zero
Knowledge Systems in alle opzichten vervangen door een totaal ander
IT-bedrijf.

Ondanks een hoopvol begin, was dit een nieuwe tegenslag voor het doel
van de Cypherpunks.

\section{Mojo Nation}\label{mojo-nation}

Een andere positieve impuls kwam van twee tot dusver onbekende tieners
die een compleet ander deel van de cyberspace ontregelden: Met de
lancering van Napster in 1999 gooiden Shawn Fanning en Sean Parker,
achttien en negentien jaar oud respectievelijk, de digitale variant van
een handgranaat recht in het hart van de muziekindustrie.

Napster was een krachtig idee om één specifieke, technische reden. In
tegenstelling tot de meeste internetdiensten tot dat moment, die
vertrouwden op een centrale server om gebruikers te voorzien van wat ze
nodig hadden, was Napster ontworpen als een \emph{peer-to-peer} (P2P)
netwerk. De peers (Napster-gebruikers) op het Napster-netwerk fungeerden
als gelijken, ze hielpen elkaar waar nodig: specifieker, ze deelden hun
eigen muziekbestanden met elkaar. Omdat Napster zelf geen
muziekbestanden verspreidde, dachten Fanning en Parker dat ze claims
wegens inbreuk op auteursrechten konden omzeilen, terwijl gebruikers nog
steeds gratis nummers konden downloaden.

Maar toen de populariteit van Napster ontplofte, lanceerde de
muziekindustrie een succesvolle tegenaanval. Fanning en Parker deelden
misschien zelf geen muziek, maar artiesten en platenmaatschappijen
beweerden dat de service desondanks actief inbreuk maakte op het
auteursrecht: Napster bood gebruikers een platform, het beheerde en
bewaarde de indexen om alle muziekbestanden te vinden, en de service
koppelde peers dienovereenkomstig. Al snel bezweken Fanning en Parker
onder de enorme juridische druk, en in juli 2001 haalden ze Napster
offline.

Uiteindelijk was Napster een kortstondig project. Maar in een paar jaar
tijd populariseerden ze de P2P-technologie, die een hele nieuwe klasse
van vernieuwers inspireerden. Alternatieve diensten om bestanden te
delen zoals Kazaa en eDonkey doken al snel op, elk van hen ontworpen om
nog decentraler te zijn dan de creatie van Fanning en Parker. Gedurende
de volgende paar jaar waren de makers van deze nieuwe protocollen
betrokken bij een hoogtechnologisch kat-en-muis-spel met de platenlabels
die probeerden hun projecten lam te leggen.\footnote{Ori Brafman and Rod
  A. Beckstrom, `The Starfish and the Spider: The Unstoppable Power of
  Leaderless Organizations', 22--27.}

De eenendertigjarige Cypherpunk, Jim McCoy, besloot dat hij ook wilde
meespelen. Begin 2000 nam hij ontslag bij Yahoo -- `Ik werd het zat om
niets revolutionairs te doen'\footnote{Damien Cave, `The Mojo solution',
  \emph{Salon}, 9 oktober 2000,
  \href{https://www.salon.com/2000/10/09/mojo_nation/}{online}} -- en
samen met verschillende andere Cypherpunks, waaronder DigiCash-alumnus
Bryce `Zooko' Wilcox, richtte McCoy Autonomous Zone Industries op.

De naam van het bedrijf was geïnspireerd op `tijdelijke autonome zones',
een term die voor het eerst werd gebruikt in 1991 door anarchist Hakim
Bey om niet-permanente, lokale samenlevingen te beschrijven die vrij
zijn van de overheid. De start-up zou een ambitieus open-source
softwareproject genaamd Mojo Nation ontwikkelen. Net als Napster was
Mojo Nation in essentie een P2P \emph{file sharing system}. Maar McCoy,
als ervaren Cypherpunk, had een paar extra tools in zijn
crypto-gereedschapskist om het ontwerp van Fanning en Parker te
verbeteren.

Een van Mojo Nation's meest interessante innovaties was dat alle
bestanden op het netwerk in kleine stukjes werden opgedeeld, gecodeerd
en strategisch gekopieerd en verspreid over het netwerk. Als iemand een
bestand ging downloaden, downloadden ze in feite al deze kleine
gecodeerde stukjes van verschillende gebruikers over het netwerk, om
uiteindelijk deze puzzelstukjes bij elkaar te brengen en in één keer het
volledige bestand te decoderen. Omdat alle uploaders slechts een beetje
bandbreedte nodig hadden om hun stukje te delen, kon de downloadsnelheid
worden verhoogd. Dit stelde Mojo Nation-gebruikers in staat om grotere
bestanden te delen dan de typische MP3's. Bovendien bood het meer
privacy: gebruikers die de gecodeerde stukjes deelden, wisten vaak niet
wat voor soort inhoud ze deelden (of deze inhoud al dan niet
auteursrechtelijk beschermd was).

Daarnaast werden sommige taken die Napster nog steeds als een centrale
coördinator uitvoerde, in Mojo Nation overgedragen aan de gebruikers.
Gebruikers die als archivaris fungeerden zouden bijvoorbeeld de indexen
van bestanden die op het netwerk werden gehost bijhouden, terwijl andere
gebruikers, die als zoekagenten fungeerde, zoekopdrachten via deze
indexen zouden aanbieden. Door dergelijke verantwoordelijkheden in de
handen van gebruikers te leggen, geloofde McCoy dat Mojo Nation niet
vatbaar zou zijn voor het soort rechtszaken waarmee Napster te maken had
gehad. In plaats daarvan zouden de gebruikers zelf verantwoordelijk zijn
als ze wetten van hun jurisdictie overtraden --- maar al deze
individuele mensen waren natuurlijk veel moeilijker te vinden dan
Fanning en Parker.

Wat dit alles liet draaien was wellicht het meest interessante element
van Mojo Nation: een digitale valuta genaamd \emph{Mojo}.

\section{Mojo}\label{mojo}

Mojo is ontworpen als een ongedekte digitale valuta die eigenlijk alleen
nuttig was binnen de context van het bestandsuitwisselingsnetwerk.

Specifiek had Mojo de taak om een markt voor bestandsdeling en andere
taken mogelijk te maken. Waar Napster-gebruikers hun eigen bestanden
gratis deelden, konden Mojo Nation-gebruikers elkaar betalen voor de
service, en de prijzen zouden worden bepaald door vraag en aanbod.
Iemand zou bijvoorbeeld kunnen aanbieden om 1.000 Mojo te betalen voor
elk gecodeerd deel van een bestand dat weer in elkaar gezet kan worden
als een DVD-rip van \emph{The Matrix}: wie een of meerdere van deze
gecodeerde delen had, kon het aanbod accepteren als ze dachten dat het
hun tijd, moeite en bandbreedte waard zou zijn om ze te uploaden. De
verdiende Mojo's konden vervolgens worden gebruikt om andere diensten op
het netwerk te kopen, of misschien in te ruilen voor dollars op een
speciale Mojo-handelsbeurs.\footnote{Als een interessant detail werden
  de meeste Mojo-transacties aanvankelijk alleen tussen twee peers
  geregistreerd, waarbij elke peer krediet of schuld opbouwde bij de
  ander. De schuld werd pas vereffend met een daadwerkelijke Mojo-token
  wanneer een bepaalde drempel werd bereikt.}

`De mensen die betaald krijgen, zijn degenen die de diensten uitvoeren.
Dus die agenten die je hebben geholpen om dat blok {[}bestand{]} te
vinden, worden betaald', legde McCoy uit. `De verspreide zoekagenten
krijgen betaald. Alle verschillende blokservers waar je blokken van hebt
gekocht krijgen betaald, en als de gebruiker via een \emph{relay-server}
werkte, hetzij omdat ze achter een firewall zaten of omdat ze hun
privacy wilden beschermen, zou de persoon die berichten doorgaf ook een
deel van de betaling ontvangen.'\footnote{Cave, `The Mojo solution.'}

McCoy's visie was dat heel Mojo Nation zou worden gestuurd door
marktprocessen, waarbij de ene gebruiker's probleem de volgende
gebruiker's gelegenheid was om wat geld te verdienen door het op te
lossen. Dit zou het Mojo Nation-netwerk quasi-autonoom laten
functioneren, hoopte de Cypherpunk, met zeer weinig dagelijkse
betrokkenheid van Autonomous Zone Industries.

Een opvallende uitzondering op deze regel bestond echter wel. De
munteenheid van Mojo Nation werd beheerd door Autonomous Zone Industries
\emph{zelf}, via een speciale tokenserver die de rekeningsaldi bijhield
en dubbele uitgaven voorkwam. Bovendien functioneerde de server als een
gecentraliseerde muntfabriek: het kon nieuwe Mojo uitgeven wanneer McCoy
en zijn collega's geloofden dat dit nodig was, zonder technische
beperking op hoeveel ervan gecreëerd kon worden.

Het resulteerde uiteindelijk in de vernietiging van de munteenheid. Toen
sommige gebruikers slimmigheidjes ontdekten om anderen te bedriegen om
hun munten naar hen te sturen, besloot het team van Mojo Nation de
slachtoffers te compenseren met nieuw geld. Dit leidde uiteindelijk tot
de uitgifte van zoveel Mojo dat het uiteindelijk resulteerde in
hyperinflatie. Mojo was afhankelijk geweest van een vertrouwde derde
partij -- de muntfabriek -- en dat vertrouwen was geschonden.

Om iets als Mojo Nation echt te laten werken, had het waarschijnlijk een
onafhankelijke digitale munteenheid nodig.

' {[}\ldots{]} we bestudeerden MojoNation kritisch, omdat ons hoofddoel
een werkende gemeenschapsmunt voor p2p-diensten was, en tot op zekere
hoogte nog steeds is', schreef informaticus Daniel A. Nagy kort na het
einde van Mojo Nation aan Jim McCoy. `Als reden voor het falen, wezen we
hyperinflatie aan. MN had geen inflatiebeperkende maatregelen en op den
duur leidde dit ertoe dat de Mojo geheel werd geïnflateerd.' Hij voegde
eraan toe dat `Ik geloof in de visie dat de wereld dringend behoefte
heeft aan een p2p-cashsysteem. Zonder zo'n systeem zal e-commerce een
grote PITA blijven.' \footnote{Daniel A. Nagy, opmerkingen als antwoord
  op `The Mojo Nation Story --- Part 2', \emph{Financial Cryptography},
  12 oktober 2005,
  \href{https://www.financialcryptography.com/mt/archives/000572.html}{online}}

Dat gezegd zijnde, was het vertrouwen op een gecentraliseerd digitaal
valutasysteem niet het enige probleem waarmee Mojo Nation te kampen had.
Hoewel de software door meer dan 100.000 mensen werd gedownload en
gebruikt,\footnote{Bryce Wilcox-O'Hearn, `Experiences Deploying A
  Large-Scale Emergent Network', Peer-to-Peer Systems: 104--110.} bleken
meerdere onderdelen van het systeem erg moeilijk om draaiende te krijgen
(en te houden). Met problemen variërend van netwerkinstabiliteit tot
ontbrekende bestandsfracties en een gebrek aan vertrouwen tussen
gebruikers,\footnote{Wilcox-O`Hearn, `Large-Scale Emergent Network'.}
was de dienst waarschijnlijk te ambitieus voor het bescheiden budget van
Autonomous Zone Industries: het bedrijf zat binnen een paar jaar door
zijn geld heen en de negatieve publiciteit rond Napster maakte het
moeilijk om meer financiering te verkrijgen.

In 2002 zag McCoy zich gedwongen om de meeste werknemers te ontslaan.

\section{BitTorrent}\label{bittorrent}

Hoewel Mojo Nation ten onder ging, waren er enkele ontwikkelaars bij de
start-up die hun vooruitstrevende technologieën niet wilden laten
vergaan. Zo besloot Wilcox bijvoorbeeld de code van Mojo Nation te
kopiëren (forken) om een versie van het protocol genaamd Mnet uit te
brengen. Daarnaast bracht ook een andere medewerker van Autonomous Zone
Industries, de 28-jarige software-ontwikkelaar en Cypherpunk Bram Cohen,
zijn eigen op Mojo Nation geïnspireerde bestandsuitwisselingsnetwerk
uit.

Hij noemde het: BitTorrent.

Cohen had Mojo Nation in feite tot op de bot gestript. BitTorrent nam
sommige van McCoy's ideeën over, zoals het opdelen van bestanden in
kleinere fracties. Maar verder was het protocol vrij eenvoudig: er waren
geen ingebouwde archivaris (indexen, torrent-bestanden genoemd, werden
buiten het protocol onderhouden en verspreid), er waren geen zoekagenten
(gewone websites, opnieuw buiten het protocol, konden gebruikers helpen
specifieke torrent-bestanden te vinden), en er was geen eigen valuta.

BitTorrent had geen eigen munteenheid nodig, omdat niemand hoefde te
betalen voor bestanden. In plaats daarvan uploaden gebruikers, die de
verschillende delen van een bestand downloaden, deze delen gelijktijdig
naar andere downloaders. Dit betekende dat bestanden technisch gezien
altruïstisch gedeeld werden, maar op zo'n manier dat de last op
hulpbronnen voornamelijk werd gedragen door degenen die ook profiteerden
van de bestandsoverdrachten.

En zo had Cohen een echt peer-to-peer en volledig gedistribueerd
bestandsoverdrachtprotocol ontworpen. Waar Napster's P2P-netwerk
effectief kon worden uitgeschakeld door juridische druk uit te oefenen
op het bedrijf erachter en zelfs het veel ambitieuzere Mojo Nation niet
kon functioneren zonder dat de Autonomous Zone Industries een
valutasysteem voor het netwerk onderhield, was BitTorrent niet van enige
betrouwbare derde partij afhankelijk.

Vanuit een juridisch oogpunt waren gebruikers nu voor het eerst volledig
verantwoordelijk voor hun eigen file sharing activiteiten. Net als
e-mail (SMTP) of zelfs het internet zelf (IP), was BitTorrent in wezen
slechts een internetprotocol. Bram Cohen was op geen enkele wijze
aansprakelijk voor hoe mensen het protocol gebruikten, zelfs al werd er
op grootschalige wijze illegaal auteursrechtelijk beschermde bestanden
uitgewisseld via BitTorrent.

Bovendien, als Cohen om welke reden dan ook legale druk zou ondervinden,
zouden noch hijzelf, noch het later door hem opgerichte
BitTorrent-bedrijf op technisch niveau controle over het
BitTorrent-netwerk kunnen uitoefenen. Hoewel Cohen de software initieel
creëerde, werd deze bediend door mensen over de hele wereld. Het netwerk
werd al snel bijna onmogelijk om te censureren en zo goed als onstopbaar
--- zelfs de maker kon dit niet veranderen.

BitTorrent zou zich in de volgende jaren vestigen als de standaard voor
bestandsoverdrachten. Ongeveer een decennium na Cohen's eerste
software-release, in het begin van de jaren 2010, had het protocol op
elk moment van de dag minstens vijftien miljoen gelijktijdige
gebruikers,\footnote{Liang Wang, `BitTorrent Mainline DHT Measurement',
  \emph{MLDHT}, 2013,
  \href{https://www.cl.cam.ac.uk//~lw525/MLDHT/}{online}} en in een
typische maand waren er wereldwijd zo'n 150 miljoen mensen verbonden met
het netwerk.\footnote{BitTorrent, `BitTorrent and µTorrent Software
  Surpass 150 Million User Milestone; Announce New Consumer Electronics
  Partnerships', BitTorrent.com, 9 januari 2012,
  \href{https://web.archive.org/web/20140326102305/http://www.bittorrent.com/intl/es/company/about/ces_2012_150m_users}{online}}
Alles bij elkaar opgeteld, werd geschat dat BitTorrent-gebruikers
verantwoordelijk waren voor zo'n 25 tot 30 procent van al het
internetverkeer in de wereld, wat meer was dan enig ander protocol in
die tijd.\footnote{Hendrik Schulze and Klaus Mochalski, `Internet Study
  2008/2009', ipoque.}

Zonder een centrale entiteit die ze nog konden aanklagen, hadden
muziekartiesten en platenmaatschappijen weinig andere keuze dan zich ook
aanpassen aan de nieuwe realiteit. In plaats van te proberen hun muziek
van het internet te verwijderen, verschoven ze uiteindelijk hun
inspanningen om te concurreren met diensten om bestanden te delen door
hun nummers gemakkelijk beschikbaar te maken via handige
softwaretoepassingen (zoals Apple's iTunes) en later, streamingdiensten.
Slechts een paar jaar na de introductie van BitTorrent zou het kopen van
een fysieke cd (of zelfs het bezit van muziek in het algemeen) ouderwets
lijken.

Misschien had deze kennis de doorgewinterde Cypherpunks aan het einde
van de jaren 1990 enige hoop kunnen bieden. Een van `hun' technologieën
zou niet alleen de wereld veroveren, maar, nog relevanter: Cohen's code
revolutioneerde hoe mensen het internet gebruikten en ernaar keken. Dit
dwong uiteindelijk een volledige transformatie van de
entertainmentindustrie af.

Waar Wei Dai, Eric Hughes, en andere Cypherpunks dachten dat
elektronisch geld en andere crypto-tools alleen succesvol zouden zijn
als het publieke bewustzijn over het belang van online privacy toenam,
zou BitTorrent jaren later aantonen dat het ook andersom kon werken: een
krachtige genoeg technologie kon, op zichzelf, helpen de heersende
cultuur te veranderen.

\chapter{RPOW}\label{rpow}

In het begin van de jaren 2000 had de Cypherpunk-beweging het grootste
deel van zijn momentum verloren.

Terwijl sommige van de oorspronkelijke Cypherpunks gedesillusioneerd
raakten en stopten met deelnemen aan de Cypherpunk-mailinglijst, ging de
algehele kwaliteit van dit discussieplatform achteruit, met veel nieuwe
berichten die weinig meer waren dan gescheld en schreeuwpartijen, of
zelfs regelrechte spam. John Gilmore, de oorspronkelijke host van de
lijst, had eind jaren 1990 geprobeerd om een moderatiebeleid te
introduceren, maar dit werd streng afgewezen door mensen zoals Tim May,
die zich in reactie daarop uitschreef (May keerde terug toen het beleid
werd aangepast om boze berichten en andere inhoud van lage kwaliteit om
te leiden in plaats van te censureren, hoewel hij nog steeds niet blij
was met de veranderingen).\footnote{Tim May, `My Departure, Moderation,
  and 'Ownership of the List', oorspronkelijk verstuurd naar de
  Cypherpunk-mailinglijst, 2 februari 1997,
  \href{https://cypherpunks.venona.com/date/1997/02/msg02898.html}{online}}

Gilmore besloot uiteindelijk te stoppen met het hosten van de
mailinglijst, waarna enkele van de overgebleven abonnees nieuwe groepen
creëerden en overstapten naar Usenet, die op een meer verspreide manier
konden worden gehost door meerdere mensen tegelijk.\footnote{\hspace{0pt}Mark
  Frauenfelder, `Homeless Cypherpunks Turn to Usenet', \emph{Wired}, 17
  februari 1997,
  \href{https://www.wired.com/1997/02/homeless-cypherpunks-turn-to-usenet/}{online}}
Desalniettemin, zou het verval van de beweging alleen maar versnellen.
Na de terroristische aanslagen van 11 september 2001, maakte een scherpe
toename van digitale surveillance mensen huiverig om discussies over
radicale privacytools te faciliteren, en toen de enige resterende
Cypherpunk-server werd gehost vanaf het webadres al-qaeda.net, besloot
zelfs Tim May dat het tijd was om te vertrekken. Dit keer, voorgoed.

Dat betekende niet dat de Cypherpunk-ethos verloren of volledig vergeten
was. Velen van de cypherpunks behielden hun interesse in bestaande
privacytools zoals PGP, evenals in nieuwe technologieën zoals Tor (The
Onion Router): het privacy-netwerk dat in 2002 werd gelanceerd leek op
Zero-Knowledge Systems's Freedom, maar vereiste geen betaald abonnement.
Tor stelde iedereen in staat om anoniem het internet te gebruiken.

Veel van de Cypherpunks bleven ook in contact via andere middelen.
Online migreerden nogal wat van hen uiteindelijk naar de strikter
gemodereerde Cryptografie mailinglijst, die soms werd beschouwd als de
feitelijke opvolger van de Cypherpunk-lijst. Offline liepen enkele van
de Cypherpunks elkaar regelmatig tegen het lijf op
cryptografieconferenties of hacker-evenementen.

Ondertussen ontstonden vele andere initiatieven voor elektronisch geld.
Rond de overgang van het millennium werkten honderden start-ups aan
online betalingssystemen, en veel van deze bedrijven omschreven hun
oplossingen als een vorm van digitaal geld. Hoewel dat vaak gewoon
betekende dat de betaalsystemen snel, goedkoop en gemakkelijk te
gebruiken waren, waren ze niet per se privé. CyberCash, bijvoorbeeld,
trok veel media-aandacht voor zijn digitaal geldsysteem, genaamd
CyberCoin, dat zich specialiseerde in kleine betalingen in plaats van
anonimiteit. Hetzelfde gold voor het systeem van elektronisch geld van
Compaq, dat veel aandacht trok, genaamd Millicent.

Andere initiatieven die enig potentieel toonden, zoals n-Count
(medeontworpen door een voormalige werknemer van DigiCash), Proton (een
project van samenwerkende Europese banken), of Mondex (een initiatief
van de Britse bank NatWest dat later werd verkocht aan Mastercard),
waren vooral gebaseerd op het concept van fysieke smartcards. Net zoals
de smartcard die in ontwikkeling was bij de start-up van David Chaum,
moesten deze stukken hardware (gelijkend aan een kredietkaart) vooraf
geladen worden met een waarde die fiatvaluta vertegenwoordigde, om
vervolgens te worden gebruikt voor persoonlijke transacties. Hoewel de
meeste van deze privacyfuncties aanboden, waren ze vooral ontworpen om
fysiek geld te vervangen in plaats van te dienen als anonieme valuta
voor cyberspace.

Misschien nog dichter bij de visie van de Cypherpunks, richtte Robert
Hettinga, die sinds 1996 de jaarlijkse Financial Cryptography
conferenties had georganiseerd, in 1999 de Internet Bearer Underwriting
Corporation op. Na het falen van DigiCash, wilde deze Cypherpunk
financiering veiligstellen om een nieuw soort eCash-systeem te
ontwikkelen, maar deze keer geoptimaliseerd voor goedkope transacties.
Hij was van mening dat sterke privacygaranties niet alleen individuen
beschermen tegen Big Brother, maar dat ze ook de wrijving kunnen
verminderen en dus economische voordelen kunnen opleveren.

Maar geen van deze projecten heeft haar beloftes kunnen waarmaken.
Hoewel sommige technologieën banenbrekend waren in specifieke sectoren,
zoals het openbaar vervoer of voor betaalkaarten voor telefooncellen,
slaagde digitaal geld er niet in om veel aantrekkingskracht te winnen
bij het algemene publiek. Door het gebrek aan interesse begon de
financiering ook te verminderen.

`Eerlijk gezegd is het dot-com geld verdwenen', concludeerde Hettinga in
2001, nadat hij er niet in geslaagd was genoeg geld op te halen om het
elektronische geldsysteem van zijn bedrijf te ontwikkelen. `We gaan ook
over terrein waar CyberCash, DigiCash en veel andere mensen hun vingers
aan verbrand hebben.'\footnote{Declan McCullagh, `Digging Those Digicash
  Blues', \emph{Wired}, 14 juni 2001,
  \href{https://www.wired.com/2001/06/digging-those-digicash-blues/}{online}}

In plaats van het investeren in nieuwe crypto-initiatieven, richtten
traditionele banken en financiële dienstverleners zich op het verbeteren
van bestaande cashloze betalingsystemen, zoals transacties via
creditcards en betaalpassen. Intussen wonnen flamboyante nieuwe
web-gebaseerde betalingsverwerkers zoals PayPal snel aan marktaandeel en
het leek erop dat privacy (laat staan \hspace{0pt}\hspace{0pt}monetaire
hervorming) geen grote zorg was voor de meesten van hen. De dystopische
toekomst waar Chaum en veel van de Cypherpunks voor gewaarschuwd hadden
(een toekomst waarin alle financiële transacties konden worden
gemonitord, geregistreerd en mogelijk gecensureerd) werd snel realiteit.

Toch was niet iedereen bereid de hoop op te geven\ldots{}

\section{Hal Finney}\label{hal-finney}

Geboren in het voorjaar van 1956 in het kleine Californische dorpje
Coalinga, toonde Hal Finney al vroeg, nog als een jong kind, een
interesse in codes: op de basisschool, vond hij het leuk om codes met
letters en cijfers te maken voor willekeurige teksten die hij tegenkwam.

Iets later, in zijn tienerjaren, ontwikkelde Hal een fascinatie voor
computers. Gelukkig was de middelbare school die hij bezocht zijn tijd
ver vooruit: de schooladministratie maakte al gebruik van een computer
voor het beheer en de opslag van leerlinggegevens jaren voordat dit
gebruikelijk werd. Jonge Hal, enthousiast om met de machine te werken,
bood vrijwillig zijn hulp aan het schoolpersoneel aan, waardoor hij
tussen de lessen door een soort bijbaantje kreeg.

Finney behaalde in 1974 zijn middelbareschooldiploma als beste student
van zijn klas en werd toegelaten tot het California Institute of
Technology (Caltech), één van de meest prestigieuze en selectieve
universiteiten ter wereld. Omdat Caltech toen nog geen bacheloropleiding
in informatica aanbood, besloot hij een opleiding in techniek te volgen,
terwijl hij tegelijkertijd zoveel mogelijk programmeercursussen volgde.

Rond dezelfde tijd ontwikkelde Finney een sterke waardering voor logica
en omarmde hij de libertaire filosofie. Hij ging graag filosofische
discussies aan met zijn medestudenten aan de universiteit, waar een
combinatie van prikkelende ideeën, stevige argumentaties en een
bedachtzame benadering van gesprekken hem veel aandacht opleverde van
zijn leeftijdsgenoten. Onder hen was Fran, het meisje met wie hij later
zou trouwen en de rest van zijn leven zou doorbrengen.

Kort na zijn afstuderen aan Caltech in 1978, vond Finney zijn eerste
serieuze baan als programmeur bij een kleine ingenieursfirma APh
Technological Consulting. APh was zojuist een samenwerking gestart met
speelgoedfabrikant Mattel om het besturingssysteem voor hun
Intellivision-spelcomputer te ontwikkelen, naast een aantal vroege
spellen. In de daaropvolgende jaren zette Finney zijn werk voort in het
ontwikkelen van baanbrekende videogames zoals \emph{Space Battle} en
\emph{Star Strike} voor de Intellivision, alsook \emph{Adventures of
Tron}, \emph{Astroblast!} en \emph{Space Attack} voor het Atari Video
Computersysteem.

Als algemeen optimistisch mens, was Finney ervan overtuigd dat de wereld
van morgen beter zou zijn dan die van vandaag, en stond hij open voor
verandering. Dus toen de Extropiaanse-gemeenschap eind jaren '80 begon
te vormen, paste hij er ook goed in. Het vooruitzicht van technologische
vernieuwingen zoals nanotechnologie, kunstmatige intelligentie en
\emph{mind uploading} maakte hem enthousiast. En als een overtuigd
atheïst die niet in het hiernamaals geloofde, was Finney al erg
geïnteresseerd in het potentieel van cryonics sinds hij over het concept
las tijdens zijn eerste jaar op de universiteit.

Zoals Fran het later zei: `Hij geloofde niet in God. Hij geloofde in de
toekomst.'\footnote{Nicole Weinstock, `Member Profile: Hal Finney',
  \emph{Cryonics 40}, issue 2: 9.}

\section{Cypherpunkrealisme}\label{cypherpunkrealisme}

Toen het internet begin jaren '90 voor het eerst publiek toegankelijk
werd, was Finney een van de allereerste gebruikers die zich een
verbinding wist te bemachtigen.

Terwijl hij de verschillende, op dat moment enkel op tekst gebaseerde,
hoeken van de gloednieuwe informatiesnelweg verkende, herkende Finney al
snel het revolutionaire potentieel van het ontluikende digitale domein.
Voor het eerst zou de mensheid over de hele wereld verbonden zijn,
ongeacht geografische afstanden, willekeurige grenzen of culturele
verschillen. Hij geloofde dat de gevolgen hiervan de wereld zouden
veranderen.

Maar al snel besefte hij dat er ook een nadeel was aan de digitalisering
van communicatie. Als kenner van de technische architectuur van het
internet, wist Finney dat zonder beschermende maatregelen, cyberspace
rampzalige inbreuken op individuele privacy kon faciliteren: alles dat
iemand online doet, kon potentieel bespioneerd worden. Hij voorzag dat
het internet eigenlijk een bedreiging voor de menselijke vrijheid kon
worden.

Dit was het geval voor gewone communicatie, en Finney dacht dat dit net
zo goed opging voor financiële transacties.

`Er kunnen dossiers worden gemaakt die de uitgavenpatronen van ons
allemaal volgen', waarschuwde Finney. `Als ik nu iets bestel via de
telefoon of elektronisch met mijn Visa kaart, wordt er een record
bijgehouden van hoeveel ik precies heb uitgegeven en waar ik het heb
besteed. Naarmate de tijd vordert, kunnen er meer transacties op deze
manier plaatsvinden, en het uiteindelijke resultaat kan een groot
verlies aan privacy zijn.'\footnote{Hal Finney, `Re: Physical to digital
  cash, and back again', oorspronkelijk verstuurd naar de
  Cypherpunk-mailinglijst, August 19, 1993, accessed via:
  {[}https://cypherpunks.venona.com/date/1993/08/msg00581.html{]}}

Het internet had behoefte aan een ontraceerbare vorm van geld,
concludeerde Finney --- digitaal contant geld. En hij was opgetogen te
ontdekken dat zo'n systeem al in ontwikkeling was.

`Het leek me zo duidelijk', herinnerde Finney zich later. `We worden
geconfronteerd met problemen zoals het verlies van privacy, oprukkende
computarisering, enorme databases en meer centralisatie, en Chaum biedt
een volledig andere weg om in te slaan, eentje die macht in handen van
individuen legt in plaats van regeringen en bedrijven. De computer kan
eerder als instrument gebruikt worden om mensen te bevrijden en te
beschermen, in plaats van hen te controleren.'\footnote{Hal Finney, `Why
  remailers\ldots{}', oorspronkelijk verstuurd naar de
  Cypherpunk-mailinglijst, 15 november 1992,
  \href{https://cypherpunks.venona.com/date/1993/08/msg00581.html}{online}}

Finney had daarom een uitnodiging aanvaard van mede-Extropiaan Tim May,
die een ontmoeting organiseerde met een groep lokale hackers en
cryptografen uit de Bay Area, die zichzelf spoedig de Cypherpunks zouden
noemen.

Kort daarna bevond Finney zich in de positie waar hij Chaum's
eCash-project aan het promoten was onder zijn mede-Extropianen, en op
een bepaald moment daarover een zeven pagina's tellende uitleg voor het
\emph{Extropy} magazine schreef. Finney schreef aan de
techno-libertarische menigte dat cryptografie individuen kon beschermen
tegen overheidsmacht, inmenging, en controle, en legde uit hoe
elektronisch geld de Extropiaanse zaak kon bevorderen.

En, als een echte Cypherpunk, schreef Finney code. De game-ontwikkelaar
was verantwoordelijk voor een vroeg Cypherpunk-succes toen hij Eric
Hughes hielp de allereerste Chaumian remailer te ontwikkelen. Het was
ook Finney's idee om de uitdaging te organiseren om de (verzwakte)
export-grade encryptiestandaard van Netscape te kraken, een uitdaging
die voltooid werd door mede-Cypherpunk Ian Goldberg. Dit bleek een grote
overwinning tijdens de crypto-oorlogen.

Maar Finney's meest opmerkelijke bijdragen vielen ten goede aan PGP:
nadat Phil Zimmermann voor het eerst de encryptietool had uitgebracht,
werd Finney een belangrijke bijdrager aan het project. De tweede versie
van de software, een grote verbetering ten opzichte van versie 1, werd
grotendeels door hem ontwikkeld, hoewel dit een beetje stil werd
gehouden om Finney te behoeden voor mogelijke juridische problemen zoals
Zimmermann die ondervond. Een paar jaar later zou Finney de eerste
werknemer worden van Zimmermann's PGP-bedrijf.

Finney stond echter niet achter de visie van Tim May om met behulp van
Cypherpunk-hulpmiddelen een crypto-anarchistische samenleving te
stichten.

Dit was niet omdat hij het idee om de staat uit economische interacties
te verwijderen niet leuk vond, of omdat de ideeën van May op dat vlak te
radicaal voor zijn smaak waren. Als een Extropiaan en libertariër vond
Finney in feite dat May's visie in principe geweldig klonk. Hij geloofde
echter niet dat May's idee om een anarchistische samenleving te bereiken
door middel van cryptografie erg realistisch was.

`{[}\ldots{]} er bestaat niet zoiets als cyberspace', schreef Finney op
een gegeven moment aan de mailinglijst van de Cypherpunks in reactie op
een van de betogen van May. `Ik ben nu niet in cyberspace, ik ben in
Californië. Ik val onder de wetten van Californië en de Verenigde
Staten, ook al communiceer ik met een andere persoon, of het nu per post
of elektronisch is, via telefoon of TCP/IP-verbinding. Wat betekent het
om te spreken over een regering in cyberspace? Het is de regering in de
fysieke ruimte die ik vrees. Haar agenten dragen fysieke wapens die
echte kogels afvuren. Totdat ik in mijn computer kan leven en elektronen
kan eten, zie ik de relevantie van cyberspace niet in.' \footnote{Hal
  Finney, `Re: Voluntary Governments?' oorspronkelijk verstuurd naar de
  Cypherpunk-mailinglijst, 4 augustus 1994,
  \href{https://cypherpunks.venona.com/date/1994/08/msg00239.html}{online}}

Hoewel individuen Cypherpunk hulpmiddelen konden gebruiken om hun
privacy te beschermen, geloofde Finney niet dat de meeste mensen hun
hele leven zich in cyberspace zouden kunnen `verbergen'. Zelfs als
Cypherpunk-hulpmiddelen een kleine groep technisch onderlegde elite
zouden kunnen helpen om bepaalde wetten te omzeilen, verwierp hij de
gedachte dat dit de beschaving ingrijpend zou veranderen, omdat hij
uiteindelijk niet geloofde dat een libertaire samenleving gerealiseerd
kon worden zonder wijdverspreide steun van de bevolking.

In plaats van een anarchistisch utopia om naar te migreren, zag Finney
het internet eerder als een plek voor intellectuele uitwisseling: in
plaats van een Galt's Gulch, zag hij cyberspace als een plek om ideeën
vrijelijk uit te wisselen en te bediscussiëren. En dit, zo geloofde
Finney, was de echte sleutel tot het bereiken van ware vrijheid. De
beste en enige manier om een vrije samenleving te creëren was om de
massa ervan te overtuigen dat een vrije samenleving een goed idee is.

`In de kern geloof ik dat we de soort samenleving zullen hebben die de
meeste mensen wensen. Als we vrijheid en privacy willen, moeten we
anderen overtuigen dat deze waardevol zijn. Er zijn geen
snelkoppelingen. Terugtrekken in technologie is als het over je hoofd
trekken van je dekens. Het voelt even goed, totdat de realiteit je
inhaalt.'\footnote{Hal Finney, `POLI: Politics vs Technology',
  oorspronkelijk verstuurd naar de Cypherpunk-mailinglijst, 2 januari
  1994,
  \href{https://cypherpunks.venona.com/date/1994/01/msg00014.html}{online}}

\section{Elektronisch geld}\label{elektronisch-geld}

Ondanks zijn nuchtere en misschien wel meer realistische kijk op de
mogelijkheden van cryptografie, was Finney altijd gedreven om
elektronisch geld te realiseren. Hij voerde uitvoerig gesprekken over de
mogelijkheden hiervan met zowel de Extropianen als de Cypherpunks op hun
respectievelijke mailinglijsten, en ook op de Libtech mailinglijst van
Nick Szabo.

Op de Cypherpunks mailinglijst was hij bij gesprekken over digitale
valuta altijd één van de meest actieve deelnemers en nam hij soms zelfs
een soort ondersteunende rol aan. Hoewel sommige Cypherpunks hevig
konden twisten over de beste benadering van elektronisch geld, stond
Finney meer open voor verschillende ideeën. In plaats van vast te houden
aan één oplossing, gaf hij liever een overzicht van de verschillende
compromissen die elk van hen met zich meebracht.

Finney leek bijvoorbeeld grotendeels onbeslist, of misschien beter
gezegd, open-minded over het onderwerp van dekking. Hij merkte op dat
het dekken van elektronisch geld met fiatvaluta werkte, maar speculeerde
soms ook over digitale valuta's gedekt door een mandje van goederen, of
door een synthetisch gemiddelde van meerdere nationale valuta's, of
helemaal niet gedekt.

Elke keer wanneer een nieuw voorstel voor elektronisch geld opdook op de
mailinglijst, was Finney altijd enthousiast om het te beoordelen, met
een speciale nadruk op hun privacyfuncties. Na het bestuderen van het
ontwerp, koppelde hij vaak terug op de mailinglijst om in zijn eigen
woorden uit te leggen hoe het werkte, hoe het zich verhield tot eerdere
voorstellen, en wat hij van het idee vond. Naast (meestal constructieve)
feedback voor de indiener, bood Finney aan andere Cypherpunks in wezen
een openbare dienst door hen te helpen de mogelijkheden en beperkingen
van verschillende benaderingen te begrijpen.

Finney had ook een speciale interesse in de wettelijkheid van
elektronisch geld, een onderwerp dat hem in de begindagen van de
Cypherpunk-gemeenschap naar de geschiedenis van het geld leidde. Hier
kwam hij voor het eerst het werk van George Selgin over vrij bankieren
tegen. Terwijl hij wetten over wettig betaalmiddel, belastingregels,
bankregulering en meer bestudeerde, deelde Finney zijn bevindingen op de
mailinglijst van de Cypherpunks en begon hij de mogelijkheden en
risico's in kaart te brengen (het was bijvoorbeeld Finney die ontdekte
dat niet-commerciële experimenten voor systemen zoals eCash getolereerd
zouden moeten worden, zelfs als ze gebruik maakten van Chaum's
gepatenteerde blinde handtekeningenschema).

Tegelijkertijd stelde Finney zich terughoudend op tegenover enkele
uitspraken, geïnspireerd door crypto-anarchie, over de beloften van
elektronisch geld. Ook hier was hij sceptisch over enkele van de meer
radicale voorspellingen met betrekking tot massale belastingontduiking
en hoe elektronisch geld dit zou mogelijk maken.

`We zijn verblind door het beeld van monetaire stromen die over de hele
wereld flitsen. Wat ik echter nooit precies kan plaatsen is, wat precies
verhindert dat zoiets vandaag de dag wordt gedaan?' vroeg Finney aan de
mailinglijst. `Als je in goud wilt investeren, kun je toch naar de
goudhandelaar gaan en wat kopen? Of je kunt je geld in een
beleggingsfonds in goud stoppen en het als een betaalrekening gebruiken.
Als je yen of marken wilt, kun je daarin investeren. Als het punt is om
dit in het geheim te doen, waarom zou het dan gemakkelijker zijn om je
salaris per post naar de digicash-bank in de Bahamas te sturen dan naar
een bestaande bank daar?'\footnote{Hal Finney, `Re: Re: re: re: digital
  cash', oorspronkelijk verstuurd naar de Cypherpunk-mailinglijst, 16
  maart 1994,
  \href{https://cypherpunks.venona.com/date/1994/03/msg00694.html}{online}}

Ook hier was het niet zo dat Finney de meer radicale beloftes van Tim
May onaantrekkelijk vond. Hij beschouwde ze gewoon als niet erg
realistisch. Afgezien van het feit dat de meeste mensen hun belastingen
toch al rechtstreeks van hun salaris betaalden, moest iedereen
uiteindelijk in de fysieke wereld leven, waar belastingontduiking nog
steeds illegaal zou zijn. Het was voor Finney verre van duidelijk dat
het verbergen van rijkdom in cyberspace de meeste mensen in het echte
leven ten goede zou komen.

`Het lijkt mij dat de zwakte in deze plannen om de overheid te omzeilen
met digitaal geld zit in de omzetting van fysiek geld naar digitaal
geld. Dat lijkt het knelpunt te zijn waar de overheid nog steeds
controle kan houden', concludeert Finney. \footnote{Finney, `Re: Re: re:
  re: digital cash.'}

\section{Herbruikbare proofs-of-work}\label{herbruikbare-proofs-of-work}

In de jaren 2000, ongeveer tien jaar nadat Finney bij Extropianen begon
te pleiten voor elektronisch geld, was er nog steeds geen succesvol
elektronisch geldsysteem. Hoewel een reeks ideeën besproken was op de
mailinglijst van de Cypherpunks, en Finney veel van de voorstellen
persoonlijk had beoordeeld, was geen van hen van de grond gekomen. In
sommige gevallen, zoals bij de start-ups van Chaum of Hettinga, was dit
omdat het product uiteindelijk commercieel niet haalbaar bleek te zijn,
of althans zo leek het. Maar in andere gevallen, zoals bij het Bit Gold
van Nick Szabo of de b-geld voorstellen van Wei Dai, waren de systemen
in de eerste plaats nooit geïmplementeerd.

Wellicht was het omdat zijn verwachtingen voor het potentieel van
elektronisch geld meer ingetogen waren dan die van Tim May en andere
crypto-anarchisten in de eerste plaats, of misschien was het gewoon
vanwege zijn over het algemeen optimistische karakter, maar waar vele
andere Cypherpunks tegen deze tijd ontgoocheld waren geraakt, wilde
Finney het idee nog een kans geven. Hij besloot uiteindelijk om een op
proof-of-work gebaseerd elektronisch geldsysteem te ontwikkelen, zelfs
als het in een vereenvoudigde vorm moest.

In 2004 lanceerde hij \emph{Reusable Proofs of Work}, of kortweg
\emph{RPOW} (uitgesproken als `arpow'). Hij nodigde mensen uit om het
systeem te testen, adverteerde het elektronische geld op een eenvoudige
blauw-groene webpagina met een RPOW-logo in stripboekstijl (denk aan de
`POW' letters die de plek markeren waar Batman's uppercut de kaak van
een ongelukkige handlanger raakt).

`Beveiligingsonderzoeker Nick Szabo heeft de term bit gold bedacht om
een soortgelijk concept van tokens, dat inherent een bepaald niveau van
inspanning vertegenwoordigt, te omschrijven', schreef Finney op de
website van het project. `Het concept van Nick is complexer dan het
eenvoudige RPOW-systeem, maar zijn inzicht is van toepassing: op sommige
manieren kun je een RPOW-token zien als het hebben van de eigenschappen
van een zeldzame grondstof zoals goud. Het kost moeite en uitgaven om
goud te delven en munten te slaan, waardoor ze inherent zeldzaam
zijn.'\footnote{Hal Finney, `Reusable Proofs of Work', RPOW website
  index pagina,
  \href{https://web.archive.org/web/20090217090451/http://rpow.net/index.html}{online}}

Waar Bit Gold was ontworpen rondom een `eigendomsclub', zou ook RPOW
beheerd worden door specifieke servers. Voor het prototype had Finney
zelf een RPOW-server opgezet. Deze voerde de basisbewerkingen uit die
nodig zijn voor het elektronische geldsysteem: het gaf nieuwe
RPOW-tokens uit (de munteenheden), en controleerde of tokens niet twee
keer werden uitgegeven.

Het is belangrijk te vermelden dat de RPOW-server alleen nieuwe tokens
uitgaf als aan één van de twee voorwaarden was voldaan: er moest een
geldige hash worden ingediend, of een oudere token moest worden
ingeleverd als ruil.

De eerste optie was een eenvoudige proof-of-work functie. Als gebruiker
Alice een RPOW-token wilde, moest ze verbinding maken met Finney's
server (mogelijk via Tor voor optimale privacy), een aantal gegevens die
uniek zijn voor de server en voor haarzelf nemen, en beginnen met hashen
tot ze een geldige hash (beginnend met genoeg nullen) vond. Vervolgens
stuurt ze de hash naar de server, die deze op geldigheid controleert, en
(indien geldig) een unieke RPOW-token terugstuurt, in feite gewoon een
unieke gegevensreeks. De server bewaarde ook een kopie van de token in
een lokale database.

Wanneer Alice een RPOW-token wilde uitgeven, bijvoorbeeld om een
MP3-bestand te kopen, zou ze het simpelweg naar de beoogde ontvanger,
Bob, sturen. Technisch gezien maakte het voor het RPOW-systeem niet uit
hoe ze het verzond, zolang ze er maar zeker van was dat het bij Bob
terechtkwam zonder dat iemand het onderschepte (een bericht aan Bob
versleuteld met zijn publieke sleutel zou de klus klaren).

Wanneer Bob de RPOW-token ontving, zou hij deze moeten valideren en
controleren dat deze niet dubbel was uitgegeven. Hij zou daarom de token
direct doorsturen naar de RPOW-server, die naging of de token in de
interne database stond en of deze niet dubbel was uitgegeven. Als de
token geldig was, zou de server dit aan Bob bevestigen. Hierdoor kon Bob
het MP3-bestand naar Alice sturen. De server zou ook de RPOW-token als
uitgegeven bestempelen, waardoor deze in de toekomst niet meer gebruikt
kon worden.

Ten slotte zou de server Bob een nieuw RPOW-token geven, en die nieuwe
token opnemen in zijn interne database. Op deze manier kon Bob de nieuwe
token later uitgeven.

Stel je voor dat Bob zijn nieuwe RPOW-token wil gebruiken om toegang te
krijgen tot Carol's website. Als Carol de RPOW-token van Bob krijgt,
stuurt ze hem weer door naar de RPOW-server van Finney. De server
bevestigt dan dat de token echt is, markeert deze als besteed in zijn
interne database en geeft vervolgens een nieuwe RPOW-token uit aan
Carol. Deze wordt ook toegevoegd aan de interne database van de server.

Op deze manier kon het proof-of-work, vertegenwoordigd door een enkele
geldige hash (gemaakt door Alice), effectief oneindig blijven
circuleren. Het was inderdaad \emph{herbruikbaar} proof-of-work.

\section{Betrouwbaar rekenen}\label{betrouwbaar-rekenen}

Het systeem zoals tot dusver beschreven zou vrij goed werken, behalve
dat het vertrouwen vereist in de beheerder van de RPOW-server om geen
dubbele uitgaven te doen of RPOW-tokens voor zichzelf te maken zonder
een proof-of-work te leveren. Finney wilde echter niet dat de gebruikers
de beheerder van de RPOW-server moesten vertrouwen, zelfs als die
beheerder hijzelf was. Daarom voegde Finney nog een speciale eigenschap
toe aan het ontwerp.

Ten eerste zou de RPOW-server gebruik maken van gratis en open source
software. Iedereen kon online de broncode van RPOW vinden en controleren
hoe het functioneerde.

En, als hoofdinnovatie van het systeem, was de RPOW-server gehost op een
veilige hardwarecomponent, de IBM 4758. Dit maakte `betrouwbaar rekenen'
mogelijk.

Kortom, de sabotagebestendige hardware bevatte een privésleutel, ingebed
door IBM, die niemand, zelfs niet de eigenaar van de veilige
hardwarecomponent, kon manipuleren of eruit kon halen. Met behulp van
een techniek die `remote attestation' wordt genoemd, kon de privésleutel
vervolgens de gratis en open source software die op de veilige
hardwarecomponent was geïnstalleerd, cryptografisch ondertekenen. Met
deze handtekening en de bijbehorende publieke sleutel van IBM kon
iedereen verifiëren dat de veilige hardwarecomponent daadwerkelijk de
RPOW-broncode uitvoerde die Finney had gepubliceerd, zonder achterdeuren
of aanpassingen.

Zolang men IBM vertrouwde om niet mee te werken met Finney om een valse
handtekening te fabriceren (en aannemende dat de centrale server niet
volledig offline ging), konden RPOW-gebruikers er zeker van zijn dat het
elektronische geldsysteem functioneerde zoals het moest.

`{[}\ldots{]} Het RPOW-systeem is ontworpen met één overkoepelend doel:
het onmogelijk maken dat iemand, zelfs de eigenaar van de RPOW-server of
zelfs de ontwikkelaar van de RPOW-software, in staat zou zijn om de
regels van het systeem te overtreden en RPOW-tokens te vervalsen', legde
Finney uit op de RPOW-website. `Zonder zo'n garantie tegen
vervalsbaarheid, zouden RPOW-tokens het werk dat is gedaan om ze te
creëren niet geloofwaardig kunnen vertegenwoordigen. Vervalsbare tokens
zouden meer lijken op papiergeld dan op bit-goud.'\footnote{Hal Finney,
  `RPOW Theory', RPOW website theorie pagina,
  \href{https://web.archive.org/web/20040815154951/http://rpow.net/theory.html}{online}}

\section{Het lot van RPOW}\label{het-lot-van-rpow}

De eerste RPOW-release was weliswaar nog erg ruw, maar Hal Finney had
het voornemen het project in de loop van de tijd te verbeteren. Wellicht
het belangrijkst, hij plande om het systeem te upgraden zodat het op
meerdere, onafhankelijk van elkaar opererende, servers zou draaien. Zo
zou het volledige RPOW-systeem niet onderuit gaan als zijn server om
welke reden dan ook offline ging.

Ondertussen vond de Cypherpunk het ook leuk om te experimenteren en te
sleutelen aan de RPOW-software. Zo heeft hij bijvoorbeeld een
BitTorrent-client aangepast om samen te werken met zijn elektronische
geldsysteem. Dit leek op het Mojo Nation concept en stelde gebruikers in
staat om andere gebruikers te betalen als ze hun download wilden
versnellen. In een even creatieve toepassing van de RPOW-technologie
werkte hij aan een peer-to-peer pokerapplicatie. Hier konden gebruikers
tegen elkaar spelen, waarbij de RPOW-tokens automatisch naar de digitale
portemonnee van de winnaar werden overgemaakt.

Finney kreeg al snel hulp van een jongere ontwikkelaar genaamd Gregory
Maxwell, die een actieve interesse toonde in het elektronische
geldsysteem. Maxwell droeg bij aan het project met code, en overwoog om
geavanceerde bestedingsvoorwaarden zoals escrow-betalingen te
implementeren. Hij besprak met Finney ook mogelijke oplossingen voor
sommige van de meer subtiele technische uitdagingen, zoals het instellen
van vervaltermijnen voor tokens, of de relatief zwakke versleuteling die
de veilige hardwarecomponent beveiligde.

Helaas voor Finney bleek Maxwell echter een zeldzame uitzondering. Omdat
bijna niemand anders interesse toonde in het elektronische geldsysteem,
lukte het RPOW niet om door te breken.

Dit was waarschijnlijk ten minste gedeeltelijk te wijten aan het feit
dat RPOW geen heel goede vorm van geld was. Geconfronteerd met hetzelfde
probleem als Adam Back's hashcash, een probleem dat Szabo en Dai
geprobeerd hadden op een omslachtige manier op te lossen, zouden
computationele verbeteringen het na verloop van tijd goedkoper maken om
geldige hashes te genereren, wat suggereert dat de markt uiteindelijk
zou worden overspoeld met RPOW-tokens. De verwachting van hoge inflatie
werkte ontmoedigend om de RPOW-valuta-eenheden te bezitten.

`Het klopt dat als de Wet van Moore blijft gelden, de kosten voor het
vervaardigen van een 'proof-of-work'-token exponentieel zullen blijven
dalen', gaf Finney toe op de website van het project. `Maar onthou dat
dit geen geld is en niet bedoeld is als een stabiel middel om waarde op
te slaan. Het is eerder bedoeld als een gemakkelijk te ruilen
representatie van computerrekenkracht.'\footnote{Hal Finney, `RPOW
  FAQs', RPOW website FAQ pagina,
  \href{https://web.archive.org/web/20090217090439/http://rpow.net/faqs.html\#inflation}{online}}

Inderdaad, het elektronische geldsysteem van Finney fungeerde niet
zozeer als een breed geaccepteerde waardeopslag of rekeneenheid. Het
werd vooral nuttig geacht als ruilmiddel op plaatsen waar hashcash
zinvol kon zijn, bijvoorbeeld om te dienen als `postzegels' voor het
beperken van spam.

Maar waarschijnlijk slaagde het elektronische geldsysteem er ook niet in
om van de grond te komen omdat het de opstartuitdaging niet kon
overwinnen. Geld is enkel nuttig als anderen het accepteren als
betaalmiddel, maar zonder een economische stimulans om RPOW-tokens bij
te houden, hadden de meeste mensen daar geen reden toe. En zonder dat er
iemand was die de tokens als betaling accepteerde, was er ook niemand
die ze wilde uitgegeven, wat betekende dat er nog minder reden was voor
iemand om ze in eerste instantie te accepteren als betaling\ldots{}

`Het had het probleem dat er min of meer niets was om het voor te
gebruiken', concludeerde Maxwell ook, die jaren later terugkeek op het
RPOW-project, `wat het moeilijk maakte om de aandacht erop gericht te
houden.'\footnote{Gregory Maxwell, IRC message to author, 13 augustus
  2020.}

Net als eCash en hashcash voorheen, leed ook RPOW aan een
kip-en-ei-probleem.

\section{E-goud}\label{e-goud}

Ondanks de beste bedoelingen van Hal Finney is RPOW halverwege de jaren
2000 geëindigd als nog een mislukte poging om elektronisch geld te
creëren.

Het was op dit moment dat sommige techno-libertariërs wat nieuw
perspectief vonden in een alternatieve vorm van internetgeld, die met
een zeer verschillend ontwerp, succesvoller leek te zijn: e-goud.

Het project van Douglas Jackson, een digitale valuta gedekt door goud,
groeide rond het midden van de jaren 2000 snel. Het voldeed aan
verschillende van de eisen die de Cypherpunks hadden gesteld aan
elektronisch geld: transacties konden met enige mate van anonimiteit
worden uitgevoerd, het systeem ondersteunde microtransacties tot op één
tienduizendste van een gram goud en, natuurlijk, goud zelf
vertegenwoordigde onvervalsbare kostbaarheid. Naarmate de technologie
van e-goud verbeterde, konden ontwikkelaars zelfs computerprogramma's in
het systeem integreren via een Application Programming Interface (API),
waardoor oplossingen die op slimme contracten leken mogelijk werden.

Er was natuurlijk één vereiste waaraan e-goud niet kon voldoen. De
ervaring met DigiCash's Cyberbucks had de cypherpunks geleerd wat er kon
gebeuren met een digitale munteenheid als het afhankelijk was van één
enkel bedrijf, en de klanten van Douglas Jackson zouden deze les
binnenkort ook leren. Met de arrestatie van de CEO en federale agenten
die het kantoor van het bedrijf in 2006 binnenvielen, was er weer een
internetvalutaproject mislukt. Szabo's stelling klonk nogmaals luid en
duidelijk: \emph{vertrouwde derde partijen zijn veiligheidslekken}.

Na meer dan twintig jaar van niet-gerealiseerde voorstellen, verlaten
projecten en mislukte start-ups, bestond er nog steeds geen elektronisch
geld.

Intussen werd de noodzaak om een alternatief voor fiatgeld te vinden
steeds groter\ldots{}

\part{Deel III: Bitcoin}

\chapter{Fiat in de 21ste eeuw}\label{fiat-in-de-21ste-eeuw}

Voor het grootste deel van zijn leven voelde Friedrich Hayek dat zijn
ideeën waren gemarginaliseerd. Spontane orde van onderaf moest het
onderspit delven voor de economische staatsinterventie van John Maynard
Keynes, die van bovenaf functioneert. In plaats van rentetarieven die
bepaald werden door de markt, hadden centrale banken het goud vaarwel
gezegd om het manipuleren van de rente nog gemakkelijker te maken. En
geld, verre van gedenationaliseerd, was een strategisch instrument
geworden op het globale geopolitieke schaakbord.

Weinig schaakspelers waren zo sluw als de Amerikaanse president Richard
Nixon.

Door in 1971 effectief het Bretton Woodssysteem te ontbinden, had Nixon
een liquiditeitscrisis weten te voorkomen toen landen hun dollarreserves
opnieuw naar goud begonnen om te zetten. Maar dat zou natuurlijk uit
zichzelf de problemen die veroorzaakt werden door het begrotingstekort
van Amerika niet oplossen. Het vertrouwen in de dollar begon te dalen nu
die niet langer werd gedekt door edelmetaal, en het begon erop te lijken
dat de Verenigde Staten hun dominante positie in het internationale
financiële systeem mogelijks zouden verliezen.

Nixon zou een oplossing vinden tijdens een wereldwijde oliecrisis.

In 1973 riepen olieproducerende Arabische landen, verenigd in de
Organisatie van de Olie-Exporterende Landen (OPEC), bestaande uit
meerdere overheden, een olie-embargo uit over landen die Israël steunden
tijdens de Oktoberoorlog tussen de joodse staat en Egypte. De landen
voor wie de handel verboden werd, waren Amerika, het Verenigd Koninkrijk
en vele andere westerse landen. Het veroorzaakte een scherpe stijging in
de olieprijs, met verstrekkende negatieve gevolgen voor de gehele
wereldeconomie.

Als reactie hierop werd in 1973 de nieuwbenoemde Amerikaanse minister
van Financiën, William Simon, naar Saudi-Arabië gestuurd, een lidstaat
van de OPEC. Zijn taak was om olie als economisch wapen te
neutraliseren, het vestigen van een stevige greep van de Sovjet-Unie in
de regio te voorkomen en bovendien een oplossing voor de dollarcrisis te
vinden; al met al geen gemakkelijke klus. Maar Simon ging de
onderhandelingen in met een sterk drukkingsmiddel: het Amerikaanse
leger.

De overeenkomst die Simon wist te sluiten met de Saoedische koninklijke
familie zou het geopolitieke landschap voor de komende decennia
vormgeven. Kort samengevat, zouden Saudi-Arabië en andere OPEC-landen
hun olie exclusief voor Amerikaanse dollars verkopen, ongeacht welk land
petroleum wilde kopen. Landen die olie exporteren zouden alleen de
Amerikaanse valuta accepteren als betaling. Deze dollars zouden op hun
beurt grotendeels gebruikt worden om Amerikaanse staatsobligaties te
kopen en zo de Amerikaanse uitgaven te financieren. In ruil hiervoor zou
het Amerikaanse leger hulp en uitrustingen verstrekken om de Saoedische
olievelden te beschermen en de veiligheid van de koninklijke familie te
waarborgen.\footnote{Andrea Wong, `The Untold Story Behind Saudi
  Arabia's 41-Year U.S. Debt Secret', \emph{Bloomberg}, 31 mei 2016,
  \href{https://www.bloomberg.com/news/features/2016-05-30/the-untold-story-behind-saudi-arabia-s-41-year-u-s-debt-secret}{online}}

Deze deal bezorgde de Verenigde Staten een aanhoudende vraag naar hun
dollars: iedereen die olie wilde kopen van OPEC-landen, die samen meer
dan twee derde van de wereldreserves controleerden,\footnote{Energy
  Exploration \& Exploitation, `World Oil Reserves 1948--2001: Annual
  Statistics and Analysis', \emph{Energy Exploration \& Exploitation},
  vol.~19, no. 2 \& 3,
  \href{https://journals.sagepub.com/doi/pdf/10.1260/0144598011492561}{online}}
moest eerst Amerikaanse dollars verkrijgen. Gezien het centrale belang
van olie in de wereldeconomie, garandeerde dit dat de dollar in feite de
wereldreservevaluta bleef. Deze onofficiële overeenkomst zou bekend
komen te staan als het petrodollarsysteem (kort nadat deze regeling was
ingevoerd, op 30 december 1974, werd het privébezit van goud opnieuw
gelegaliseerd in de Verenigde Staten).

Het petrodollarsysteem was een geweldige overeenkomst voor de Amerikanen
- maar niet zo geweldig voor het merendeel van de rest van de wereld. Om
dollars te verkrijgen om olie te kunnen kopen, moesten de meeste landen
goederen of diensten naar de VS exporteren, of dollars kopen op de
buitenlandse valutamarkten \ldots{} terwijl de VS simpelweg dollars
konden drukken, zonder zich zorgen te maken over een
gouddekkingverhouding. Als en wanneer ze dat deden, betaalden andere
landen echt de prijs, omdat de waarde van hun dollarreserves daalde.

Het petrodollarsysteem implementeerde in feite het Cantillon-effect op
wereldwijde schaal, met de Amerikaanse regering en Amerikaanse
financiële instellingen in het hart van dit monetaire
paradigma.\footnote{Alex Gladstein, `Uncovering The Hidden Costs of the
  Petrodollar', \emph{Bitcoin Magazine}, 28 april 2021,
  \href{https://bitcoinmagazine.com/culture/the-hidden-costs-of-the-petrodollar}{online}}

\section{De Hayekiaanse heropleving}\label{de-hayekiaanse-heropleving}

Intussen had stagflatie de economische wereld tot wanorde gebracht,
aangezien de Keynesiaanse assumptie dat inflatie werkloosheid zou
tegengaan, onjuist bleek. Als het gebruik van inflatie om de economie te
stimuleren verslavend was, zoals Hayek had betoogd, waren de positieve
effecten van deze drug nu uitgewerkt en ondervond de samenleving de
pijnlijke ontwenningsverschijnselen. Na een bestuur van veertig jaar
bevond het Keynesianisme zich in een existentiële crisis.

Het was in deze context dat Hayeks ideeën werden herontdekt om de basis
te vormen voor een heropleving van klassieke economische ideeën.

Deze \emph{neoliberale} herleving begon in het Verenigd Koninkrijk, waar
Margaret Thatcher in 1975 de leiding nam over de Conservatieve Partij.
Nadat ze als student \emph{The Road to Serfdom} had gelezen, werd ze een
aanhanger van het werk van Hayek en een sterke voorstander van vrije
markten en een kleine overheid. Thatcher verwierp de gewoonte van de
Conservatieve Partij om een compromis te sluiten over deze idealen om de
centristische stem te winnen, en koos in plaats daarvan voor de harde
aanpak. Op een gegeven moment, tijdens een vergadering met de
onderzoeksafdeling van haar partij, haalde ze op beroemde wijze zelfs
Hayeks boek \emph{The Constitution of Liberty} uit haar tas en sloeg het
neer op tafel. `Dit is waar we in geloven!' verklaarde ze.\footnote{Wapshott,
  Keynes Hayek, 258--59.}

Het plan had succes. De Iron Lady, zoals Thatcher vaak werd genoemd,
werd in 1979 verkozen tot de eerste vrouwelijke premier van het Verenigd
Koninkrijk. Eenmaal in functie voerde ze haar plannen uit om de omvang
van de overheid te beperken en de vrije markt meer ruimte te geven. Ze
deed dit door belastingen te verlagen, regulerende obstakels weg te
nemen en een nationale golf van privatisering door staatsbedrijven te
verkopen. Dit beleid leverde haar in de daaropvolgende jaren meerdere
herverkiezingen op. Uiteindelijk werd Thatcher hierdoor de
langstzittende Britse premier van de twintigste eeuw.

Het succes van Thatcher in het VK diende als inspiratie voor een
geestverwant in de Verenigde Staten. De Republikeinse kandidaat voor de
presidentsverkiezingen van 1980, Ronald Reagan, beloofde eveneens een
vermindering van de overheidsuitgaven. Zijn campagneslogan, `We kunnen
de overheid van onze ruggen halen, uit onze zakken', sloeg aan bij de
Amerikaanse kiezers. De voormalige filmster versloeg de zittende
president Jimmy Carter met een grandioze overwinning.

Als president verlaagde Reagan inderdaad de belastingen en zette hij de
sociale programma's stop. Om zijn economische beleid verder te
ontwikkelen, dat later bekend kwam te staan als `Reaganomics', creëerde
de president een nieuwe adviesraad voor economisch beleid. Hierin
benoemde hij economen die een sterke voorkeur hadden voor de vrije
markt, waaronder opmerkelijk Milton Friedman, een vroege bewonderaar van
Hayek.\footnote{Wapshott, Keynes Hayek, 261.}

Decennia eerder, in 1947, nodigde Hayek Friedman, die net was begonnen
als economieprofessor aan de Universiteit van Chicago, uit voor een
tiendaagse conferentie in Zwitserland. Ongeveer 60 toonaangevende
voorstanders van de vrije markt, bekende libertaire denkers en andere
invloedrijke vrijdenkende personen, hadden hier tien dagen besteed aan
het bespreken hoe een vrije samenleving kan worden behouden en hoe te
voorkomen dat het Westen zou terugvallen naar fascisme of afglijden naar
socialisme. Voor Hayek vertegenwoordigde de topbijeenkomst (die een
jaarlijks terugkerend evenement zou worden) `de wedergeboorte van een
liberale beweging in Europa'.\footnote{Wapshott, Keynes Hayek, 211--214.}

Friedman had zich in de decennia na die reis naar Zwitserland weten te
vestigen als een van de meest invloedrijke economen van de twintigste
eeuw. Net als Hayek verwierp Friedman de destijds dominante Keynesiaanse
doctrine van staatsinterventie en overheidsuitgaven. Naarmate hij
beroemder werd, werd hij in veel opzichten een meer effectief
voorvechter van Hayeks ideeën over markten en het prijssysteem dan Hayek
zelf ooit was geweest.

Echter, op bepaalde punten zouden Friedmans gedachten ook afwijken van
de inzichten van Hayek. Zijn bijdragen op het gebied van economie
hielpen bij het vormgeven van een aparte denkrichting die bekend staat
als de `Chicago School of Economics'.

De Chicago School verschilt op een aantal belangrijke punten van de
Oostenrijkse School.

Een fundamenteel verschil was de methodologie van de Chicago School.
Waar Carl Menger de basis had gelegd voor de Oostenrijkse school van
economie in praxeologie, de methode gebaseerd op logica, redenering en
basisprincipes, koos de Chicago School voor een meer traditioneel
empirisme, waarin hypotheses worden geformuleerd en getoetst aan de hand
van reële wereldgegevens en statistieken.

Het tweede grote verschil was bijna net zo fundamenteel als het eerste:
de Chicago School was het niet eens met de Oostenrijkers over het
onderwerp geld.

\section{Monetarisme}\label{monetarisme}

Hoewel Friedman een grote voorvechter was van Hayeks ideeën over markt
en prijzen, omvatte dit niet Hayeks perspectief op de stadia van
productie, of hoe rentetarieven en het intertemporele prijssysteem de
toewijzing van middelen door de tijd heen konden sturen. Friedman was
zeker geen voorstander van Hayeks voorstel om geld te denationaliseren.
In plaats daarvan pleitte Friedman voor strikte overheidsregulatie van
geld.

Friedman heeft, samen met econome Anna Schwartz, een onderzoek
uitgevoerd naar economische hoogte- en dieptepunten in de Verenigde
Staten. \footnote{Milton Friedman and Anna Schwartz, A Monetary History
  of the United States.} Hun studie bestreek de periode vanaf het midden
van de negentiende eeuw tot het midden van de twintigste eeuw, met extra
aandacht voor de Grote Depressie van de jaren 1930. Friedman en Schwartz
kwamen tot de conclusie dat elke economische recessie voorafging aan een
afname van de geldvoorraad, of op z'n minst een vertraging in haar
groei.

Volgens economen, inclusief Hayek, kan de geldvoorraad het duidelijkst
stijgen en dalen door het wijdverspreide gebruik van fractioneel
bankieren. Telkens wanneer iemand bij een fractionele reservebank een
lening afsluit, brengt de bank nieuw geld in omloop (als krediet), wat
de geldvoorraad vergroot. En vice versa: elke keer dat iemand een lening
terugbetaalt bij een fractionele reservebank, wordt er geld uit de
omloop genomen, waardoor de geldvoorraad afneemt.

Friedman en Schwartz legden uit dat toen de Federal Reserve de
rentetarieven in 1928 had verhoogd, de bereidheid van mensen om leningen
af te sluiten vanzelfsprekend verminderde. Maar naarmate oude leningen
werden afbetaald en geld daardoor uit de omloop werd gehaald, terwijl er
niet zoveel nieuwe leningen werden uitgegeven om dit te compenseren, nam
de totale geldvoorraad af. Dit zette, zoals Hayek ook had uitgelegd, een
deflatoire schuldenspiraal in gang.

Maar Friedman nam Hayeks uitleg van hoe rentemanipulatie deze deflatoire
schuldenspiraal had ingeleid niet over, en hij was vooral het niet eens
met Hayeks voorstel om deze schuldenspiraal zijn gang te laten gaan. In
plaats daarvan stelde hij een oplossing voor die veel meer leek op wat
Irving Fisher en de stabilisatoren in de jaren 1920 hadden voorgesteld
(maar waarvan hij geloofde dat het niet goed was uitgevoerd in de jaren
1930).

De econoom van de Chicago School redeneerde dat de geldvoorraad op een
gestage en voorspelbare manier zou moeten groeien, zodat de economie
meegroeit en het geaggregeerde prijsniveau stabiel kan blijven of
misschien heel langzaam kan stijgen. Zolang de prijzen stabiel werden
gehouden, kon de markt voor de rest zorgen. \emph{Monetarisme}, zoals
deze monetaire doctrine werd genoemd, werd een fundamenteel onderdeel
van de Chicago School van Economie.

De voorgestelde hulpmiddelen om de geldvoorraad te beheren waren
dezelfde als die van Fischer: de consumentenprijsindex zou fungeren als
de maatstaf voor stabiliteit en rentetarieven konden worden gereguleerd
om de geldvoorraad te verhogen of te verlagen. Als de prijzen zouden
dalen, wilden monetaristen dat centrale banken de rentetarieven zouden
verlagen om zo de kredietverlening aan te wakkeren, uitgaven te
stimuleren, en opwaartse druk op de prijzen te zetten. Als de prijzen te
snel zouden stijgen, wilden monetaristen dat centrale banken de
rentetarieven zouden verhogen. Als de prijzen stabiel bleven, zou de
hoeveelheid geld in de economie gestaag groeien samen met de economie
zelf, en was de rente precies goed ingesteld.

Monetaristen waren het grotendeels oneens met Keynesianen over de rol
van de overheid. Zij waren niet van mening dat regeringen lage
rentetarieven zouden moeten gebruiken om de uitgaven te verhogen, maar
in plaats daarvan geloofden zij dat overheidsuitgaven uitsluitend
gefinancierd zouden moeten worden door middel van fiscaal beleid, met
andere woorden, belastingen. Hoewel monetaristen het erover eens waren
dat geaggregeerde uitgaven essentieel waren om een economische neergang
te vermijden, beweerden zij dat de particuliere sector de uitgaven even
goed kon doen, en in feite de uitgaven zou doen als mensen goedkoop
genoeg geld konden lenen. Op een bepaalde manier combineerden
monetaristen ideeën van zowel Oostenrijkers als Keynesianen in een
nieuwe benadering.

De theorie overtuigde Hayek echter niet, om veel van dezelfde redenen
dat de stabilisatoren hem voorheen nooit overtuigd hadden: hij geloofde
dat het manipuleren van rentetarieven de spontane orde door de tijd heen
verstoorde, en geloofde niet in het monetaristische idee van
stabiliteit. (Eerder in zijn carrière had hij aangehaald dat stabiele
prijzen eigenlijk niet stabiel zijn, aangezien deflatie juist de
natuurlijke staat is van een gezonde economie, terwijl hij later in zijn
carrière toevoegde dat de vrije markt, en niet een centraal geleid
overheidscomité, zou moeten bepalen wat stabiel is.)

Hoewel Hayeks ideeën over markten en het prijssysteem in de jaren
tachtig een heropleving genoten, bleven zijn ideeën over geld
grotendeels onopgemerkt. Geld bleef onder staatscontrole en zelfs de
nieuwe stroom vrije markteconomen in Chicago geloofde niet dat hier
verandering in moest komen.

\section{Spaargeld en leningen}\label{spaargeld-en-leningen}

In 1987 benoemde president Reagan een ander lid van zijn Economische
Beleidsadviesraad tot de nieuwe voorzitter van de Federal Reserve, met
name Alan Greenspan. Net zoals Friedman was ook Greenspan een fervent
monetarist. Toen de Amerikaanse Senaat kort daarna de nominatie
bevestigde, stond de monetaire theorie van de Chicago School op het punt
om voor het eerst in de praktijk te worden gebracht.

Nauwelijks in zijn nieuwe functie benoemd, werd Greenspan vrijwel meteen
geconfronteerd met de ergste bankencrisis sinds 1929. In een economie
met zowel hoge inflatie als hoge rentetarieven om de inflatie te dempen,
hadden de spaarbanken het moeilijk. Veel van deze samenwerkende
bankachtige financiële instellingen, die langlopende leningen zoals
hypotheken verstrekten met een vaste rente, hadden nu moeite om
voldoende geld aan te trekken om aan alle opnameverzoeken van spaarders
te voldoen. Uiteindelijk dwong dit veel spaar- en leenverenigingen om in
gebreke te blijven en faillissement aan te vragen.

Toen de bezorgdheid bij economen en beleidsmakers toenam dat deze
faillissementen een domino-effect op de Amerikaanse economie konden
hebben, verlaagde Greenspan in 1989 de rentetarieven. Hierdoor werd het
goedkoper voor de spaar- en leenverenigingen om geld te verkrijgen,
terwijl tegelijkertijd de eerste tekenen van een economische neergang
werden tegengegaan en beheerst.

Desondanks moest de Federal Savings and Loan Insurance Corporation
(FSLIC) uiteindelijk ingrijpen om de mislopende sector te redden,
waarbij spaarders in totaal 125 miljard dollar werd terugbetaald. De
FSLIC, alsook de Federal Deposit Insurance Corporation (FDIC), waren
tijdens de crisis van de jaren 1930 (Great Depression) in het leven
geroepen om het vertrouwen van het volk in de banken te herstellen: deze
overheidsinstellingen garandeerden dat bankklanten hun deposito's (tot
een zekere limiet) terugbetaald kregen in geval van een
bankfaillissement. Aangezien de Federal Reserve in haar rol als
kredietverstrekker in uiterste nood tijdens de economische crisis van de
jaren 1930 had gefaald, gaf het deposanten een tweede reden om zich geen
zorgen te maken over fractioneel reservebankieren.

Het klopt dat de reddingsacties de omvang van de spaar- en leencrisis
beperkten en veel persoonlijke drama's voorkwamen. Maar dit ging gepaard
met aanzienlijke kosten: de 125 miljard dollar moest worden betaald door
de overheid, dus in werkelijkheid door de Amerikaanse belastingbetaler.
Zelfs de Amerikanen die zorgvuldig en voorzichtig waren met hun
spaargeld, moesten indirect een deel van de last dragen, terwijl de
spaar- en leningverenigingen en hun klanten er relatief makkelijk vanaf
kwamen.

Aan het begin van zijn carrière had Hayek al zijn zorgen uitgesproken
over het morele risico dat centrale banken in de economie
introduceerden. Dit werd met de FDIC en FSLIC alleen maar explicieter.
Tijdens de spaar- en leencrisis werd het duidelijk dat financiële
instellingen grote risico's konden nemen; de Amerikaanse overheid zou de
rekening betalen als het fout liep.

\section{Dot-Com}\label{dot-com}

Ongeveer tien jaar na de crisis van de spaar- en leenverenigingen, tegen
het einde van de jaren 1990, deden de aandeelmarkten het enorm goed.

Dit kwam deels door een algemeen gevoel van optimisme in de westerse
wereld: eerder in dat decennium was de Sovjet-Unie eindelijk ingestort.
Hoewel Ludwig von Mises al in 1973 overleden was, leek zijn economische
rekenprobleem eindelijk bevestigd te zijn. Nu de dwaasheid van centrale
planning eindelijk bevestigd leek, omarmden voormalige Sovjetlanden de
vrije markteconomie.

Bovenop dit alles werden de Verenigde Staten overmand door een enorme
tech-euforie, wat het duidelijkst weerspiegeld werd op de Nasdaq
-aandelenbeurs. Door bedrijven uit Silicon Valley, zoals Netscape, die
openbaar werden met waarderingen van meerdere miljarden dollars, schoten
technologieaandelen over de hele linie omhoog. Zelfs internetstart-ups,
die vaak niet meer dan een domeinnaam hadden, werden in sommige gevallen
gewaardeerd op tientallen, of zelfs honderden, miljoenen dollars. Het
internet was de toekomst en iedereen wilde er een stuk van hebben.

Maar studenten van Hayeks werk hadden reden om te denken dat er ook iets
anders aan de hand was. De Federal Reserve onder Greenspan had namelijk
in de nasleep van de spaar- en leencrisis de rentetarieven verlaagd naar
de laagste niveaus sinds de jaren 1960. Net zoals in de jaren 1920, was
geld goedkoop en mensen waren maar al te blij om te lenen en te
investeren in de aandelenmarkt. Kunstmatig lage rentetarieven waren de
drijvende kracht achter de economische boom.

En, deze leerlingen van Hayek zouden weten dat de economische realiteit
vroeg of laat een inhaalslag ging maken. En dat deed ze uiteindelijk.
Net voor de eeuwwisseling, besloot Greenspan om de rentetarieven te
verhogen, en plofte de dot-com-bubbel, en de Nasdaq stortte naar
beneden. De razernij was voorbij.

Als Hayek nog in leven was geweest, dan had hij waarschijnlijk betoogd
dat de beste weg voorwaarts zou zijn om op de tanden te bijten en de
markt te laten normaliseren. De economie zou door een pijnlijke recessie
moeten gaan naarmate onrendabele bedrijven zouden sneuvelen, en middelen
langzaam maar zeker herverdeeld konden worden naar meer duurzame
inspanningen.

Maar Greenspan had een ander idee. De monetarist was vastbesloten een
deflatoire schuldenspiraal te voorkomen, dus besloot hij opnieuw de
rente te verlagen. Deze keer liet hij ze ver onder het niveau van de
jaren 1990 zakken, waardoor krediet in de vroege jaren 2000 zelfs
goedkoper was dan het tijdens de opmars van de dot-com-bubbel was.

Op het eerste zicht leek het te werken. In de daaropvolgende jaren begon
de aandelenmarkt langzaam te herstellen. Voor veel economische
commentatoren diende dit als een bevestiging dat het monetarisme naar
behoren had gewerkt. Greenspan had de Amerikaanse economie met minimale
schade door de dot-com-crash geloodst, wat hem zelfs een nieuwe bijnaam
opleverde: `de Maestro'.

Eén sector in het bijzonder beleefde in het midden van de jaren 2000
niets minder dan een volledige economische opleving: de huizenmarkt.

\section{`Te groot om te falen'}\label{te-groot-om-te-falen}

Deze bloei in de huizenmarkt baarde Greenspan, met het oog op monetaire
stabiliteit, nauwelijks zorgen. Hoewel sommige afgeleide prijzen, zoals
de kosten van woninghuur en onderhoud, in acht werden genomen, waren de
werkelijke huizenprijzen zelf niet opgenomen in de CPI
(Consumentenprijsindex); ze waren in 1983 uit de index gehaald. Vastgoed
wordt sindsdien grotendeels beschouwd als een vorm van investering,
hetgeen bijzonder handig was omdat politici in die tijd de
inflatiecijfers wilden verlagen.\footnote{The Economist, `Why don't
  rising house prices count towards inflation?' \emph{The Economist}, 29
  juli 2021,
  \href{https://www.economist.com/the-economist-explains/2021/07/29/why-dont-rising-house-prices-count-towards-inflation}{online}}

Desalniettemin kon de bloei in grote mate worden toegeschreven aan het
beleid van Greenspan. Lage rentetarieven hadden in het begin van de
jaren 2000 de hypotheekrente naar ongekende dieptepunten gestuurd, en de
Amerikaanse woningmarkt floreerde als direct gevolg hiervan. De prijs
van een nieuw huis steeg jaar na jaar, aangezien iedereen leek te willen
profiteren van de kans om goedkoop in te stappen.

En er was nog een andere, verborgen reden voor deze bloei. Financiële
instellingen hadden, voornamelijk sinds de late jaren 1980, complexe
soorten van hypotheekgedekte effecten gebruikt, namelijk gedekte
schuldobligaties. Dit stelde hen in staat om hypotheekschulden in
stukken te hakken en door te verkopen; in plaats van de bank die het
uitgaf, waren de hypotheekschulden steeds meer eigendom van
investeerders, waardoor ook andere financiële instellingen zoals banken,
verzekeringsmaatschappijen en pensioenfondsen betrokken raakten.

Het probleem was echter dat elke hypotheekschuld kon worden voorgesteld
als een praktisch risicovrij activum. Daarom waren sommige financiële
instellingen die hypotheken verstrekten maar al te graag bereid nieuwe
hypotheken te verstrekken aan vrijwel iedereen die er een aanvroeg.
Controles op inkomen, baanzekerheid of kredietwaardigheid werden
grotendeels over het hoofd gezien. De risico's die inherent waren aan
deze hypotheken werden verhuld om ze opnieuw te kunnen verkopen.

Maar de risico's konden niet eeuwig verborgen blijven. Midden jaren 2000
begon Alan Greenspan, de toenmalige voorzitter van de Federal Reserve,
de rente weer te verhogen, en zijn opvolger in 2006, Ben Bernanke,
volgde zijn voorbeeld. Nogmaals voor degenen die het werk van Hayek
hadden bestudeerd: wat er vervolgens gebeurde kwam niet als een
verrassing.

Toen het duurder werd om te lenen, begon de huizenmarkt op te drogen,
terwijl tegelijkertijd steeds meer Amerikanen in gebreke bleven bij de
hypotheken die hen zo vrij waren verstrekt. Toen de huizenprijzen in de
Verenigde Staten begonnen te dalen, ontdekten degenen die hadden
geïnvesteerd in hypotheekgebonden effecten dat ze lang niet zo veilig
waren als geadverteerd, waardoor sommigen van hen in geldnood kwamen om
hun eigen schulden af te betalen.

Toen wanbetalingen (leningen die niet kunnen worden terugbetaald)
doorheen de Amerikaanse financiële sector begonnen te verspreiden als
een epidemie en steeds grotere bedrijven beïnvloedden, werd de omvang
van de crisis steeds duidelijker. Toen de financiële reus Lehman
Brothers in september 2008 het grootste bedrijf werd dat ooit
faillissement indiende in de geschiedenis van de Verenigde Staten,
wisten financiële professionals, beleidsmakers en iedereen die oplette
dat verdere escalatie tot een volledige economische neerval kon leiden.

Toen het erop leek dat de grote verzekeringsmaatschappij AIG wellicht de
volgende zou zijn om te bezwijken, heeft de Federal Reserve, ondersteund
door een nieuwe noodwet, een werkelijk opmerkelijke stap genomen. De
centrale bank verklaarde de verzekeraar `te groot om te falen', en samen
met de Amerikaanse schatkist redde ze AIG door middel van een injectie
van \$ 68 miljard (plus nog eens \$ 112 miljard aan garanties).

Dit luidde een nieuw beleidstijdperk in, zowel in de VS als daarbuiten,
omdat de crisis internationaal om zich heen greep. In de daaropvolgende
weken coördineerde de Federal Reserve nogmaals met het Amerikaanse
Ministerie van Financiën om het voor \$ 405 miljard aan noodlijdende
activa te laten kopen, terwijl aan de andere kant van de Atlantische
Oceaan de Britse minister van Financiën, Alistair Darling, ook een
noodmaatregel uitrolde in de vorm van een bankreddingsplan ter waarde
van £ 137 miljard (\$ 230 miljard). Vergelijkbare maatregelen werden in
andere Europese landen genomen.\footnote{US Department of the Treasury,
  `Troubled Asset Relief Program',
  \href{https://home.treasury.gov/data/troubled-asset-relief-program}{online};
  Federico Mor, `Bank Rescues of 2007-09: Outcomes and Cost',
  \emph{House of Commons Research Briefing}, 8 oktober 2018,
  \href{https://commonslibrary.parliament.uk/research-briefings/sn05748/}{online}}

Een onmiddellijke financiële instorting werd afgewend, maar alleen omdat
grote delen van de financiële sector werden gered door openbare
instellingen en de creatie van enorme hoeveelheden nieuw geld uit het
niets; het morele risico stond nu volop in de schijnwerpers.

\section{Een nieuwe wereld}\label{een-nieuwe-wereld}

De overheidsinterventies zouden zich niet alleen tot reddingsoperaties
beperken.

Midden in een crisis had de Federal Reserve opnieuw de rente verlaagd
met als doel deflatie te vermijden en de economie weer op de rails te
krijgen, deze keer tot bijna nul procent: geld lenen werd bijna gratis.
Maar het leek weinig effect te hebben.

In november 2008 kondigde de Federal Reserve daarom een grootschalig
programma aan ter aankoop van activa, genaamd Quantitative Easing
(Kwantitatieve versoepeling). De Fed zou nog eens voor \$ 600 miljard
aan hypotheekgedekte effecten opkopen met vers gecreëerde dollars, en
breidde dit kort daarna uit naar het opkopen van bank- en
overheidsschuld. De centrale bank kreeg al snel voor meer dan \$ 2000
miljard van deze drie activa in handen, waardoor de totale waarde van
haar balans bijna verdrievoudigde.

Quantitative Easing (QE) stelt centrale banken in staat risicovolle
activa van de markt te halen, maar het hoofddoel van deze programma's is
om het geldaanbod te vergroten en deflatie tegen te gaan wanneer
traditionele instrumenten de klus niet klaren. Met andere woorden, QE
kan helpen om inflatie te stimuleren wanneer de rentetarieven tot nul
(of bijna nul) zijn gedaald, door wanhopig duizenden miljarden
rechtstreeks in de economie te pompen; een voorbeeld dat al snel door
andere centrale banken over de hele wereld werd gevolgd.

En het \emph{was} wanhopig. Volgens de toonaangevende theorie van
Bernanke over geld en rente, zou QE zelfs helemaal geen inflatie moeten
stimuleren als een rentetarief van nul procent dat al niet deed. Toch
bleek het enigszins te werken. Zoals de voorzitter van de Fed grappend
zei: `Het probleem met QE is dat het in de praktijk goed werkt, maar
niet in theorie.'\footnote{Rupal Patel and Jack Meaning, `Can't We Just
  Print More Money? Economics in Ten Simple Questions', The Bank of
  England, 238.}

Toch betraden de centrale banken met de invoering van QE bijna volledig
onbekend terrein. Niemand die leidinggaf aan deze opkoopprogramma's wist
precies wat de effecten zouden zijn, of hoe lang het zou duren voordat
dergelijke effecten zichtbaar werden. Sterker nog, de financiële crisis
van 2008 en de nasleep daarvan markeerden het begin van een nieuwe en
onzekere wereld van geld en financiën.

Oorspronkelijk opgericht als een leningverstrekker in absolute nood,
ongeveer een eeuw geleden, was de Federal Reserve al bijna even lang de
rentevoeten aan het manipuleren. Nu begon het ook winnaars en verliezers
in de markt te kiezen met reddingspakketten en begon het zelfs middelen
toe te wijzen doorheen de vermogensmarkten via QE. Het mandaat van de
centrale bank was na verloop van tijd uitgebreid om taken te omvatten
die even goed de verantwoordelijkheden zouden kunnen beschrijven die
doorgaans aan centrale planners van de Sovjet-Unie werden toegeschreven.

Ondertussen bood de crisis --- net als de crisis van de jaren 1930 ---
een klimaat waarin Keynesiaanse ideeën een heropleving konden maken: een
groep economen, bekend als de `Neo-Keynesianen'\footnote{De
  Neo-Keynesianen integreerden meer van de neoklassieke ideeën over
  vrije markten, maar namen ook meer mogelijke marktfalen aan, met name
  door het onvermogen van sommige prijzen om zich snel aan te passen aan
  nieuwe omstandigheden, wat overheidsingrijpen rechtvaardigt. Dit geldt
  vooral voor de arbeidsprijs, vanwege psychologische factoren.},
drongen er bij overheden wereldwijd op aan om het begrotingstekort te
verhogen om zo nationale economieën te stimuleren. Toen IMF-directeur
Dominique Strauss-Kahn in 2008 deze `Keynesiaanse heropleving'
aanbeveelde, stemden wereldleiders, onder wie de Amerikaanse president
George W. Bush en de Britse premier Gordon Brown, in met het steunen van
de plannen. Financiële stimuleringspakketten werden al snel over de hele
wereld uitgerold.

Het grootste deel van Hayeks leven waren zijn ideeën gemarginaliseerd.
Hayek had al in de jaren 1920 kritiek geuit op de rol van centrale
banken, en sinds de jaren 1930 waarschuwde hij dat Keynesiaanse
maatregelen weinig deden buiten het verlengen van onhoudbare economische
oplevingen. Toch leken de zaken in de loop van de tijd alleen maar erger
te worden. Geld was geëvolueerd tot een geopolitiek schaakstuk voor
nationalisten, een machtsmiddel voor economische planners van centrale
banken en vooral, een bron van economische onrust --- dat op zijn beurt
de heropleving van de Keynesiaanse doctrine bevorderde.

Het monetaire systeem was in de vroege eenentwintigste eeuw zo ver
verwijderd van Hayeks ideaal als maar mogelijk kon zijn, en er was geen
oplossing in zicht.

Tenzij je toevallig geabonneerd was op een technologiemailinglijst in
een bijna vergeten hoekje van het internet\ldots{}

\chapter{Het witboek}\label{het-witboek}

\textbf{De financiële markten} waren al in rep en roer toen Adam Back in
de zomer van 2008 een e-mail ontving van iemand die zichzelf `Satoshi
Nakamoto' noemde. Nakamoto legde uit dat hij een digitaal geldsysteem
had ontworpen dat gebaseerd was op het proof-of-work-systeem dat Back
meer dan tien jaar eerder had geïntroduceerd via hashcash. De e-mail
bevatte een link naar een conceptversie van een witboek, getiteld
`Elektronisch Geld Zonder een Vertrouwde Derde Partij.' Nakamoto was
geïnteresseerd in feedback.

Back had nog nooit van de naam Satoshi Nakamoto gehoord, en, als
betrokken lid van de Cypherpunkbeweging sinds midden jaren 1990, had hij
al te veel mislukte pogingen gezien om digitale valuta te creëren om
hoge verwachtingen te hebben van het nieuwe voorstel in zijn inbox. Toch
had Nakamoto genoeg interesse bij de Britse Cypherpunk gewekt om het
document te lezen, en hij merkte enkele overeenkomsten met het
b-money-voorstel van Wei Dai op. Back wees Nakamoto hierop in zijn
antwoord, maar liet het daar bij.

Ook Wei Dai kreeg al snel bericht van Satoshi Nakamoto.

`Ik had veel interesse om jouw b-money pagina te lezen', stond er in de
e-mail van Nakamoto. `Ik sta op het punt om een paper te publiceren dat
jouw ideeën uitbreidt tot een volledig functioneel systeem.'\footnote{Satoshi
  Nakamoto and Wei Dai, `Wei Dai/Satoshi Nakamoto 2009 Bitcoin emails',
  \emph{gwern.net}, laatste update op 14 september 2017,
  \href{https://gwern.net/doc/bitcoin/2008-nakamoto}{online}}

Nakamoto vroeg vervolgens wanneer het ontwerp van b-money voor het eerst
werd gepubliceerd: hij wilde naar Dais voorstel voor elektronisch geld
verwijzen in de definitieve versie van zijn witboek. Deze e-mail bevatte
ook weer een link naar het concept.

Dai--- die net als Back, Satoshi Nakamoto niet kende --- reageerde met
links naar een webarchief met de originele b-money e-mail en de
relevante discussies op de Cypherpunks-mailinglijst van tien jaar
eerder. `Bedankt dat je me over je paper hebt verteld', voegde Dai toe.
`Ik zal het eens bekijken en je laten weten of ik opmerkingen of vragen
heb.'\footnote{Nakamoto and Dai, `Wei Dai/Satoshi Nakamoto 2009 Bitcoin
  emails.'}

Dai gaf geen vervolg aan hun correspondentie. Tegen die tijd had hij de
hoop opgegeven dat digitale geldsystemen, geïnspireerd door de
Cypherpunkbeweging, genoeg gebruikers konden aantrekken om een zinvol
verschil in de wereld te maken. Hij schonk nauwelijks aandacht aan
Nakamoto zijn ontwerp. In plaats daarvan koos hij ervoor om zijn tijd te
besteden aan beslissingstheorie en andere benaderingen van AI-veiligheid
(de oorspronkelijke reden waarom hij geïnteresseerd raakte in
cryptografie).

Ondertussen leek Nakamoto de informatie te hebben vergaard die hij nodig
had. Terwijl de financiële crisis van 2008 in de daaropvolgende weken en
maanden volledig tot uiting kwam, werd er niets meer gehoord van
Nakamoto.

Tot en met 31 oktober. Alle abonnees van de Cryptography-mailinglijst
ontvingen op dit moment een e-mail van Satoshi Nakamoto. Doordat deze
lijst fungeerde als de feitelijke opvolger van de Cypherpunks-lijst,
werden veel van de originele Cypherpunks nu voorgesteld aan het nieuwe
voorstel voor een digitale valuta. `Bitcoin P2P e-cash paper', luidde
het onderwerp. \footnote{Satoshi Nakamoto, `Bitcoin P2P e-cash paper.'
  oorspronkelijk verstuurd naar de Cypherpunk-mailinglijst. 31 oktober
  2008.
  \href{https://www.metzdowd.com/pipermail/cryptography/2008-October/014810.html}{online}}

Inderdaad, het elektronische geldproject van Nakamoto had nu een naam:
\emph{Bitcoin}.

`Ik heb mij toegelegd om een nieuw elektronisch geldsysteem te creëren
dat volledig van persoon tot persoon functioneert en geen vertrouwde
derde partij vereist', luidde Nakamotos e-mail, alvorens de
hoofdeigenschappen samen te vatten:

\begin{itemize}
\item
  \emph{Dubbele uitgaven worden voorkomen met een peer-to-peer netwerk.}
\item
  \emph{Geen mint of andere vertrouwde partijen.}
\item
  \emph{Deelnemers kunnen anoniem zijn.}
\item
  \emph{Nieuwe munten worden gemaakt op basis van proof-of-work in de
  stijl van Hashcash.}
\item
  \emph{Het proof-of-work voor het genereren van nieuwe munten geeft het
  netwerk ook de energie om dubbele uitgaven te voorkomen.}
\item
\end{itemize}

Bijgevoegd was een link naar de actuele versie van het
witboek.\footnote{Satoshi Nakamoto, `Bitcoin: A Peer-to-Peer Electronic
  Cash System', \href{https://bitcoin.org/bitcoin.pdf}{online}}
`Bitcoin: Een Peer-to-Peer Elektronisch Geldsysteem', luidde nu de
titel. In amper negen pagina's (inclusief een bladzijde voor externe
referenties), schetste Nakamoto de kernmechanismen van zijn digitale
valuta. Het maakte een compacte, maar zeer efficiënte beschrijving van
--- zoals de inleiding van het paper omschreef --- `een elektronisch
betalingssysteem gebaseerd op cryptografisch bewijs in plaats van
vertrouwen.'

\section{De block chain}\label{de-block-chain}

Het systeem dat Satoshi Nakamoto in zijn whitepaper beschreef leek
daadwerkelijk op b-money --- dat nu als eerste van acht verwijzingen
wordt genoemd --- op meerdere manieren.

Net als in Wei Dais ontwerp voor elektronisch geld zouden de
munteenheden van Bitcoin (ook wel \emph{bitcoin} genoemd, maar meestal
geschreven met een kleine letter b) niet worden gedekt, terwijl eigendom
van bitcoin zou worden toegeschreven aan publieke sleutels. Transacties
zouden in essentie cryptografisch ondertekende berichten zijn die
aangeven dat de aan deze publieke sleutels toegeschreven munten
overgedragen worden aan andere publieke sleutels. Het eigendom van
Bitcoin zou dan cryptografisch gegarandeerd zijn; bitcoin zou alleen
kunnen worden verplaatst met geldige handtekeningen.

Dit heeft als doel de privacy van de gebruikers te beschermen, hoewel
Nakamoto in zijn witboek toegeeft dat dit niet helemaal perfect zou
zijn.

`Sommige connecties zijn nog steeds onvermijdelijk met multi-input
transacties, aangezien deze noodzakelijkerwijs onthullen dat de inputs
toebehoorden aan dezelfde eigenaar', schreef hij. `Het risico bestaat
als de eigenaar van een sleutel wordt onthuld, dat connecties andere
transacties kunnen onthullen die toebehoren aan dezelfde eigenaar.'

Maar misschien nog wel het interessantste is dat Nakamoto Bitcoin heeft
ontworpen volgens de meest ambitieuze variant van b-geld, waarbij alle
gebruikers een grootboek bijhouden om het eigendom van de valuta binnen
het systeem te volgen. Elke node (elke deelnemer op het netwerk) zou
alle nieuwe transacties zien, hun geldigheid controleren en hun
grootboeken dienovereenkomstig bijwerken --- terwijl ongeldige
transacties (waaronder dubbele uitgaven) zouden worden afgewezen.

Tien jaar eerder had Dai de conclusie getrokken dat zo'n gedistribueerde
aanpak niet praktisch was. Zonder een synchrone en onverstoorbare
anonimiteit van het uitzendkanaal, konden verschillende delen van het
netwerk verschillende transacties als eerste waarnemen. Hierdoor zouden
transacties die dubbel worden uitgegeven, in principe het netwerk
opsplitsen, waar verschillende nodes conflicterende basisregisters
bijhouden. Met potentieel oneerlijke deelnemers die uit zijn op het
zaaien van verdeeldheid, en zonder een leider die beslist welke versie
van het grootboek de daadwerkelijke toestand van het netwerk
vertegenwoordigt, zag Dai geen manier om dergelijke splitsingen op te
lossen. Het was in feite een perfect voorbeeld van het Byzantijnse
Generaalsprobleem, en Satoshi Nakamoto geloofde dat hij dit probleem had
opgelost.

Bij het ontwerpen van dit systeem, leek het erop dat Nakamoto inspiratie
had geput uit het werk van Scott Stornetta en Stuart Haber: drie van de
volgende vier referenties in het witboek wezen naar de vertrouwenloze
tijdstempelpapers van Stornetta en Haber (inclusief degene die ze samen
schreven met Dave Bayer), terwijl de vierde refereerde naar een voorstel
voor tijdstempels door Belgische onderzoekers die sterk leunden op het
werk van Stornetta en Haber. In een aparte referentie, gaf Nakamoto ook
erkenning aan de fundamentele paper van Ralph Merkle die Hash-bomen
(Merkle Trees) voor het eerst beschreef.

Het paste als gegoten. Wei Dais idee om het grootboek van eigendom over
alle gebruikers te verdelen stond filosofisch zeer dicht bij het concept
van Haber en Stornetta, waarin kopieën van het basisregister van een
tijdstempelprotocol werden gedeeld. In beide gevallen zou elke deelnemer
de belangrijke informatie zelf verifiëren, zodat ze zeker zouden weten
dat niemand kon valsspelen. Elk individu zou een ander eerlijk houden.

In het meest geavanceerde voorstel van Stornetta en Haber verwerkten
gebruikers documenten samen in een grote Merkle-boom, zodat de
Merkle-wortel opgenomen kon worden in de volgende Merkle-boom. Dit
resulteerde in een wiskundig verifieerbare chronologische volgorde van
Hash-bomen en dus ook een chronologische volgorde van documenten
opgenomen in de Hash-bomen. Zolang de gebruikers van het systeem hun
eigen basisregister bijhielden, konden ze altijd bewijzen dat een
bepaald document was opgenomen vóór een ander document en op welk
tijdstip.

Het elektronische geldsysteem van Nakamoto was op een zeer vergelijkbare
manier ontworpen: de documenten van Haber en Stornettas
tijdstempeloplossingen werden in Bitcoin in feite vervangen door
transacties. Terwijl transacties via het peer-to-peer netwerk van
Bitcoin werden verzonden, zouden gebruikers ze met enige regelmaat
samenvoegen in een Merkle-boom. Een dergelijke bundel van transacties
zou het grootste deel uitmaken van wat een `blok' werd genoemd. Als twee
of meer conflicterende transacties in het netwerk circuleerden, kon
slechts een van hen in een blok worden opgenomen, en alleen transacties
die in een blok waren opgenomen zouden als bevestigd worden beschouwd.

En, net zoals in Haber en Stornettas basisregisters, zou elk nieuw
Bitcoinblok ook de hash bevatten van het vorige blok, ook wel de
`blok-hash' genoemd. Dit zou het onmogelijk maken om een ouder blok te
wijzigen en het nog steeds wiskundig te laten passen in de ketting van
alle blokken, aangezien dit noodzakelijkerwijs de hele ketting vanaf dat
blok zou veranderen, iets wat alle gebruikers zouden opmerken.

Inderdaad, de inhoud van Bitcoinblokken, hun chronologische volgorde en
daarmee de volgorde van alle transacties daarin, zouden in feite
cryptografisch zijn verzegeld.

Nakamoto noemde deze ketting van blokken in de mailinglijst: de block
chain.

\section{Minen}\label{minen}

Omdat een Bitcoinblok alleen als geldig zou worden beschouwd als het
geen conflicterende transacties bevat, bood de `block chain' het begin
van een oplossing voor het probleem van dubbele uitgaven. Zelfs als
verschillende gebruikers de transacties in een andere volgorde zouden
zien, zouden hun eigendomsregisters nog steeds overeenkomen zolang ze
alle blokken in dezelfde volgorde ontvingen. Bij een poging tot dubbele
uitgave zou alleen de transactie die in een blok was opgenomen worden
gebruikt om alle grootboeken bij te werken.

Toch loste dit het probleem van dubbele uitgaven niet volledig op. Als
verschillende gebruikers namelijk verschillende blokken creëren
(mogelijk met conflicterende transacties in deze verschillende blokken),
en de verschillende blokken worden gelijktijdig naar verschillende delen
van het netwerk gestuurd, zou exact hetzelfde probleem weer opduiken:
het netwerk zou opsplitsen.

En aangezien elk volgend blok de hash van het voorgaande blok zou
bevatten, zou dit zelfs betekenen dat verschillende delen van het
netwerk uiteindelijk compleet verschillende, onverenigbare block chains
zouden creëren. Hierdoor zouden gebruikers op termijn hun
eigendomsrechten op diverse manieren bijhouden en permanent buiten
consensus vallen. Gebruikers van de verschillende delen van het netwerk
zouden niet meer met elkaar kunnen handelen.

Nakamoto wist dat alle Bitcoingebruikers zich moesten verenigen rond
dezelfde block chain, ook al betekende dit dat sommige gebruikers af en
toe de blokken moesten opgeven die ze als eerste hadden ontvangen in
geval van een conflict. Maar met mogelijk onbetrouwbare deelnemers en
niemand die de leiding heeft, was het bepalen van op welke block chain
zich te vestigen alweer een uitstekend voorbeeld van het Byzantijnse
Generaalsprobleem.

Hier komt het proof-of-work-systeem van Adam Back bij kijken, met zijn
originele hashcashvoorstel dat de volgende referentie is in het
Bitcoin-witboek. proof-of-work vormde op dat moment al de basis voor
verschillende digitale muntontwerpen van Cypherpunks, waaronder b-money,
Bit Gold en RPOW. Maar waar deze ontwerpen gewoonlijk de
proof-of-work-hashes \emph{zelf} gebruikten als een vorm van geld, had
Nakamoto, in wat een van zijn belangrijkste inzichten was, er een
vernuftig nieuw gebruik voor bedacht.

proof-of-work wordt in Bitcoin gebruikt als consensusmechanisme.

Naast een set van transacties en de hash van het voorgaande blok, zouden
Bitcoinblokken een derde ingrediënt bevatten: een nonce. Dit
willekeurige getal zou samen met de rest van de inhoud van een blok
gehasht worden om de blok-hash te genereren. De truc van het
proof-of-work-systeem was dan ook dat niet elk blok als geldig zou
worden beschouwd. Alleen blokken met een blok-hash die begint met een
vooraf bepaald aantal nullen zouden worden geaccepteerd door het netwerk
van gebruikers.

Net zoals bij de productie van hashcash, zou de enige manier om een
geldig blok te vinden, via vallen en opstaan zijn: een proces dat
Satoshi Nakamoto later \emph{mining} zou noemen. De mensen die aan het
\emph{minen} zijn (of \emph{de miners}) zouden willekeurig veel
verschillende nonces in een gewenst blok moeten proberen op te nemen,
totdat één van hen een geldige blok-hash zou genereren. Een geldig blok
--- dat door niemand zou kunnen worden aangepast nadat het geproduceerd
was, omdat dat de blok-hash ook zou veranderen --- zou dan over het
netwerk worden verzonden, waarbij elke gebruiker hun eigendomsregister
zou bijwerken met de transacties in dit blok.

Miners zouden ondertussen hun inspanningen aanpassen om de nieuwe
blok-hash op te nemen in een mogelijk volgend blok, dat op zijn beurt
zijn eigen geldige blok-hash zou vereisen. Vergelijkbaar met hoe Bit
Goldgebruikers (en vermoedelijk b-moneygebruikers) een cryptografische
ketting van hashes zouden produceren om valuta te creëren, waarbij elke
geldige hash diende als potentiële tekenreeks voor de volgende, zouden
Bitcoinminers een cryptografische ketting van blok-hashes produceren.

En wat cruciaal is: de lengte van deze ketting zou als beslissende
factor dienen in het geval van een conflict.

Als twee conflicterende blokken over het Bitcoinnetwerk zouden
circuleren, zou elke gebruiker in eerste instantie het blok accepteren
dat ze eerst ontvangen, en miners zouden de hash van dat blok opnemen in
het volgende blok dat ze zouden proberen te vinden. In zekere zin zou
het Bitcoinnetwerk inderdaad splitsen. Maar deze splitsing zou tijdelijk
zijn. Zodra de ene kant van de splitsing sneller het volgende blok delft
dan de andere kant van de splitsing, en hun versie van de block chain
langer wordt dan het alternatief, zouden Bitcoingebruikers en -miners
van beide kanten van de splitsing deze langste ketting accepteren,
waarbij ze de kortere ketting verlaten en daarmee de splitsing oplossen.

Om het Byzantijnse Generaalsprobleem te overkomen, bedacht de sluwe
Satoshi Nakamoto een nieuwe toepassing voor proof-of-work door het in te
zetten als een decentrale beslisser om consensus te bereiken. Omdat
iedereen een proof-of-work kan uitoefenen, en omdat iedereen eenvoudig
de geldigheid ervan kan controleren zonder anderen te hoeven vertrouwen,
paste dit perfect binnen het leiderloze ontwerp van Bitcoin.

`Het netwerk is robuust in zijn ongestructureerde eenvoud', concludeerde
Nakamoto in zijn witboek. `De nodes werken allemaal tegelijk met weinig
coördinatie.'

\section{Muntuitgifte}\label{muntuitgifte}

Als Bitcoingebruikers proof-of-work zouden produceren, en dus blokken
minen, hadden ze een aansporing nodig.

Miners zouden dus beloond worden met bitcoineenheden als ze een geldig
blok vonden, legde Nakamoto uit in zijn witboek. Deze
\emph{blokbeloning} zou deels bestaan uit transactiekosten, betaald door
andere gebruikers om hun transactie in een nieuw blok op te nemen. Maar
het grootste deel van de blokbeloning zou aanvankelijk bestaan uit
gloednieuwe munten.

Dit loste op een elegante manier twee problemen tegelijk op: ten eerste
bood het een aansporing om te minen (waardoor het netwerk over de staat
van het grootboek kon overeenkomen), en ten tweede fungeerde het als
methode om nieuwe valuta in omloop te brengen zonder een centrale
uitgever. Bovendien deed het dit op een bijzonder slimme manier: hoewel
proof-of-work kon worden gebruikt om nieuwe valuta te verdienen, was de
hoeveelheid bitcoin die bij elk nieuw blok in omloop kwam in feite
vastgelegd. Ongeacht hoeveel energie het had gekost om een geldig blok
te produceren, het aantal nieuwe munten die per blok werd beloond, zou
hetzelfde blijven.

Daarenboven --- en dit kan wel eens een van Nakamoto's belangrijkste
originele innovaties zijn, die niet gebaseerd zijn op eerdere digitale
geldsystemen --- zou een \emph{aanpassingsalgoritme} voor de
moeilijkheidsgraad ervoor zorgen dat nieuwe blokken op een zo
gelijkmatig mogelijke snelheid zouden worden gevonden. Als er te veel
blokken te snel geproduceerd zouden worden, zoals aangegeven door de
tijdstempels in elk nieuw blok, zouden alle nodes op het netwerk
automatisch gaan eisen dat nieuwe blokken meer proof-of-work zouden
moeten bevatten. (Met als resultaat dat er meer rekenkracht nodig is om
een geldige blok-hash te vinden, omdat nieuwe hashes meer nulwaarden aan
het begin zouden vereisen.) Ook als blokken te langzaam gevonden zouden
worden, zouden alle nodes beginnen met het accepteren van nieuwe blokken
die minder proof-of-work bevatten (minder nulwaarden aan het begin).

Met blokken die tegen een redelijk gelijkmatig tempo worden gevonden, en
elk blok een vastgesteld aantal nieuwe munten uitgeeft, zou de snelheid
van muntaanmaak voorspelbaar zijn --- ongeacht de hoeveelheid hashkracht
die aan het netwerk wordt toegeschreven. Waar systemen zoals hashcash en
RPOW te maken zouden hebben gehad met hyperinflatie, omdat de kosten om
een geldige hash te produceren in de loop van de tijd bleven dalen door
hardwareverbeteringen, was Bitcoin ontworpen om zich aan een vooraf
geprogrammeerd uitgifteschema te houden.

Door de hoeveelheid proof-of-work te ontkoppelen van het tempo van
valutacreatie, heeft Nakamoto het inflatieprobleem opgelost. Dit maakt
dat de uitgifte van Bitcoin een grotere gelijkenis heeft met die van een
edelmetaal.

`Het stabiel toevoegen van een vastgesteld aantal nieuwe munten is te
vergelijken met gouddelvers die middelen verbruiken om goud in omloop te
brengen', legt het Bitcoin- witboek uit. `In ons geval gaat het om
CPU-tijd en elektriciteit die/dat verbruikt wordt.'

\section{Positieve aansporing}\label{positieve-aansporing}

Nakamoto was bovendien van mening dat het uitgiftemodel van Bitcoin
potentiële aanvallers zou ontmoedigen.

Het meest voor de hand liggende is dat oneerlijke deelnemers niet
gemakkelijk dubbele uitgaven in transacties kunnen doen, aangezien
slechts één van de conflicterende transacties in de block chain kan
worden opgenomen.

De enige manier om een dubbele uitgaveaanval uit te voeren, zou zijn als
de aanvaller een van zijn transacties in een blok weet op te nemen en
deze als betaling wordt geaccepteerd door de ontvanger, om vervolgens
zelf een conflicterend blok te minen met de conflicterende transactie,
en doorgaan met het minen van deze alternatieve block chain totdat hij
langer is dan de originele ketting. Als het hem inderdaad zou lukken om
de langste ketting te maken (met de dubbele uitgavetransactie erin),
zouden alle Bitcoingebruikers overschakelen naar deze alternatieve
ketting, en iedereen zou hun grootboeken dienovereenkomstig bijwerken.
De originele transactie zou worden ingetrokken en de dubbele uitgave zou
gelukt zijn.

Echter, zolang de aanvaller niet over meer rekenkracht beschikt dan de
rest van het netwerk tezamen, zou de kans dat hij de eerlijke ketting
inhaalt, exponentieel afnemen voor elk blok dat hij achterloopt, legt
Nakamoto uit. De eerlijke ketting zou vrijwel zeker sneller groeien. Hij
onderbouwde zijn uitleg met de achtste en laatste verwijzing in het
witboek, en ook de oudste: het handboek \emph{An Introduction to
Probability Theory and Its Applications}' uit 1957, geschreven door
wiskundige William Feller.

Om een dubbele uitgave te voorkomen, zou de eenvoudigste oplossing zijn
om te wachten tot er enkele blokken zijn ontgonnen boven op het blok dat
een binnenkomende transactie bevat, voordat de betaling als definitief
wordt beschouwd, schreef Nakamoto. Elk nieuw blok zou een extra
bevestiging van de betreffende transactie vertegenwoordigen en met
slechts een paar bevestigingen zou het in de meeste gevallen uiterst
onwaarschijnlijk zijn dat een aanvaller ooit zou kunnen inhalen. En
aangezien een aanvaller in middelen zou moeten investeren om rekenkracht
te verkrijgen om het zelfs maar te proberen, zou een aanvalspoging
meestal niet de moeite waard zijn.

Dit gezegd zijnde: het wachten op meer bevestigingen zou niet helpen als
een aanvaller daadwerkelijk meer rekenkracht had dan de rest van het
netwerk tezamen. In dat scenario zou de aanvaller uiteindelijk altijd
kunnen inhalen en de langste ketting kunnen genereren, waarmee hij naar
believen dubbel kon uitgeven.

Maar zelfs als dat het geval is, kan de aanvaller zijn aanval niet
kosteloos uitvoeren; hij zou nog steeds het benodigde proof-of-work
moeten leveren om geldige blokken te creëren.

Nakamoto veronderstelde dat de blokbeloningen die door het
Bitcoinprotocol worden toegekend op zich al een mogelijke aanvaller
zouden kunnen weerhouden om een dubbele uitgave te wagen:

`De stimulans kan wellicht nodes ertoe aanzetten om eerlijk te blijven.
Als een hebzuchtige aanvaller erin slaagt meer CPU-vermogen te
verzamelen dan alle eerlijke nodes, dan zou hij moeten kiezen tussen
mensen te bedriegen door zijn betalingen terug te stelen, of om nieuwe
munten te genereren. Het zou voor hem winstgevender zijn om volgens de
regels te spelen, regels die hem verrijken met meer nieuwe munten dan de
rest, in plaats van het systeem en de geldigheid van zijn eigen vermogen
te ondermijnen.'

Zelfs in het ergste geval, zijn de stimuli van Bitcoin waarschijnlijk
zodanig afgestemd dat iedereen eerlijk handelt.

\section{Peer-To-Peer}\label{peer-to-peer}

Bitcoin was, zoals Nakamoto in zijn e-mailaankondiging had beloofd,
ontworpen om een echt peer-to-peer-systeem te zijn.

Alle gebruikers zouden gelijk zijn binnen het netwerk, ze helpen elkaar
het systeem draaiende te houden door het creëren en doorsturen van
transacties en blokken, zonder enige speciale privileges of vertrouwde
entiteiten. Er zou geen bedrijf zoals Digicash zijn om failliet te gaan,
geen Bit Gold-eigendomsclub om te beslissen wie wat bezit, en ook geen
vertrouwenloze RPOW-server om stop te zetten. Net als BitTorrent was
Bitcoin in essentie ontworpen om een nieuw internetprotocol te zijn dat
iedereen kon gebruiken, maar dat niemand zou beheersen.

Om dit klaar te spelen, moest Satoshi Nakamoto enkele van de meest
hardnekkige problemen oplossen waar eerdere ontwerpen van
gedecentraliseerde elektronische valuta mee worstelden: door dubbele
uitgaven te voorkomen zonder een centrale partij, vond hij een oplossing
voor het Byzantijnse Generaalsprobleem en ontdekte hoe inflatie in een
proof-of-work-systeem beperkt kan worden ondanks voortdurende
verbeteringen van de hardware. En, op het tijdstempelgebaseerde
aanpassingsalgoritme van de moeilijkheid na, had hij dit gedaan zonder
dat er baanbrekende technologieën nodig waren. Nakamoto gebruikte
verschillende instrumenten uit de wereld van elektronisch geld en
cryptografie die al minstens een decennium eerder waren ontwikkeld, en
combineerde ze op een slimme manier.

Daarbovenop was zijn timing ook verbazingwekkend. Net toen centrale
banken over de hele wereld ongekende maatregelen in het financiële
systeem implementeerden in een wanhopige poging om een totale
economische instorting te voorkomen, stelde Satoshi Nakamoto een nieuw
soort geld voor. Dit geld kon volledig zonder financiële instellingen
functioneren --- een digitaal valutasysteem dat geheel op wiskunde was
gebaseerd.

Toch was de reactie op de publieke aankondiging van Bitcoin grotendeels
zonder erkenning of waardering.

\section{\texorpdfstring{\textbf{Schaalcapaciteit}}{Schaalcapaciteit}}\label{schaalcapaciteit}

De eerste reactie op Nakamotos aankondiging kwam ongeveer een dag later,
van James A. Donald, een Cypherpunk die toevallig bijna klaar was met
het ontwerpen van zijn eigen digitaal geldsysteem.

`Zo'n systeem hebben we zeer, zeer hard nodig, maar zoals ik jouw
voorstel begrijp, lijkt het niet op te schalen naar de vereiste omvang',
schreef Donald. `Om tijdig dubbele uitgaven te detecteren en te
verwerpen, moet men de meeste vorige transacties van de munten in de
transactie hebben, wat, naïef geïmplementeerd, vereist dat elk persoon
de meeste van eerdere transacties heeft, of de meeste transacties die
recentelijk hebben plaatsgevonden. Als honderden miljoenen mensen
transacties uitvoeren, is dat heel veel bandbreedte --- iedereen moet
alles weten, of een aanzienlijk deel daarvan.'\footnote{James A. Donald,
  `Bitcoin P2P e-cash paper', oorspronkelijk verstuurd naar de
  Cypherpunk-mailinglijst, 2 november 2008,
  \href{https://www.metzdowd.com/pipermail/cryptography/2008-November/014814.html}{online}}

Inderdaad, Nakamotos ontwerp vereiste dat gebruikers alle transacties op
het Bitcoinnetwerk bijhielden, om zo hun lokale versies van het
eigendomsgrootboek te kunnen bijwerken. Ze zouden precies moeten weten
welke munten al waren uitgegeven en welke niet, om er zeker van te zijn
dat een munt die ze als betaling ontvingen nog niet aan iemand anders
was uitgegeven. Als het Bitcoinnetwerk groot genoeg zou worden, kon dit
onuitvoerbaar worden voor de meeste standaardgebruikers.

Het was een probleem waarover Satoshi Nakamoto nagedacht had. In zijn
witboek stelde hij een oplossing bestaande uit twee stappen voor,
genaamd `Simplified Payment Verification' (SPV). Ten eerste, om de
hoeveelheid benodigde schijfruimte voor het draaien van Bitcoin op de
gemiddelde computer te minimaliseren, konden oudere blokken worden
verwijderd van de computers, zodat ze alleen de blok-hashes hoefden op
te slaan. En ten tweede, door gebruik te maken van hashbewijzen, konden
gebruikers controleren met behulp van deze blok-hashes dat transacties
naar hen waren opgenomen in de block chain --- waarbij ze alle andere
transacties grotendeels negeerden. Het draaien van een volledig
validerende netwerknode `zou meer en meer worden overgelaten aan
specialisten met serverfarms van gespecialiseerde hardware', schreef
Nakamoto als antwoord op Donald in de mailinglijst. \footnote{Satoshi
  Nakamoto, `Bitcoin P2P e-cash paper', oorspronkelijk verstuurd naar de
  Cypherpunk-mailinglijst, 2 november 2008,
  \href{https://www.metzdowd.com/pipermail/cryptography/2008-November/014815.html}{online}}

Helaas loste deze oplossing het probleem niet volledig op, zoals
Nakamoto zelf ook erkende in het witboek. Het introduceerde ook nieuwe
problemen. SPV liet het beveiligingsmodel waarin iedereen elkaar in de
gaten houdt varen, omdat in plaats daarvan alleen toegewijde miners
zouden controleren of de regels van het systeem altijd worden nageleefd.

Donald reageerde op Nakamoto in de mailinglijst en waarschuwde dat als
slechts een klein deel van de Bitcoingebruikers over voldoende middelen
kon beschikken om een miner te zijn, deze een doelwit konden worden en
een drukpunt voor regelgevende instanties.

In een lange en gedetailleerde repliek beschreef hij hoe overheden
financiële netwerken stap voor stap zouden overnemen, uiteindelijk met
als doel de gelduitgevende instantie te beheersen: `Net zoals
bijvoorbeeld de Federal Reservewet van 1913, is het doel altijd om het
netwerk op te rollen in een enkele 'te groot om te falen' entiteit, en
zij zijn steeds groter, serieuzer en rampzaliger geworden.'

Bitcoin, voorspelde Donald, zou onderworpen worden aan hetzelfde type
druk.

`Als een klein aantal instanties nieuwe munten uitgeeft, is dit beter
bestand tegen staatsaanvallen dan bij een enkele uitgever, maar de
overheid valt regelmatig financiële netwerken aan, met de financiële
instorting die voortkomt uit de meest recente aanval die nog steeds aan
de gang is terwijl ik dit schrijf', betoogde hij.

Om het systeem gedecentraliseerd te houden en ervoor te zorgen dat de
meeste gebruikers alle transacties in het netwerk kunnen verwerken,
stelde de Cypherpunk voor dat Bitcoin baat zou kunnen hebben bij een
betalingslaag voor transacties van lage waarde. Alleen grote transacties
zouden dan door alle gebruikers verwerkt en opgeslagen hoeven te worden
in de block chain.

`Ik denk dat we ons moeten bezighouden met het minimaliseren van de
gegevens en bandbreedte die gelduitgevers nodig hebben --- voor kleine
munten lijkt het protocol verspillend. Het zou mooi zijn om het
volledige protocol voor grote munten te hebben, en een soort
snelkoppeling voor kleine munten waarbij mensen accountgebaseerd geld
vertrouwen voor kleine bedragen totdat ze worden omgezet in grote
munten', schreef hij. `Hoe kleiner de gegevensopslag en bandbreedte die
gelduitgevers nodig hebben, hoe sterker het systeem bestand is tegen het
soort overheidsaanvallen op financiële netwerken die we recentelijk
hebben gezien.' \footnote{James A. Donald, `Bitcoin P2P e-cash paper',
  oorspronkelijk verstuurd naar de Cypherpunk-mailinglijst, 3 november
  2008,
  \href{https://www.metzdowd.com/pipermail/cryptography/2008-November/014819.html}{online}}

Nakamoto heeft niet specifiek gereageerd op Donalds suggestie over de
betalingslaag, maar hij heeft wel zeker de mogelijke aanvallen op
staatsniveau aangepakt die Bitcoin uiteindelijk zou kunnen tegenkomen.
Hoewel de mysterieuze auteur van het nieuwe witboek in zijn geschriften
een tamelijk wetenschappelijke en zakelijke benadering van het onderwerp
heeft uitgedragen, bevestigde Nakamoto nu onmiskenbaar de motivatie
achter het gedecentraliseerde ontwerp van het systeem.

`Ja, maar we kunnen een grote slag winnen in de wapenwedloop en enkele
jaren een nieuw territorium van vrijheid verwerven', schreef hij.
`Overheden zijn goed in het elimineren van de leiders van {[}\ldots{]}
centraal gecontroleerde netwerken zoals Napster, maar pure peer-to-peer
netwerken zoals Gnutella en Tor lijken zich goed staande te
houden.'\footnote{Satoshi Nakamoto, `Bitcoin P2P e-cash paper',
  oorspronkelijk verstuurd naar de Cypherpunk-mailinglijst, 6 november
  2008,
  \href{https://www.metzdowd.com/pipermail/cryptography/2008-November/014823.html}{online}}

\section{Zorgen en verwarring}\label{zorgen-en-verwarring}

De eerste reactie die Nakamoto kreeg --- de bezorgdheid van James A.
Donald over schaalbaarheid --- was zeker niet onterecht; het opschalen
van Bitcoin om miljoenen, of zelfs miljarden gebruikers te dienen, zou
inderdaad een grote uitdaging worden. Maar veel van de feedback die
volgde op deze reactie was meer uiteenlopend , waarbij sommigen
duidelijk in de war waren over het ontwerp van Bitcoin.

Ray Dillinger, een informaticus en Cypherpunk die een van de eerste
bijdragers was aan de Cryptography-mailinglijst, wees bijvoorbeeld
Bitcoin af vanwege het inflatiepercentage van 35\% --- hoewel er in het
witboek geen inflatieschema werd genoemd. Hij ging er onterecht van uit
dat de uitgifte van nieuwe munten zou toenemen naarmate
computerhardwareprestatie door de jaren heen zou verbeteren, zoals het
geval was geweest bij een systeem zoals RPOW.\footnote{Ray Dillinger,
  `Bitcoin P2P e-cash paper', oorspronkelijk verstuurd naar de
  Cypherpunk-mailinglijst, 6 november 2008,
  \href{https://www.metzdowd.com/pipermail/cryptography/2008-November/014822.html}{online}}

Als mogelijke oplossing voor het inflatieprobleem, stelde Dillinger in
een latere e-mail voor dat Bitcoin een moeilijkheidaanpassingsalgoritme
zou moeten hebben. Hij leek zich er echter niet van bewust dat dit al
onderdeel was van Nakamotos ontwerp.\footnote{Ray Dillinger, `Bitcoin
  P2P e-cash paper', oorspronkelijk verstuurd naar de
  Cypherpunk-mailinglijst, 14 november 2008,
  \href{https://www.metzdowd.com/pipermail/cryptography/2008-November/014857.html}{online}}

Ondertussen beweerde Donald dat Nakamotos
moeilijkheidaanpassingsalgoritme helemaal niet zou werken. Hij leek te
geloven dat dit volledig de prikkel zou wegnemen om nieuwe blokken te
minen, hoewel hij in zijn e-mail niet uitlegde waarom.\footnote{James A.
  Donald, `Bitcoin P2P e-cash paper', oorspronkelijk verstuurd naar de
  Cypherpunk-mailinglijst, 9 november 2008,
  \href{https://www.metzdowd.com/pipermail/cryptography/2008-November/014837.html}{online}}

Zowel Dillinger als Donald waren het er echter over eens dat het
proof-of-work-consensusmechanisme van Bitcoin niet robuust of snel
genoeg was. Ze hielden niet van het idee dat transacties omkeerbaar
konden zijn, in het geval dat de block chain wordt ingehaald door een
langere concurrerende ketting, en vonden het wachten op meerdere
blokbevestigingen geen degelijke oplossing.

`Hoe weet iemand wanneer een transactie onomkeerbaar is geworden?' vroeg
Dillinger, retorisch. `Is 'een paar' blokken drie? Dertig?
Honderd?'\footnote{Ray Dillinger, `Bitcoin P2P e-cash paper',
  oorspronkelijk verstuurd naar de Cypherpunk-mailinglijst, 15 november
  2008,
  \href{https://www.metzdowd.com/pipermail/cryptography/2008-November/014859.html}{online}}

Wat het `juiste' nummer ook zou zijn, het zou niet werken, voorspelde
hij: noch consumenten noch verkopers zouden bereid zijn om `een uur' te
wachten tot transacties waren afgerond.

Donald deelde die assumptie: `We willen dat mensen zeker zijn dat hun
transactie geldig is, en dat dit van dezelfde duur is als tijdens een
uitgave om het netwerk te overspoelen, niet op het moment dat het nodig
is om vertakkingswedstrijden op te lossen.'\footnote{James A. Donald,
  `Bitcoin P2P e-cash paper', oorspronkelijk verstuurd naar de
  Cypherpunk-mailinglijst, 9 november 2008,
  \href{https://www.metzdowd.com/pipermail/cryptography/2008-November/014841.html}{online}}

John Levine, een andere informaticus die eerder de
Cypherpunks-mailinglijst bezocht, maar later overstapte naar de
Cryptography-mailinglijst, zette ook vraagtekens bij het consensusmodel
van Bitcoin, maar om een andere reden. Levine voorspelde dat het
proof-of-work-consensusmodel niet erg veilig zou zijn tegen aanvallers.

`Slechteriken hebben regelmatig controle over zombienetwerken van meer
dan 100.000 machines. Mensen die ik ken houden een zwarte lijst bij van
computers die spam verspreiden, en zij vertellen me dat er vaak wel een
miljoen nieuwe zombies per dag zijn', schreef Levine. `Dit is dezelfde
reden waarom hashcash niet kan werken op het internet van vandaag --- de
brave mensen hebben aanzienlijk minder rekenkracht dan de
slechteriken.'\footnote{John Levine, `Bitcoin P2P e-cash paper',
  oorspronkelijk verstuurd naar de Cypherpunk-mailinglijst, 3 november
  2008,
  \href{https://www.metzdowd.com/pipermail/cryptography/2008-November/014817.html}{online}}

Echter, niet iedereen op de Cryptography-mailinglijst was bereid Bitcoin
af te doen als een foutief ontwerp.

\section{Optimisme}\label{optimisme}

Op 7 november, ongeveer een week nadat Nakamoto zijn witboek openbaar
had gemaakt, kwam er ook een opvallend optimistische reactie naar voren
op de lijst.

`Bitcoin lijkt een veelbelovend idee te zijn', begint Hal Finney zijn
e-mail, en duidt vervolgens nauwkeurig de twee voornaamste innovaties
aan in vergelijking met voorgaande elektronische betaalsystemen. `Ik
waardeer het idee dat de beveiliging is gebaseerd op de veronderstelling
dat de CPU-kracht van eerlijke deelnemers die van de aanvaller
overstijgt', schreef hij. `Ik denk ook dat er potentieel is in de vorm
van een onvervalsbare token waarvan de productiesnelheid voorspelbaar is
en niet kan worden beïnvloed door corrupte partijen. Dit zou meer
vergelijkbaar zijn met goud dan met fiatvaluta. Nick Szabo schreef al
vele jaren geleden over wat hij 'bit gold' noemde en dit kan een
implementatie van dat concept zijn.'\footnote{Hal Finney, `Bitcoin P2P
  e-cash paper', oorspronkelijk verstuurd naar de
  Cypherpunk-mailinglijst, 7 november 2008,
  \href{https://www.metzdowd.com/pipermail/cryptography/2008-November/014827.html}{online}}

Net als Donald haalde Finney ook meteen aan dat Bitcoin wellicht baat
kon hebben bij een lichtgewicht betalingssysteem boven op het bestaande
protocol. Naast verbeterde schaalcapaciteit gaf de doorgewinterde expert
op het gebied van elektronisch geld aan dat dit het systeem sterker zou
maken en meer privacyfuncties zou kunnen bieden.

`Er zijn ook voorstellen gedaan om lichtgewicht anonieme
betalingssystemen te bouwen bovenop zwaargewicht niet-anonieme systemen,
zodat Bitcoin zou kunnen worden ingezet om anonimiteit mogelijk te
maken, zelfs verder dan de mechanismen die in het witboek worden
besproken', schreef hij.\footnote{Hal Finney, `Bitcoin P2P e-cash
  paper', oorspronkelijk verstuurd naar de Cypherpunk-mailinglijst, 7
  november 2008,
  \href{https://www.metzdowd.com/pipermail/cryptography/2008-November/014827.html}{online}}

Een paar dagen later, in een afzonderlijke e-mail, merkte Finney op dat
een bron van verwarring in de verschillende reacties op de mailinglijst
voortkwam uit het feit dat Bitcoin in feite twee verschillende ideeën
bundelde in één voorstel. Hij legde uit dat Bitcoin allereerst een
poging was om een wereldwijd consistente, maar gedecentraliseerde
database te creëren. Op zijn beurt werd deze database dan gebruikt om
een elektronisch geldsysteem te realiseren. Waar verschillende
deelnemers aan de mailinglijst zich meer op het ene of het andere aspect
focusten en benadrukten dat het aspect waarop ze zich focusten enigszins
imperfect was opgelost, was het vindingrijke aan Bitcoin dat het beide
deed, en dat op een manier waarop ze elkaar aanvulden.

`Het oplossen van het wereldwijde, sterk gedecentraliseerde
databaseprobleem is misschien wel het moeilijkste deel', schreef Finney.
`Het gebruik van 'proof-of-work' als hulpmiddel voor dit doel is een
nieuw idee dat volgens mij zeker verdere evaluatie verdient.'

Zijn eigen e-mail bevatte een deel van deze beoordeling. Terwijl hij
nadacht over de veiligheid van het systeem, bedacht hij dat gebruikers
het netwerk draaiende konden houden, zelfs als het enkel om het
ondersteunen als sociaal nuttig project ging, niet anders dan de soorten
internetprojecten waarbij mensen rekenkracht doneren om medisch
onderzoek te ondersteunen of radiosignalen te analyseren op zoek naar
tekenen van buitenaards leven. `In dit geval lijkt het me dat simpel
altruïsme kan volstaan om het netwerk goed te laten functioneren',
concludeerde Finney. \footnote{Hal Finney, `Bitcoin P2P e-cash paper',
  oorspronkelijk verstuurd naar de Cypherpunk-mailinglijst, 13 november
  2008,
  \href{https://www.metzdowd.com/pipermail/cryptography/2008-November/014848.html}{online}}

Nakamoto was het hiermee eens.

`Het is erg aantrekkelijk vanuit het libertaire standpunt als we het
goed kunnen uitleggen', antwoordde de uitvinder van Bitcoin. `Ik ben
echter beter met code dan met woorden.'\footnote{Satoshi Nakamoto,
  `Bitcoin P2P e-cash paper', oorspronkelijk verstuurd naar de
  Cypherpunk-mailinglijst, 14 november 2008,
  \href{https://www.metzdowd.com/pipermail/cryptography/2008-November/014853.html}{online}}

\section{Het pseudoniem}\label{het-pseudoniem}

De voornamelijk sceptische reacties op het witboek in de
Cryptography-mailinglijst kunnen waarschijnlijk deels worden verklaard
doordat de auteur van het witboek, Satoshi Nakamoto, compleet onbekend
was.

Satoshi Nakamoto had tot dan toe nog geen actieve rol gespeeld op de
Cryptography-mailinglijst, de Cypherpunk-mailinglijst of enige andere
relevante mailinglijst, en niemand onder die naam was ooit naar een
Cypherpunk-bijeenkomst gekomen. Satoshi Nakamoto had nog geen eerdere
digitale geldsystemen voorgesteld, en hij had ook geen andere
opmerkelijke artikels over cryptografie of computertechnologie
gepubliceerd. Wat dat betreft, was Satoshi Nakamoto een onbekende in
Cypherpunk en cryptokringen, en elke keer een onbekende een nieuw
elektronisch geldsysteem aankondigde, had het doorgaans weinig
betekenis.

Maar de uitvinder van Bitcoin had mogelijks meer ervaring in het
vakgebied dan hij liet blijken. Hoewel de meeste abonnees van de
Cryptography-mailinglijst waarschijnlijk aannamen dat ze werden
gecontacteerd door een Japanse man, of op zijn minst een man van Japanse
afkomst, was `Satoshi Nakamoto' hoogstwaarschijnlijk een pseudoniem.
Degene die achter deze schuilnaam zat, was mogelijk wel één of meerdere
van de vooraanstaande bijdragers aan de Cryptography-mailinglijst, of de
Cypherpunk-lijst daarvoor.

Aan de andere kant, hij, zij, of zij (als groep) hadden net zo nieuw en
onervaren kunnen zijn in het domein van elektronisch geld als hun
pseudoniem aanduidde.

Wat de waarheid ook is, de entiteit simpelweg bekend als Satoshi
Nakamoto leek niet bijzonder gehinderd te zijn door de sceptische
reacties. Hoewel zijn witboek een zeer beknopt overzicht was van het
Bitcoinprotocol, erkende hij dat hij veel functionele details weggelaten
had. Daarom beantwoordde hij geduldig de meeste zorgen en verwarring
over zijn voorstel en nam hij de tijd om alle delen van het ontwerp die
misschien onduidelijk waren opnieuw uit te leggen.

Hoewel hij niet elk detail in het witboek had opgenomen, had hij over de
meeste ervan nagedacht. In een van zijn e-mailreacties verduidelijkte
Nakamoto dat hij de ontwikkeling van Bitcoin `achterstevoren' had
benaderd: hij had zelfs de meeste Bitcoincode geschreven nog voor het
opstellen van het witboek.\footnote{Satoshi Nakamoto, `Bitcoin P2P
  e-cash paper', oorspronkelijk verstuurd naar de
  Cypherpunk-mailinglijst, 8 november 2008,
  \href{https://www.metzdowd.com/pipermail/cryptography/2008-November/014832.html}{online}}

Inderdaad, Bitcoin was niet zomaar een voorstel, zoals dat bij Bit Gold
en b-money het geval was. Satoshi Nakamoto had al twee jaar gespendeerd
aan de implementatie van het idee in code. Na iets meer dan twee weken
discussies met een handvol respondenten, bood hij aan om de
belangrijkste bestanden te sturen naar abonnees van de
Cryptography-mailinglijst als zij daarom vroegen. De volledige release,
zo beloofde hij, zou spoedig volgen.

Voor lijstbeheerder Perry Metzger --- een andere vroege Cypherpunk ---
was het een goed moment om een pauze in te lassen voor Bitcoin.

`Ik zou graag voorlopig een einde willen maken aan de bitcoin
e-cash-discussie --- er wordt veel gediscussieerd en dat zou beter
kunnen als mensen op dit moment op zichzelf schrijven in plaats van
dingen heen en weer te herhalen', schreef Metzger. `Misschien kunnen we
hier later op terugkomen wanneer Satoshi (of iemand anders) iets in
detail opstelt en het publiceert.'\footnote{Perry E. Metzger, `ADMIN:
  end of bitcoin discussion for now', oorspronkelijk verstuurd naar de
  Cypherpunk-mailinglijst, 17 november 2008,
  \href{https://www.metzdowd.com/pipermail/cryptography/2008-November/014867.html}{online}}

\chapter{De release}\label{de-release}

\begin{quote}
`\textbf{Wet, taal, geld}: de drie paradigma's van spontaan ontstane
instituties. Gelukkig hebben wet en taal zich mogen ontwikkelen. Geld is
ontstaan in zijn oorspronkelijke vorm, maar zodra het er in zijn meest
primitieve vorm was, werd het bevroren. Overheden zeiden dat het niet
verder mocht ontwikkelen. En wat we sinds die ontwikkeling hebben gehad,
waren zaken van overheidsuitvindingen, meestal verkeerde, meestal
misbruik van geld, en ik ben tot het punt gekomen dat ik me afvraag:
heeft monetair beleid ooit enig goed gedaan? Ik denk het niet. Ik denk
dat het alleen maar schade heeft aangericht. Dat is waarom ik nu pleit
voor wat ik 'denationalisering van geld' heb genoemd.'\footnote{Friedrich
  A. Hayek, `F. A. Hayek on Monetary Policy, the Gold Standard,
  Deficits, Inflation, and John Maynard Keynes', interview door James U.
  Blanchard III, opnieuw gepubliceerd door \emph{Libertarianism.org} op
  19 april 2015,
  \href{https://www.youtube.com/watch?v=EYhEDxFwFRU}{online}}
\end{quote}

In een van zijn laatst opgenomen interviews in 1984 aan de Universiteit
van Freiburg bleef de op leeftijd geraakte Friedrich Hayek pleiten voor
radicale monetaire hervormingen. De econoom was er nog steeds van
overtuigd dat fiatvaluta en het rentebeleid van centrale banken de
economie vergiftigden, en dat geld uiteindelijk het beste kon worden
overgelaten aan de vrije markt.

Echter, in de acht jaar sinds de publicatie van \emph{Denationalisation
of Money}, was de Oostenrijker nog minder hoopvol geworden dat bestaande
regeringen bereid zouden zijn om wetten aan te passen om concurrentie
tussen valuta's mogelijk te maken. Hij vermoedde dat ze te veel voordeel
haalden uit de status quo.

`Ik geloof nog steeds dat mijn oorspronkelijke plan juist is, maar ik
vrees dat ik tot de conclusie ben gekomen dat het politiek gezien
volledig utopisch is', legde Hayek nuchter uit. `Overheden zullen het
nooit toestaan, en zelfs bankiers begrijpen het idee niet, omdat ze
allemaal zijn opgegroeid in een systeem waarin ze zo volledig
afhankelijk zijn van centrale banken, overheidsinstellingen, als
kredietverstrekkers in uiterste nood.'\footnote{Hayek, interview.}

En toch koesterde de toen vierentachtigjarige econoom nog steeds de hoop
dat geld gerepareerd kon worden. Maar het vereiste een andere aanpak dan
het soort burgerbeweging dat hij in zijn boek beschreef. Aangezien
overheden de beperkingen die vrijemarktcompetitie voor valuta
belemmerden, niet zouden wegnemen, suggereerde hij dat mensen creatief
moesten zijn en een manier moesten vinden om deze beperkingen te
omzeilen.

In plaats van te proberen overheden te overtuigen hun feitelijke
monopolie op geld op te geven, zouden mensen \emph{op een of andere
sluwe, indirecte manier iets moeten introduceren dat ze niet kunnen
stoppen}.

Toen Satoshi Nakamoto bijna vijfentwintig jaar later, op 8 januari 2009,
terugkeerde naar de Cryptography-mailinglijst om een volledig
vertrouwensloos en volledig peer-to-peer elektronisch geldsysteem te
starten, deed hij exact wat Hayek suggereerde. Hij maakte gebruik van
tientallen jaren onderzoek in privacytechnologie, de architectuur van
gedecentraliseerde netwerken en digitale valutasystemen.

Satoshi Nakamoto introduceerde iets wat regeringen niet kunnen stoppen.

\section{De Codebasis}\label{de-codebasis}

\emph{Bitcoin v.0.1 is uitgebracht}, luidde de titel van Nakamotos
e-mail deze keer.\footnote{Satoshi Nakamoto, `Bitcoin v0.1 released',
  oorspronkelijk verstuurd naar de Cypherpunk-mailinglijst, 8 januari
  2009,
  \href{https://www.mail-archive.com/cryptography@metzdowd.com/msg10142.html}{online}}

De abonnees van de Cryptography-mailinglijst die de e-mail openden,
troffen er een tweezinnige beschrijving van Satoshi Nakamoto aan over
het net gelanceerde project:

\begin{quote}
`Hierbij kondig ik de eerste release van Bitcoin aan, een nieuw
elektronisch geldsysteem dat een peer-to-peer netwerk gebruikt om
dubbele uitgaven te voorkomen. Het is volledig gedecentraliseerd zonder
server of centrale autoriteit.'
\end{quote}

Naast de korte beschrijving, bevatte de e-mail een downloadlink voor de
software, de link naar de website van het project --- bitcoin.org --- en
verschillende alinea's met aanvullende informatie, disclaimers (`de
software is nog in alfaversie en experimenteel') en basisinstructies
voor hoe het te gebruiken.

Iets meer dan twee maanden nadat hij zijn witboek aan de
Cryptography-mailinglijst had voorgelegd, maakte Satoshi Nakamoto de
eerste versie van de Bitcoinsoftware openbaar. Het programma, bekend als
Bitcoin versie 0.1, was klaar om gedownload en gebruikt te worden: men
kon sleutelparen aanmaken, transacties uitvoeren, en blokken minen.

De release van de software onthulde ook belangrijke nieuwe informatie
over het project.

Wat meteen opviel --- hoewel het geen grote verrassing was --- was dat
Satoshi Nakamoto Bitcoin had vrijgegeven als vrije en open
source-software. Iedereen was vrij om de code te kopiëren, te gebruiken,
te delen en te wijzigen. Gepubliceerd onder de MIT-licentie, konden
zelfs commerciële projecten Nakamotos werk integreren (dit maakt de
MIT-licentie minder restrictief dan Richard Stallmans GPL-licentie, die
deze vrijheid enkel toekent aan andere vrije softwareprojecten).

Het was essentieel dat Bitcoin vrije en open source-software was, omdat
de code noodzakelijkerwijs controleerbaar moest zijn: om het systeem
echt vertrouwensloos te laten functioneren, zouden gebruikers moeten
kunnen verifiëren dat het werkt zoals beloofd. Dit was wellicht nog
crucialer voor Bitcoin dan voor vele andere softwareprojecten, aangezien
de code letterlijk geld vertegenwoordigde. Passend bij de filosofie van
Stallman voor vrije software, zouden mensen Satoshi Nakamoto niet moeten
vertrouwen om geen malware in te bouewen om munten te stelen of een
geheime achterdeur voor het bijdrukken van geld te introduceren.

Meer in het algemeen maakte Nakamotos vrije en open source-code,
geschreven in de programmeertaal C++ en jaren later door de eerste
voltijdse Bitcoinontwikkelaar beschreven als `briljant maar slordig',
voor het eerst volledig inzichtelijk hoe het elektronische geldsysteem
intern werkte.\footnote{Michael J. Casey, `Bitcoin Foundation's Andresen
  on Working With Satoshi Nakamoto', \emph{The Wall Street Journal}, 6
  maart 2014, \href{https://www.wsj.com/articles/BL-MBB-17626}{online}}

Transacties bleken bijvoorbeeld gebruik te maken van `Script', een
nieuwe programmeertaal voor Bitcoin die geïnspireerd was op Forth, een
programmeertaal origineel ontworpen in de jaren 1960 om radiotelescopen
te bedienen. Met enkele aanpassingen aan de functionaliteit van Forth,
kon Script gebruikt worden om eenvoudige slimme contracten op Bitcoin te
schrijven. Munten konden zodanig worden opgeslagen dat ze alleen
verplaatst konden worden als er aan bepaalde programmeerbare condities
werd voldaan (een eenvoudig voorbeeld hiervan was `multisignature', of
\emph{multisig}, waarbij niet één, maar meerdere cryptografische
handtekeningen vereist waren om de munten uit te geven).

Het handtekeningsysteem dat ingebed is in Bitcoin was de Elliptic Curve
Digital Signature Algorithm (ECDSA), die, zoals de naam al suggereert,
wiskundig gegenereerde elliptische curven gebruikte om sleutelparen te
berekenen. Het was uitgevonden in 1985, zo'n acht jaar na RSA. De
elliptische curve-cryptografie bood hetzelfde beveiligingsniveau als de
oplossing van Rivest, Shamir en Adlemen, maar vereiste veel kleinere
sleutelgroottes en was in de loop der jaren een gangbaar alternatief
geworden.

Tegelijkertijd had Nakamoto een aantal functies toegevoegd om Bitcoin
wat gebruiksvriendelijker te maken. Hoewel betalingen technisch gezien
nog altijd naar publieke sleutels gedaan werden, konden gebruikers hun
publieke sleutel (of de hash van hun publieke sleutel) coderen in een
Bitcoin-\emph{adres}. Wanneer ze geld ontvingen, deelden ze doorgaans
alleen deze adressen met andere gebruikers.

De codebasis van Bitcoin onthulde ook veel van de min of meer
willekeurige parameters die Nakamoto had gekozen. Zoals hij al eerder op
de Cryptography-mailinglijst had gesuggereerd, zou er gemiddeld elke
tien minuten een nieuw blok gevonden moeten worden. Clusters van 2.016
blokken zouden vervolgens worden gebruikt om de moeilijkheidsgraad van
minen aan te passen: als de 2.016 blokken in minder dan twee weken
werden gevonden, zou de moeilijkheidsgraad van Bitcoin proportioneel
worden verhoogd, en als het meer dan twee weken duurde om de 2.016
blokken te vinden, zou de moeilijkheidsgraad van het proof-of-work naar
beneden worden bijgesteld.

En om het Bitcoinnetwerk daadwerkelijk op gang te brengen, bevatte de
codebasis ook de allereerste blok: het `Genesisblok'. Dit blok moest
inderdaad ingebed worden in de release zelf; de block chain had een
startpunt nodig. Een interessant detail is echter dat de beloning voor
dit Genesisblok in feite waardeloos was: de protocolregels stonden niet
toe dat deze specifieke munten onder enige voorwaarde uitgegeven konden
worden. In Bitcoin kunnen nieuwe munten alleen worden verdiend door
competitief te minen, en Nakamoto weigerde een voorsprong van één blok
voor zichzelf te accepteren. Als hij munten wilde hebben, moest zelfs
hij, de maker van het systeem, ze verdienen --- net als alle anderen.

Nakamoto benadrukte nog eens zijn expliciete weigering om enig oneerlijk
voordeel te genieten boven andere Bitcoingebruikers. Hij had ook een
bewijs toegevoegd dat hij in de weken of maanden voorafgaand aan het
openbaar maken van de code niet privé had gemined. Hij had een kop van
de voorpagina van de Engelse krant \emph{The Times} van 3 januari in het
Genesisblok opgenomen, wat aantoont dat deze blok niet voor die datum
kon gecreëerd zijn. Dit betekent op zijn beurt dat elk daaropvolgend
blok later gemined moest zijn.\footnote{Aangezien Bitcoin uiteindelijk
  op 8 januari werd uitgebracht, zou dit de maker van het systeem een
  maximum van zes dagen hebben gegeven om comfortabel munten te
  genereren zonder enige concurrentie, maar vroege block-timestamps
  geven aan dat hij dit ook niet heeft gedaan.}

Als kers op de taart leek de specifieke kop ook niet willekeurig voor
dit doel gekozen te zijn:

\begin{quote}
The Times 03/Jan/2009 Chancellor on brink of second bailout for banks
\end{quote}

De wereld van geld en financiën was een puinhoop geworden. Met Bitcoin
stelde Satoshi Nakamoto een alternatief voor.

\section{Eenentwintig miljoen}\label{eenentwintig-miljoen}

Het interessantste nieuw onthulde kenmerk was echter het `monetaire
beleid' van Bitcoin.

Het witboek beschreef hoe Bitcoin een voorspelbaar uitgifteschema kon
garanderen, dankzij vaste blokbeloningen en het
moeilijkheidsaanpassingsalgoritme. Maar het document gaf nog niet
precies aan hoe dit schema eruit zou zien.

Het bleek nu dat de code van Bitcoin zo was geprogrammeerd dat het
aantal nieuwe munten die per blok werden toegekend halveerde na elke
210.000 blokken, of ongeveer eens in de vier jaar. In de eerste vier
jaar zouden miners 50 nieuwe munten per blok verdienen, maar in de vier
jaar daarna zouden ze slechts 25 nieuwe munten per blok verdienen. In de
volgende vier jaar zou dat 12,5 zijn, daarna 6,25 in de vier
daaropvolgende jaren, en zo verder.

Nakamoto kondigde in zijn e-mail aan:

De totale circulatie zal bestaan uit 21.000.000 munten. Ze worden
verdeeld onder de netwerknodes wanneer ze blokken maken, waarbij de
hoeveelheid elke 4 jaar gehalveerd wordt.

\emph{In de eerste 4 jaren: 10,500,000 munten}

\emph{In de volgende 4 jaren: 5.250.000 munten.}

\emph{In de volgende 4 jaren : 2.625.000 munten.}

\emph{In de volgende 4 jaren : 1.312.500 munten.}

\emph{Wanneer dit tot zijn einde komt, kan het systeem transactiekosten
invoeren, indien nodig.}

Eenentwintig miljoen munten.\footnote{Het is echter belangrijk op te
  merken dat elke bitcoin tot acht decimalen kan worden opgedeeld,
  waardoor er in zekere zin 2,1 biljard valutadelen zijn. De kleinste
  eenheid, 0,00000001 bitcoin, wordt tegenwoordig meestal een `satoshi'
  genoemd, of kortweg `sat'.} Bitcoin was ontworpen met een vaste
voorraad.

De gevolgen hiervan waren waarschijnlijk groter dan veel abonnees van de
Cryptography-mailinglijst zich realiseerden.

Slechts enkele jaren eerder was het RPOW-project van Hal Finney mislukt,
grotendeels omdat mensen geen economische prikkel hadden (zelfs een
negatieve economische prikkel) om RPOW-tokens te bezitten. Dit betekende
dat vrijwel niemand bereid was om ze als betaling te accepteren. Doordat
er bijna geen plaatsen waren om ze uit te geven, waren de tokens
praktisch nutteloos en daardoor waardeloos, wat betekende dat ze niet
echt als geld gebruikt konden worden. Net als eCash en hashcash, leed
RPOW onder een kip-en-ei-probleem dat het niet had kunnen overwinnen.

Bitcoin moest zichzelf ook opstarten vanaf nul. Toen Nakamoto zijn code
voor het eerst vrijgaf, werd bitcoin natuurlijk nergens als betaalmiddel
geaccepteerd en hadden deze munten geen monetaire waarde.

Maar het was net Hal Finney die besefte dat de stimulansen deze keer
enigszins anders zaten.

Op 10 januari, twee dagen na de release van Bitcoin, was Finney de
eerste persoon op de Cryptography-mailinglijst die reageerde op de
aankondigingsmail. Na Nakamoto te feliciteren met de release en te
beloven het te proberen, richtte de veteraan in de wereld van
elektronisch geld zijn aandacht snel op de vaste geldvoorraad van
Bitcoin.

`Een direct probleem met elke nieuwe munteenheid is hoe deze te
waarderen', schreef hij. `Zelfs als we het praktische probleem negeren
dat vrijwel niemand het in het begin zal accepteren, is er nog steeds
een uitdaging om een redelijk argument te vinden ten gunste van een
specifieke niet-nul waarde voor de munten.'

Maar Finney, die goed onderlegd was in de wereld van statistiek en
kansberekening, geloofde dat de vaste voorraad van Bitcoin de oplossing
kon bieden. Het stelde mensen in staat om eenvoudige inschattingen te
maken over de potentiële toekomstige waarde van de munten.

`Stel je voor dat Bitcoin succesvol is en het dominante betalingssysteem
in de wereld wordt. Dan zou de totale waarde van de valuta gelijk moeten
zijn aan de totale waarde van alle rijkdom ter wereld. Actuele
schattingen van het totale wereldwijde vermogen liggen tussen de 100 en
300 biljoen dollar. Met 20 miljoen munten zou elke munt een waarde
hebben van ongeveer 10 miljoen dollar', berekende hij.

`Dus de kans om vandaag de dag munten te genereren met slechts enkele
centen aan computertijd kan een zeer goede 'gok' zijn, met een mogelijke
uitbetaling van iets als 100 miljoen tegen 1! Zelfs als de kans klein is
dat Bitcoin in deze mate succesvol wordt, zijn ze echt 100 miljoen tegen
één? Iets om over na te denken\ldots'\footnote{Hal Finney, `Bitcoin v0.1
  released', oorspronkelijk verstuurd naar de Cypherpunk-mailinglijst,
  10 januari 2009,
  \href{https://www.metzdowd.com/pipermail/cryptography/2009-January/015004.html}{online}}

Het was inderdaad iets om over na te denken. De schattingen van Finney
waren natuurlijk ruw, het was slechts wat rekenwerk op een kladblad.
Maar zolang de kans dat Bitcoin in de toekomst zou slagen niet nul was,
zou het inderdaad rationeel zijn om voor een goedkope prijs wat munten
te bemachtigen.

Als anderen dezelfde redenering volgden, zou dat meteen de vraag naar de
munten doen toenemen, ruwweg tot het punt waar de markt inschat dat de
verhouding tussen risico en beloning het nog steeds waard zou zijn. De
potentiële toekomstige waarde van een bitcoin, en de geschatte kans dat
deze toekomst werkelijkheid wordt, zou in feite in de huidige marktprijs
moeten worden weerspiegeld.

En zodra er een marktprijs voor de munten is vastgesteld --- dat wil
zeggen, \emph{elke} niet-nul marktprijs --- kunnen ze daadwerkelijk ook
als een vorm van geld gebruikt beginnen worden, waarschijnlijk eerst in
plaatsen zonder enige alternatieven.

In een vervolgmail suggereerde Nakamoto: `{[}\ldots{]} zoals
beloningspunten, donatietokens, valuta voor een spel of micropayments
voor volwassenensites.'\footnote{Satoshi Nakamoto, `Bitcoin v0.1
  released', oorspronkelijk verstuurd naar de Cypherpunk-mailinglijst,
  16 januari 2009,
  \href{https://www.metzdowd.com/pipermail/cryptography/2009-January/015014.html}{online}}

Dit zou de vraag \emph{nog} meer moeten stimuleren. Hierdoor zou bitcoin
in feite het klassieke kip-en-ei-probleem, waar eerdere digitale
geldprojecten onder te lijden hadden, kunnen overwinnen. Bitcoin zou
zelfs een \emph{positieve} feedbacklus kunnen vertonen!

In zekere zin had Finney het regressietheorema van Ludwig von Mises
ondersteboven gekeerd: in plaats van de waarde van een valuta te
herleiden uit de koopkracht in het \emph{verleden}, stelde de cypherpunk
voor dat een valuta's waarde in eerste instantie kan worden afgeleid uit
de verwachte koopkracht in de \emph{toekomst}.

`Het zou logisch kunnen zijn om er gewoon wat te bemachtigen in het
geval dat het aanslaat', stemde Satoshi Nakamoto toe. `Als genoeg mensen
op dezelfde manier denken, wordt dat een \emph{self-fulfilling
prophecy}.'\footnote{Satoshi Nakamoto, `Bitcoin v0.1 released',
  oorspronkelijk verstuurd naar de Cypherpunk-mailinglijst, 16 januari
  2009,
  \href{https://www.metzdowd.com/pipermail/cryptography/2009-January/015014.html}{online}}

\section{Hayeks ideaal}\label{hayeks-ideaal}

Het oplossen van het kip-en-ei-probleem was niet het enige potentiële
voordeel van de limiet van eenentwintig miljoen. Het was misschien zelfs
niet het grootste voordeel, zeker niet op een grotere tijdschaal. Als de
analyse van Friedrich Hayek over de monetaire economie aan het begin van
zijn carrière correct was, \emph{zou Bitcoin de economie kunnen helpen
te herstellen.}

Bitcoin, als een ongedekte valuta die zonder centrale bank werkt, was
een volledig homogene vorm van geld. Iedereen kon zijn eigen munten
beheren, en er waren geen reserveverhoudingen om zich zorgen over te
maken. En aangezien bitcoin op het internet bestond, kende het geen
grenzen. Iedereen met een internetverbinding, waar ook ter wereld, kon
de software downloaden en beginnen met het verzenden en ontvangen van
transacties naar wie dan ook.

Deze combinatie --- een homogeen, grenzeloos geld met een vaste voorraad
--- is wat Hayek ooit beschreef als \emph{neutraal geld}.

Toen hij hier voor het eerst over schreef, beschouwde Hayek neutraal
geld als een onbereikbaar ideaal, een perfecte valuta die eigenlijk niet
gerealiseerd kon worden. Belangrijker nog, hij geloofde niet dat er een
internationale autoriteit was die te vertrouwen was om zo'n valuta uit
te geven. De econoom dacht dat, op een zeer fundamenteel niveau, naties
niet op elkaar konden vertrouwen om de vaste voorraad te eren. En hij
had waarschijnlijk gelijk. In voldoende extreme omstandigheden (zoals
oorlog), zouden degenen die controle over de geldprinter hadden altijd
in de verleiding komen om dit privilege te misbruiken, ongeacht eerdere
afspraken, conventies of verdragen.

Maar Bitcoin werd niet uitgegeven door dergelijke internationale
autoriteit: er was geen geldprinter om controle over te hebben. Satoshi
Nakamoto had het systeem zo ontworpen dat er absoluut geen vertrouwde
derde partij nodig was. Bitcoin was vanaf het begin niet afhankelijk van
monetaire afspraken, conventies of verdragen, dus er waren ook geen
afspraken, conventies of verdragen om te verbreken. Althans in theorie,
verwezenlijkte Bitcoin wat Hayek onmogelijk achtte.

Dit betekende dat als bitcoin enige kans zou hebben om de wereldwijde
valuta te worden, het potentieel --- zoals geschetst door Hayek in de
jaren 1920 en 1930 --- enorm zou zijn.

Om te beginnen zou Bitcoin een einde kunnen maken aan monetair
nationalisme. Als de elektronische valuta van Nakamoto op grote schaal
werd geaccepteerd in internationale handel, zouden gezamenlijke
prijswijzigingen tussen landen eindelijk nauwkeurige signalen aan de
markt kunnen geven, wat de optimale toewijzing van middelen over
landsgrenzen heen mogelijk zou maken, los van nationaliteiten. Bijgevolg
zou Bitcoin ook de internationale handel op een veel directere manier
vergemakkelijken, waarbij alleen de koper en verkoper en (de prijzen
van) hun respectieve producten worden beïnvloed --- niet de prijsniveaus
over hun hele landen.

Bovendien kon Bitcoin een einde maken aan valutaoorlogen. Als de hele
wereld hetzelfde, neutrale geld zou gebruiken, zouden onderlinge
devaluaties en de economische ellende die daaruit voortkomt, voorgoed
tot het verleden behoren.

Bitcoin kon ook een einde maken aan het Cantillon-effect. Vooral als
alle eenentwintig miljoen munten in omloop zijn, zou niemand profiteren
van nieuw geld in omloop te brengen, wat zou leiden tot een
wanverhouding van middelen in hun voordeel. Maar zelfs wanneer er nog
nieuwe munten worden gemined, zou dit in feite geen enkel individu,
groep of specifieke sector bevoordelen. Iedereen zou vrij zijn om te
minen, waardoor (en door het moeilijkheidsaanpassingsalgoritme) vrije
concurrentie de winstmarges tot nul moest drijven: het proof-of-work dat
het zou kosten om een blok te vinden zou ongeveer gelijk moeten zijn aan
de waarde van de blokbeloning.

Maar misschien heeft Bitcoin nog een grotere impact dan dat. Voor het
eerst zou het intertemporele prijssysteem zonder belemmeringen kunnen
functioneren. Als gevolg hiervan zouden de rentetarieven
\emph{eindelijk} de gezamenlijke tijdsvoorkeuren in de maatschappij
weergeven. Dit zou producenten informeren in welk productiestadium ze
moeten investeren, wat een efficiënte toewijzing van middelen doorheen
de tijd vergemakkelijkt. Door zich te verzetten tegen een beleid van
kunstmatige rentetarieven, zou Bitcoin een eind kunnen maken aan het
centraal beheerde monetaire beleid. Volgens de Oostenrijkse
conjunctuurcyclustheorie zou dit de daarmee samenhangende economische
op- en neergangen stoppen.

En dankzij de vaste geldhoeveelheid zou tot slot een verandering in
productiekosten precies weerspiegeld worden in veranderende prijzen. Als
de productiekosten zouden dalen, dan zouden de prijzen hetzelfde doen:
dit was het soort deflatie dat Hayek als natuurlijk en gezond
beschouwde.

Met ongeveer 32.000 regels code legde Satoshi Nakamoto de basis om de
monetaire dogma's van de stabilisatoren, die de dominante Keynesiaanse
en Monetaristische monetaire theorieën bijna een eeuw lang hebben
beïnvloed, te verdringen.

\ldots{} \emph{Als} het alom aanvaard zou worden.

\section{Het laatste punt van
centralisatie}\label{het-laatste-punt-van-centralisatie}

Op 8 januari 2009, begon Bitcoin zijn reis met slechts één gebruiker: de
ontwerper van het digitale valutasysteem zelf. Hoewel het waarschijnlijk
is dat honderden mensen, na de aankondiging van Nakamoto op de
Cryptography-mailinglijst, over het project hadden gehoord, wijzen de
weinige reacties op zijn e-mail en de eerder sceptische reacties erop,
dat in de begindagen slechts een handvol mensen de software
daadwerkelijk hadden geprobeerd (met Hal Finney die op 12 januari de
allereerste transactie ooit van Nakamoto ontving).

Desondanks was het zaadje geplant. Met een vastgestelde geldvoorraad,
semi-anonimiteit, betalingen die resistent zijn tegen censuur, de
mogelijkheid tot simpele slimme contracten, en relatief snelle en
goedkope wereldwijde transacties als inherent onderdeel van Bitcoin, was
het klaar om gebruikt te worden door iedereen die er baat bij dacht te
hebben.

Eén ding was duidelijk: als Bitcoin gebruikers zou aantrekken, zouden
die uit eigen initiatief komen. Bitcoin was een valuta die men gebruikte
als en wanneer men ervoor koos om het te gebruiken, niet omdat iemand
hen daartoe dwong. Terwijl het gebruik van fiatgeld bij wet verplicht
was --- het was het geld dat tenminste voor het betalen van belasting
moest worden gebruikt --- zou het verzenden en ontvangen van bitcoin
volledig vrijwillig zijn.

Uiteindelijk kwamen er inderdaad gebruikers. Ondanks een zeer traag
beginjaar, begon Bitcoin in de loop van 2010 best wat aantrekkingskracht
te vertonen. Het transactievolume begon langzaam toe te nemen, nieuwe
ontwikkelaars ontdekten het project, en er vormde zich een kleine
onlinegemeenschap op een internetforum dat volledig gewijd was aan het
project voor elektronisch geld.

Dat is het moment waarop Satoshi Nakamoto het laatste aanzienlijke punt
van centralisatie uit het project verwijderde: zichzelf.

De pseudonieme maker van Bitcoin had aanvankelijk een leidende rol in de
voortzetting van de softwareontwikkeling en had een grote invloed op de
richting van het vrije en open source-project. Maar toen de digitale
valuta in populariteit begon te groeien, begon Nakamoto zich langzaam
terug te trekken. Uiteindelijk, tegen het einde van 2010, stopte hij
volledig met reageren op berichten en verwijderde hij zijn
contactgegevens van de bitcoin.org-website.

Technisch gezien was het verdwijnen van Nakamoto onbelangrijk. De
mysterieuze ontwikkelaar had eigenlijk geen controle over Bitcoin: het
bestond als een peer-to-peer netwerk dat werd bediend door gebruikers
over de hele wereld. Maar in de praktijk had de maker van het project de
natuurlijke autoriteit om aanpassingen aan de code door te voeren.

Bijgevolg kon Satoshi Nakamoto de regels van het systeem bepalen, en
tijdens zijn periode als hoofdontwikkelaar heeft hij inderdaad enkele
wijzigingen in deze regels aangebracht. Hij verwijderde bijvoorbeeld
functionaliteit uit Script die hij als potentieel gevaarlijk beschouwde,
terwijl hij tegelijkertijd bepaalde beperkingen toevoegde aan het
protocol om de systeemvereisten te beperken en een soepele werking te
garanderen.\footnote{Dit omvatte het meest bekend de 1-megabyte limiet
  voor de blokgrootte, die voorkwam dat de blockchain zo snel zou
  groeien dat de meeste gewone gebruikers geen Bitcoin-node op hun
  thuiscomputer zouden kunnen draaien.}

In de begindagen was dit type leiderschap waarschijnlijk noodzakelijk.
Bitcoin was een klein project met experimentele software en het was
handig om cruciale oplossingen snel en eenzijdig uit te rollen. Maar op
lange termijn zou Nakamotos invloed een risico kunnen vormen: als
projectleider kon hij het doelwit worden van toezichthouders, afpersers
of verschillende vormen van corruptie. Of hij zou zijn verstand kunnen
verliezen en Bitcoin in gevaar brengen door enkel zijn eigen
gemoedstoestand.

Zonder Nakamoto had niemand een vergelijkbaar natuurlijk gezag over het
project. Aan het eind van 2010 werd Bitcoin echt gedecentraliseerd.

\section{Het Bitcoin-project}\label{het-bitcoin-project}

Vandaag de dag wordt de code van Bitcoin onderhouden en ontwikkeld door
een open gemeenschap van vrijwillige programmeurs van over de hele
wereld. Of ze nu bewogen zijn door ideologische redenen, geïnteresseerd
zijn in de technologie, worden gesponsord door een bedrijf dat een
belang heeft in het project, of om een andere reden: ze nemen het op
zich om Bitcoin up-to-date te houden en te verbeteren waar ze maar
kunnen.

Een deel van dit werk bestaat simpelweg uit het updaten van de software.
Iedereen met de juiste vaardigheden kan de bestaande code verbeteren,
nieuwe code toevoegen, of het werk van anderen controleren. Volgens de
Wet van Linus, zou de kwaliteit van de Bitcoinsoftware moeten verbeteren
naarmate meer ontwikkelaars hieraan werken: \emph{met genoeg ogen, zijn
alle bugs oppervlakkig}.

Daarnaast kunnen aan Bitcoins protocol ook nieuwe functies worden
toegevoegd om zo het systeem enkele van de originele beperkingen op
domeinen te laten overwinnen, zoals schaalbaarheid en privacy.
Bijvoorbeeld kan Script uitgebreid worden om nieuwe uitgavevoorwaarden
te bieden, meer soorten slimme contracten mogelijk te maken, en zelfs
volledig nieuwe betalingslagen boven op het basisprotocol mogelijk te
maken, vergelijkbaar met wat James A. Donald en Hal Finney suggereerden
in hun eerste reacties op Nakamotos whitepaper.\footnote{Op dit moment
  is het meest bekende voorbeeld van zo'n betalingslaag het Lightning
  Network.}

Dit betekent niet dat iedereen zomaar elke wijziging kan aanbrengen aan
de code van Bitcoin en die naar eigen inzicht over het netwerk kan
uitrollen. In een samenwerkingsverband moet de ontwikkelingsgemeenschap
over het algemeen akkoord gaan met een wijziging van de originele
codebasis (nu `Bitcoin Core' genoemd), en dit is nog meer het geval
wanneer de wijzigingen effect hebben op regels van het Bitcoinprotocol.
Zonder ruime consensus, zal een verandering meestal niet doorgevoerd
worden.

Dit terzijde, omdat Bitcoin bestaat uit vrije en open source-software,
kan elke ontwikkelaar een kopie (\emph{fork}) van deze codebasis maken
en wijzigingen aanbrengen in deze nieuwe versie van de software. Ze zijn
ook vrij om deze software te gebruiken en te verspreiden onder anderen.
Maar geen enkele ontwikkelaar --- of ze nu bijdragen aan Bitcoin Core of
aan een kopie --- heeft de macht om hun software aan gebruikers op te
dringen.

Inderdaad, het zijn de gebruikers, niet de ontwikkelaars die
uiteindelijk beslissen welke code ze willen gebruiken. Ze kunnen altijd
besluiten een verandering niet te accepteren door te weigeren om een
nieuwe release te downloaden en te installeren, en in plaats daarvan te
blijven werken met de Bitcoinsoftware die ze al gebruikten. Het
tegenovergestelde is ook waar: gebruikers kunnen een nieuwe versie van
de software (of eender welke \emph{fork}) aanvaarden die een verandering
bevat. Bij Bitcoin is niemand de baas\ldots{} en is iedereen de baas.

Het Bitcoin-ecosysteem heeft in de loop der jaren zeker de introductie
van verschillende nieuwe softwareversies gezien. Sommige daarvan zijn
volledige herimplementaties van het Bitcoinprotocol, met een geheel
nieuwe code. Anderen zijn afsplitsingen van Bitcoin Core met enkele
relatief kleine aanpassingen om beter aan persoonlijke voorkeuren te
voldoen. Weer anderen zijn gespecialiseerde programma's die zich
concentreren op een specifieke taak, zoals het minen. En er zijn zelfs
versies van de software die bewust afwijken van de bestaande regels van
het Bitcoinprotocol.

Toch heeft dit niet tot chaos geleid. De meeste gebruikers willen geen
veranderingen aanbrengen die de waarde van hun munten zouden
verminderen, zoals code die inflatie van de valuta introduceert voorbij
de limiet van eenentwintig miljoen, of een versie van de software die
het niet eens zou kunnen worden met de rest van het netwerk.
Integendeel: gebruikers, in hun eigen belang handelend, hebben de
neiging alleen waardevolle veranderingen te accepteren: upgrades die het
protocol sterker maken, nodes efficiënter, en het netwerk
betrouwbaarder.

In de meer dan tien jaar sinds Satoshi Nakamoto vertrok, hebben
ontwikkelaars en gebruikers zich op eigen kracht georganiseerd om
gezamenlijk tot een zeer betrouwbaar Bitcoinprotocol te komen: nieuwe
blokken worden ongeveer elke tien minuten gevonden, splitsingen in de
block chain zijn zeldzaam en kortstondig, en dubbele uitgaven zijn
onbestaande. Ondertussen zijn het aantal gebruiksmogelijkheden, het
totale transactievolume en de marktwaarde van Bitcoin spectaculair
toegenomen.

In een wereld met van bovenaf, centraal beheerde fiatvaluta's en al hun
problemen, vertegenwoordigt Bitcoin een krachtig voorbeeld van spontane
orde.

\bookmarksetup{startatroot}

\chapter*{Erkenningen}\label{erkenningen}
\addcontentsline{toc}{chapter}{Erkenningen}

\markboth{Erkenningen}{Erkenningen}

Ik had dit boek niet kunnen schrijven zonder de hulp die ik van zoveel
mensen heb gekregen.

Allereerst, grote dank aan David Bailey, die me de tijd en kans heeft
gegeven om aan dit boek te werken tijdens mijn tijd bij \emph{Bitcoin
Magazine}.

Vervolgens wil ik mijn redacteuren bedanken. Pete Rizzo, die geduldig
genoeg was om de zeer slordige vroege concepten te lezen en hielp met
het structureren van het verhaal, en Joakim Book, die de tekst liet
glanzen, een aantal fouten ving die niemand anders zag, en me hielp om
alles over de finish te krijgen.

Ik ben ook erg dankbaar voor de steun die ik heb gekregen van andere
collega's bij \emph{Bitcoin Magazine}, namelijk Ellen Sullivan en
Christian Keroles.

Ik had het grote geluk dat een aantal mensen die in het boek voorkomen
beschikbaar waren voor interviews en/of feedback, waaronder (in
alfabetische volgorde) Adam Back, David Chaum, Douglas Jackson, Gregory
Maxwell, Martin Hellman, Nick Szabo, Richard Stallman, Scott Stornetta,
Tom Morrow, Wei Dai en Whitfield Diffie. Hartelijk bedankt!

Speciale dank gaat naar de domeindeskundigen die zo vriendelijk waren om
de vroege hoofdstukconcepten na te kijken, met name Adam Gibson, Bryan
Bishop, Eduard de Jong, Jan Burgers, Tony Klausing, Vijay Boyapati, en
Wolf von Laer.

Om verschillende redenen wil ik ook Austin Hill, Andreas Antonopoulos,
Ferdinando Ametrano, Jurjen Bos, LENA, Marcel van der Peijl, Tuur
Demeester, en Wouter Habraken bedanken.

In mei 2023 heb ik dit boek `open source' gemaakt door de tekst op
Google Docs te publiceren en zo aan iedereen de gelegenheid te geven om
het te lezen en suggesties voor verbeteringen aan te kaarten. Gedurende
de volgende maanden hadden daadwerkelijk een aantal mensen bijgedragen,
sommigen klein, anderen groot. De deelnemers waren onder meer:
0x3phemeralsoul, Antoine Poinsot, Ben Murdock, Bitcoin Graffiti, Fractal
Encrypt, Giacomo Zucco, Haarman Haarman, Info Scholarium, Jake Franklin,
Jake Thomas, Jan-Paul Franken, Joao Bordalo, Jonathan Bier, John Doe,
Leonhard Weese, Ludovic Lars, Marc Bonenberger, Mengu Gulmen, Muhammad
Saqib Arfeen, Nadir Khan, Nick Nell, Pieter Meulenhoff, Richard Hogan,
Thomas Farstrike, Will Wohler, en Zionfuo.

Tot slot wil ik mijn familie en vrienden (in het bijzonder Frederique
Mol) bedanken voor hun steun gedurende de jaren, evenals iedereen die
mij heeft geholpen op mijn Bitcoinreis sinds 2013.

En natuurlijk, bedankt Satoshi Nakamoto, wie je dan ook mag zijn.

Mijn excuses aan iedereen die ik ben vergeten te noemen.

Free Ross.

\bookmarksetup{startatroot}

\chapter*{Over de auteur}\label{over-de-auteur}
\addcontentsline{toc}{chapter}{Over de auteur}

\markboth{Over de auteur}{Over de auteur}

Aaron van Wirdum studeerde Journalistiek aan de Hogeschool Utrecht en
Politiek en Maatschappij in Historisch Perspectief aan de Hogeschool
Utrecht, waar hij zich specifiek richtte op de historische impact van
nieuwe technologieën op maatschappelijke structuren. Hij stuitte op
Bitcoin in 2013, en sindsdien schrijft hij over 's werelds eerste
succesvolle elektronische geldproject. Voor het grootste deel van deze
jaren deed hij dit voor \emph{Bitcoin Magazine}: eerst als journalist,
daarna als technisch redacteur, en uiteindelijk als hoofdredacteur van
de gedrukte editie. \emph{Het Genesis Boek} is zijn eerste boek.


\backmatter


\end{document}
